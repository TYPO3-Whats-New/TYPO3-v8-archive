% ------------------------------------------------------------------------------
% TYPO3 CMS 8.3 - What's New - Chapter "Extbase & Fluid" (Serbian Version)
%
% @author	Patrick Lobacher <patrick@lobacher.de> and Michael Schams <schams.net>
% @license	Creative Commons BY-NC-SA 3.0
% @link		http://typo3.org/download/release-notes/whats-new/
% @language	Serbian
% ------------------------------------------------------------------------------
% LTXE-CHAPTER-UID:		846d40ce-66fdc2ec-750dcf95-33ce93e0
% LTXE-CHAPTER-NAME:	Extbase & Fluid
% ------------------------------------------------------------------------------

\section{Extbase i Fluid}
\begin{frame}[fragile]
	\frametitle{Extbase i Fluid}

	\begin{center}\huge{Poglavlje 4:}\end{center}
	\begin{center}\huge{\color{typo3darkgrey}\textbf{Extbase i Fluid}}\end{center}

\end{frame}

% ------------------------------------------------------------------------------
% LTXE-SLIDE-START
% LTXE-SLIDE-UID:		edadbc54-5255d09c-77f75e6a-25220274
% LTXE-SLIDE-ORIGIN:	06a9e027-1c0e09af-d41d1c4c-c0f07273 English
% LTXE-SLIDE-TITLE:		Feature: #76531 - Add IconForRecordViewHelper
% LTXE-SLIDE-REFERENCE:	Feature-76531-AddIconForRecordViewHelper.rst
% ------------------------------------------------------------------------------
\begin{frame}[fragile]
	\frametitle{Extbase i Fluid}
	\framesubtitle{Dodat IconForRecordViewHelper}

	% decrease font size for code listin
	\lstset{basicstyle=\tiny\ttfamily}

	\begin{itemize}

		\item Dodat je novi ViewHelper za renderanje ikonica za zapise

			\begin{lstlisting}
				<core:iconForRecord table="sys_template" row="{templateRecord}" ></core:iconForRecord>

				// output:
				<span class="t3js-icon icon icon-size-small icon-state-default icon-mimetypes-x-content-template"
				  data-identifier="mimetypes-x-content-template">
				  <span class="icon-markup">
				    <img src="/typo3/sysext/core/Resources/Public/Icons/T3Icons/mimetypes/mimetypes-x-content-template.svg" width="16" height="16">
				  </span>
				</span>
			\end{lstlisting}

	\end{itemize}

\end{frame}

% ------------------------------------------------------------------------------
% LTXE-SLIDE-START
% LTXE-SLIDE-UID:		7ea9a114-e2c561a0-262d2007-91f5f32b
% LTXE-SLIDE-ORIGIN:	c88b95e6-9808a38a-747272d5-19346080 English
% LTXE-SLIDE-TITLE:		Feature #76107: Add Fluid Interceptor Registration (1)
% LTXE-SLIDE-REFERENCE:	Feature-76107-AddFluidInterceptorRegistration.rst
% ------------------------------------------------------------------------------
\begin{frame}[fragile]
	\frametitle{Extbase i Fluid}
	\framesubtitle{Dodat Fluid Interceptor Registration (1)}

	% decrease font size for code listin
	\lstset{basicstyle=\tiny\ttfamily}

	\begin{itemize}

		\item Interceptor-i u \textit{Fluid Standalone} dodati su kako bi mogli da izmene prikaz sablona

		\item Fluid API vec dozvoljava registrovanje prilagodjenih interceptor-a.
			Sada je moguce definisati prilagodjene interceptor-e koristeci sledecu opciju:

			\small
				\texttt{\$GLOBALS['TYPO3\_CONF\_VARS']['fluid']['interceptors']}
			\normalsize

		\item Interceptor-i registrovani ovde su dodati u Fluid parser konfiguraciju

	\end{itemize}

\end{frame}

% ------------------------------------------------------------------------------
% LTXE-SLIDE-START
% LTXE-SLIDE-UID:		1ffa4580-ccce3c8e-770e1e5a-a4f6afb8
% LTXE-SLIDE-ORIGIN:	bec08e51-3b4dccca-f2060fd8-35d46d4b English
% LTXE-SLIDE-TITLE:		Feature #76107: Add Fluid Interceptor Registration (2)
% LTXE-SLIDE-REFERENCE:	Feature-76107-AddFluidInterceptorRegistration.rst
% ------------------------------------------------------------------------------
\begin{frame}[fragile]
	\frametitle{Extbase i Fluid}
	\framesubtitle{Dodat Fluid Interceptor Registration (2)}

	% decrease font size for code listin
	\lstset{basicstyle=\tiny\ttfamily}

	\begin{itemize}

		\item Registrovanje sopstvenog interceptor-a u fluid parser konfiguraciju

			\begin{lstlisting}
				$GLOBALS['TYPO3_CONF_VARS']['SYS']['fluid']['interceptors']
			      [\TYPO3\CMS\Fluid\Core\Parser\Interceptor\DebugInterceptor::class] =
				  \TYPO3\CMS\Fluid\Core\Parser\Interceptor\DebugInterceptor::class;
			\end{lstlisting}

		\item Kod klase::

			\begin{lstlisting}
				use TYPO3Fluid\Fluid\Core\Parser\InterceptorInterface;
				use TYPO3Fluid\Fluid\Core\Parser\ParsingState;
				use TYPO3Fluid\Fluid\Core\Parser\SyntaxTree\NodeInterface;

				class DebugInterceptor implements InterceptorInterface
				{
				  public function process(NodeInterface $node, $interceptorPosition, ParsingState $parsingState)
				  {
				    return $node;
				  }

				  public function getInterceptionPoints()
				  {
				    return [];
				  }
				}
			\end{lstlisting}

	\end{itemize}

\end{frame}

% ------------------------------------------------------------------------------

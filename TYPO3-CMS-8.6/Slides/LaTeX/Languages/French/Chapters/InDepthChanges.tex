% ------------------------------------------------------------------------------
% TYPO3 CMS 8.6 - What's New - Chapter "Changements en profondeur" (French Version)
%
% @author	Michael Schams <schams.net>
% @license	Creative Commons BY-NC-SA 3.0
% @link		http://typo3.org/download/release-notes/whats-new/
% @language	French
% ------------------------------------------------------------------------------
% LTXE-CHAPTER-UID:		5ebcecbe-66abfa57-cf38bc00-aa637965
% LTXE-CHAPTER-NAME:	Changements en profondeur
% ------------------------------------------------------------------------------

\section{Changements en profondeur}
\begin{frame}[fragile]
	\frametitle{Changements en profondeur}

	\begin{center}\huge{Chapitre 3~:}\end{center}
	\begin{center}\huge{\color{typo3darkgrey}\textbf{Changements en profondeur}}\end{center}

\end{frame}

% ------------------------------------------------------------------------------
% LTXE-SLIDE-START
% LTXE-SLIDE-UID:		2937be8a-f6119f51-7fb50c6b-edf1e170
% LTXE-SLIDE-ORIGIN:	5ae3ff92-e2f03af9-8f5caf79-b7e0aca1 English
% LTXE-SLIDE-TITLE:		Page Browser for scheduler tasks
% LTXE-SLIDE-REFERENCE:	!Feature: #12211 - Usability: Scheduler provide page browser to choose start page
% ------------------------------------------------------------------------------

\begin{frame}[fragile]
	\frametitle{Changements en profondeur}
	\framesubtitle{Navigateur de page pour le planificateur}

	\begin{itemize}
		\item Les tâches planifiées nécessitant un identifiant de page (\texttt{uid}) peuvent
			ajouter un bouton pour le navigateur de page.

		\item Dans \texttt{ValidatorTaskAdditionalFieldProvider}, deux champs supplémentaires
			sont ajoutés.

		\item Si le champ supplémentaire \texttt{browser} est défini à \texttt{page} alors le
			\texttt{SchedulerModuleController} ajoute un bouton pour appeler le navigateur de
			page pour le champ.

			\begin{lstlisting}
				'browser' => 'page',
			\end{lstlisting}

		\item Le champ \texttt{pageTitle} contient le titre de la page qui est affiché à côté
		 	du bouton de navigation.

			\begin{lstlisting}
				'pageTitle' => $pageTitle,
			\end{lstlisting}

	\end{itemize}

\end{frame}

% ------------------------------------------------------------------------------
% LTXE-SLIDE-START
% LTXE-SLIDE-UID:		b81e3859-05a0a2c8-a335d708-f1cd01b2
% LTXE-SLIDE-ORIGIN:	02a7fe32-ae3fd901-b04372a2-7d3726f4 English
% LTXE-SLIDE-TITLE:		Synchronized field values in localized records (1/2)
% LTXE-SLIDE-REFERENCE:	!Feature: #51291 - Synchronized field values in localized records
% ------------------------------------------------------------------------------

\begin{frame}[fragile]
	\frametitle{Changements en profondeur}
	\framesubtitle{Valeurs de champs synchronisées dans les traductions (1/2)}

	% decrease font size for code listing
	\lstset{basicstyle=\tiny\ttfamily}

	\begin{itemize}
		\item Le comportement de recouvrement des enregistrements traduits est changé pour
			rendre les traductions indépendantes.

		\item Précédemment, si des champs dans les colonnes \texttt{TCA} étaient définies à
			\texttt{l10n\_mode} \texttt{exclude} ou \texttt{mergeIfNotBlank}, l'enregistrement
			de traduction ne contenait pas de valeurs, celles-ci étant «~remontées~» depuis
			l'enregistrement de langue par défaut.

	\end{itemize}

\end{frame}

% ------------------------------------------------------------------------------
% LTXE-SLIDE-START
% LTXE-SLIDE-UID:		72607275-94e0bf62-4ddc99c8-999fa4ad
% LTXE-SLIDE-ORIGIN:	cf030e7c-03fb0ac4-9b9ff848-ac69581b English
% LTXE-SLIDE-TITLE:		Synchronized field values in localized records (2/2)
% LTXE-SLIDE-REFERENCE:	!Feature: #51291 - Synchronized field values in localized records
% ------------------------------------------------------------------------------

\begin{frame}[fragile]
	\frametitle{Changements en profondeur}
	\framesubtitle{Valeurs de champs synchronisées dans les traductions (2/2)}

	% decrease font size for code listing
	\lstset{basicstyle=\tiny\ttfamily}

	\begin{itemize}
		\item C'est changé, le \texttt{DataHandler} copie ces valeurs dans l'enregistrement
			traduit et les synchronises quand l'enregistrement de langue par défaut est
			changé.

			\begin{lstlisting}
				'columns' => [
				  ...
				  'header' => [
				    'label' => 'My header',
				    'config' => [
				      'type' => 'input',
				      'behaviour' => [
				        'allowLanguageSynchronization' => true,
				      ],
				    ],
				  ],
				],
			\end{lstlisting}

	\end{itemize}

\end{frame}

% ------------------------------------------------------------------------------
% LTXE-SLIDE-START
% LTXE-SLIDE-UID:		a1792ff2-4f02d3f7-b9c56010-fa62dfa1
% LTXE-SLIDE-ORIGIN:	18c4f826-440f3e00-a854e8e8-ddc7cbc0 English
% LTXE-SLIDE-TITLE:		Image Manipulation Tool (1/6)
% LTXE-SLIDE-REFERENCE:	!Feature: #75880 - Implement multiple cropping variants in image manipulation tool
% ------------------------------------------------------------------------------

\begin{frame}[fragile]
	\frametitle{Changements en profondeur}
	\framesubtitle{Outil de manipulation d'images (1/6)}

	% decrease font size for code listing
	\lstset{basicstyle=\tiny\ttfamily}

	\begin{itemize}
		\item Le type TCA \texttt{imageManipulation} est capable de gérer plusieurs variantes
			de cadrage si configuré.

		\item Il est également possible de définir une zone de cadrage initiale. S'il n'y
			en a pas de définie, la zone de cadrage sélectionnée par défaut couvre toute
			l'image.
		\item Les zones de cadrage sont définies relativement avec des nombres réels. Les
			coordonnées et tailles doivent être définies pour cela.

	\end{itemize}

\end{frame}

% ------------------------------------------------------------------------------
% LTXE-SLIDE-START
% LTXE-SLIDE-UID:		99e4d805-3fa3947e-3e6a7f1d-5c3b911b
% LTXE-SLIDE-ORIGIN:	36ea74c0-3b92dbf6-e35c9e25-af0a0abf English
% LTXE-SLIDE-TITLE:		Image Manipulation Tool (2/6)
% LTXE-SLIDE-REFERENCE:	!Feature: #75880 - Implement multiple cropping variants in image manipulation tool
% ------------------------------------------------------------------------------

\begin{frame}[fragile]
	\frametitle{Changements en profondeur}
	\framesubtitle{Outil de manipulation d'images (2/6)}

	% decrease font size for code listing
	\lstset{basicstyle=\tiny\ttfamily}

	\begin{itemize}

		\item L'exemple suivant configure deux zones de cadrage, avec les
			identifiants «~mobile~» et «~desktop~». La clé du tableau sera
			utilisée lors du rendu avec le ViewHelper Image comme identifiant.

			\begin{lstlisting}
				'config' => [
				  'type' => 'imageManipulation',
				  'cropVariants' => [
				    'mobile' => [
				      'title' => 'Mobile',
				      'allowedAspectRatios' => [
				        '4:3' => [
				          'title' => '4:3',
				          'value' => 4 / 3
				        ],
				        ...
				      ],
				    ],
				    'desktop' => [
				      ...
				    ],
				  ],
				]
			\end{lstlisting}

	\end{itemize}

\end{frame}

% ------------------------------------------------------------------------------
% LTXE-SLIDE-START
% LTXE-SLIDE-UID:		84b69aef-b6cdef68-829458b8-f2ddea42
% LTXE-SLIDE-ORIGIN:	27b20030-548f80c4-3540aabb-41a83498 English
% LTXE-SLIDE-TITLE:		Image Manipulation Tool (3/6)
% LTXE-SLIDE-REFERENCE:	!Feature: #75880 - Implement multiple cropping variants in image manipulation tool
% ------------------------------------------------------------------------------

\begin{frame}[fragile]
	\frametitle{Changements en profondeur}
	\framesubtitle{Outil de manipulation d'images (3/6)}

	% decrease font size for code listing
	\lstset{basicstyle=\tiny\ttfamily}

	\begin{itemize}

		\item L'exemple suivant a une zone initiale de cadrage de la taille du
		 	cadreur d'image précédent fourni par défaut.

			\begin{lstlisting}
				'config' => [
				  'type' => 'imageManipulation',
				  'cropVariants' => [
				    'mobile' => [
				      'title' => 'LLL:EXT:ext_key/Resources/Private/Language/locallang.xlf:imageManipulation.mobile',
				      'cropArea' => [
				        'x' => 0.1,
				        'y' => 0.1,
				        'width' => 0.8,
				        'height' => 0.8,
				      ],
				    ],
				  ],
				]
			\end{lstlisting}

	\end{itemize}

\end{frame}

% ------------------------------------------------------------------------------
% LTXE-SLIDE-START
% LTXE-SLIDE-UID:		97315913-05e4d690-758aa9e5-c32eb8f8
% LTXE-SLIDE-ORIGIN:	58cebeda-4ee109f4-5e9eb728-b7703b40 English
% LTXE-SLIDE-TITLE:		Image Manipulation Tool (4/6)
% LTXE-SLIDE-REFERENCE:	!Feature: #75880 - Implement multiple cropping variants in image manipulation tool
% ------------------------------------------------------------------------------

\begin{frame}[fragile]
	\frametitle{Changements en profondeur}
	\framesubtitle{Outil de manipulation d'images (4/6)}

	% decrease font size for code listing
	\lstset{basicstyle=\tiny\ttfamily}

	\begin{itemize}
		\item Les utilisateurs peuvent aussi sélectionner une zone de mise en avant,
			si configuré.
		\item Cette zone est toujours dans la zone de cadrage et marque une zone de
			l'image qui doit être visible pour que l'image transporte son sens.

			\begin{lstlisting}
				'config' => [
				  'type' => 'imageManipulation',
				  'cropVariants' => [
				    'mobile' => [
				      'title' =>
				        'LLL:EXT:ext_key/Resources/Private/Language/locallang.xlf:imageManipulation.mobile',
				      'focusArea' => [
				        'x' => 1 / 3,
				        'y' => 1 / 3,
				        'width' => 1 / 3,
				        'height' => 1 / 3,
				      ],
				    ],
				  ],
				]
			\end{lstlisting}

	\end{itemize}

\end{frame}

% ------------------------------------------------------------------------------
% LTXE-SLIDE-START
% LTXE-SLIDE-UID:		91dfd8de-721188b8-e82a6d5c-54ffa11d
% LTXE-SLIDE-ORIGIN:	94d809e0-3369bf18-616710ee-93f6a9da English
% LTXE-SLIDE-TITLE:		Image Manipulation Tool (5/6)
% LTXE-SLIDE-REFERENCE:	!Feature: #75880 - Implement multiple cropping variants in image manipulation tool
% ------------------------------------------------------------------------------

\begin{frame}[fragile]
	\frametitle{Changements en profondeur}
	\framesubtitle{Outil de manipulation d'images (5/6)}

	% decrease font size for code listing
	\lstset{basicstyle=\tiny\ttfamily}

	\begin{itemize}
		\item Souvent, les images sont utilisées dans un contexte, où elles sont recouvertes
			par autre élément tel qu'un titre.
		\item Pour fournir aux éditeurs un indice de quelle zone de l'image est affectée,
			lors de la sélection d'une zone de cadrage, il est possible de définir plusieurs
			zones de recouvrement.
		\item Ces zones sont affichées dans la zone de cadrage. La zone de mise en avant
			ne peux pas croiser les zones de recouvrement.

			\begin{lstlisting}
				'config' => [
				  'type' => 'imageManipulation',
				  'coverAreas' => [
				    [
				      'x' => 0.05, 'y' => 0.85,
				      'width' => 0.9, 'height' => 0.1,
				    ],
				  ],
				]
			\end{lstlisting}

	\end{itemize}

\end{frame}

% ------------------------------------------------------------------------------
% LTXE-SLIDE-START
% LTXE-SLIDE-UID:		eb098ff9-f1f95408-1f61d30f-cb05def5
% LTXE-SLIDE-ORIGIN:	c9df3305-35544e3c-6a073be7-1cec40f4 English
% LTXE-SLIDE-TITLE:		Image Manipulation Tool (6/6)
% LTXE-SLIDE-REFERENCE:	!Feature: #75880 - Implement multiple cropping variants in image manipulation tool
% ------------------------------------------------------------------------------

\begin{frame}[fragile]
	\frametitle{Changements en profondeur}
	\framesubtitle{Outil de manipulation d'images (6/6)}

	% decrease font size for code listing
	\lstset{basicstyle=\smaller\ttfamily}

	\begin{itemize}
		\item Pour effectuer le rendu d'une variante de cadrage, son identifiant peut être
			spécifié comme argument du ViewHelper Image~:

			\begin{lstlisting}
				<f:image image="{data.image}" cropVariant="mobile" width="800" >
				</f:image>
			\end{lstlisting}

	\end{itemize}

\end{frame}

% ------------------------------------------------------------------------------
% LTXE-SLIDE-START
% LTXE-SLIDE-UID:		a68a04da-84042ebf-7af38a2f-1b136e1d
% LTXE-SLIDE-ORIGIN:	a71d77d1-6a419c8d-0fbb35d0-e1ce57e4 English
% LTXE-SLIDE-TITLE:		Default Content Element Changed for Fluid Styled Content
% LTXE-SLIDE-REFERENCE:	!Breaking: #79622 - Default content element changed for Fluid Styled Content
% ------------------------------------------------------------------------------

\begin{frame}[fragile]
	\frametitle{Changements en profondeur}
	\framesubtitle{Élément de contenu par défaut changé pour Fluid Styled Content}

	% decrease font size for code listing
	\lstset{basicstyle=\tiny\ttfamily}

	\begin{itemize}
		\item L'élément de contenu par défaut est mis en accord avec CSS Styled Content
			et changé à «~Text~»
		\item Pour revenir à la configuration précédente, il est nécessaire de définir
			manuellement l'élément de contenu par défaut au choix préféré. Il suffit de
			surcharger de nouveau la configuration dans un fichier
			\texttt{Configuration/TCA/Overrides/tt\_content.php}.

			\begin{lstlisting}
				$GLOBALS['TCA']['tt_content']['columns']['CType']['config']['default'] = 'textmedia';
				$GLOBALS['TCA']['tt_content']['columns']['CType']['config']['default'] = 'header';
			\end{lstlisting}

	\end{itemize}

\end{frame}

% ------------------------------------------------------------------------------
% LTXE-SLIDE-START
% LTXE-SLIDE-UID:		9f6a9230-def43e1d-a1256b50-645c8b38
% LTXE-SLIDE-ORIGIN:	035f6a19-9be2f793-bc0ebb3e-c6a4cca4 English
% LTXE-SLIDE-TITLE:		TCA Changes (1/2)
% LTXE-SLIDE-REFERENCE:	!Deprecation: #79440 - TCA Changes
% ------------------------------------------------------------------------------

\begin{frame}[fragile]
	\frametitle{Changements en profondeur}
	\framesubtitle{Changements TCA (1/2)}

	\begin{itemize}

		\item Le \texttt{TCA} au niveau champ est changé.

		\item Quasiment tous les types de colonne sont touchés.

		\item En général, la sous-section \texttt{wizards} est retirée et remplacée
			par une combinaison de nouveaux \texttt{renderType} et un nouvel ensemble
			d'options de configuration.

		\item Les assistants sont scindés en trois catégories différentes~:
			\texttt{fieldInformation}, \texttt{fieldControl} et \texttt{fieldWizard}.

	\end{itemize}

\end{frame}
% ------------------------------------------------------------------------------
% LTXE-SLIDE-START
% LTXE-SLIDE-UID:		c4fd0dea-6c134299-dd7fa185-9eba1a0a
% LTXE-SLIDE-ORIGIN:	55c8113e-b234daa0-e701f035-8e1b58d1 English
% LTXE-SLIDE-TITLE:		TCA Changes (2/2)
% LTXE-SLIDE-REFERENCE:	!Deprecation: #79440 - TCA Changes
% ------------------------------------------------------------------------------

\begin{frame}[fragile]
	\frametitle{Changements en profondeur}
	\framesubtitle{Changements TCA (2/2)}

	% decrease font size for code listing
	\lstset{basicstyle=\tiny\ttfamily}

	\begin{itemize}
		\item Exemple~:

			\begin{lstlisting}
				'fieldControl' => [
				  'editPopup' => [
				    'disabled' => false,
				  ],
				  'addRecord' => [
				    'disabled' => false,
				    'options' => [
				      'setValue' => 'prepend',
				    ],
				  ],
				  'listModule' => [
				    'disabled' => false,
				  ],
				],
			\end{lstlisting}

		\item Plus de détails à
			\href{https://docs.typo3.org/typo3cms/extensions/core/8-dev/singlehtml/Index.html#deprecation-79440-formengine-element-expansion}{docs.typo3.org}

	\end{itemize}

\end{frame}


% ------------------------------------------------------------------------------
% LTXE-SLIDE-START
% LTXE-SLIDE-UID:		cd3e706d-a8f3eab2-babf4ecc-f211050c
% LTXE-SLIDE-ORIGIN:	da922107-1604619b-66f713fe-9508d9d7 English
% LTXE-SLIDE-TITLE:		Introduce Session Storage Framework
% LTXE-SLIDE-REFERENCE:	!Feature: #70316 - Introduce Session Storage Framework
% ------------------------------------------------------------------------------

\begin{frame}[fragile]
	\frametitle{Changements en profondeur}
	\framesubtitle{Introduction du framework d'enregistrement de session}

	\begin{itemize}
		\item Un framework d'enregistrement de session est introduit (Session Storage Framework)
		\item Le but de ce framework est de créer l'interopérabilité entre les
			différents stockages de session (appelés «~backends~») comme la base
			de données, les fichiers, Redis, etc.
		\item Les backends de session suivants sont disponibles par défaut~:

			\begin{itemize}
				\item
					\smaller\texttt{\textbackslash
						TYPO3\textbackslash
						CMS\textbackslash
						Core\textbackslash
						Session\textbackslash
						Backend\textbackslash
						DatabaseSessionBackend}

				\item
					\texttt{\textbackslash
						TYPO3\textbackslash
						CMS\textbackslash
						Core\textbackslash
						Session\textbackslash
						Backend\textbackslash
						RedisSessionBackend}
			\end{itemize}
	\end{itemize}

\end{frame}

% ------------------------------------------------------------------------------
% LTXE-SLIDE-START
% LTXE-SLIDE-UID:		d1520e48-f4ab4d96-b1482593-7c61f70a
% LTXE-SLIDE-ORIGIN:	2d65beca-277a2f35-a1bc5ae7-c71712aa English
% LTXE-SLIDE-TITLE:		CLI Support for T3D Imports
% LTXE-SLIDE-REFERENCE:	!Feature: #72749 - CLI support for T3D import
% ------------------------------------------------------------------------------

\begin{frame}[fragile]
	\frametitle{Changements en profondeur}
	\framesubtitle{Support CLI pour imports T3D}

	\begin{itemize}
		\item \texttt{EXT:impexp} permet d'importer des fichiers de données (T3D ou XML)
			depuis l'interface en ligne de commande à travers une commande Symfony.

		\item Usage~:\newline
			\smaller
				\texttt{./typo3/sysext/core/bin/typo3 impexp:import [<options>] <file> <pageId>}
			\normalsize

		\item Options~:
			\begin{itemize}
				\item \texttt{-}\texttt{-updateRecords}~: Forcer la mise à jour des
					enregistrements existants
				\item \texttt{-}\texttt{-ignorePid}~: Ne pas corriger l'identifiant de page
					des enregistrements mis à jour
				\item \texttt{-}\texttt{-enableLog}~: journaliser l'ensemble des opérations
			\end{itemize}

	\end{itemize}

\end{frame}

% ------------------------------------------------------------------------------
% LTXE-SLIDE-START
% LTXE-SLIDE-UID:		fedf06b1-97cdeb74-fb28b36f-a058b192
% LTXE-SLIDE-ORIGIN:	38064844-53baa89a-7f73669e-2bb96cab English
% LTXE-SLIDE-TITLE:		Hook in typolink for Modification of Page Params
% LTXE-SLIDE-REFERENCE:	!Feature: #79121 - Implement hook in typolink for modification of page params
% ------------------------------------------------------------------------------

\begin{frame}[fragile]
	\frametitle{Changements en profondeur}
	\framesubtitle{Hook dans \texttt{typolink} pour la modification des paramètres de page}

	% decrease font size for code listing
	\lstset{basicstyle=\tiny\ttfamily}

	\begin{itemize}
		\item Un nouveau hook est intégré à \texttt{ContentObjectRenderer::typoLink} pour les liens vers les pages.
			Celui-ci permet la modification de la configuration du lien, par exemple l'enrichir avec des paramètres
			supplémentaires ou des métadonnées depuis l'enregistrement de page.

		\item L'enregistrement s'effectue via~:

			\begin{lstlisting}
				$GLOBALS['TYPO3_CONF_VARS']['SC_OPTIONS']['typolinkProcessing']
				  ['typolinkModifyParameterForPageLinks'][] = \Your\Namespace\Hooks\MyHook::class;
			\end{lstlisting}

		\item Usage~:

			\begin{lstlisting}
				public function modifyPageLinkConfiguration(
				  array $linkConfiguration, array $linkDetails, array $pageRow) : array
				{
				  $linkConfiguration['additionalParams'] .= $pageRow['myAdditionalParamsField'];
				  return $linkConfiguration;
				}
			\end{lstlisting}

	\end{itemize}

\end{frame}

% ------------------------------------------------------------------------------
% LTXE-SLIDE-START
% LTXE-SLIDE-UID:		ed7db278-02de8e66-88ed3457-dfccdd8b
% LTXE-SLIDE-ORIGIN:	290e1c2e-e708ede1-82aaf127-e69b2338 English
% LTXE-SLIDE-TITLE:		Hook to Add Custom TypoScript Templates (1/2)
% LTXE-SLIDE-REFERENCE:	!Feature: #79140 - Add hook to add custom TypoScript templates
% ------------------------------------------------------------------------------

\begin{frame}[fragile]
	\frametitle{Changements en profondeur}
	\framesubtitle{Hook pour ajouter des template TypoScript personnalisés (1/2)}

	% decrease font size for code listing
	\lstset{basicstyle=\tiny\ttfamily}

	\begin{itemize}
		\item Un nouveau hook dans TemplateService permet l'ajout ou la modification de template
			TypoScript.

		\item L'enregistrement au hook s'effectue via le code suivant dans le fichier
			\texttt{ext\_localconf.php} des extensions~:

			\begin{lstlisting}
				$GLOBALS['TYPO3_CONF_VARS']['SC_OPTIONS']['Core/TypoScript/TemplateService']
				  ['runThroughTemplatesPostProcessing']
			\end{lstlisting}

		\item \texttt{EXT:my\_site/Classes/Hooks/TypoScriptHook.php} (1/2)

			\begin{lstlisting}
				namespace MyVendor\MySite\Hooks;
				class TypoScriptHook
				{
				  /**
				   * Hooks into TemplateService after
				   * @param array $parameters
				   * @param \TYPO3\CMS\Core\TypoScript\TemplateService $parentObject
				   * @return void
				   */
				...
			\end{lstlisting}

	\end{itemize}

\end{frame}

% ------------------------------------------------------------------------------
% LTXE-SLIDE-START
% LTXE-SLIDE-UID:		d2690d5b-bb4c70ea-326b487b-5118f376
% LTXE-SLIDE-ORIGIN:	d64368f4-0f5c43e3-92a546bf-75d50a61 English
% LTXE-SLIDE-TITLE:		Hook to Add Custom TypoScript Templates (2/2)
% LTXE-SLIDE-REFERENCE:	!Feature: #79140 - Add hook to add custom TypoScript templates
% ------------------------------------------------------------------------------

\begin{frame}[fragile]
	\frametitle{Changements en profondeur}
	\framesubtitle{Hook pour ajouter des template TypoScript personnalisés (2/2)}

	% decrease font size for code listing
	\lstset{basicstyle=\tiny\ttfamily}

	\begin{itemize}
		\item \texttt{EXT:my\_site/Classes/Hooks/TypoScriptHook.php} (2/2)

			\begin{lstlisting}
				...
				  public function addCustomTypoScriptTemplate($parameters, $parentObject)
				  {
				    // Disable the inclusion of default TypoScript set via TYPO3_CONF_VARS
				    $parameters['isDefaultTypoScriptAdded'] = true;
				    // Disable the inclusion of ext_typoscript_setup.txt of all extensions
				    $parameters['processExtensionStatics'] = false;

				    // No template was found in rootline so far, so a custom "fake" sys_template record is added
				    if ($parentObject->outermostRootlineIndexWithTemplate === 0) {
				      $row = [
				        'uid' => 'my_site_template',
				        'config' =>
					      '<INCLUDE_TYPOSCRIPT: source="FILE:EXT:my_site/Configuration/TypoScript/site_setup.t3s">',
				        'root' => 1,
				        'pid' => 0
				      ];
				      $parentObject->processTemplate($row, 'sys_' . $row['uid'], 0, 'sys_' . $row['uid']);
				    }
				  }
				}
			\end{lstlisting}

	\end{itemize}

\end{frame}

% ------------------------------------------------------------------------------
% LTXE-SLIDE-START
% LTXE-SLIDE-UID:		bbaba923-c2ae8f17-86e3391a-ce69e3fb
% LTXE-SLIDE-ORIGIN:	ad9f66e3-bcd548f4-8e14e04c-de29a79c English
% LTXE-SLIDE-TITLE:		Plugin Preview with Fluid
% LTXE-SLIDE-REFERENCE:	!Feature: #79225 - Plugin preview with Fluid
% ------------------------------------------------------------------------------
\begin{frame}[fragile]
	\frametitle{Changements en profondeur}
	\framesubtitle{Prévisualisation de plugin avec Fluid}

	% decrease font size for code listing
	\lstset{basicstyle=\tiny\ttfamily}

	\begin{itemize}
		\item La configuration TypoScript de page pour prévisualiser un élément de contenu dans le
			Backend est améliorée en permettant aussi le rendu des plugins via Fluid

		\item Toutes les propriétés de l'enregistrement \texttt{tt\_content} sont disponibles directement
			dans le template (i.e. UID via \{uid\})

		\item Les données du champ flexform \texttt{pi\_flexform} sont disponibles dans la propriété
			\texttt{pi\_flexform\_transformed} comme tableau.

			\begin{lstlisting}
				mod.web_layout.tt_content.preview.list.simpleblog_bloglisting =
				  EXT:simpleblog/Resources/Private/Templates/Preview/SimpleblogPlugin.html
			\end{lstlisting}

	\end{itemize}

\end{frame}

% ------------------------------------------------------------------------------
% LTXE-SLIDE-START
% LTXE-SLIDE-UID:		64d746ed-4caef190-d7ceac0f-a7c2b636
% LTXE-SLIDE-ORIGIN:	245de018-04aed929-a3b3c380-6c822a14 English
% LTXE-SLIDE-TITLE:		Template Paths in BackendTemplateView
% LTXE-SLIDE-REFERENCE:	!Feature: #79124 - Allow overwriting of template paths in BackendTemplateView
% ------------------------------------------------------------------------------

\begin{frame}[fragile]
	\frametitle{Changements en profondeur}
	\framesubtitle{Chemins des templates dans BackendTemplateView}

	% decrease font size for code listing
	\lstset{basicstyle=\tiny\ttfamily}

	\begin{itemize}
		\item BackendTemplateView permet de surcharger le chemin des templates pour ajouter son
			propre emplacement pour templates, partials et layouts dans un module backend basé
			sur BackendTemplateView.

			\begin{lstlisting}
				$frameworkConfiguration =
				  $this->configurationManager->getConfiguration(
				    ConfigurationManagerInterface::CONFIGURATION_TYPE_FRAMEWORK
				  );
				$viewConfiguration = [
				  'view' => [
				    'templateRootPaths' => ['EXT:myext/Resources/Private/Backend/Templates'],
				    'partialRootPaths' => ['EXT:myext/Resources/Private/Backend/Partials'],
				    'layoutRootPaths' => ['EXT:myext/Resources/Private/Backend/Layouts'],
				  ],
				];
				$this->configurationManager->setConfiguration(
				  array_merge($frameworkConfiguration, $viewConfiguration)
				);
			\end{lstlisting}

	\end{itemize}

\end{frame}

% ------------------------------------------------------------------------------
% LTXE-SLIDE-START
% LTXE-SLIDE-UID:		dd8f25c6-9d8b3599-3908419d-c6e13546
% LTXE-SLIDE-ORIGIN:	82514d33-f6c72709-2052880d-4d753fe8 English
% LTXE-SLIDE-TITLE:		Miscellaneous
% LTXE-SLIDE-REFERENCE:	!Feature: #78899 - TCA maxitems optional
% LTXE-SLIDE-REFERENCE:	!Feature: #79240 - Single cli user for cli commands
% ------------------------------------------------------------------------------

\begin{frame}[fragile]
	\frametitle{Changements en profondeur}
	\framesubtitle{Divers}

	\begin{itemize}
		\item La configuration \texttt{TCA} \texttt{maxitems} des champs de type \texttt{select} et
			\texttt{group} est optionnelle et sa valeur par défaut est élevée (99999) au lieu de 1.

		\item L'accès aux fonctionnalités de TYPO3 depuis la ligne de commande est simplifié. Les commandes
			n'ont plus besoin d'utilisateur dans la base de données. Au lieu de cela, les commandes utilisent
			le nom d'utilisateur \texttt{\_cli\_}. L'utilisateur est créé à la demande par le framework s'il
			n'existe pas déjà lors de la première exécution d'une commande. Cet utilisateur possède les
			droits d'administrateur et ne nécessite plus d'avoir ses autorisations assignées explicitement
			pour effectuer les tâches comme la manipulation de la base de données avec \texttt{DataHandler}.

	\end{itemize}

\end{frame}

% ------------------------------------------------------------------------------

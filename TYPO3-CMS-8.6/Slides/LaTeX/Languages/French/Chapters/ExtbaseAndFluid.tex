% ------------------------------------------------------------------------------
% TYPO3 CMS 8.6 - What's New - Chapter "Extbase & Fluid" (French Version)
%
% @author	Michael Schams <schams.net>
% @license	Creative Commons BY-NC-SA 3.0
% @link		http://typo3.org/download/release-notes/whats-new/
% @language	French
% ------------------------------------------------------------------------------
% LTXE-CHAPTER-UID:		846d40ce-66fdc2ec-750dcf95-33ce93e0
% LTXE-CHAPTER-NAME:	Extbase & Fluid
% ------------------------------------------------------------------------------

\section{Extbase \& Fluid}
\begin{frame}[fragile]
	\frametitle{Extbase \& Fluid}

	\begin{center}\huge{Chapitre 4~:}\end{center}
	\begin{center}\huge{\color{typo3darkgrey}\textbf{Extbase \& Fluid}}\end{center}

\end{frame}

% ------------------------------------------------------------------------------
% LTXE-SLIDE-START
% LTXE-SLIDE-UID:		9fe8c279-b20e7166-8c5c6436-d351e397
% LTXE-SLIDE-ORIGIN:	a9a5442b-0f7ff21f-9e8812b1-9531a9c1 English
% LTXE-SLIDE-TITLE:		Widget Identifier Extended
% LTXE-SLIDE-REFERENCE:	!Feature: #47006 - Extend the widget identifier with custom string
% ------------------------------------------------------------------------------

\begin{frame}[fragile]
	\frametitle{Extbase \& Fluid}
	\framesubtitle{Extension de l'identificateur de Widget}

	\begin{itemize}
		\item Le paramètre \texttt{customWidgetId} est introduit pour le widget fluid.
			Cette chaîne est utilisée dans l'identificateur de widget en plus de
			\texttt{nextWidgetNumber}.

		\item L'identificateur de widget est utilisé pour créer les noms des paramètres GET.

		\item Une bonne valeur pour \texttt{customWidgetId} est \texttt{{contentObjectData.uid}} pour assurer l'absence de collision.

		\item Il permet d'utiliser un même widget fluid plusieurs fois sur une page dans des
		 	éléments de contenu différents.

			\begin{lstlisting}
				<f:widget.paginate customWidgetId="{contentObjectData.uid}" ...>
				</f:widget.paginate>
			\end{lstlisting}

	\end{itemize}

\end{frame}

% ------------------------------------------------------------------------------
% LTXE-SLIDE-START
% LTXE-SLIDE-UID:		22bf0045-0d8dd48a-b4c02ca9-e8cf2076
% LTXE-SLIDE-ORIGIN:	cbc1b52f-68132620-c5f08dbb-14723e1f English
% LTXE-SLIDE-TITLE:		FlashMessageViewHelper
% LTXE-SLIDE-REFERENCE:	!Breaking: #78477 - FlashMessageViewHelper no longer inherits from TagBasedViewHelper
% ------------------------------------------------------------------------------

\begin{frame}[fragile]
	\frametitle{Extbase \& Fluid}
	\framesubtitle{FlashMessagesViewHelper}

	% decrease font size for code listing
	\lstset{basicstyle=\tiny\ttfamily}

	\begin{itemize}
		\item Le \texttt{FlashMessagesViewHelper} est réécrit et n'hérite plus de
			\texttt{TagBasedViewHelper}

		\item Sont donc retirés les attributs spécifiques des balises et les styles par défaut
		 	de sortie. Si vous avez toujours besoin de personnaliser la sortie, il reste possible
		 	de personnaliser le contenu, par exemple~:

			\begin{lstlisting}
				<f:flashMessages as="flashMessages">
				  <dl class="messages">
				    <f:for each="{flashMessages}" as="flashMessage">
				      <dt>CODE!! {flashMessage.code}</dt>
				      <dd>MESSAGE:: {flashMessage.message}</dd>
				    </f:for>
				  </dl>
				</f:flashMessages>
			\end{lstlisting}

	\end{itemize}

\end{frame}

% ------------------------------------------------------------------------------
% LTXE-SLIDE-START
% LTXE-SLIDE-UID:		7b2ef7c4-b84bbe75-f81fb112-78f77bbb
% LTXE-SLIDE-ORIGIN:	91358547-99cab1b1-208f0f24-a5737bee English
% LTXE-SLIDE-TITLE:		Removal of Fluid Styled Content Menu ViewHelpers (1/3)
% LTXE-SLIDE-REFERENCE:	!Breaking: #79622 - Removal of Fluid Styled Content Menu ViewHelpers
% ------------------------------------------------------------------------------

\begin{frame}[fragile]
	\frametitle{Extbase \& Fluid}
	\framesubtitle{Retrait des Menu ViewHelpers de Fluid Styled Content (1/3)}

	\begin{itemize}
		\item Récupérer des données directement dans la vue n'est pas recommandé.
			La solution temporaire des ViewHelpers Menu est remplacée par son
			successeur, le processeur de menu basé sur HMENU.

		\item Les Menu ViewHelpers sont déplacés dans l'extension \texttt{compatibility7},
			et sont remplacés dans les éléments de contenu du noyau.

	\end{itemize}

\end{frame}

% ------------------------------------------------------------------------------
% LTXE-SLIDE-START
% LTXE-SLIDE-UID:		9f1c25e1-4b66c87d-45d7c194-9533894d
% LTXE-SLIDE-ORIGIN:	8515a53c-1e60545a-7e99c75e-2638a0b1 English
% LTXE-SLIDE-TITLE:		Removal of Fluid Styled Content Menu ViewHelpers (2/3)
% LTXE-SLIDE-REFERENCE:	!Breaking: #79622 - Removal of Fluid Styled Content Menu ViewHelpers
% ------------------------------------------------------------------------------

\begin{frame}[fragile]
	\frametitle{Extbase \& Fluid}
	\framesubtitle{Retrait du MenuViewHelper de Fluid Styled Content (2/3)}

	% decrease font size for code listing
	\lstset{basicstyle=\tiny\ttfamily}

	\begin{itemize}

		\item Avant~:

			\begin{lstlisting}
				tt_content.menu_subpages.dataProcessing {
				  10 = TYPO3\CMS\Frontend\DataProcessing\SplitProcessor
				  10 {
				    if.isTrue.field = pages
				    fieldName = pages
				    delimiter = ,
				    removeEmptyEntries = 1
				    filterIntegers = 1
				    filterUnique = 1
				    as = pageUids
				  }
				}

				<ce:menu.directory pageUids="{pageUids}" as="pages" levelAs="level">
				  <f:for each="{pages}" as="page">
				    ...
				  </f:for>
				</ce.menu.directory>
			\end{lstlisting}

	\end{itemize}

\end{frame}

% ------------------------------------------------------------------------------
% LTXE-SLIDE-START
% LTXE-SLIDE-UID:		975b67b0-3a27cd35-af76d4f0-ae45878e
% LTXE-SLIDE-ORIGIN:	53461585-875b27dd-f00c174b-9b459763 English
% LTXE-SLIDE-TITLE:		Removal of Fluid Styled Content Menu ViewHelpers (3/3)
% LTXE-SLIDE-REFERENCE:	!Breaking: #79622 - Removal of Fluid Styled Content Menu ViewHelpers
% ------------------------------------------------------------------------------

\begin{frame}[fragile]
	\frametitle{Extbase \& Fluid}
	\framesubtitle{Retrait du MenuViewHelper de Fluid Styled Content (3/3)}

	% decrease font size for code listing
	\lstset{basicstyle=\tiny\ttfamily}

	\begin{itemize}

		\item Après~:

			\begin{lstlisting}
				tt_content.menu_subpages.dataProcessing {
				  10 = TYPO3\CMS\Frontend\DataProcessing\MenuProcessor
				  10.special = directory
				  10.special.value.field = pages
				}

				<f:for each="{menu}" as="page">
				  ...
				</f:for>
			\end{lstlisting}

	\end{itemize}

\end{frame}

% ------------------------------------------------------------------------------
% LTXE-SLIDE-START
% LTXE-SLIDE-UID:		af5fa693-867bf088-b3382e5b-bf80916b
% LTXE-SLIDE-ORIGIN:	0bdb89f8-d302c7ef-2037711d-a3706e69 English
% LTXE-SLIDE-TITLE:		Fluid ViewHelper <f:variable>
% LTXE-SLIDE-REFERENCE:	!Feature: #79402 - New Fluid ViewHelper f:variable added
% ------------------------------------------------------------------------------

\begin{frame}[fragile]
	\frametitle{Extbase \& Fluid}
	\framesubtitle{Nouveau ViewHelper Fluid \texttt{f:variable}}

	\begin{itemize}

		\item Le ViewHelper \texttt{f:variable} a été ajouté à Fluid 2.2.0,
			qui est maintenant la version minimum requise par TYPO3 CMS

		\item Le ViewHelper permet d'assigner des variables dans le modèle~:

			\begin{lstlisting}
				<f:variable name="myvariable">My variable's content</f:variable>
				<f:variable name="myvariable" value="My variable's content"></f:variable>
				{f:variable(name: 'myvariable', value: 'My variable\'s content')}
				{myoriginalvariable -> f:variable(name: 'mynewvariable')}
			\end{lstlisting}

	\end{itemize}

\end{frame}

% ------------------------------------------------------------------------------
% LTXE-SLIDE-START
% LTXE-SLIDE-UID:		3552ea10-101c1f80-fbc11e91-c8cf6179
% LTXE-SLIDE-ORIGIN:	c459d3a9-e6fb03fe-cc79880c-d86a18d1 English
% LTXE-SLIDE-TITLE:		New Default Layout for Fluid Styled Content (1/2)
% LTXE-SLIDE-REFERENCE:	!Feature: #79622 - New default layout for Fluid Styled Content
% ------------------------------------------------------------------------------

\begin{frame}[fragile]
	\frametitle{Extbase \& Fluid}
	\framesubtitle{Nouvelle disposition Default pour Fluid Styled Content (1/2)}

	\begin{itemize}
		\item Précédemment, il était possible de choisir entre trois dispositions
			lorsque l'on définissait son propre élément de contenu ou lors de la
			surcharge d'un modèle existant.

		\item Afin de fournir une meilleure maintenabilité et faciliter la surcharge,
			nous les avons réduits à une seule disposition nommée \texttt{Default}
			avec toutes les sections en optionnel avec repli si la section n'est
			pas définie. Le concept de «~DropIn~» est introduit.

		\item La disposition \texttt{Default} consiste en cinq sections prédéfinies qui
		 	peuvent être utilisées pour agencer la sortie pour le rendu de votre contenu.
		 	Dans la plupart des cas, les sections autres que \texttt{Main} ne seront pas
		 	utilisées. Le rendu des sections est effectué dans cet ordre~:
			\texttt{Before}, \texttt{Header}, \texttt{Main}, \texttt{Footer}, \texttt{After}
	\end{itemize}

\end{frame}

% ------------------------------------------------------------------------------
% LTXE-SLIDE-START
% LTXE-SLIDE-UID:		3f43e251-50a4a86a-f2fa357a-3fcbf3ff
% LTXE-SLIDE-ORIGIN:	da3b6df5-a15241de-c724dfb6-765885d4 English
% LTXE-SLIDE-TITLE:		New Default Layout for Fluid Styled Content (2/2)
% LTXE-SLIDE-REFERENCE:	!Feature: #79622 - New default layout for Fluid Styled Content
% ------------------------------------------------------------------------------

\begin{frame}[fragile]
	\frametitle{Extbase \& Fluid}
	\framesubtitle{Nouvelle disposition Default pour Fluid Styled Content (2/2)}

	\begin{itemize}
		\item Les sections \texttt{Before} et \texttt{After} sont les sections «~DropIn~»
		\item Les DropIns sont introduits pour permettre l'ajout de fonctionnalité à l'ensemble
			des éléments de contenu sans surcharger les dispositions ou modèles
		\item Les DropIns sont des emplacements de substitution/partial vides qui sont à
			surcharger lorsque nécessaire
		\item Emplacement des DropIn~:
			\begin{itemize}
				\item \texttt{Resources/Private/Partials/DropIn/Before/All.html}
				\item \texttt{Resources/Private/Partials/DropIn/After/All.html}
			\end{itemize}
	\end{itemize}

\end{frame}

% ------------------------------------------------------------------------------

% ------------------------------------------------------------------------------
% TYPO3 CMS 8.6 - What's New - Chapter "In-Depth Changes" (Spanish Version)
%
% @author	Michael Schams <schams.net>
% @license	Creative Commons BY-NC-SA 3.0
% @link		http://typo3.org/download/release-notes/whats-new/
% @language	English
% ------------------------------------------------------------------------------
% LTXE-CHAPTER-UID:		5ebcecbe-66abfa57-cf38bc00-aa637965
% LTXE-CHAPTER-NAME:	In-Depth Changes
% ------------------------------------------------------------------------------

\section{Cambios en Profundidad}
\begin{frame}[fragile]
	\frametitle{Cambios en Profundidad}

	\begin{center}\huge{Capítulo 3:}\end{center}
	\begin{center}\huge{\color{typo3darkgrey}\textbf{Cambios en Profundidad}}\end{center}

\end{frame}

% ------------------------------------------------------------------------------
% LTXE-SLIDE-START
% LTXE-SLIDE-UID:		ca5794ca-889539b4-ea3cab24-4c6d1ff7
% LTXE-SLIDE-ORIGIN:	5ae3ff92-e2f03af9-8f5caf79-b7e0aca1 English
% LTXE-SLIDE-TITLE:		Page Browser for scheduler tasks
% LTXE-SLIDE-REFERENCE:	!Feature: #12211 - Usability: Scheduler provide page browser to choose start page
% ------------------------------------------------------------------------------

\begin{frame}[fragile]
	\frametitle{Cambios en Profundidad}
	\framesubtitle{Paginador de Página para tareas del programador}

	\begin{itemize}
		\item Tareas del programador que necesitan un \texttt{uid} de página pueden ahora añadir un botón para el popup de paginador de página.

		\item Han sido añadidos dos campos adicionales en el \texttt{ValidatorTaskAdditionalFieldProvider}.

		\item Si el campo adicional \texttt{browser} es configurado a \texttt{page} entonces el
			\texttt{SchedulerModuleController} añade un botón para llamar al popup de paginador de página para el campo.

			\begin{lstlisting}
				'browser' => 'page',
			\end{lstlisting}

		\item El \texttt{pageTitle} contiene el título de la página que es mostrado cerca del botón de paginado.

			\begin{lstlisting}
				'pageTitle' => $pageTitle,
			\end{lstlisting}

	\end{itemize}

\end{frame}

% ------------------------------------------------------------------------------
% LTXE-SLIDE-START
% LTXE-SLIDE-UID:		fb9c394f-ccadf351-6c409ac5-a2c37cf5
% LTXE-SLIDE-ORIGIN:	02a7fe32-ae3fd901-b04372a2-7d3726f4 English
% LTXE-SLIDE-TITLE:		Synchronized field values in localized records (1/2)
% LTXE-SLIDE-REFERENCE:	!Feature: #51291 - Synchronized field values in localized records
% ------------------------------------------------------------------------------

\begin{frame}[fragile]
	\frametitle{Cambios en Profundidad}
	\framesubtitle{Valores de campo sincronizados en registros localizados (1/2)}

	% decrease font size for code listing
	\lstset{basicstyle=\tiny\ttfamily}

	\begin{itemize}
		\item El comportamiento de la capa de registro localizado ha sido cambiado para hacer autónomas las filas de localización.

		\item Previamente, si los campos en las columnas de \texttt{TCA} fueron puestos a \texttt{l10n\_mode} \texttt{exclude} o
			\texttt{mergeIfNotBlank}, la capa de registro localizado no contenía valores, y estos valores eran
			"arrancados" de los registros de lenguaje por defecto subyacentes.

	\end{itemize}

\end{frame}

% ------------------------------------------------------------------------------
% LTXE-SLIDE-START
% LTXE-SLIDE-UID:		8dfd2c5d-6db7041b-a2fb3911-bf9c81dc
% LTXE-SLIDE-ORIGIN:	cf030e7c-03fb0ac4-9b9ff848-ac69581b English
% LTXE-SLIDE-TITLE:		Synchronized field values in localized records (2/2)
% LTXE-SLIDE-REFERENCE:	!Feature: #51291 - Synchronized field values in localized records
% ------------------------------------------------------------------------------

\begin{frame}[fragile]
	\frametitle{Cambios en Profundidad}
	\framesubtitle{Valores de campo sincronizados en registros localizados (2/2)}

	% decrease font size for code listing
	\lstset{basicstyle=\tiny\ttfamily}

	\begin{itemize}
		\item Esto ha sido cambiado, el \texttt{DataHandler} ahora copia esos valores sobre el registro
			localizado y los sincroniza si el registro de lenguaje por defecto es cambiado.

			\begin{lstlisting}
				'columns' => [
				  ...
				  'header' => [
				    'label' => 'My header',
				    'config' => [
				      'type' => 'input',
				      'behaviour' => [
				        'allowLanguageSynchronization' => true,
				      ],
				    ],
				  ],
				],
			\end{lstlisting}

	\end{itemize}

\end{frame}

% ------------------------------------------------------------------------------
% LTXE-SLIDE-START
% LTXE-SLIDE-UID:		8e752a91-cd1b225b-079122a1-146069f6
% LTXE-SLIDE-ORIGIN:	18c4f826-440f3e00-a854e8e8-ddc7cbc0 English
% LTXE-SLIDE-TITLE:		Image Manipulation Tool (1/6)
% LTXE-SLIDE-REFERENCE:	!Feature: #75880 - Implement multiple cropping variants in image manipulation tool
% ------------------------------------------------------------------------------

\begin{frame}[fragile]
	\frametitle{Cambios en Profundidad}
	\framesubtitle{Herramienta de Manipulación de Imágenes (1/6)}

	% decrease font size for code listing
	\lstset{basicstyle=\tiny\ttfamily}

	\begin{itemize}
		\item El tipo de TCA \texttt{imageManipulation} es ahora capaz de manejar múltiples variantes de cortado si se configura.

		\item También es posible definir un área de recorte inicial. Si no se define un área de recorte inicial,
			el área de recorte seleccionada por defecto cubrirá la imagen completa.
		\item Las áreas de cortado son definidas relativamente con números flotantes. Las coordinadas y
			tamaños deben ser especificados para ello.

	\end{itemize}

\end{frame}

% ------------------------------------------------------------------------------
% LTXE-SLIDE-START
% LTXE-SLIDE-UID:		a8910499-bf1c96d8-8ed35b71-7d15c5db
% LTXE-SLIDE-ORIGIN:	36ea74c0-3b92dbf6-e35c9e25-af0a0abf English
% LTXE-SLIDE-TITLE:		Image Manipulation Tool (2/6)
% LTXE-SLIDE-REFERENCE:	!Feature: #75880 - Implement multiple cropping variants in image manipulation tool
% ------------------------------------------------------------------------------

\begin{frame}[fragile]
	\frametitle{Cambios en Profundidad}
	\framesubtitle{Herramienta de Manipulación de Imágenes (2/6)}

	% decrease font size for code listing
	\lstset{basicstyle=\tiny\ttfamily}

	\begin{itemize}

		\item El siguiente ejemplo configura dos variantes de cortado,
			uno con el id "mobile", uno con el id "desktop".
			La clave del array define el id, que será usado al
			renderizar una image con el view helper image.

			\begin{lstlisting}
				'config' => [
				  'type' => 'imageManipulation',
				  'cropVariants' => [
				    'mobile' => [
				      'title' => 'Mobile',
				      'allowedAspectRatios' => [
				        '4:3' => [
				          'title' => '4:3',
				          'value' => 4 / 3
				        ],
				        ...
				      ],
				    ],
				    'desktop' => [
				      ...
				    ],
				  ],
				]
			\end{lstlisting}

	\end{itemize}

\end{frame}

% ------------------------------------------------------------------------------
% LTXE-SLIDE-START
% LTXE-SLIDE-UID:		39e3d7fa-4f3ffed5-38074ca2-6b2fb396
% LTXE-SLIDE-ORIGIN:	27b20030-548f80c4-3540aabb-41a83498 English
% LTXE-SLIDE-TITLE:		Image Manipulation Tool (3/6)
% LTXE-SLIDE-REFERENCE:	!Feature: #75880 - Implement multiple cropping variants in image manipulation tool
% ------------------------------------------------------------------------------

\begin{frame}[fragile]
	\frametitle{Cambios en Profundidad}
	\framesubtitle{Herramienta de Manipulación de Imágenes (3/6)}

	% decrease font size for code listing
	\lstset{basicstyle=\tiny\ttfamily}

	\begin{itemize}

		\item El ejemplo siguiente tiene un área de recorte inicial en el tamaño
			del cortado de la imagen previa proporcionado por defecto.

			\begin{lstlisting}
				'config' => [
				  'type' => 'imageManipulation',
				  'cropVariants' => [
				    'mobile' => [
				      'title' => 'LLL:EXT:ext_key/Resources/Private/Language/locallang.xlf:imageManipulation.mobile',
				      'cropArea' => [
				        'x' => 0.1,
				        'y' => 0.1,
				        'width' => 0.8,
				        'height' => 0.8,
				      ],
				    ],
				  ],
				]
			\end{lstlisting}

	\end{itemize}

\end{frame}

% ------------------------------------------------------------------------------
% LTXE-SLIDE-START
% LTXE-SLIDE-UID:		a8fc6cbe-251e60e4-e12e422c-f8191403
% LTXE-SLIDE-ORIGIN:	58cebeda-4ee109f4-5e9eb728-b7703b40 English
% LTXE-SLIDE-TITLE:		Image Manipulation Tool (4/6)
% LTXE-SLIDE-REFERENCE:	!Feature: #75880 - Implement multiple cropping variants in image manipulation tool
% ------------------------------------------------------------------------------

\begin{frame}[fragile]
	\frametitle{Cambios en Profundidad}
	\framesubtitle{Herramienta de Manipulación de Imágenes (4/6)}

	% decrease font size for code listing
	\lstset{basicstyle=\tiny\ttfamily}

	\begin{itemize}
		\item Los usuarios pueden también seleccionar un área de enfoque, si se configura.
		\item El área de enfoque está siempre dentro del área de recorte y marca el área en la imagen
			que debe ser visible en la imagen para transportar su significado.

			\begin{lstlisting}
				'config' => [
				  'type' => 'imageManipulation',
				  'cropVariants' => [
				    'mobile' => [
				      'title' =>
				        'LLL:EXT:ext_key/Resources/Private/Language/locallang.xlf:imageManipulation.mobile',
				      'focusArea' => [
				        'x' => 1 / 3,
				        'y' => 1 / 3,
				        'width' => 1 / 3,
				        'height' => 1 / 3,
				      ],
				    ],
				  ],
				]
			\end{lstlisting}

	\end{itemize}

\end{frame}

% ------------------------------------------------------------------------------
% LTXE-SLIDE-START
% LTXE-SLIDE-UID:		0d05d01d-befa889d-c7cec8dd-9ac4f799
% LTXE-SLIDE-ORIGIN:	94d809e0-3369bf18-616710ee-93f6a9da English
% LTXE-SLIDE-TITLE:		Image Manipulation Tool (5/6)
% LTXE-SLIDE-REFERENCE:	!Feature: #75880 - Implement multiple cropping variants in image manipulation tool
% ------------------------------------------------------------------------------

\begin{frame}[fragile]
	\frametitle{Cambios en Profundidad}
	\framesubtitle{Herramienta de Manipulación de Imágenes (5/6)}

	% decrease font size for code listing
	\lstset{basicstyle=\tiny\ttfamily}

	\begin{itemize}
		\item Muy a menudo las imágenes son usadas en un contexto, donde hay elementos superpuestos con otros elementos DOM como una cabecera.
		\item Para dar a los editores una pista de qué área de la imagen se ve afectada, al seleccionar un área de recorte,
			es posible definir múltiples áreas de envoltura.
		\item Estas áreas son mostradas dentro del área de recorte. El área de enfoque no puede interseccionar co ninguna de las áreas de envoltura.

			\begin{lstlisting}
				'config' => [
				  'type' => 'imageManipulation',
				  'coverAreas' => [
				    [
				      'x' => 0.05, 'y' => 0.85,
				      'width' => 0.9, 'height' => 0.1,
				    ],
				  ],
				]
			\end{lstlisting}

	\end{itemize}

\end{frame}

% ------------------------------------------------------------------------------
% LTXE-SLIDE-START
% LTXE-SLIDE-UID:		528f2058-1c0b5bb3-35f86e8f-d232351f
% LTXE-SLIDE-ORIGIN:	c9df3305-35544e3c-6a073be7-1cec40f4 English
% LTXE-SLIDE-TITLE:		Image Manipulation Tool (6/6)
% LTXE-SLIDE-REFERENCE:	!Feature: #75880 - Implement multiple cropping variants in image manipulation tool
% ------------------------------------------------------------------------------

\begin{frame}[fragile]
	\frametitle{Cambios en Profundidad}
	\framesubtitle{Herramienta de Manipulación de Imágenes (6/6)}

	% decrease font size for code listing
	\lstset{basicstyle=\smaller\ttfamily}

	\begin{itemize}
		\item Para renderizar variantes de recorte, las variantes pueden ser especificadas como argumento del view helper image:

			\begin{lstlisting}
				<f:image image="{data.image}" cropVariant="mobile" width="800" >
				</f:image>
			\end{lstlisting}

	\end{itemize}

\end{frame}

% ------------------------------------------------------------------------------
% LTXE-SLIDE-START
% LTXE-SLIDE-UID:		1a6846cd-6106b88a-8995ef64-d462fd92
% LTXE-SLIDE-ORIGIN:	a71d77d1-6a419c8d-0fbb35d0-e1ce57e4 English
% LTXE-SLIDE-TITLE:		Default Content Element Changed for Fluid Styled Content
% LTXE-SLIDE-REFERENCE:	!Breaking: #79622 - Default content element changed for Fluid Styled Content
% ------------------------------------------------------------------------------

\begin{frame}[fragile]
	\frametitle{Cambios en Profundidad}
	\framesubtitle{Elemento de Contenido por Defecto Cambiado para Fluid Styled Content}

	% decrease font size for code listing
	\lstset{basicstyle=\tiny\ttfamily}

	\begin{itemize}
		\item El elemento de contenido por defecto ha sido coordinado con CSS Styled Content y ha sido cambiado a "Text"
		\item Para restaurar la configuración necesitas configurar el elemento de contenido por defecto manualmente a tu opción preferida.
			Puedes hacer esto simplemente sobreescribiendo la configuración otra vez en tu
			fichero \texttt{Configuration/TCA/Overrides/tt\_content.php}.

			\begin{lstlisting}
				$GLOBALS['TCA']['tt_content']['columns']['CType']['config']['default'] = 'textmedia';
				$GLOBALS['TCA']['tt_content']['columns']['CType']['config']['default'] = 'header';
			\end{lstlisting}

	\end{itemize}

\end{frame}

% ------------------------------------------------------------------------------
% LTXE-SLIDE-START
% LTXE-SLIDE-UID:		19a35432-8e423d90-7c286bbd-926f7032
% LTXE-SLIDE-ORIGIN:	035f6a19-9be2f793-bc0ebb3e-c6a4cca4 English
% LTXE-SLIDE-TITLE:		TCA Changes (1/2)
% LTXE-SLIDE-REFERENCE:	!Deprecation: #79440 - TCA Changes
% ------------------------------------------------------------------------------

\begin{frame}[fragile]
	\frametitle{Cambios en Profundidad}
	\framesubtitle{Cambios de TCA (1/2)}

	\begin{itemize}

		\item El \texttt{TCA} a nivel de campo ha sido cambiado.

		\item Casi todos los tipos de columnas están afectados.

		\item En general, la subsección \texttt{wizards} se ha eliminado y reemplazado por una combinación de nuevo
			\texttt{renderType} y un nuevo conjunto de opciones de configuración.

		\item Los asistentes están ahora divididos en tres tipos diferentes: \texttt{fieldInformation}, \texttt{fieldControl}
			y \texttt{fieldWizard}.

	\end{itemize}

\end{frame}
% ------------------------------------------------------------------------------
% LTXE-SLIDE-START
% LTXE-SLIDE-UID:		09579b9d-670b7ecf-5e79756a-d29ae57d
% LTXE-SLIDE-ORIGIN:	55c8113e-b234daa0-e701f035-8e1b58d1 English
% LTXE-SLIDE-TITLE:		TCA Changes (2/2)
% LTXE-SLIDE-REFERENCE:	!Deprecation: #79440 - TCA Changes
% ------------------------------------------------------------------------------

\begin{frame}[fragile]
	\frametitle{Cambios en Profundidad}
	\framesubtitle{Cambios de TCA (2/2)}

	% decrease font size for code listing
	\lstset{basicstyle=\tiny\ttfamily}

	\begin{itemize}
		\item Ejemplo:

			\begin{lstlisting}
				'fieldControl' => [
				  'editPopup' => [
				    'disabled' => false,
				  ],
				  'addRecord' => [
				    'disabled' => false,
				    'options' => [
				      'setValue' => 'prepend',
				    ],
				  ],
				  'listModule' => [
				    'disabled' => false,
				  ],
				],
			\end{lstlisting}

		\item Puedes encontrar más detalles en
			\href{https://docs.typo3.org/typo3cms/extensions/core/8-dev/singlehtml/Index.html#deprecation-79440-formengine-element-expansion}{docs.typo3.org}

	\end{itemize}

\end{frame}


% ------------------------------------------------------------------------------
% LTXE-SLIDE-START
% LTXE-SLIDE-UID:		3adbd753-0972bd54-7e149d7f-3c3349ee
% LTXE-SLIDE-ORIGIN:	da922107-1604619b-66f713fe-9508d9d7 English
% LTXE-SLIDE-TITLE:		Introduce Session Storage Framework
% LTXE-SLIDE-REFERENCE:	!Feature: #70316 - Introduce Session Storage Framework
% ------------------------------------------------------------------------------

\begin{frame}[fragile]
	\frametitle{Cambios en Profundidad}
	\framesubtitle{Introducir Marco de Trabajo de Almacenamiento de Sesión}

	\begin{itemize}
		\item Un nuevo marco de trabajo de almacenamiento de sesión ha sido introducido
		\item El objetivo de este marco de trabajo es crear interoperabilidad entre los diferentes almacenamientos de sesión
			(llamados "backends") como la base de datos, almacenamiento de ficheros, Redis, etc.
		\item Los siguientes backends de sesión están disponibles por defecto:

			\begin{itemize}
				\item
					\smaller\texttt{\textbackslash
						TYPO3\textbackslash
						CMS\textbackslash
						Core\textbackslash
						Session\textbackslash
						Backend\textbackslash
						DatabaseSessionBackend}

				\item
					\texttt{\textbackslash
						TYPO3\textbackslash
						CMS\textbackslash
						Core\textbackslash
						Session\textbackslash
						Backend\textbackslash
						RedisSessionBackend}
			\end{itemize}
	\end{itemize}

\end{frame}

% ------------------------------------------------------------------------------
% LTXE-SLIDE-START
% LTXE-SLIDE-UID:		a97ba207-618aa302-6c2ff099-e0055de2
% LTXE-SLIDE-ORIGIN:	2d65beca-277a2f35-a1bc5ae7-c71712aa English
% LTXE-SLIDE-TITLE:		CLI Support for T3D Imports
% LTXE-SLIDE-REFERENCE:	!Feature: #72749 - CLI support for T3D import
% ------------------------------------------------------------------------------

\begin{frame}[fragile]
	\frametitle{Cambios en Profundidad}
	\framesubtitle{Soporte CLI para Importaciones T3D}

	\begin{itemize}
		\item \texttt{EXT:impexp} ahora permite importar ficheros de datos (T3D o XML) vía el interfaz de la línea
			de comandos a través de Symfony Command.

		\item Uso:\newline
			\smaller
				\texttt{./typo3/sysext/core/bin/typo3 impexp:import [<options>] <file> <pageId>}
			\normalsize

		\item Opciones:
			\begin{itemize}
				\item \texttt{-}\texttt{-updateRecords}: Fuerza la actualización de registros existentes
				\item \texttt{-}\texttt{-ignorePid}: No corregir los ids de página de registros actualizados
				\item \texttt{-}\texttt{-enableLog}: Registra toda la acción de base de datos
			\end{itemize}

	\end{itemize}

\end{frame}

% ------------------------------------------------------------------------------
% LTXE-SLIDE-START
% LTXE-SLIDE-UID:		5a07520b-b86a6583-b68e9a1f-4452b5c2
% LTXE-SLIDE-ORIGIN:	38064844-53baa89a-7f73669e-2bb96cab English
% LTXE-SLIDE-TITLE:		Hook in typolink for Modification of Page Params
% LTXE-SLIDE-REFERENCE:	!Feature: #79121 - Implement hook in typolink for modification of page params
% ------------------------------------------------------------------------------

\begin{frame}[fragile]
	\frametitle{Cambios en Profundidad}
	\framesubtitle{Implementar Hook en \texttt{typolink} para Modificación de Parámetros de Página}

	% decrease font size for code listing
	\lstset{basicstyle=\tiny\ttfamily}

	\begin{itemize}
		\item Un nuevo hook ha sido implementado en \texttt{ContentObjectRenderer::typoLink} para enlaces a páginas.
			Con este hook puedes modificar la configuración de enlace, por ejemplo enriqueciéndola con parámetros
			adicionales or meta datos de la fila de página.

		\item Ahora puedes registrar un hook vía:

			\begin{lstlisting}
				$GLOBALS['TYPO3_CONF_VARS']['SC_OPTIONS']['typolinkProcessing']
				  ['typolinkModifyParameterForPageLinks'][] = \Your\Namespace\Hooks\MyHook::class;
			\end{lstlisting}

		\item Uso:

			\begin{lstlisting}
				public function modifyPageLinkConfiguration(
				  array $linkConfiguration, array $linkDetails, array $pageRow) : array
				{
				  $linkConfiguration['additionalParams'] .= $pageRow['myAdditionalParamsField'];
				  return $linkConfiguration;
				}
			\end{lstlisting}

	\end{itemize}

\end{frame}

% ------------------------------------------------------------------------------
% LTXE-SLIDE-START
% LTXE-SLIDE-UID:		63b1b507-668d03c8-21b3ba58-4130a94b
% LTXE-SLIDE-ORIGIN:	290e1c2e-e708ede1-82aaf127-e69b2338 English
% LTXE-SLIDE-TITLE:		Hook to Add Custom TypoScript Templates (1/2)
% LTXE-SLIDE-REFERENCE:	!Feature: #79140 - Add hook to add custom TypoScript templates
% ------------------------------------------------------------------------------

\begin{frame}[fragile]
	\frametitle{Cambios en Profundidad}
	\framesubtitle{Hook para Añadir Templates TypoScript Templates Personalizados (1/2)}

	% decrease font size for code listing
	\lstset{basicstyle=\tiny\ttfamily}

	\begin{itemize}
		\item Un nuevo hook en TemplateService permite añadir o modificar templates existentes de TypoScript.

		\item Ahora puedes registrar un hook a través del siguiente código en el fichero de extensiones \texttt{ext\_localconf.php}:

			\begin{lstlisting}
				$GLOBALS['TYPO3_CONF_VARS']['SC_OPTIONS']['Core/TypoScript/TemplateService']
				  ['runThroughTemplatesPostProcessing']
			\end{lstlisting}

		\item \texttt{EXT:my\_site/Classes/Hooks/TypoScriptHook.php} (1/2)

			\begin{lstlisting}
				namespace MyVendor\MySite\Hooks;
				class TypoScriptHook
				{
				  /**
				   * Hooks into TemplateService after
				   * @param array $parameters
				   * @param \TYPO3\CMS\Core\TypoScript\TemplateService $parentObject
				   * @return void
				   */
				...
			\end{lstlisting}

	\end{itemize}

\end{frame}

% ------------------------------------------------------------------------------
% LTXE-SLIDE-START
% LTXE-SLIDE-UID:		ac13b681-50e27e60-95d91367-bd42a1ca
% LTXE-SLIDE-ORIGIN:	d64368f4-0f5c43e3-92a546bf-75d50a61 English
% LTXE-SLIDE-TITLE:		Hook to Add Custom TypoScript Templates (2/2)
% LTXE-SLIDE-REFERENCE:	!Feature: #79140 - Add hook to add custom TypoScript templates
% ------------------------------------------------------------------------------

\begin{frame}[fragile]
	\frametitle{Cambios en Profundidad}
	\framesubtitle{Hook para Añadir Templates TypoScript Templates Personalizados (2/2)}

	% decrease font size for code listing
	\lstset{basicstyle=\tiny\ttfamily}

	\begin{itemize}
		\item \texttt{EXT:my\_site/Classes/Hooks/TypoScriptHook.php} (2/2)

			\begin{lstlisting}
				...
				  public function addCustomTypoScriptTemplate($parameters, $parentObject)
				  {
				    // Disable the inclusion of default TypoScript set via TYPO3_CONF_VARS
				    $parameters['isDefaultTypoScriptAdded'] = true;
				    // Disable the inclusion of ext_typoscript_setup.txt of all extensions
				    $parameters['processExtensionStatics'] = false;

				    // No template was found in rootline so far, so a custom "fake" sys_template record is added
				    if ($parentObject->outermostRootlineIndexWithTemplate === 0) {
				      $row = [
				        'uid' => 'my_site_template',
				        'config' =>
					      '<INCLUDE_TYPOSCRIPT: source="FILE:EXT:my_site/Configuration/TypoScript/site_setup.t3s">',
				        'root' => 1,
				        'pid' => 0
				      ];
				      $parentObject->processTemplate($row, 'sys_' . $row['uid'], 0, 'sys_' . $row['uid']);
				    }
				  }
				}
			\end{lstlisting}

	\end{itemize}

\end{frame}

% ------------------------------------------------------------------------------
% LTXE-SLIDE-START
% LTXE-SLIDE-UID:		0daae405-823db90d-4c315b67-2d5a1e9d
% LTXE-SLIDE-ORIGIN:	ad9f66e3-bcd548f4-8e14e04c-de29a79c English
% LTXE-SLIDE-TITLE:		Plugin Preview with Fluid
% LTXE-SLIDE-REFERENCE:	!Feature: #79225 - Plugin preview with Fluid
% ------------------------------------------------------------------------------
\begin{frame}[fragile]
	\frametitle{Cambios en Profundidad}
	\framesubtitle{Previsualización de Plugin con Fluid}

	% decrease font size for code listing
	\lstset{basicstyle=\tiny\ttfamily}

	\begin{itemize}
		\item El TSconfig de página para renderizar una previsualización de un único elemento de contenido en el Backend ha sido mejorado
			permitiendo el renderizado de plugins vía Fluid también

		\item Todas las propiedades del registro \texttt{tt\_content} están disponibles en el template directamente (p.e. UID vía \{uid\})

		\item Cualquier dato del campo flexform \texttt{pi\_flexform} está disponible con la propiedad
			\texttt{pi\_flexform\_transformed} como un vector.

			\begin{lstlisting}
				mod.web_layout.tt_content.preview.list.simpleblog_bloglisting =
				  EXT:simpleblog/Resources/Private/Templates/Preview/SimpleblogPlugin.html
			\end{lstlisting}

	\end{itemize}

\end{frame}

% ------------------------------------------------------------------------------
% LTXE-SLIDE-START
% LTXE-SLIDE-UID:		1009cc55-b91b210f-944b927e-df732cdb
% LTXE-SLIDE-ORIGIN:	245de018-04aed929-a3b3c380-6c822a14 English
% LTXE-SLIDE-TITLE:		Template Paths in BackendTemplateView
% LTXE-SLIDE-REFERENCE:	!Feature: #79124 - Allow overwriting of template paths in BackendTemplateView
% ------------------------------------------------------------------------------

\begin{frame}[fragile]
	\frametitle{Cambios en Profundidad}
	\framesubtitle{Rutas de Template en BackendTemplateView}

	% decrease font size for code listing
	\lstset{basicstyle=\tiny\ttfamily}

	\begin{itemize}
		\item BackendTemplateView ahora permite la sobreescritura de rutas de template para añadir tus propias localizaciones para templates,
			partials y layouts en un módulo de backend basado en BackendTemplateView.

			\begin{lstlisting}
				$frameworkConfiguration =
				  $this->configurationManager->getConfiguration(
				    ConfigurationManagerInterface::CONFIGURATION_TYPE_FRAMEWORK
				  );
				$viewConfiguration = [
				  'view' => [
				    'templateRootPaths' => ['EXT:myext/Resources/Private/Backend/Templates'],
				    'partialRootPaths' => ['EXT:myext/Resources/Private/Backend/Partials'],
				    'layoutRootPaths' => ['EXT:myext/Resources/Private/Backend/Layouts'],
				  ],
				];
				$this->configurationManager->setConfiguration(
				  array_merge($frameworkConfiguration, $viewConfiguration)
				);
			\end{lstlisting}

	\end{itemize}

\end{frame}

% ------------------------------------------------------------------------------
% LTXE-SLIDE-START
% LTXE-SLIDE-UID:		2ae04d43-ac1352aa-fc97b80e-0d9ca287
% LTXE-SLIDE-ORIGIN:	82514d33-f6c72709-2052880d-4d753fe8 English
% LTXE-SLIDE-TITLE:		Miscellaneous
% LTXE-SLIDE-REFERENCE:	!Feature: #78899 - TCA maxitems optional
% LTXE-SLIDE-REFERENCE:	!Feature: #79240 - Single cli user for cli commands
% ------------------------------------------------------------------------------

\begin{frame}[fragile]
	\frametitle{Cambios en Profundidad}
	\framesubtitle{Miscelánea}

	\begin{itemize}
		\item El ajuste de configuración de \texttt{TCA} \texttt{maxitems} para los campos \texttt{type=select} y \texttt{type=group}
			es ahora un ajuste opcional que por defecto tiene un valor alto (99999) en lugar de 1 como antes.

		\item El acceso a la funcionalidad TYPO3 desde la línea de comandos ha sido simplificado. Comandos únicos no requieren más
			usuarios únicos en la base de datos, en su lugar todos los comandos cli usan el usuario \texttt{\_cli\_}.
			Este usuario es creado por demanda del marco de trabajo si no existe en la primera llamada de la línea de comandos.
			El usuario \texttt{\_cli\_} tiene derechos de administración y no necesita más derechos específicos de acceso para
			realizar tareas específicas como manipular contenido de la base de datos usando el \texttt{DataHandler}.

	\end{itemize}

\end{frame}

% ------------------------------------------------------------------------------

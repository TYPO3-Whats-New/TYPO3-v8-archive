% ------------------------------------------------------------------------------
% TYPO3 CMS 8.6 - What's New - Chapter "In-Depth Changes" (German Version)
%
% @author	Michael Schams <schams.net>
% @license	Creative Commons BY-NC-SA 3.0
% @link		http://typo3.org/download/release-notes/whats-new/
% @language	English
% ------------------------------------------------------------------------------
% LTXE-CHAPTER-UID:		5ebcecbe-66abfa57-cf38bc00-aa637965
% LTXE-CHAPTER-NAME:	In-Depth Changes
% ------------------------------------------------------------------------------

\section{In-Depth Changes}
\begin{frame}[fragile]
	\frametitle{In-Depth Changes}

	\begin{center}\huge{Chapter 3:}\end{center}
	\begin{center}\huge{\color{typo3darkgrey}\textbf{In-Depth Changes}}\end{center}

\end{frame}

% ------------------------------------------------------------------------------
% LTXE-SLIDE-START
% LTXE-SLIDE-UID:		4c1682d4-b57fad8e-ff8134b5-2c5581ea
% LTXE-SLIDE-ORIGIN:	5ae3ff92-e2f03af9-8f5caf79-b7e0aca1 English
% LTXE-SLIDE-TITLE:		Page Browser for scheduler tasks
% LTXE-SLIDE-REFERENCE:	!Feature: #12211 - Usability: Scheduler provide page browser to choose start page
% ------------------------------------------------------------------------------

\begin{frame}[fragile]
	\frametitle{In-Depth Changes}
	\framesubtitle{Page Browser for scheduler tasks}

	\begin{itemize}
		\item Scheduler tasks that need a page \texttt{uid} can now add a button for the page browser popup.

		\item In the \texttt{ValidatorTaskAdditionalFieldProvider} two additional fields have to be added.

		\item If the additional field \texttt{browser} is set to \texttt{page} then the
			\texttt{SchedulerModuleController} adds a button for calling the page browser popup to the field.

			\begin{lstlisting}
				'browser' => 'page',
			\end{lstlisting}

		\item The \texttt{pageTitle} contains the title of the page that is shown next to the browse button.

			\begin{lstlisting}
				'pageTitle' => $pageTitle,
			\end{lstlisting}

	\end{itemize}

\end{frame}

% ------------------------------------------------------------------------------
% LTXE-SLIDE-START
% LTXE-SLIDE-UID:		9752c961-3f327c60-06089304-24c71a0c
% LTXE-SLIDE-ORIGIN:	02a7fe32-ae3fd901-b04372a2-7d3726f4 English
% LTXE-SLIDE-TITLE:		Synchronized field values in localized records (1/2)
% LTXE-SLIDE-REFERENCE:	!Feature: #51291 - Synchronized field values in localized records
% ------------------------------------------------------------------------------

\begin{frame}[fragile]
	\frametitle{In-Depth Changes}
	\framesubtitle{Synchronized field values in localized records (1/2)}

	% decrease font size for code listing
	\lstset{basicstyle=\tiny\ttfamily}

	\begin{itemize}
		\item The localized record overlay behaviour has been changed to make localization-rows standalone.

		\item Previously, if fields in \texttt{TCA} columns were set to \texttt{l10n\_mode} \texttt{exclude} or
			\texttt{mergeIfNotBlank}, the localized record overlay did not contain values, and those values were
			"pulled up" from the underlying default language records.

	\end{itemize}

\end{frame}

% ------------------------------------------------------------------------------
% LTXE-SLIDE-START
% LTXE-SLIDE-UID:		e7323e00-69dcfe25-bac99617-984bcd3c
% LTXE-SLIDE-ORIGIN:	cf030e7c-03fb0ac4-9b9ff848-ac69581b English
% LTXE-SLIDE-TITLE:		Synchronized field values in localized records (2/2)
% LTXE-SLIDE-REFERENCE:	!Feature: #51291 - Synchronized field values in localized records
% ------------------------------------------------------------------------------

\begin{frame}[fragile]
	\frametitle{In-Depth Changes}
	\framesubtitle{Synchronized field values in localized records (2/2)}

	% decrease font size for code listing
	\lstset{basicstyle=\tiny\ttfamily}

	\begin{itemize}
		\item This has been changed, the \texttt{DataHandler} now copies those values over to the localized
			record and synchronizes them if the default language record is changed.

			\begin{lstlisting}
				'columns' => [
				  ...
				  'header' => [
				    'label' => 'My header',
				    'config' => [
				      'type' => 'input',
				      'behaviour' => [
				        'allowLanguageSynchronization' => true,
				      ],
				    ],
				  ],
				],
			\end{lstlisting}

	\end{itemize}

\end{frame}

% ------------------------------------------------------------------------------
% LTXE-SLIDE-START
% LTXE-SLIDE-UID:		7fcb8a9e-12e499af-a6d0fdac-0ee13912
% LTXE-SLIDE-ORIGIN:	18c4f826-440f3e00-a854e8e8-ddc7cbc0 English
% LTXE-SLIDE-TITLE:		Image Manipulation Tool (1/6)
% LTXE-SLIDE-REFERENCE:	!Feature: #75880 - Implement multiple cropping variants in image manipulation tool
% ------------------------------------------------------------------------------

\begin{frame}[fragile]
	\frametitle{In-Depth Changes}
	\framesubtitle{Image Manipulation Tool (1/6)}

	% decrease font size for code listing
	\lstset{basicstyle=\tiny\ttfamily}

	\begin{itemize}
		\item The \texttt{imageManipulation} TCA type is now capable of handling multiple crop variants if configured.

		\item It is also possible to define an initial crop area. If no initial crop area is defined,
			the default selected crop area will cover the complete image.
		\item Crop areas are defined relatively with floating point numbers. The coordinates and
			sizes must be specified for that.

	\end{itemize}

\end{frame}

% ------------------------------------------------------------------------------
% LTXE-SLIDE-START
% LTXE-SLIDE-UID:		2331dc5e-44d6bceb-01dc0c9e-0df35003
% LTXE-SLIDE-ORIGIN:	36ea74c0-3b92dbf6-e35c9e25-af0a0abf English
% LTXE-SLIDE-TITLE:		Image Manipulation Tool (2/6)
% LTXE-SLIDE-REFERENCE:	!Feature: #75880 - Implement multiple cropping variants in image manipulation tool
% ------------------------------------------------------------------------------

\begin{frame}[fragile]
	\frametitle{In-Depth Changes}
	\framesubtitle{Image Manipulation Tool (2/6)}

	% decrease font size for code listing
	\lstset{basicstyle=\tiny\ttfamily}

	\begin{itemize}

		\item The following example configures two crop variants,
			one with the id "mobile", one with the id "desktop".
			The array key defines the id, which will be used when
			rendering an image with the image view helper.

			\begin{lstlisting}
				'config' => [
				  'type' => 'imageManipulation',
				  'cropVariants' => [
				    'mobile' => [
				      'title' => 'Mobile',
				      'allowedAspectRatios' => [
				        '4:3' => [
				          'title' => '4:3',
				          'value' => 4 / 3
				        ],
				        ...
				      ],
				    ],
				    'desktop' => [
				      ...
				    ],
				  ],
				]
			\end{lstlisting}

	\end{itemize}

\end{frame}

% ------------------------------------------------------------------------------
% LTXE-SLIDE-START
% LTXE-SLIDE-UID:		415f9209-fb4c1fb8-096fb2e2-64de63a7
% LTXE-SLIDE-ORIGIN:	27b20030-548f80c4-3540aabb-41a83498 English
% LTXE-SLIDE-TITLE:		Image Manipulation Tool (3/6)
% LTXE-SLIDE-REFERENCE:	!Feature: #75880 - Implement multiple cropping variants in image manipulation tool
% ------------------------------------------------------------------------------

\begin{frame}[fragile]
	\frametitle{In-Depth Changes}
	\framesubtitle{Image Manipulation Tool (3/6)}

	% decrease font size for code listing
	\lstset{basicstyle=\tiny\ttfamily}

	\begin{itemize}

		\item The below example has an initial crop area in the size
			of the previous image cropper provided by default.

			\begin{lstlisting}
				'config' => [
				  'type' => 'imageManipulation',
				  'cropVariants' => [
				    'mobile' => [
				      'title' => 'LLL:EXT:ext_key/Resources/Private/Language/locallang.xlf:imageManipulation.mobile',
				      'cropArea' => [
				        'x' => 0.1,
				        'y' => 0.1,
				        'width' => 0.8,
				        'height' => 0.8,
				      ],
				    ],
				  ],
				]
			\end{lstlisting}

	\end{itemize}

\end{frame}

% ------------------------------------------------------------------------------
% LTXE-SLIDE-START
% LTXE-SLIDE-UID:		f4937a7b-c01025e7-569e5a7f-1c252fc5
% LTXE-SLIDE-ORIGIN:	58cebeda-4ee109f4-5e9eb728-b7703b40 English
% LTXE-SLIDE-TITLE:		Image Manipulation Tool (4/6)
% LTXE-SLIDE-REFERENCE:	!Feature: #75880 - Implement multiple cropping variants in image manipulation tool
% ------------------------------------------------------------------------------

\begin{frame}[fragile]
	\frametitle{In-Depth Changes}
	\framesubtitle{Image Manipulation Tool (4/6)}

	% decrease font size for code listing
	\lstset{basicstyle=\tiny\ttfamily}

	\begin{itemize}
		\item Users can also select a focus area, when configured.
		\item The focus area is always inside the crop area and mark the area in the image
			which must be visible for the image to transport its meaning.

			\begin{lstlisting}
				'config' => [
				  'type' => 'imageManipulation',
				  'cropVariants' => [
				    'mobile' => [
				      'title' =>
				        'LLL:EXT:ext_key/Resources/Private/Language/locallang.xlf:imageManipulation.mobile',
				      'focusArea' => [
				        'x' => 1 / 3,
				        'y' => 1 / 3,
				        'width' => 1 / 3,
				        'height' => 1 / 3,
				      ],
				    ],
				  ],
				]
			\end{lstlisting}

	\end{itemize}

\end{frame}

% ------------------------------------------------------------------------------
% LTXE-SLIDE-START
% LTXE-SLIDE-UID:		238217fb-82b38785-6d9d5540-570dbeb6
% LTXE-SLIDE-ORIGIN:	94d809e0-3369bf18-616710ee-93f6a9da English
% LTXE-SLIDE-TITLE:		Image Manipulation Tool (5/6)
% LTXE-SLIDE-REFERENCE:	!Feature: #75880 - Implement multiple cropping variants in image manipulation tool
% ------------------------------------------------------------------------------

\begin{frame}[fragile]
	\frametitle{In-Depth Changes}
	\framesubtitle{Image Manipulation Tool (5/6)}

	% decrease font size for code listing
	\lstset{basicstyle=\tiny\ttfamily}

	\begin{itemize}
		\item Very often images are used in a context, where there are overlayed with other DOM elements like a headline.
		\item To give editors a hint which area of the image is affected, when selecting a crop area,
			it is possible to define multiple cover areas.
		\item These areas are shown inside the crop area. The focus area cannot intersect with any of the cover areas.

			\begin{lstlisting}
				'config' => [
				  'type' => 'imageManipulation',
				  'coverAreas' => [
				    [
				      'x' => 0.05, 'y' => 0.85,
				      'width' => 0.9, 'height' => 0.1,
				    ],
				  ],
				]
			\end{lstlisting}

	\end{itemize}

\end{frame}

% ------------------------------------------------------------------------------
% LTXE-SLIDE-START
% LTXE-SLIDE-UID:		1a653344-f918e8a6-4d52f5ea-56cb7efc
% LTXE-SLIDE-ORIGIN:	c9df3305-35544e3c-6a073be7-1cec40f4 English
% LTXE-SLIDE-TITLE:		Image Manipulation Tool (6/6)
% LTXE-SLIDE-REFERENCE:	!Feature: #75880 - Implement multiple cropping variants in image manipulation tool
% ------------------------------------------------------------------------------

\begin{frame}[fragile]
	\frametitle{In-Depth Changes}
	\framesubtitle{Image Manipulation Tool (6/6)}

	% decrease font size for code listing
	\lstset{basicstyle=\smaller\ttfamily}

	\begin{itemize}
		\item To render crop variants, the variants can be specified as argument to the image view helper:

			\begin{lstlisting}
				<f:image image="{data.image}" cropVariant="mobile" width="800" >
				</f:image>
			\end{lstlisting}

	\end{itemize}

\end{frame}

% ------------------------------------------------------------------------------
% LTXE-SLIDE-START
% LTXE-SLIDE-UID:		0f5b26bd-e9a0de2d-460e254c-2d6b6122
% LTXE-SLIDE-ORIGIN:	a71d77d1-6a419c8d-0fbb35d0-e1ce57e4 English
% LTXE-SLIDE-TITLE:		Default Content Element Changed for Fluid Styled Content
% LTXE-SLIDE-REFERENCE:	!Breaking: #79622 - Default content element changed for Fluid Styled Content
% ------------------------------------------------------------------------------

\begin{frame}[fragile]
	\frametitle{In-Depth Changes}
	\framesubtitle{Default Content Element Changed for Fluid Styled Content}

	% decrease font size for code listing
	\lstset{basicstyle=\tiny\ttfamily}

	\begin{itemize}
		\item The default content element has been streamlined with CSS Styled Content and has been changed to "Text"
		\item To restore the configuration you need to set the default content element manually to your preferred choice.
			You can do this by simply overriding the configuration again in your
			\texttt{Configuration/TCA/Overrides/tt\_content.php} file.

			\begin{lstlisting}
				$GLOBALS['TCA']['tt_content']['columns']['CType']['config']['default'] = 'textmedia';
				$GLOBALS['TCA']['tt_content']['columns']['CType']['config']['default'] = 'header';
			\end{lstlisting}

	\end{itemize}

\end{frame}

% ------------------------------------------------------------------------------
% LTXE-SLIDE-START
% LTXE-SLIDE-UID:		4ab9c9f8-8f3c0662-320a0e90-a2f94918
% LTXE-SLIDE-ORIGIN:	035f6a19-9be2f793-bc0ebb3e-c6a4cca4 English
% LTXE-SLIDE-TITLE:		TCA Changes (1/2)
% LTXE-SLIDE-REFERENCE:	!Deprecation: #79440 - TCA Changes
% ------------------------------------------------------------------------------

\begin{frame}[fragile]
	\frametitle{In-Depth Changes}
	\framesubtitle{TCA Changes (1/2)}

	\begin{itemize}

		\item The \texttt{TCA} on field level has been changed.

		\item Nearly all column types are affected.

		\item In general, the sub-section \texttt{wizards} is gone and replaced by a combination of new
			\texttt{renderType} and a new set of configuration options.

		\item Wizards are now divided into three different kinds: \texttt{fieldInformation}, \texttt{fieldControl}
			and \texttt{fieldWizard}.

	\end{itemize}

\end{frame}
% ------------------------------------------------------------------------------
% LTXE-SLIDE-START
% LTXE-SLIDE-UID:		e95e5ec7-3357c68a-35fb2c65-bf49e9a7
% LTXE-SLIDE-ORIGIN:	55c8113e-b234daa0-e701f035-8e1b58d1 English
% LTXE-SLIDE-TITLE:		TCA Changes (2/2)
% LTXE-SLIDE-REFERENCE:	!Deprecation: #79440 - TCA Changes
% ------------------------------------------------------------------------------

\begin{frame}[fragile]
	\frametitle{In-Depth Changes}
	\framesubtitle{TCA Changes (2/2)}

	% decrease font size for code listing
	\lstset{basicstyle=\tiny\ttfamily}

	\begin{itemize}
		\item Example:

			\begin{lstlisting}
				'fieldControl' => [
				  'editPopup' => [
				    'disabled' => false,
				  ],
				  'addRecord' => [
				    'disabled' => false,
				    'options' => [
				      'setValue' => 'prepend',
				    ],
				  ],
				  'listModule' => [
				    'disabled' => false,
				  ],
				],
			\end{lstlisting}

		\item You find further details at
			\href{https://docs.typo3.org/typo3cms/extensions/core/8-dev/singlehtml/Index.html#deprecation-79440-formengine-element-expansion}{docs.typo3.org}

	\end{itemize}

\end{frame}


% ------------------------------------------------------------------------------
% LTXE-SLIDE-START
% LTXE-SLIDE-UID:		b038ce08-003f2d14-c1afaabf-cb05bca8
% LTXE-SLIDE-ORIGIN:	da922107-1604619b-66f713fe-9508d9d7 English
% LTXE-SLIDE-TITLE:		Introduce Session Storage Framework
% LTXE-SLIDE-REFERENCE:	!Feature: #70316 - Introduce Session Storage Framework
% ------------------------------------------------------------------------------

\begin{frame}[fragile]
	\frametitle{In-Depth Changes}
	\framesubtitle{Introduce Session Storage Framework}

	\begin{itemize}
		\item A new session storage framework has been introduced
		\item The goal of this framework is to create interoperability between different session storages
			(called "backends") like database, file storage, Redis, etc.
		\item The following session backends are available by default:

			\begin{itemize}
				\item
					\smaller\texttt{\textbackslash
						TYPO3\textbackslash
						CMS\textbackslash
						Core\textbackslash
						Session\textbackslash
						Backend\textbackslash
						DatabaseSessionBackend}

				\item
					\texttt{\textbackslash
						TYPO3\textbackslash
						CMS\textbackslash
						Core\textbackslash
						Session\textbackslash
						Backend\textbackslash
						RedisSessionBackend}
			\end{itemize}
	\end{itemize}

\end{frame}

% ------------------------------------------------------------------------------
% LTXE-SLIDE-START
% LTXE-SLIDE-UID:		f3a5ca84-72eb8a73-c8b8e66b-cfad14bb
% LTXE-SLIDE-ORIGIN:	2d65beca-277a2f35-a1bc5ae7-c71712aa English
% LTXE-SLIDE-TITLE:		CLI Support for T3D Imports
% LTXE-SLIDE-REFERENCE:	!Feature: #72749 - CLI support for T3D import
% ------------------------------------------------------------------------------

\begin{frame}[fragile]
	\frametitle{In-Depth Changes}
	\framesubtitle{CLI Support for T3D Imports}

	\begin{itemize}
		\item \texttt{EXT:impexp} now allows to import data files (T3D or XML) via the command line
			interface through a Symfony Command.

		\item Usage:\newline
			\smaller
				\texttt{./typo3/sysext/core/bin/typo3 impexp:import [<options>] <file> <pageId>}
			\normalsize

		\item Options:
			\begin{itemize}
				\item \texttt{-}\texttt{-updateRecords}: Force updating existing records
				\item \texttt{-}\texttt{-ignorePid}: Don't correct page ids of updated records
				\item \texttt{-}\texttt{-enableLog}: log all database action
			\end{itemize}

	\end{itemize}

\end{frame}

% ------------------------------------------------------------------------------
% LTXE-SLIDE-START
% LTXE-SLIDE-UID:		530f99c0-e0950c25-8f12e822-3ad2f9ba
% LTXE-SLIDE-ORIGIN:	38064844-53baa89a-7f73669e-2bb96cab English
% LTXE-SLIDE-TITLE:		Hook in typolink for Modification of Page Params
% LTXE-SLIDE-REFERENCE:	!Feature: #79121 - Implement hook in typolink for modification of page params
% ------------------------------------------------------------------------------

\begin{frame}[fragile]
	\frametitle{In-Depth Changes}
	\framesubtitle{Implement Hook in \texttt{typolink} for Modification of Page Params}

	% decrease font size for code listing
	\lstset{basicstyle=\tiny\ttfamily}

	\begin{itemize}
		\item A new hook has been implemented in \texttt{ContentObjectRenderer::typoLink} for links to pages.
			With this hook you can modify the link configuration, for example enriching it with additional
			parameters or meta data from the page row.

		\item You can now register a hook via:

			\begin{lstlisting}
				$GLOBALS['TYPO3_CONF_VARS']['SC_OPTIONS']['typolinkProcessing']
				  ['typolinkModifyParameterForPageLinks'][] = \Your\Namespace\Hooks\MyHook::class;
			\end{lstlisting}

		\item Usage:

			\begin{lstlisting}
				public function modifyPageLinkConfiguration(
				  array $linkConfiguration, array $linkDetails, array $pageRow) : array
				{
				  $linkConfiguration['additionalParams'] .= $pageRow['myAdditionalParamsField'];
				  return $linkConfiguration;
				}
			\end{lstlisting}

	\end{itemize}

\end{frame}

% ------------------------------------------------------------------------------
% LTXE-SLIDE-START
% LTXE-SLIDE-UID:		f5b56ed0-04436105-2525b076-1f551f79
% LTXE-SLIDE-ORIGIN:	290e1c2e-e708ede1-82aaf127-e69b2338 English
% LTXE-SLIDE-TITLE:		Hook to Add Custom TypoScript Templates (1/2)
% LTXE-SLIDE-REFERENCE:	!Feature: #79140 - Add hook to add custom TypoScript templates
% ------------------------------------------------------------------------------

\begin{frame}[fragile]
	\frametitle{In-Depth Changes}
	\framesubtitle{Hook to Add Custom TypoScript Templates (1/2)}

	% decrease font size for code listing
	\lstset{basicstyle=\tiny\ttfamily}

	\begin{itemize}
		\item A new hook in TemplateService allows to add or modify existing TypoScript templates.

		\item You can now register a hook via the following code in the extensions \texttt{ext\_localconf.php} file:

			\begin{lstlisting}
				$GLOBALS['TYPO3_CONF_VARS']['SC_OPTIONS']['Core/TypoScript/TemplateService']
				  ['runThroughTemplatesPostProcessing']
			\end{lstlisting}

		\item \texttt{EXT:my\_site/Classes/Hooks/TypoScriptHook.php} (1/2)

			\begin{lstlisting}
				namespace MyVendor\MySite\Hooks;
				class TypoScriptHook
				{
				  /**
				   * Hooks into TemplateService after
				   * @param array $parameters
				   * @param \TYPO3\CMS\Core\TypoScript\TemplateService $parentObject
				   * @return void
				   */
				...
			\end{lstlisting}

	\end{itemize}

\end{frame}

% ------------------------------------------------------------------------------
% LTXE-SLIDE-START
% LTXE-SLIDE-UID:		4551cadf-8df1f5ff-81d7417f-f7b03cdc
% LTXE-SLIDE-ORIGIN:	d64368f4-0f5c43e3-92a546bf-75d50a61 English
% LTXE-SLIDE-TITLE:		Hook to Add Custom TypoScript Templates (2/2)
% LTXE-SLIDE-REFERENCE:	!Feature: #79140 - Add hook to add custom TypoScript templates
% ------------------------------------------------------------------------------

\begin{frame}[fragile]
	\frametitle{In-Depth Changes}
	\framesubtitle{Hook to Add Custom TypoScript Templates (2/2)}

	% decrease font size for code listing
	\lstset{basicstyle=\tiny\ttfamily}

	\begin{itemize}
		\item \texttt{EXT:my\_site/Classes/Hooks/TypoScriptHook.php} (2/2)

			\begin{lstlisting}
				...
				  public function addCustomTypoScriptTemplate($parameters, $parentObject)
				  {
				    // Disable the inclusion of default TypoScript set via TYPO3_CONF_VARS
				    $parameters['isDefaultTypoScriptAdded'] = true;
				    // Disable the inclusion of ext_typoscript_setup.txt of all extensions
				    $parameters['processExtensionStatics'] = false;

				    // No template was found in rootline so far, so a custom "fake" sys_template record is added
				    if ($parentObject->outermostRootlineIndexWithTemplate === 0) {
				      $row = [
				        'uid' => 'my_site_template',
				        'config' =>
					      '<INCLUDE_TYPOSCRIPT: source="FILE:EXT:my_site/Configuration/TypoScript/site_setup.t3s">',
				        'root' => 1,
				        'pid' => 0
				      ];
				      $parentObject->processTemplate($row, 'sys_' . $row['uid'], 0, 'sys_' . $row['uid']);
				    }
				  }
				}
			\end{lstlisting}

	\end{itemize}

\end{frame}

% ------------------------------------------------------------------------------
% LTXE-SLIDE-START
% LTXE-SLIDE-UID:		fcac3246-c02790cb-82913817-c94b812d
% LTXE-SLIDE-ORIGIN:	ad9f66e3-bcd548f4-8e14e04c-de29a79c English
% LTXE-SLIDE-TITLE:		Plugin Preview with Fluid
% LTXE-SLIDE-REFERENCE:	!Feature: #79225 - Plugin preview with Fluid
% ------------------------------------------------------------------------------
\begin{frame}[fragile]
	\frametitle{In-Depth Changes}
	\framesubtitle{Plugin Preview with Fluid}

	% decrease font size for code listing
	\lstset{basicstyle=\tiny\ttfamily}

	\begin{itemize}
		\item The page TSconfig to render a preview of a single content element in the Backend has been improved
			by allowing the rendering of plugins via Fluid as well

		\item All properties of the \texttt{tt\_content} record are available in the template directly (e.g. UID via \{uid\})

		\item Any data of the flexform field \texttt{pi\_flexform} is available with the property
			\texttt{pi\_flexform\_transformed} as an array.

			\begin{lstlisting}
				mod.web_layout.tt_content.preview.list.simpleblog_bloglisting =
				  EXT:simpleblog/Resources/Private/Templates/Preview/SimpleblogPlugin.html
			\end{lstlisting}

	\end{itemize}

\end{frame}

% ------------------------------------------------------------------------------
% LTXE-SLIDE-START
% LTXE-SLIDE-UID:		daf423fa-ce418a18-bd0a4581-5fe3388f
% LTXE-SLIDE-ORIGIN:	245de018-04aed929-a3b3c380-6c822a14 English
% LTXE-SLIDE-TITLE:		Template Paths in BackendTemplateView
% LTXE-SLIDE-REFERENCE:	!Feature: #79124 - Allow overwriting of template paths in BackendTemplateView
% ------------------------------------------------------------------------------

\begin{frame}[fragile]
	\frametitle{In-Depth Changes}
	\framesubtitle{Template Paths in BackendTemplateView}

	% decrease font size for code listing
	\lstset{basicstyle=\tiny\ttfamily}

	\begin{itemize}
		\item BackendTemplateView now allows overwriting of template paths to add your own locations for templates,
			partials and layouts in a BackendTemplateView based backend module.

			\begin{lstlisting}
				$frameworkConfiguration =
				  $this->configurationManager->getConfiguration(
				    ConfigurationManagerInterface::CONFIGURATION_TYPE_FRAMEWORK
				  );
				$viewConfiguration = [
				  'view' => [
				    'templateRootPaths' => ['EXT:myext/Resources/Private/Backend/Templates'],
				    'partialRootPaths' => ['EXT:myext/Resources/Private/Backend/Partials'],
				    'layoutRootPaths' => ['EXT:myext/Resources/Private/Backend/Layouts'],
				  ],
				];
				$this->configurationManager->setConfiguration(
				  array_merge($frameworkConfiguration, $viewConfiguration)
				);
			\end{lstlisting}

	\end{itemize}

\end{frame}

% ------------------------------------------------------------------------------
% LTXE-SLIDE-START
% LTXE-SLIDE-UID:		da4b1203-138c9f78-665d18ad-41359002
% LTXE-SLIDE-ORIGIN:	82514d33-f6c72709-2052880d-4d753fe8 English
% LTXE-SLIDE-TITLE:		Miscellaneous
% LTXE-SLIDE-REFERENCE:	!Feature: #78899 - TCA maxitems optional
% LTXE-SLIDE-REFERENCE:	!Feature: #79240 - Single cli user for cli commands
% ------------------------------------------------------------------------------

\begin{frame}[fragile]
	\frametitle{In-Depth Changes}
	\framesubtitle{Miscellaneous}

	\begin{itemize}
		\item The \texttt{TCA} config setting \texttt{maxitems} for \texttt{type=select} and \texttt{type=group}
			fields is now an optional setting that defaults to a high value (99999) instead of 1 as before.

		\item Accessing TYPO3 functionality from the command line has been simplified. Single commands no longer
			require single users in the database, instead all cli command use the username \texttt{\_cli\_}.
			This user is created on demand by the framework if it does not exist at the first command line call.
			The \texttt{\_cli\_} user has admin rights and no longer needs specific access rights assigned to
			perform specific tasks like manipulating database content using the \texttt{DataHandler}.

	\end{itemize}

\end{frame}

% ------------------------------------------------------------------------------

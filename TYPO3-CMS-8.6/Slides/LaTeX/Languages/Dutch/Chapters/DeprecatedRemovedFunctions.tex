% ------------------------------------------------------------------------------
% TYPO3 CMS 8.6 - What's New - Chapter "Deprecated Functions" (Dutch Version)
%
% @author	Michael Schams <schams.net>
% @license	Creative Commons BY-NC-SA 3.0
% @link		http://typo3.org/download/release-notes/whats-new/
% @language	English
% ------------------------------------------------------------------------------
% LTXE-CHAPTER-UID:		3f842373-9262b8d3-f9c8de76-cf29ce17
% LTXE-CHAPTER-NAME:	Deprecated Functions
% ------------------------------------------------------------------------------

\section{Verouderde/verwijderde functies}
\begin{frame}[fragile]
	\frametitle{Verouderde/verwijderde functies}

	\begin{center}\huge{Hoofdstuk 5:}\end{center}
	\begin{center}\huge{\color{typo3darkgrey}\textbf{Verouderde/verwijderde functies}}\end{center}

\end{frame}

% ------------------------------------------------------------------------------
% LTXE-SLIDE-START
% LTXE-SLIDE-UID:		6082015f-7f3edf46-fc14d53e-3114f100
% LTXE-SLIDE-ORIGIN:	f6dccaab-1b20a7cb-3b708187-74c96db9 English
% LTXE-SLIDE-TITLE:		!Breaking: #78988 - Remove optional Fluid TypoScript template
% ------------------------------------------------------------------------------

\begin{frame}[fragile]
	\frametitle{Verouderde/verwijderde functies}
	\framesubtitle{Optionele Fluid TypoScript-sjabloon verwijderd}

	% decrease font size for code listing
	\lstset{basicstyle=\tiny\ttfamily}

	\begin{itemize}
		\item Het bestand voor statische opname "Fluid: (Optioneel) standaard ajax configuratie (fluid)" is
			bedoeld als voorbeeld van Fluid Widgets in FE. Het is verouderd en daardoor verwijderd.
		\item Neem in plaats hiervan het benodigde bestand handmatig op in het TypoScript-sjabloon:

			\begin{lstlisting}
				page.includeJSLibs {
				  jquery = https://code.jquery.com/jquery-3.1.1.slim.min.js
				  jquery.external = 1
				  jquery.integrity = sha256-/SIrNqv8h6QGKDuNoLGA4iret+kyesCkHGzVUUV0shc=
				  jqueryUi = https://code.jquery.com/ui/1.12.1/jquery-ui.min.js
				  jqueryUi.external = 1
				  jqueryUi.integrity = sha256-VazP97ZCwtekAsvgPBSUwPFKdrwD3unUfSGVYrahUqU=
				}

				page.includeCSSLibs {
				  jqueryUI = https://code.jquery.com/ui/1.12.1/themes/smoothness/jquery-ui.css
				  jqueryUi.external = 1
				}
			\end{lstlisting}

	\end{itemize}

\end{frame}

% ------------------------------------------------------------------------------
% LTXE-SLIDE-START
% LTXE-SLIDE-UID:		6b072cc1-b9dd3f9d-289fcee2-cec72962
% LTXE-SLIDE-ORIGIN:	2f81893a-60617439-c59f6e53-1c79a41f English
% LTXE-SLIDE-TITLE:		!Breaking: #79109 - Lowlevel VersionsCommand parameters changed
% ------------------------------------------------------------------------------

\begin{frame}[fragile]
	\frametitle{Verouderde/verwijderde functies}
	\framesubtitle{Basale VersionsCommand parameters gewijzigd (1/2)}

	\begin{itemize}
		\item Het bestaande CLI commando in \texttt{EXT:lowlevel} voor het tonen en opschonen van versies (van
			\texttt{EXT:version} / \texttt{EXT:workspaces}) is gemigreerd naar een Symfony Console commando.

		\item Het commando was beschikbaar via \texttt{./typo3/cli\_dispatch.phpsh lowlevel\_cleaner versions}
			en nu via \texttt{./typo3/sysext/core/bin/typo3 cleanup:versions}. De volgende CLI opties zijn
			beschikbaar:

			\begin{itemize}
				\item \texttt{-v} en \texttt{-vv} tonen detailinformatie voor de betrokken records
				\item \texttt{-}\texttt{-pid=23} of \texttt{-p=23} om alleen versies van pagina met ID 23 (anders wordt "0" gebruikt)
			\end{itemize}

	\end{itemize}

	\small\textit{Verder op volgende pagina}\normalsize

\end{frame}

% ------------------------------------------------------------------------------
% LTXE-SLIDE-START
% LTXE-SLIDE-UID:		1f2e4522-668b3401-2eccb7b8-f10f5609
% LTXE-SLIDE-ORIGIN:	2104e84b-c80621c1-0774f43b-33d86b6b English
% LTXE-SLIDE-TITLE:		!Breaking: #79109 - Lowlevel VersionsCommand parameters changed
% ------------------------------------------------------------------------------

\begin{frame}[fragile]
	\frametitle{Verouderde/verwijderde functies}
	\framesubtitle{Basale VersionsCommand parameters gewijzigd (2/2)}

	\small\textit{Vervolg}\normalsize

	\begin{itemize}
		\item ...
			\begin{itemize}
				\item \texttt{-}\texttt{-depth=4} of \texttt{-d=4} om recursief op te schonen tot een bepaalde diepte
				\item \texttt{-}\texttt{-dry-run} om alleen de records die gewijzigd/verwijderd zouden worden te tonen
				\item \texttt{-}\texttt{-action=naamvanactie} om records met versies op te schonen,
					de mogelijke acties:
					\begin{itemize}
						\item \texttt{versions\_in\_live}: Versies van records in de live werkruimte verwijderen
						\item \texttt{published\_versions}: Versies van gepubliceerde records verwijderen
						\item \texttt{invalid\_workspace}: Records in een niet-bestaande werkruimte verplaatsen naar live
						\item \texttt{unused\_placeholders}: Plaatshouders die niet meer gebruikt worden verwijderen
					\end{itemize}
			\end{itemize}
	\end{itemize}

\end{frame}

% ------------------------------------------------------------------------------
% LTXE-SLIDE-START
% LTXE-SLIDE-UID:		2868ce84-4a4d897d-a9f865b8-507b17d1
% LTXE-SLIDE-ORIGIN:	35714a8f-ad99f4e7-ccda0780-c0aa44aa English
% LTXE-SLIDE-TITLE:		Default Layouts for Fluid Styled Content Changed
% LTXE-SLIDE-REFERENCE:	!Breaking: #79622 - Default layouts for Fluid Styled Content changed
% ------------------------------------------------------------------------------

\begin{frame}[fragile]
	\frametitle{Verouderde/verwijderde functies}
	\framesubtitle{Wijziging standaard layouts voor Fluid Styled Content}

	% decrease font size for code listing
	\lstset{basicstyle=\tiny\ttfamily}

	\begin{itemize}
		\item De layouts voor inhoudselementen voor Fluid Styled Content zijn nu flexibeler en kunnen
			eenvoudiger onderhouden worden.

		\item De layouts \texttt{ContentFooter}, \texttt{HeaderFooter} en \texttt{HeaderContentFooter}
			zijn verwijderd en vervangen door een enkele \texttt{Default} layout die flexibeler is.

			\begin{lstlisting}
				$GLOBALS['TCA']['tt_content']['columns']['CType']['config']['default'] = 'textmedia';
				$GLOBALS['TCA']['tt_content']['columns']['CType']['config']['default'] = 'header';
			\end{lstlisting}

	\end{itemize}

\end{frame}

% ------------------------------------------------------------------------------
% LTXE-SLIDE-START
% LTXE-SLIDE-UID:		9793cba2-6d39198e-bae77e7e-33527309
% LTXE-SLIDE-ORIGIN:	0e36fc45-799c2e27-3f058488-8173f554 English
% LTXE-SLIDE-TITLE:		TypoScript Standard Header (1/2)
% LTXE-SLIDE-REFERENCE:	!Breaking: #79622 - TypoScript Standard Header has been removed from Fluid Styled Content
% ------------------------------------------------------------------------------

\begin{frame}[fragile]
	\frametitle{Verouderde/verwijderde functies}
	\framesubtitle{TypoScript standaard header (1/2)}

	\begin{itemize}
		\item De renderdefinitie van de TypoScript standaard header \texttt{lib.stdHeader} is aangebracht in
			CSS Styled Content voor gebruik in diverse inhoudselementen om onderhoud te versimpelen.

		\item Voor Fluid Styled Content is een workaround gemaakt voor compatibiliteit met CMS 7 om migratie
		 	te versimpelen. Echter, alleen de kop wordt gerenderd en alle frames ontbreken. Additionele opties
		 	zijn nodig om gestroomlijnde uitvoer te genereren als de layout van het inhoudselement niet correct
		 	is geïmplementeerd.

	\end{itemize}

\end{frame}

% ------------------------------------------------------------------------------
% LTXE-SLIDE-START
% LTXE-SLIDE-UID:		514a5251-19ffe794-818c2c70-fa2d35f3
% LTXE-SLIDE-ORIGIN:	ca67c35f-cf7489b1-dd2ae6a2-f3557c53 English
% LTXE-SLIDE-TITLE:		TypoScript Standard Header (2/2)
% LTXE-SLIDE-REFERENCE:	!Breaking: #79622 - TypoScript Standard Header has been removed from Fluid Styled Content
% ------------------------------------------------------------------------------

\begin{frame}[fragile]
	\frametitle{Verouderde/verwijderde functies}
	\framesubtitle{TypoScript standaard header (2/2)}

	% decrease font size for code listing
	\lstset{basicstyle=\tiny\ttfamily}

	\begin{itemize}

		\item huidige uitvoer:

			\begin{lstlisting}
				tt_content.simple_content = COA
				tt_content.simple_content {
				  10 < lib.stdHeader
				  20 = TEXT
				  20.field = bodytext
				}

				<header>
				  <h1>Nunc vel libero dignissim</h1>
				</header>
				<p>
				  ...
				</p>
			\end{lstlisting}

	\end{itemize}

\end{frame}

% ------------------------------------------------------------------------------
% LTXE-SLIDE-START
% LTXE-SLIDE-UID:		904cd5ae-4f7e9bd7-626770ac-1adae712
% LTXE-SLIDE-ORIGIN:	64df6081-eb1f0a13-7f8292d3-9c69a293 English
% LTXE-SLIDE-TITLE:		Miscellaneous (1/4)
% ------------------------------------------------------------------------------

\begin{frame}[fragile]
	\frametitle{Verouderde/verwijderde functies}
	\framesubtitle{Divers (1/4)}

	% !Breaking: #78477 - Remove method FlashMessage->getMessageAsMarkup()
	% !Breaking: #79100 - ext:felogin: Remove default CSS
	% !Breaking: #79201 - EXT:form: Split TypoScript Includes
	% !Breaking: #79242 - Remove l10n_mode noCopy
	% !Breaking: #79243 - Remove l10n_mode mergeIfNotBlank
	% !Breaking: #79243 - Remove sys_language_softMergeIfNotBlank

	\begin{itemize}
		\item De volgende functie is verwijderd:\newline
			\small\texttt{FlashMessage->getMessageAsMarkup()}\normalsize
		\item \texttt{EXT:felogin} voegt niet meer standaard CSS toe omdat het de
			frontend kapot kan maken bij gebruik van CSS-frameworks.
		\item De TypoScript van \texttt{EXT:form} specifiek voor de frontend wordt niet meer
		 	automatisch geladen en moet handmatig toegevoegd worden met statische includes.
			Hiermee kan een TYPO3 integrator eenvoudiger besluiten waar de TypoScript wordt ingeladen.
		\item De optie \texttt{noCopy} is verwijderd zonder vervanging uit de lijst mogelijke waarden
			voor de TCA-kolomeigenschap \texttt{l10n\_mode}.
		\item De optie \texttt{mergeIfNotBlank} is verwijderd zonder vervanging uit de lijst mogelijke
			waarden voor de TCA-kolomeigenschap \texttt{l10n\_mode}.

	\end{itemize}

\end{frame}


% ------------------------------------------------------------------------------
% LTXE-SLIDE-START
% LTXE-SLIDE-UID:		4ef2470d-49ba3ba4-238d2c9a-7ddba019
% LTXE-SLIDE-ORIGIN:	9b7c4a92-17f663e5-b890e0b9-1b039f82 English
% LTXE-SLIDE-TITLE:		Miscellaneous (2/4)
% ------------------------------------------------------------------------------

\begin{frame}[fragile]
	\frametitle{Verouderde/verwijderde functies}
	\framesubtitle{Divers (2/4)}

	% !Breaking: #79302 - Moved pages.url_scheme to compatibility7 extension
	% !Breaking: #79364 - Move page module function `QuickEdit` to compatibility7
	% !Breaking: #79622 - CSS Styled Content and TypoScript

	\begin{itemize}
		\item De TypoScript optie \texttt{config.sys\_language\_softMergeIfNotBlank}
			is verwijderd zonder vervanging. Dit is het gevolg van het verwijderen van de
			TCA optie \texttt{mergeIfNotBlank} uit de lijst mogelijke waarden voor \texttt{l10n\_mode}.

		\item De functionaliteit van het databaseveld \texttt{pages.url\_scheme} is verplaatst
			naar de extensie compatibility7. Met het veld kan het protocol HTTP of HTTPS door
			een redacteur geforceerd worden in de paginaeigenschappen van een specifiek pagina.
			Echter, het is tegenwoordig gewoon om (als een SSL-certificaat beschikbaar is) HTTPS te
			gebruiken voor een hele website of eventueel voor een specifiek onderdeel (inclusief
			onderliggende pagina's) dit te forceren.

	\end{itemize}

\end{frame}

% ------------------------------------------------------------------------------
% LTXE-SLIDE-START
% LTXE-SLIDE-UID:		217722b1-70cb9bba-4fa411bc-bcc0a373
% LTXE-SLIDE-ORIGIN:	6b0c8695-0ad6167d-2eda453e-5b16dd49 English
% LTXE-SLIDE-TITLE:		Miscellaneous (3/4)
% LTXE-SLIDE-REFERENCE:	!Deprecation: #77934 - Deprecate tt_content field select_key
% LTXE-SLIDE-REFERENCE:	!Deprecation: #78477 - Refactoring of FlashMessage rendering
% LTXE-SLIDE-REFERENCE:	!Deprecation: #79316 - Deprecate ArrayUtility::inArray()
% LTXE-SLIDE-REFERENCE:	!Deprecation: #79622 - Deprecation of CSS Styled Content
% ------------------------------------------------------------------------------

\begin{frame}[fragile]
	\frametitle{Verouderde/verwijderde functies}
	\framesubtitle{Divers (3/4)}

	\begin{itemize}
		\item De functie \texttt{QuickEdit} in de module pagina is verplaatst naar
			\texttt{EXT:compatibility7} en zal niet verder ontwikkeld worden.\newline
			Zie \href{https://typo3.org/extensions/repository}{TYPO3 Extension Repository (TER)}.

		\item Om CSS Styled Content en Fluid Styled Content te stroomlijnen zijn diverse opties
		 	van CSS Styled Content verwijderd zonder vervanging:
			\texttt{TCA image\_compression}, \texttt{TCA image\_effects}, \texttt{TCA image\_noRows},
			\texttt{TypoScript IMAGE noRows}, \texttt{TypoScript IMAGE noCols},
			\texttt{TypoScript IMAGE noRowsStdWrap}, \texttt{TypoScript IMGTEXT captionAlign}

		\item Het veld \texttt{select\_key} van de tabel \texttt{tt\_content} wordt niet gebruikt
			door de core en is verwijderd.

	\end{itemize}

\end{frame}

% ------------------------------------------------------------------------------
% LTXE-SLIDE-START
% LTXE-SLIDE-UID:		1d7d2d58-c3dc09ca-1cf73b87-eb1746ca
% LTXE-SLIDE-ORIGIN:	3fe05cb6-a2648210-5486e0fa-93f0f02a English
% LTXE-SLIDE-TITLE:		Miscellaneous (4/4)
% LTXE-SLIDE-REFERENCE:	!Deprecation: #77934 - Deprecate tt_content field select_key
% LTXE-SLIDE-REFERENCE:	!Deprecation: #78477 - Refactoring of FlashMessage rendering
% LTXE-SLIDE-REFERENCE:	!Deprecation: #79316 - Deprecate ArrayUtility::inArray()
% LTXE-SLIDE-REFERENCE:	!Deprecation: #79622 - Deprecation of CSS Styled Content
% ------------------------------------------------------------------------------

\begin{frame}[fragile]
	\frametitle{Verouderde/verwijderde functies}
	\framesubtitle{Divers (4/4)}

	\begin{itemize}

		\item De volgende methodes en eigenschappen binnen \texttt{FlashMessage::class}
			zijn aangemerkt als verouderd:

			\begin{itemize}
				\item \texttt{FlashMessage->classes}
				\item \texttt{FlashMessage->icons}
				\item \texttt{FlashMessage->getClass()}
				\item \texttt{FlashMessage->getIconName()}
			\end{itemize}

		\item Methode \texttt{ArrayUtility::inArray()} is aangemerkt als verouderd

		\item CSS Styled Content is nu verouderd\newline
			\small(zal verwijderd worden in TYPO3 CMS versie 9)\normalsize

	\end{itemize}

\end{frame}

% ------------------------------------------------------------------------------

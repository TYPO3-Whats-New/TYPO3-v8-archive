% ------------------------------------------------------------------------------
% TYPO3 CMS 8.6 - What's New - Chapter "In-Depth Changes" (Dutch Version)
%
% @author	Michael Schams <schams.net>
% @license	Creative Commons BY-NC-SA 3.0
% @link		http://typo3.org/download/release-notes/whats-new/
% @language	English
% ------------------------------------------------------------------------------
% LTXE-CHAPTER-UID:		5ebcecbe-66abfa57-cf38bc00-aa637965
% LTXE-CHAPTER-NAME:	In-Depth Changes
% ------------------------------------------------------------------------------

\section{Systeemwijzigingen}
\begin{frame}[fragile]
	\frametitle{Systeemwijzigingen}

	\begin{center}\huge{Hoofdstuk 3:}\end{center}
	\begin{center}\huge{\color{typo3darkgrey}\textbf{Systeemwijzigingen}}\end{center}

\end{frame}

% ------------------------------------------------------------------------------
% LTXE-SLIDE-START
% LTXE-SLIDE-UID:		25ed5c73-1f41669e-2c92c353-60d57f60
% LTXE-SLIDE-ORIGIN:	5ae3ff92-e2f03af9-8f5caf79-b7e0aca1 English
% LTXE-SLIDE-TITLE:		Page Browser for scheduler tasks
% LTXE-SLIDE-REFERENCE:	!Feature: #12211 - Usability: Scheduler provide page browser to choose start page
% ------------------------------------------------------------------------------

\begin{frame}[fragile]
	\frametitle{Systeemwijzigingen}
	\framesubtitle{Paginaselector bij geplande taken}

	\begin{itemize}
		\item Taken in de taakplanner die een pagina \texttt{uid} nodig hebben kunnen een knop toevoegen
			voor een paginaselectiepopup.

		\item In de \texttt{ValidatorTaskAdditionalFieldProvider} zijn twee velden toegevoegd.

		\item Als het extra veld \texttt{browser} de waarde \texttt{page} heeft dan voegt de
			\texttt{SchedulerModuleController} een knop toe voor de paginaselectie bij het veld.

			\begin{lstlisting}
				'browser' => 'page',
			\end{lstlisting}

		\item De \texttt{pageTitle} bevat de titel van de pagina die naast de bladerknop getoond wordt.

			\begin{lstlisting}
				'pageTitle' => $pageTitle,
			\end{lstlisting}

	\end{itemize}

\end{frame}

% ------------------------------------------------------------------------------
% LTXE-SLIDE-START
% LTXE-SLIDE-UID:		e763052d-941b7def-24e74719-ee00d467
% LTXE-SLIDE-ORIGIN:	02a7fe32-ae3fd901-b04372a2-7d3726f4 English
% LTXE-SLIDE-TITLE:		Synchronized field values in localized records (1/2)
% LTXE-SLIDE-REFERENCE:	!Feature: #51291 - Synchronized field values in localized records
% ------------------------------------------------------------------------------

\begin{frame}[fragile]
	\frametitle{Systeemwijzigingen}
	\framesubtitle{Velden in vertaalde records synchroniseren (1/2)}

	% decrease font size for code listing
	\lstset{basicstyle=\tiny\ttfamily}

	\begin{itemize}
		\item Het gedrag van de vertalingsoverlay is gewijzigd om vertalingsrecords zelfstandig te maken.

		\item Voorheen, als velden in \texttt{TCA} velden ingesteld waren op \texttt{l10n\_mode} \texttt{exclude}
		 	of \texttt{mergeIfNotBlank}, bevatte het vertaalde record geen waarden en werden die waarden opgehaald
		 	uit het onderliggende record voor de standaard taal.

	\end{itemize}

\end{frame}

% ------------------------------------------------------------------------------
% LTXE-SLIDE-START
% LTXE-SLIDE-UID:		220961f5-f0c5f736-772ae4c3-4e0fcfa8
% LTXE-SLIDE-ORIGIN:	cf030e7c-03fb0ac4-9b9ff848-ac69581b English
% LTXE-SLIDE-TITLE:		Synchronized field values in localized records (2/2)
% LTXE-SLIDE-REFERENCE:	!Feature: #51291 - Synchronized field values in localized records
% ------------------------------------------------------------------------------

\begin{frame}[fragile]
	\frametitle{Systeemwijzigingen}
	\framesubtitle{Velden in vertaalde records synchroniseren (2/2)}

	% decrease font size for code listing
	\lstset{basicstyle=\tiny\ttfamily}

	\begin{itemize}
		\item Dit is gewijzigd. De \texttt{DataHandler} kopieert nu de waardes naar het vertaalde record
		 	en synchroniseert ze als het record in de standaard taal gewijzigd wordt.

			\begin{lstlisting}
				'columns' => [
				  ...
				  'header' => [
				    'label' => 'My header',
				    'config' => [
				      'type' => 'input',
				      'behaviour' => [
				        'allowLanguageSynchronization' => true,
				      ],
				    ],
				  ],
				],
			\end{lstlisting}

	\end{itemize}

\end{frame}

% ------------------------------------------------------------------------------
% LTXE-SLIDE-START
% LTXE-SLIDE-UID:		362baa50-0e308936-e522e688-87cb70b6
% LTXE-SLIDE-ORIGIN:	18c4f826-440f3e00-a854e8e8-ddc7cbc0 English
% LTXE-SLIDE-TITLE:		Image Manipulation Tool (1/6)
% LTXE-SLIDE-REFERENCE:	!Feature: #75880 - Implement multiple cropping variants in image manipulation tool
% ------------------------------------------------------------------------------

\begin{frame}[fragile]
	\frametitle{Systeemwijzigingen}
	\framesubtitle{Afbeeldingsmanipulatiehulp (1/6)}

	% decrease font size for code listing
	\lstset{basicstyle=\tiny\ttfamily}

	\begin{itemize}
		\item Het \texttt{imageManipulation} TCA-type kan nu meerdere cropvarianten aan als dit geconfigureerd is.

		\item Ook kan een initieel cropgebied ingesteld worden. Als geen initieel cropgebied is ingesteld,
		 	is standaard de hele afbeelding geselecteerd.
		\item Cropgebieden worden relatief ingesteld met drijvende-kommagetallen. De coördinaten en
		 	groottes moeten daarvoor gespecificeerd worden.

	\end{itemize}

\end{frame}

% ------------------------------------------------------------------------------
% LTXE-SLIDE-START
% LTXE-SLIDE-UID:		9bc1a82a-efbe5824-5f4832a3-2fdabcce
% LTXE-SLIDE-ORIGIN:	36ea74c0-3b92dbf6-e35c9e25-af0a0abf English
% LTXE-SLIDE-TITLE:		Image Manipulation Tool (2/6)
% LTXE-SLIDE-REFERENCE:	!Feature: #75880 - Implement multiple cropping variants in image manipulation tool
% ------------------------------------------------------------------------------

\begin{frame}[fragile]
	\frametitle{Systeemwijzigingen}
	\framesubtitle{Afbeeldingsmanipulatiehulp (2/6)}

	% decrease font size for code listing
	\lstset{basicstyle=\tiny\ttfamily}

	\begin{itemize}

		\item Dit voorbeeld stelt twee cropvarianten in, een met
		 	het ID "mobiel", de andere met het ID "desktop".
		 	De arraysleutel bepaalt het ID dat wordt gebruikt bij
		 	het maken van de afbeelding met de afbeeldingsviewhelper.

			\begin{lstlisting}
				'config' => [
				  'type' => 'imageManipulation',
				  'cropVariants' => [
				    'mobiel' => [
				      'title' => 'Mobiel',
				      'allowedAspectRatios' => [
				        '4:3' => [
				          'title' => '4:3',
				          'value' => 4 / 3
				        ],
				        ...
				      ],
				    ],
				    'desktop' => [
				      ...
				    ],
				  ],
				]
			\end{lstlisting}

	\end{itemize}

\end{frame}

% ------------------------------------------------------------------------------
% LTXE-SLIDE-START
% LTXE-SLIDE-UID:		7af4a43b-22ceb13f-452f830c-da92cfc6
% LTXE-SLIDE-ORIGIN:	27b20030-548f80c4-3540aabb-41a83498 English
% LTXE-SLIDE-TITLE:		Image Manipulation Tool (3/6)
% LTXE-SLIDE-REFERENCE:	!Feature: #75880 - Implement multiple cropping variants in image manipulation tool
% ------------------------------------------------------------------------------

\begin{frame}[fragile]
	\frametitle{Systeemwijzigingen}
	\framesubtitle{Afbeeldingsmanipulatiehulp (3/6)}

	% decrease font size for code listing
	\lstset{basicstyle=\tiny\ttfamily}

	\begin{itemize}

		\item Onderstaand voorbeeld heeft een intieel cropgebied ter
		 	grootte van de standaard van de vorige crophulp.

			\begin{lstlisting}
				'config' => [
				  'type' => 'imageManipulation',
				  'cropVariants' => [
				    'mobiel' => [
				      'title' => 'LLL:EXT:ext_key/Resources/Private/Language/locallang.xlf:imageManipulation.mobile',
				      'cropArea' => [
				        'x' => 0.1,
				        'y' => 0.1,
				        'width' => 0.8,
				        'height' => 0.8,
				      ],
				    ],
				  ],
				]
			\end{lstlisting}

	\end{itemize}

\end{frame}

% ------------------------------------------------------------------------------
% LTXE-SLIDE-START
% LTXE-SLIDE-UID:		f2115dff-1c6dfffb-c78f3420-3d6ae448
% LTXE-SLIDE-ORIGIN:	58cebeda-4ee109f4-5e9eb728-b7703b40 English
% LTXE-SLIDE-TITLE:		Image Manipulation Tool (4/6)
% LTXE-SLIDE-REFERENCE:	!Feature: #75880 - Implement multiple cropping variants in image manipulation tool
% ------------------------------------------------------------------------------

\begin{frame}[fragile]
	\frametitle{Systeemwijzigingen}
	\framesubtitle{Afbeeldingsmanipulatiehulp (4/6)}

	% decrease font size for code listing
	\lstset{basicstyle=\tiny\ttfamily}

	\begin{itemize}
		\item Gebruikers kunnen ook een focusgebied selecteren, indien geconfigureerd.
		\item Het focusgebied valt altijd binnen het cropgebied en bepaalt het gebied dat
		 	altijd zichtbaar moet blijven zodat de afbeelding z'n mening behoud.

			\begin{lstlisting}
				'config' => [
				  'type' => 'imageManipulation',
				  'cropVariants' => [
				    'mobiel' => [
				      'title' =>
				        'LLL:EXT:ext_key/Resources/Private/Language/locallang.xlf:imageManipulation.mobile',
				      'focusArea' => [
				        'x' => 1 / 3,
				        'y' => 1 / 3,
				        'width' => 1 / 3,
				        'height' => 1 / 3,
				      ],
				    ],
				  ],
				]
			\end{lstlisting}

	\end{itemize}

\end{frame}

% ------------------------------------------------------------------------------
% LTXE-SLIDE-START
% LTXE-SLIDE-UID:		bb105611-574eaeca-216aa7ec-e83c5bf1
% LTXE-SLIDE-ORIGIN:	94d809e0-3369bf18-616710ee-93f6a9da English
% LTXE-SLIDE-TITLE:		Image Manipulation Tool (5/6)
% LTXE-SLIDE-REFERENCE:	!Feature: #75880 - Implement multiple cropping variants in image manipulation tool
% ------------------------------------------------------------------------------

\begin{frame}[fragile]
	\frametitle{Systeemwijzigingen}
	\framesubtitle{Afbeeldingsmanipulatiehulp (5/6)}

	% decrease font size for code listing
	\lstset{basicstyle=\tiny\ttfamily}

	\begin{itemize}
		\item Vaak worden afbeeldingen gebruikt in een context waar andere elementen over de afbeelding vallen.
		\item Om redacteuren een hint te geven over dit soort gebieden bij het kiezen van een cropgebied kan men
		 	diverse bedekte gebieden vastlegeen.
		\item Deze gebieden worden binnen het cropgebied getoond. Het focusgebied kan niet samenvallen met een van
		 	de bedekte gebieden.

			\begin{lstlisting}
				'config' => [
				  'type' => 'imageManipulation',
				  'coverAreas' => [
				    [
				      'x' => 0.05, 'y' => 0.85,
				      'width' => 0.9, 'height' => 0.1,
				    ],
				  ],
				]
			\end{lstlisting}

	\end{itemize}

\end{frame}

% ------------------------------------------------------------------------------
% LTXE-SLIDE-START
% LTXE-SLIDE-UID:		1a78012c-7dc19b83-0693dde2-740f1359
% LTXE-SLIDE-ORIGIN:	c9df3305-35544e3c-6a073be7-1cec40f4 English
% LTXE-SLIDE-TITLE:		Image Manipulation Tool (6/6)
% LTXE-SLIDE-REFERENCE:	!Feature: #75880 - Implement multiple cropping variants in image manipulation tool
% ------------------------------------------------------------------------------

\begin{frame}[fragile]
	\frametitle{Systeemwijzigingen}
	\framesubtitle{Afbeeldingsmanipulatiehulp (6/6)}

	% decrease font size for code listing
	\lstset{basicstyle=\smaller\ttfamily}

	\begin{itemize}
		\item Om cropvarianten weer te geven kunnen de varianten als argument aan de afbeeldings-viewhelper
		 	meegegeven worden:

			\begin{lstlisting}
				<f:image image="{data.image}" cropVariant="mobiel" width="800" >
				</f:image>
			\end{lstlisting}

	\end{itemize}

\end{frame}

% ------------------------------------------------------------------------------
% LTXE-SLIDE-START
% LTXE-SLIDE-UID:		c3402e6f-50716ce7-b0a9ec3a-8843c88d
% LTXE-SLIDE-ORIGIN:	a71d77d1-6a419c8d-0fbb35d0-e1ce57e4 English
% LTXE-SLIDE-TITLE:		Default Content Element Changed for Fluid Styled Content
% LTXE-SLIDE-REFERENCE:	!Breaking: #79622 - Default content element changed for Fluid Styled Content
% ------------------------------------------------------------------------------

\begin{frame}[fragile]
	\frametitle{Systeemwijzigingen}
	\framesubtitle{Standaard inhoudselement gewijzigd voor Fluid Styled Content}

	% decrease font size for code listing
	\lstset{basicstyle=\tiny\ttfamily}

	\begin{itemize}
		\item Het standaard inhoudselement voor CSS Styled Content is gewijzigd in "Tekst"
		\item Om de configuratie terug te zetten moet het standaard inhoudselement handmatig op de voorkeur ingesteld worden.
		 	Dit kan gedaan worden door het overschrijven van de configuratie in een
		 	\texttt{Configuration/TCA/Overrides/tt\_content.php} bestand.

			\begin{lstlisting}
				$GLOBALS['TCA']['tt_content']['columns']['CType']['config']['default'] = 'textmedia';
				$GLOBALS['TCA']['tt_content']['columns']['CType']['config']['default'] = 'header';
			\end{lstlisting}

	\end{itemize}

\end{frame}

% ------------------------------------------------------------------------------
% LTXE-SLIDE-START
% LTXE-SLIDE-UID:		4af80236-794860c3-0e5893a3-ca99a193
% LTXE-SLIDE-ORIGIN:	035f6a19-9be2f793-bc0ebb3e-c6a4cca4 English
% LTXE-SLIDE-TITLE:		TCA Changes (1/2)
% LTXE-SLIDE-REFERENCE:	!Deprecation: #79440 - TCA Changes
% ------------------------------------------------------------------------------

\begin{frame}[fragile]
	\frametitle{Systeemwijzigingen}
	\framesubtitle{TCA wijzigingen (1/2)}

	\begin{itemize}

		\item De \texttt{TCA} is op veldniveau gewijzigd.

		\item Dit betreft bijna alle kolomtypes.

		\item In het kort is het onderdeel \texttt{wizards} verdwenen en vervangen door een combinatie
		 	van nieuwe \texttt{renderType} en een nieuwe set configuratie-opties.

		\item Assistenten (wizards) zijn onderverdeeld in drie soorten: \texttt{fieldInformation},
		 	\texttt{fieldControl} en \texttt{fieldWizard}.

	\end{itemize}

\end{frame}
% ------------------------------------------------------------------------------
% LTXE-SLIDE-START
% LTXE-SLIDE-UID:		c54c0ce6-3866d42b-54944787-8cb2cecd
% LTXE-SLIDE-ORIGIN:	55c8113e-b234daa0-e701f035-8e1b58d1 English
% LTXE-SLIDE-TITLE:		TCA Changes (2/2)
% LTXE-SLIDE-REFERENCE:	!Deprecation: #79440 - TCA Changes
% ------------------------------------------------------------------------------

\begin{frame}[fragile]
	\frametitle{Systeemwijzigingen}
	\framesubtitle{TCA wijzigingen (2/2)}

	% decrease font size for code listing
	\lstset{basicstyle=\tiny\ttfamily}

	\begin{itemize}
		\item Voorbeeld:

			\begin{lstlisting}
				'fieldControl' => [
				  'editPopup' => [
				    'disabled' => false,
				  ],
				  'addRecord' => [
				    'disabled' => false,
				    'options' => [
				      'setValue' => 'prepend',
				    ],
				  ],
				  'listModule' => [
				    'disabled' => false,
				  ],
				],
			\end{lstlisting}

		\item Meer details:
			\href{https://docs.typo3.org/typo3cms/extensions/core/8-dev/singlehtml/Index.html#deprecation-79440-formengine-element-expansion}{docs.typo3.org}

	\end{itemize}

\end{frame}


% ------------------------------------------------------------------------------
% LTXE-SLIDE-START
% LTXE-SLIDE-UID:		832c09ff-e22f2367-1f488aa3-800744df
% LTXE-SLIDE-ORIGIN:	da922107-1604619b-66f713fe-9508d9d7 English
% LTXE-SLIDE-TITLE:		Introduce Session Storage Framework
% LTXE-SLIDE-REFERENCE:	!Feature: #70316 - Introduce Session Storage Framework
% ------------------------------------------------------------------------------

\begin{frame}[fragile]
	\frametitle{Systeemwijzigingen}
	\framesubtitle{Nieuw sessieopslagraamwerk}

	\begin{itemize}
		\item Er is een nieuw opslagraamwerk voor sessies
		\item Het doel is samenwerking tussen diverse sessieopslagen ("backends" genoemd)
			zoals database, bestandsopslag, Redis, etc.
		\item De volgende sessie-backends zijn standaard beschikbaar:

			\begin{itemize}
				\item
					\smaller\texttt{\textbackslash
						TYPO3\textbackslash
						CMS\textbackslash
						Core\textbackslash
						Session\textbackslash
						Backend\textbackslash
						DatabaseSessionBackend}

				\item
					\texttt{\textbackslash
						TYPO3\textbackslash
						CMS\textbackslash
						Core\textbackslash
						Session\textbackslash
						Backend\textbackslash
						RedisSessionBackend}
			\end{itemize}
	\end{itemize}

\end{frame}

% ------------------------------------------------------------------------------
% LTXE-SLIDE-START
% LTXE-SLIDE-UID:		093bd0fc-db9c4ea9-46425aa8-8625fef6
% LTXE-SLIDE-ORIGIN:	2d65beca-277a2f35-a1bc5ae7-c71712aa English
% LTXE-SLIDE-TITLE:		CLI Support for T3D Imports
% LTXE-SLIDE-REFERENCE:	!Feature: #72749 - CLI support for T3D import
% ------------------------------------------------------------------------------

\begin{frame}[fragile]
	\frametitle{Systeemwijzigingen}
	\framesubtitle{CLI ondersteuning voor T3D import}

	\begin{itemize}
		\item \texttt{EXT:impexp} kan nu bestanden (T3D of XML) importeren via de opdrachtregel
			met een Symfony commando.

		\item Gebruik:\newline
			\smaller
				\texttt{./typo3/sysext/core/bin/typo3 impexp:import [<options>] <file> <pageId>}
			\normalsize

		\item Opties:
			\begin{itemize}
				\item \texttt{-}\texttt{-updateRecords}: Bestaande records geforceerd bijwerken
				\item \texttt{-}\texttt{-ignorePid}: Pagina ID's van bijgewerkte records niet corrigeren
				\item \texttt{-}\texttt{-enableLog}: alle databasebewerkingen loggen
			\end{itemize}

	\end{itemize}

\end{frame}

% ------------------------------------------------------------------------------
% LTXE-SLIDE-START
% LTXE-SLIDE-UID:		fcb3f6cd-a5660c94-8c4d196b-567d73d7
% LTXE-SLIDE-ORIGIN:	38064844-53baa89a-7f73669e-2bb96cab English
% LTXE-SLIDE-TITLE:		Hook in typolink for Modification of Page Params
% LTXE-SLIDE-REFERENCE:	!Feature: #79121 - Implement hook in typolink for modification of page params
% ------------------------------------------------------------------------------

\begin{frame}[fragile]
	\frametitle{Systeemwijzigingen}
	\framesubtitle{Hook in \texttt{typolink} voor wijzigen paginaparameters}

	% decrease font size for code listing
	\lstset{basicstyle=\tiny\ttfamily}

	\begin{itemize}
		\item Er is een nieuwe hook in \texttt{ContentObjectRenderer::typoLink} voor links naar pagina's.
			Hiermee kan de linkconfiguratie aangepast worden, bijv. door het toevoegen van extra parameters
			of metadata uit de pagina.

		\item De hook is te registreren via:

			\begin{lstlisting}
				$GLOBALS['TYPO3_CONF_VARS']['SC_OPTIONS']['typolinkProcessing']
				  ['typolinkModifyParameterForPageLinks'][] = \Your\Namespace\Hooks\MyHook::class;
			\end{lstlisting}

		\item Gebruik:

			\begin{lstlisting}
				public function modifyPageLinkConfiguration(
				  array $linkConfiguration, array $linkDetails, array $pageRow) : array
				{
				  $linkConfiguration['additionalParams'] .= $pageRow['myAdditionalParamsField'];
				  return $linkConfiguration;
				}
			\end{lstlisting}

	\end{itemize}

\end{frame}

% ------------------------------------------------------------------------------
% LTXE-SLIDE-START
% LTXE-SLIDE-UID:		e35c430d-b7ab8f47-7a33c57f-20e9a23c
% LTXE-SLIDE-ORIGIN:	290e1c2e-e708ede1-82aaf127-e69b2338 English
% LTXE-SLIDE-TITLE:		Hook to Add Custom TypoScript Templates (1/2)
% LTXE-SLIDE-REFERENCE:	!Feature: #79140 - Add hook to add custom TypoScript templates
% ------------------------------------------------------------------------------

\begin{frame}[fragile]
	\frametitle{Systeemwijzigingen}
	\framesubtitle{Hook voor maatwerk TypoScript-sjablonen (1/2)}

	% decrease font size for code listing
	\lstset{basicstyle=\tiny\ttfamily}

	\begin{itemize}
		\item Een nieuwe hook in TemplateService kan TypoScript-sjablonen toevoegen of bestaande wijzigen.

		\item De hook is te registreren met de volgende code in een \texttt{ext\_localconf.php} bestand:

			\begin{lstlisting}
				$GLOBALS['TYPO3_CONF_VARS']['SC_OPTIONS']['Core/TypoScript/TemplateService']
				  ['runThroughTemplatesPostProcessing']
			\end{lstlisting}

		\item \texttt{EXT:my\_site/Classes/Hooks/TypoScriptHook.php} (1/2)

			\begin{lstlisting}
				namespace MyVendor\MySite\Hooks;
				class TypoScriptHook
				{
				  /**
				   * Hooks into TemplateService after
				   * @param array $parameters
				   * @param \TYPO3\CMS\Core\TypoScript\TemplateService $parentObject
				   * @return void
				   */
				...
			\end{lstlisting}

	\end{itemize}

\end{frame}

% ------------------------------------------------------------------------------
% LTXE-SLIDE-START
% LTXE-SLIDE-UID:		d1ac77d2-4f82fd3a-7711fe78-e3810d8a
% LTXE-SLIDE-ORIGIN:	d64368f4-0f5c43e3-92a546bf-75d50a61 English
% LTXE-SLIDE-TITLE:		Hook to Add Custom TypoScript Templates (2/2)
% LTXE-SLIDE-REFERENCE:	!Feature: #79140 - Add hook to add custom TypoScript templates
% ------------------------------------------------------------------------------

\begin{frame}[fragile]
	\frametitle{Systeemwijzigingen}
	\framesubtitle{Hook voor maatwerk TypoScript-sjablonen (2/2)}

	% decrease font size for code listing
	\lstset{basicstyle=\tiny\ttfamily}

	\begin{itemize}
		\item \texttt{EXT:my\_site/Classes/Hooks/TypoScriptHook.php} (2/2)

			\begin{lstlisting}
				...
				  public function addCustomTypoScriptTemplate($parameters, $parentObject)
				  {
				    // Disable the inclusion of default TypoScript set via TYPO3_CONF_VARS
				    $parameters['isDefaultTypoScriptAdded'] = true;
				    // Disable the inclusion of ext_typoscript_setup.txt of all extensions
				    $parameters['processExtensionStatics'] = false;

				    // No template was found in rootline so far, so a custom "fake" sys_template record is added
				    if ($parentObject->outermostRootlineIndexWithTemplate === 0) {
				      $row = [
				        'uid' => 'my_site_template',
				        'config' =>
					      '<INCLUDE_TYPOSCRIPT: source="FILE:EXT:my_site/Configuration/TypoScript/site_setup.t3s">',
				        'root' => 1,
				        'pid' => 0
				      ];
				      $parentObject->processTemplate($row, 'sys_' . $row['uid'], 0, 'sys_' . $row['uid']);
				    }
				  }
				}
			\end{lstlisting}

	\end{itemize}

\end{frame}

% ------------------------------------------------------------------------------
% LTXE-SLIDE-START
% LTXE-SLIDE-UID:		946400cc-06e4dc7f-bf61a573-6cb7d093
% LTXE-SLIDE-ORIGIN:	ad9f66e3-bcd548f4-8e14e04c-de29a79c English
% LTXE-SLIDE-TITLE:		Plugin Preview with Fluid
% LTXE-SLIDE-REFERENCE:	!Feature: #79225 - Plugin preview with Fluid
% ------------------------------------------------------------------------------
\begin{frame}[fragile]
	\frametitle{Systeemwijzigingen}
	\framesubtitle{Plug-in-voorvertoning met Fluid}

	% decrease font size for code listing
	\lstset{basicstyle=\tiny\ttfamily}

	\begin{itemize}
		\item De pagina TSconfig voor een voorvertoning van een los inhoudselement in de backend is verbeterd
		 	met het ook kunnen renderen van plug-ins via Fluid

		\item Alle eigenschappen van het \texttt{tt\_content} record zijn direct beschikbaar (bij. UID via \{uid\})

		\item Data van het flexformveld \texttt{pi\_flexform} is beschikbaar via de eigenschap
			\texttt{pi\_flexform\_transformed} als een array.

			\begin{lstlisting}
				mod.web_layout.tt_content.preview.list.simpleblog_bloglisting =
				  EXT:simpleblog/Resources/Private/Templates/Preview/SimpleblogPlugin.html
			\end{lstlisting}

	\end{itemize}

\end{frame}

% ------------------------------------------------------------------------------
% LTXE-SLIDE-START
% LTXE-SLIDE-UID:		539cc20d-2a56ac7d-ca99c5d0-a5d70671
% LTXE-SLIDE-ORIGIN:	245de018-04aed929-a3b3c380-6c822a14 English
% LTXE-SLIDE-TITLE:		Template Paths in BackendTemplateView
% LTXE-SLIDE-REFERENCE:	!Feature: #79124 - Allow overwriting of template paths in BackendTemplateView
% ------------------------------------------------------------------------------

\begin{frame}[fragile]
	\frametitle{Systeemwijzigingen}
	\framesubtitle{Sjabloonpaden in BackendTemplateView}

	% decrease font size for code listing
	\lstset{basicstyle=\tiny\ttfamily}

	\begin{itemize}
		\item BackendTemplateView ondersteunt nu het overschrijven van sjabloonpaden om eigen locaties
			voor sjablonen, partials en layouts toe te voegen voor een backend module gebaseerd op
			BackendTemplateView.

			\begin{lstlisting}
				$frameworkConfiguration =
				  $this->configurationManager->getConfiguration(
				    ConfigurationManagerInterface::CONFIGURATION_TYPE_FRAMEWORK
				  );
				$viewConfiguration = [
				  'view' => [
				    'templateRootPaths' => ['EXT:myext/Resources/Private/Backend/Templates'],
				    'partialRootPaths' => ['EXT:myext/Resources/Private/Backend/Partials'],
				    'layoutRootPaths' => ['EXT:myext/Resources/Private/Backend/Layouts'],
				  ],
				];
				$this->configurationManager->setConfiguration(
				  array_merge($frameworkConfiguration, $viewConfiguration)
				);
			\end{lstlisting}

	\end{itemize}

\end{frame}

% ------------------------------------------------------------------------------
% LTXE-SLIDE-START
% LTXE-SLIDE-UID:		5563ba29-eadeff00-4cbbc6b5-fb1f14b3
% LTXE-SLIDE-ORIGIN:	82514d33-f6c72709-2052880d-4d753fe8 English
% LTXE-SLIDE-TITLE:		Miscellaneous
% LTXE-SLIDE-REFERENCE:	!Feature: #78899 - TCA maxitems optional
% LTXE-SLIDE-REFERENCE:	!Feature: #79240 - Single cli user for cli commands
% ------------------------------------------------------------------------------

\begin{frame}[fragile]
	\frametitle{Systeemwijzigingen}
	\framesubtitle{Divers}

	\begin{itemize}
		\item De \texttt{TCA} optie \texttt{maxitems} voor \texttt{type=select} en \texttt{type=group}
			velden is nu optioneel met standaard een hoge waarde (99999) i.p.v. 1 zoals eerder het geval was.

		\item TYPO3 functionaliteit aanspreken van de opdrachtregel is versimpeld. Bepaalde commando's hebben nu
		 	niet meer eigen gebruikers nodig in de database; alle CLI commando's gebruiken nu de gebruiker \texttt{\_cli\_}.
			Deze gebruiker wordt aangemaakt indien nodig als het niet bestaat bij de eerst aanroep van een commando.
			De \texttt{\_cli\_} gebruiker heeft admin rechten en hoeft niet meer specifieke rechten te krijgen om bepaalde
			taken zoals databasebewerkingen via \texttt{DataHandler} uit te kunnen voeren.

	\end{itemize}

\end{frame}

% ------------------------------------------------------------------------------

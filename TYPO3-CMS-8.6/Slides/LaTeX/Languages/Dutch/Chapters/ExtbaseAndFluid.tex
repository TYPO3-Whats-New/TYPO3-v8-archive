% ------------------------------------------------------------------------------
% TYPO3 CMS 8.6 - What's New - Chapter "Extbase & Fluid" (Dutch Version)
%
% @author	Michael Schams <schams.net>
% @license	Creative Commons BY-NC-SA 3.0
% @link		http://typo3.org/download/release-notes/whats-new/
% @language	English
% ------------------------------------------------------------------------------
% LTXE-CHAPTER-UID:		846d40ce-66fdc2ec-750dcf95-33ce93e0
% LTXE-CHAPTER-NAME:	Extbase & Fluid
% ------------------------------------------------------------------------------

\section{Extbase \& Fluid}
\begin{frame}[fragile]
	\frametitle{Extbase \& Fluid}

	\begin{center}\huge{Hoofdstuk 4:}\end{center}
	\begin{center}\huge{\color{typo3darkgrey}\textbf{Extbase \& Fluid}}\end{center}

\end{frame}

% ------------------------------------------------------------------------------
% LTXE-SLIDE-START
% LTXE-SLIDE-UID:		033ac047-0d609ac4-1a4386c9-ac8beab7
% LTXE-SLIDE-ORIGIN:	a9a5442b-0f7ff21f-9e8812b1-9531a9c1 English
% LTXE-SLIDE-TITLE:		Widget Identifier Extended
% LTXE-SLIDE-REFERENCE:	!Feature: #47006 - Extend the widget identifier with custom string
% ------------------------------------------------------------------------------

\begin{frame}[fragile]
	\frametitle{Extbase \& Fluid}
	\framesubtitle{Widget ID uitgebreid}

	\begin{itemize}
		\item De parameter \texttt{customWidgetId} is ingevoerd voor Fluid widgets.
			Deze tekenreeks wordt gebruikt in de widget ID naast de \texttt{nextWidgetNumber}.

		\item De widget ID wordt gebruikt voor GET parameter-namen.

		\item Een goede waarde voor de \texttt{customWidgetId} is \texttt{{contentObjectData.uid}} zodat geen dubbele waarden voorkomen.

		\item Hiermee kan dezelfde Fluid widget meerdere malen op een pagina in verschillende inhoudselementen gebruikt worden.

			\begin{lstlisting}
				<f:widget.paginate customWidgetId="{contentObjectData.uid}" ...>
				</f:widget.paginate>
			\end{lstlisting}

	\end{itemize}

\end{frame}

% ------------------------------------------------------------------------------
% LTXE-SLIDE-START
% LTXE-SLIDE-UID:		e669be9d-0b21184b-4f929716-18760fd9
% LTXE-SLIDE-ORIGIN:	cbc1b52f-68132620-c5f08dbb-14723e1f English
% LTXE-SLIDE-TITLE:		FlashMessageViewHelper
% LTXE-SLIDE-REFERENCE:	!Breaking: #78477 - FlashMessageViewHelper no longer inherits from TagBasedViewHelper
% ------------------------------------------------------------------------------

\begin{frame}[fragile]
	\frametitle{Extbase \& Fluid}
	\framesubtitle{FlashMessageViewHelper}

	% decrease font size for code listing
	\lstset{basicstyle=\tiny\ttfamily}

	\begin{itemize}
		\item De \texttt{FlashMessageViewHelper} is aangepast en overerft niet langer van de \texttt{TagBasedViewHelper}

		\item Verwijder de tag-specifieke attributen en stijl de standaard uitvoer. Als maatwerk uitvoer nodig is, gebruik
			dan de mogelijkheid om zelf FlashMessages af te beelden. Voorbeeld:

			\begin{lstlisting}
				<f:flashMessages as="flashMessages">
				  <dl class="messages">
				    <f:for each="{flashMessages}" as="flashMessage">
				      <dt>CODE!! {flashMessage.code}</dt>
				      <dd>MESSAGE:: {flashMessage.message}</dd>
				    </f:for>
				  </dl>
				</f:flashMessages>
			\end{lstlisting}

	\end{itemize}

\end{frame}

% ------------------------------------------------------------------------------
% LTXE-SLIDE-START
% LTXE-SLIDE-UID:		7dd2a3be-38b79b01-81e9a2e8-15a1c027
% LTXE-SLIDE-ORIGIN:	91358547-99cab1b1-208f0f24-a5737bee English
% LTXE-SLIDE-TITLE:		Removal of Fluid Styled Content Menu ViewHelpers (1/3)
% LTXE-SLIDE-REFERENCE:	!Breaking: #79622 - Removal of Fluid Styled Content Menu ViewHelpers
% ------------------------------------------------------------------------------

\begin{frame}[fragile]
	\frametitle{Extbase \& Fluid}
	\framesubtitle{Fluid Styled Content Menu ViewHelpers verwijderd (1/3)}

	\begin{itemize}
		\item Data direct ophalen in de 'view' is niet aan te raden en de tijdelijke oplossing
			van menu ViewHelpers is vervangen door de opvolger, de menu-processor
			gebaseerd op HMENU.

		\item Menu ViewHelpers zijn verplaatst naar de \texttt{compatibility7}
			extensie en vervangen in de core menu-inhoudselementen.

	\end{itemize}

\end{frame}

% ------------------------------------------------------------------------------
% LTXE-SLIDE-START
% LTXE-SLIDE-UID:		c9313969-4baa212c-a45d8206-fdd0c308
% LTXE-SLIDE-ORIGIN:	8515a53c-1e60545a-7e99c75e-2638a0b1 English
% LTXE-SLIDE-TITLE:		Removal of Fluid Styled Content Menu ViewHelpers (2/3)
% LTXE-SLIDE-REFERENCE:	!Breaking: #79622 - Removal of Fluid Styled Content Menu ViewHelpers
% ------------------------------------------------------------------------------

\begin{frame}[fragile]
	\frametitle{Extbase \& Fluid}
	\framesubtitle{Fluid Styled Content Menu ViewHelpers verwijderd (2/3)}

	% decrease font size for code listing
	\lstset{basicstyle=\tiny\ttfamily}

	\begin{itemize}

		\item Voor:

			\begin{lstlisting}
				tt_content.menu_subpages.dataProcessing {
				  10 = TYPO3\CMS\Frontend\DataProcessing\SplitProcessor
				  10 {
				    if.isTrue.field = pages
				    fieldName = pages
				    delimiter = ,
				    removeEmptyEntries = 1
				    filterIntegers = 1
				    filterUnique = 1
				    as = pageUids
				  }
				}

				<ce:menu.directory pageUids="{pageUids}" as="pages" levelAs="level">
				  <f:for each="{pages}" as="page">
				    ...
				  </f:for>
				</ce.menu.directory>
			\end{lstlisting}

	\end{itemize}

\end{frame}

% ------------------------------------------------------------------------------
% LTXE-SLIDE-START
% LTXE-SLIDE-UID:		3ad95e89-366f3d7c-6fd85a39-969f1d8b
% LTXE-SLIDE-ORIGIN:	53461585-875b27dd-f00c174b-9b459763 English
% LTXE-SLIDE-TITLE:		Removal of Fluid Styled Content Menu ViewHelpers (3/3)
% LTXE-SLIDE-REFERENCE:	!Breaking: #79622 - Removal of Fluid Styled Content Menu ViewHelpers
% ------------------------------------------------------------------------------

\begin{frame}[fragile]
	\frametitle{Extbase \& Fluid}
	\framesubtitle{Fluid Styled Content Menu ViewHelpers verwijderd (3/3)}

	% decrease font size for code listing
	\lstset{basicstyle=\tiny\ttfamily}

	\begin{itemize}

		\item Na:

			\begin{lstlisting}
				tt_content.menu_subpages.dataProcessing {
				  10 = TYPO3\CMS\Frontend\DataProcessing\MenuProcessor
				  10.special = directory
				  10.special.value.field = pages
				}

				<f:for each="{menu}" as="page">
				  ...
				</f:for>
			\end{lstlisting}

	\end{itemize}

\end{frame}

% ------------------------------------------------------------------------------
% LTXE-SLIDE-START
% LTXE-SLIDE-UID:		eea799b3-c580a6c8-1cf57513-a68e8e63
% LTXE-SLIDE-ORIGIN:	0bdb89f8-d302c7ef-2037711d-a3706e69 English
% LTXE-SLIDE-TITLE:		Fluid ViewHelper <f:variable>
% LTXE-SLIDE-REFERENCE:	!Feature: #79402 - New Fluid ViewHelper f:variable added
% ------------------------------------------------------------------------------

\begin{frame}[fragile]
	\frametitle{Extbase \& Fluid}
	\framesubtitle{Niewue Fluid ViewHelper \texttt{f:variable}}

	\begin{itemize}

		\item Een nieuwe ViewHelper \texttt{f:variable} is toegevoegd in Fluid 2.2.0,
			die nu de minimale afhankelijkheid is voor TYPO3 CMS

		\item Met de ViewHelper kunnen variabelen toegewezen worden in het sjabloon:

			\begin{lstlisting}
				<f:variable name="myvariable">Inhoud van mijn variabele</f:variable>
				<f:variable name="myvariable" value="Inhoud van mijn variabele"></f:variable>
				{f:variable(name: 'myvariable', value: 'Inhoud van mijn variabele')}
				{myoriginalvariable -> f:variable(name: 'mijnnieuwevariabele')}
			\end{lstlisting}

	\end{itemize}

\end{frame}

% ------------------------------------------------------------------------------
% LTXE-SLIDE-START
% LTXE-SLIDE-UID:		ffd9f64e-dcad4da5-2e6193b6-b752e24f
% LTXE-SLIDE-ORIGIN:	c459d3a9-e6fb03fe-cc79880c-d86a18d1 English
% LTXE-SLIDE-TITLE:		New Default Layout for Fluid Styled Content (1/2)
% LTXE-SLIDE-REFERENCE:	!Feature: #79622 - New default layout for Fluid Styled Content
% ------------------------------------------------------------------------------

\begin{frame}[fragile]
	\frametitle{Extbase \& Fluid}
	\framesubtitle{Nieuwe standaard layout voor Fluid Styled Content (1/2)}

	\begin{itemize}
		\item Voorheen waren er drie layouts waaruit te kiezen was bij het maken
			van eigen maatwerk-inhoudselementen of het overschrijven van
			een bestaand sjabloon.

		\item Voor eenvoudiger onderhoud en het eenvoudig overschrijven zijn deze
		 	teruggebracht tot een enkele layout met de naam \texttt{Default}.
		 	Alle secties zijn optioneel met een fallback als de sectie niet ingesteld
		 	is. Ook wordt het concept "DropIn" geïntroduceerd.

		\item De \texttt{Default} layout bestaat uit vijf voorgedefinieerde secties
			die gebruikt kunnen worden om de uitvoer voor de inhoud te vormen.
			In de meeste gevallen is alleen de sectie \texttt{Main} nodig. De secties
			worden gebruikt in de exacte volgorde:
			\texttt{Before}, \texttt{Header}, \texttt{Main}, \texttt{Footer}, \texttt{After}
	\end{itemize}

\end{frame}

% ------------------------------------------------------------------------------
% LTXE-SLIDE-START
% LTXE-SLIDE-UID:		41459476-d8019de7-ad5b172d-8208219d
% LTXE-SLIDE-ORIGIN:	da3b6df5-a15241de-c724dfb6-765885d4 English
% LTXE-SLIDE-TITLE:		New Default Layout for Fluid Styled Content (2/2)
% LTXE-SLIDE-REFERENCE:	!Feature: #79622 - New default layout for Fluid Styled Content
% ------------------------------------------------------------------------------

\begin{frame}[fragile]
	\frametitle{Extbase \& Fluid}
	\framesubtitle{Nieuwe standaard layout voor Fluid Styled Content (2/2)}

	\begin{itemize}
		\item De secties \texttt{Before} en \texttt{After} zijn zogenaamde "DropIn" secties
		\item DropIns zijn ingevoerd om extra functionaliteit in alle inhoudselementen te plaatsen
			zonder layouts en sjablonen te overschrijven
		\item DropIns zijn in feite lege partials die indien nodig overschreven kunnen worden
		\item DropIn locaties:
			\begin{itemize}
				\item \texttt{Resources/Private/Partials/DropIn/Before/All.html}
				\item \texttt{Resources/Private/Partials/DropIn/After/All.html}
			\end{itemize}
	\end{itemize}

\end{frame}

% ------------------------------------------------------------------------------
% ------------------------------------------------------------------------------
% TYPO3 CMS 8.7 - What's New - Chapter "In-Depth Changes" (Italian Version)
%
% @author	Michael Schams <schams.net>
% @license	Creative Commons BY-NC-SA 3.0
% @link		http://typo3.org/download/release-notes/whats-new/
% @language	English
% ------------------------------------------------------------------------------
% LTXE-CHAPTER-UID:		5ebcecbe-66abfa57-cf38bc00-aa637965
% LTXE-CHAPTER-NAME:	In-Depth Changes
% ------------------------------------------------------------------------------

\section{In-Depth Changes}
\begin{frame}[fragile]
	\frametitle{In-Depth Changes}

	\begin{center}\huge{Chapter 2:}\end{center}
	\begin{center}\huge{\color{typo3darkgrey}\textbf{In-Depth Changes}}\end{center}

\end{frame}

% ------------------------------------------------------------------------------
% LTXE-SLIDE-START
% LTXE-SLIDE-UID:		5a53a0fb-1dfa4c90-6822735c-081a260c
% LTXE-SLIDE-ORIGIN:	4af511c0-37c8bba2-c0e9fecc-290510c4 English
% LTXE-SLIDE-TITLE:		Feature: #79343 - Allow overriding PATH_site via environment variable
% ------------------------------------------------------------------------------

\begin{frame}[fragile]
	\frametitle{In-Depth Changes}
	\framesubtitle{Allow overriding \texttt{PATH\_site} via environment variable}

	\begin{itemize}
		\item It is now possible to define the \texttt{PATH\_site} constant, which acts as a basis
			for any entry point running a TYPO3 system, via the environment variable
			\texttt{TYPO3\_PATH\_ROOT}.

		\item This variable is automatically calculated and set for any TYPO3 installation set up
			via composer, making it possible to run the TYPO3 command line interface from any
			location of the system.

	\end{itemize}

\end{frame}


% ------------------------------------------------------------------------------
% LTXE-SLIDE-START
% LTXE-SLIDE-UID:		8cd3345d-84eb0a9e-3c59779e-3922c32e
% LTXE-SLIDE-ORIGIN:	276f9224-49ac8cf7-840470f3-8e32c0a1 English
% LTXE-SLIDE-TITLE:		Feature: #80126 maximum field length not set as attribute "maxlength"
% ------------------------------------------------------------------------------

\begin{frame}[fragile]
	\frametitle{In-Depth Changes}
	\framesubtitle{Maximum field length not set as attribute \texttt{maxlength}}

	% decrease font size for code listing
	\lstset{basicstyle=\tiny\ttfamily}

	\begin{itemize}
		\item If a form element is set to be using the \texttt{String length} server side validation through
			the form editor, the client side validation properties \texttt{minlength} and
			\texttt{maxlength} will be rendered.

			\begin{lstlisting}
				renderables:
				    -
				        type: <formElementType>
				        ...
				        properties:
				            fluidAdditionalAttributes:
				            minlength: 2
				            maxlength: 3
				            ...
				        validators:
				            -
				                identifier: StringLength
				                options:
				                    minimum: 2
				                    maximum: 3
			\end{lstlisting}
	\end{itemize}

\end{frame}

% ------------------------------------------------------------------------------
% LTXE-SLIDE-START
% LTXE-SLIDE-UID:		fdb08712-bc1a66d4-084c6c5c-92c88aa6
% LTXE-SLIDE-ORIGIN:	0fcb1156-153ae841-291c74a1-cda5af0d English
% LTXE-SLIDE-TITLE:		Feature: #80196 - EXT:form - support multiple form elements per row
% ------------------------------------------------------------------------------

\begin{frame}[fragile]
	\frametitle{In-Depth Changes}
	\framesubtitle{EXT:form - support multiple form elements per row}

	% decrease font size for code listing
	\lstset{basicstyle=\tiny\ttfamily}

	\begin{columns}[T]
		\begin{column}{0.5\textwidth}
			\begin{itemize}
				\item Two new form element types have been added to the form framework: \texttt{GridContainer}
					and \texttt{GridRow}

				\item Using these "container" form elements will enable you to define multiple form elements per row.
			\end{itemize}
		\end{column}
		\begin{column}{0.5\textwidth}
			\begin{lstlisting}
				type: Form
				identifier: example-form-gridcontainer
				label: 'Form Grid Container'
				prototypeName: standard
				renderables:
				    -
				        type: Page
				        identifier: page-1
				        label: Page
				        renderables:
				            -
				                type: GridContainer
				                identifier: gridcontainer-2
				                label: 'Grid: Container'
				                renderables:
				                    -
				                        type: GridRow
				                        identifier: gridrow-2
				                        label: 'Grid: Row'
				                        renderables:
				                        ...
			\end{lstlisting}
		\end{column}
	\end{columns}

\end{frame}


% ------------------------------------------------------------------------------
% LTXE-SLIDE-START
% LTXE-SLIDE-UID:		d6f441f0-be5f6eae-ad6e285c-00bcb97b
% LTXE-SLIDE-ORIGIN:	5eaa97c2-1f92b65e-ab668373-6bf97c43 English
% LTXE-SLIDE-TITLE:		Feature: #80196 - EXT:form - support multiple form elements per row
% ------------------------------------------------------------------------------
\begin{frame}[fragile]
	\frametitle{In-Depth Changes}
	\framesubtitle{EXT:form - support multiple form elements per row}

	\begin{itemize}
		\item Any Extbase Command Controller can now be accessed via the new Symfony Console
			CLI entrypoint by simply calling
			\texttt{typo3/sysext/core/bin/typo3 controller:command}.

		\item Using the existing CLI entrypoint via
			\texttt{typo3/cli\_dispatch.phpsh extbase controller:command} still works as expected.
	\end{itemize}

\end{frame}


% ------------------------------------------------------------------------------
% LTXE-SLIDE-START
% LTXE-SLIDE-UID:		ad7d64c4-47f6c7dc-21d6b55c-a473fb9c
% LTXE-SLIDE-ORIGIN:	bc5412de-1c7420ab-56b21fac-20d39822 English
% LTXE-SLIDE-TITLE:		Important: #80391 - Css Styled Content will not reset TypoScript Constants
% ------------------------------------------------------------------------------

\begin{frame}[fragile]
	\frametitle{In-Depth Changes}
	\framesubtitle{CSS Styled Content will not reset TypoScript Constants}

	\begin{itemize}
		\item Previously the TypoScript definition from CSS Styled Content resetted all constants
			that were set before the static template was included to preserve the namespace
			\texttt{styles.content}.

		\item Since there is no need to reset the constants, this behaviour is removed.
	\end{itemize}

\end{frame}

% ------------------------------------------------------------------------------

% ------------------------------------------------------------------------------
% TYPO3 CMS 8.7 - What's New - Chapter "TypoScript" (French Version)
%
% @author	Michael Schams <schams.net>
% @license	Creative Commons BY-NC-SA 3.0
% @link		http://typo3.org/download/release-notes/whats-new/
% @language	French
% ------------------------------------------------------------------------------
% LTXE-CHAPTER-UID:		3a9852ea-e2360d9d-1ff5eec1-a7de3f9f
% LTXE-CHAPTER-NAME:	TypoScript
% ------------------------------------------------------------------------------

\section{TSconfig \& TypoScript}
\begin{frame}[fragile]
	\frametitle{TSconfig \& TypoScript}

	\begin{center}\huge{Chapitre 1~:}\end{center}
	\begin{center}\huge{\color{typo3darkgrey}\textbf{TSconfig \& TypoScript}}\end{center}

\end{frame}

% ------------------------------------------------------------------------------
% LTXE-SLIDE-START
% LTXE-SLIDE-UID:		2421046d-b86f4bd5-1b57de04-f2ff792f
% LTXE-SLIDE-ORIGIN:	27017c26-9a66e642-46da3a71-4b18702a English
% LTXE-SLIDE-TITLE:		Breaking: #80412 - New shared content element typoscript libary object for Fluid Styled Content
% ------------------------------------------------------------------------------

\begin{frame}[fragile]
	\frametitle{TSconfig \& TypoScript}
	\framesubtitle{Nouvel objet TS d'élément de contenu partagé pour FSC (1/3)}

	\begin{itemize}
		\item Pour résoudre une incohérence entre les inscriptions d'élément de contenu entre
			CSS Styled Content (CSC) et Fluid Styled Content (FSC) avec \texttt{Extbase} ou
			\texttt{addPItoST43}, nous introduisons un objet de contenu partagé pour les
			éléments de contenu et retirons l'usage de \texttt{lib.fluidContent}.

		\item Le code se basait sur l'existence retirée de \texttt{lib.stdheader} et ignorait
			les différentes informations de disposition dans le contexte de Fluid Styled Content.

		\item Code généré avant le changement~:

			\begin{lstlisting}
				tt_content.myce = COA
				tt_content.myce {
				    10 =< lib.stdheader
				    20 =< plugin.myContent
				}
			\end{lstlisting}

	\end{itemize}

\end{frame}

% ------------------------------------------------------------------------------
% LTXE-SLIDE-START
% LTXE-SLIDE-UID:		5bad9a0e-62ce5938-e6805a97-97814c56
% LTXE-SLIDE-ORIGIN:	1db56916-738b57dc-34c59088-f05c64e0 English
% LTXE-SLIDE-TITLE:		Breaking: #80412 - New shared content element typoscript libary object for Fluid Styled Content
% ------------------------------------------------------------------------------

\begin{frame}[fragile]
	\frametitle{TSconfig \& TypoScript}
	\framesubtitle{Nouvel objet TS d'élément de contenu partagé pour FSC (2/3)}

	\begin{itemize}
		\item Pour l'inscription des éléments de contenu, le TypoScript \texttt{lib.contentElement}
			est utilisé par "CSC" et "FSC" et remplace l'utilisation de \texttt{lib.fluidContent}.
			Le code généré est légèrement ajusté pour correspondre aux besoins de toutes les
			définitions de rendu de contenu et s'adapte pour les besoins spécifiques puisqu'une
			référence est utilisée au lieu d'une définition directe.

		\item Code généré après le changement~:

			\begin{lstlisting}
				tt_content.myce =< lib.contentElement
				tt_content.myce {
				    templateName = Generic
				    20 =< plugin.myContent
				}
			\end{lstlisting}
	\end{itemize}

\end{frame}


% ------------------------------------------------------------------------------
% LTXE-SLIDE-START
% LTXE-SLIDE-UID:		dd3c6358-7be3923d-14613c0a-fea064f8
% LTXE-SLIDE-ORIGIN:	d96ab2c6-978a155e-837800a0-02b7772a English
% LTXE-SLIDE-TITLE:		Breaking: #80412 - New shared content element typoscript libary object for Fluid Styled Content
% ------------------------------------------------------------------------------
\begin{frame}[fragile]
	\frametitle{TSconfig \& TypoScript}
	\framesubtitle{Nouvel objet TS d'élément de contenu partagé pour FSC (3/3)}

	\begin{itemize}
		\item CSS Styled Content ajoute l'objet \texttt{lib.stdheader} manquant et tout fonctionne
			comme précédemment, aucune migration ou ajustement du code est nécessaire. Comme
			\texttt{COA} ne comprend pas l'option \texttt{templateName}, elle sera simplement ignorée.

		\item Fluid Styled Content ajoute la logique nécessaire à travers \texttt{lib.contentElement}.
			Tous les éléments de contenu inscrits au travers des APIs TYPO3 partagent maintenant un
			template multifonction \texttt{Generic}. Celui-ci fourni les dispositions et surcharges des
			options connues de "FSC".

	\end{itemize}
\end{frame}

% ------------------------------------------------------------------------------
% LTXE-SLIDE-START
% LTXE-SLIDE-UID:		1bf59708-469356bf-41ece0b9-25e8f7ae
% LTXE-SLIDE-ORIGIN:	0dfec33b-e8368c0a-9e16d774-edafcbf9 English
% LTXE-SLIDE-TITLE:		Feature: #79812 - Allow overriding cropVariants for Image Manipulation
% ------------------------------------------------------------------------------
\begin{frame}[fragile]
	\frametitle{TSconfig \& TypoScript}
	\framesubtitle{Surcharge de \texttt{cropVariants} pour la manipulation d'images (1/3)}

	\begin{itemize}
		\item L'introduction des variantes de recadrage (Ticket \texttt{75880}) a cassé la
			gestion des images recadrées lors de l'utilisation de TypoScript pour le rendu des
			fichiers et références. La fonctionnalité est corrigée et une option TypoScript est
			introduite pour spécifier une \texttt{cropVariant} différente.

		\item Pour utiliser une \texttt{cropVariant} différente de celle par défaut, il est
			possible de spécifier un nom de \texttt{cropVariant} dans la configuration TypoScript.

			\begin{lstlisting}
				# Use specific cropVariant for the images
				tt_content.image.20.1.file.cropVariant = mobile
			\end{lstlisting}
	\end{itemize}

\end{frame}


% ------------------------------------------------------------------------------
% LTXE-SLIDE-START
% LTXE-SLIDE-UID:		478c4e48-59a221ff-c4fc5292-0d7fd032
% LTXE-SLIDE-ORIGIN:	1beb3276-d79fb9a6-b53dcde1-fcacdec3 English
% LTXE-SLIDE-TITLE:		Feature: #79812 - Allow overriding cropVariants for Image Manipulation
% ------------------------------------------------------------------------------

\begin{frame}[fragile]
	\frametitle{TSconfig \& TypoScript}
	\framesubtitle{Surcharge de \texttt{cropVariants} pour la manipulation d'images (2/3)}

	\begin{itemize}
		\item Avec l'introduction du Ticket \texttt{75880}, il est possible de définir de multiples
			\texttt{cropVariants} en TCA. La possibilité de changer ou surcharger ces \texttt{cropVariants}
			par TSconfig est ajoutée.

		\item Définir une option FormEngine par
			\texttt{TCEFORM.sys\_file\_reference.crop.config.cropVariants.*} ne fonctionne pas.

	\end{itemize}

\end{frame}


% ------------------------------------------------------------------------------
% LTXE-SLIDE-START
% LTXE-SLIDE-UID:		9e908bba-e991a1a2-53898d45-c9f68c51
% LTXE-SLIDE-ORIGIN:	1bfe3276-d79acb9a6-b5dbcde1-fc95dec3 English
% LTXE-SLIDE-TITLE:		Feature: #79812 - Allow overriding cropVariants for Image Manipulation
% ------------------------------------------------------------------------------

\begin{frame}[fragile]
	\frametitle{TSconfig \& TypoScript}
	\framesubtitle{Surcharge de \texttt{cropVariants} pour la manipulation d'images (3/3)}

	% decrease font size for code listing
	\lstset{basicstyle=\tiny\ttfamily}

	\begin{itemize}
		\item Exemple~:

			\begin{lstlisting}
				TCEFORM.sys_file_reference.crop.config.cropVariants {
				    default {
				        title = Default desktop
				        selectedRatio = NaN
				        allowedAspectRatios {
				            NaN {
				                title = free
				                value = 0.0
				            }
				        }
				    }
				    specialMobile {
				        title = Our special mobile variant
				        selectedRatio = NaN
				        allowedAspectRatios {
				            4:3 {
				                title = ratio 4/3
				                value = 1.3333333
				            }
				        }
				    }
				}
			\end{lstlisting}

	\end{itemize}

\end{frame}


% ------------------------------------------------------------------------------
% LTXE-SLIDE-START
% LTXE-SLIDE-UID:		78effb07-1a8a93fb-cbf000da-566dc720
% LTXE-SLIDE-ORIGIN:	47d7657b-8d44ad91-6d089504-42a0a201 English
% LTXE-SLIDE-TITLE:		Feature: #80452 - Extbase CLI commands available via new CLI API
% ------------------------------------------------------------------------------

\begin{frame}[fragile]
	\frametitle{TSconfig \& TypoScript}
	\framesubtitle{Configuration de l'authentification Frontend disponible par constantes TypoScript}

	\begin{itemize}
		\item Les options de configurations les plus communes de l'authentification Frontend
		  	sont disponibles en constantes TypoScript et déplacées dans une section "Authentification
		  	Frontend" dans l'éditeur de constantes.

		\item Pour la liste des constantes, voir~:
			\href{https://docs.typo3.org/typo3cms/extensions/core/8-dev/singlehtml/Index.html#feature-80374-frontend-login-configuration-now-available-through-typoscript-constants}{docs.typo3.org}
	\end{itemize}

\end{frame}

% ------------------------------------------------------------------------------
% LTXE-SLIDE-START
% LTXE-SLIDE-UID:		6cfddfd1-40840aef-7e6d8051-b2311809
% LTXE-SLIDE-ORIGIN:	37c364c5-ea04bd46-34e95917-172744bf English
% LTXE-SLIDE-TITLE:		Important: #80506 - Dbal compatible field quoting in TypoScript
% ------------------------------------------------------------------------------

\begin{frame}[fragile]
	\frametitle{TSconfig \& TypoScript}
	\framesubtitle{Protection des champs en TypoScript compatible avec DBAL}

	\begin{itemize}
		\item Les propriétés TypoScript contenant des fragments SQL nécessitent la bonne protection
		 	des noms de champ pour être compatible avec les différents pilotes de base de données.
		 	Le framework de base de données du noyau applique la bonne protection si les noms de champ
		 	sont entourés comme \texttt{\{\#fieldName\}}

		\item Il est conseillé d'adapter les extensions pour la bonne exécution sur les base de données comme
		 	PostgreSQL.

			\begin{lstlisting}
				select.where = {#colPos}=0
			\end{lstlisting}


	\end{itemize}

\end{frame}

% ------------------------------------------------------------------------------

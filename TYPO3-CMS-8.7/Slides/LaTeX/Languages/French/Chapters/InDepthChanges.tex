% ------------------------------------------------------------------------------
% TYPO3 CMS 8.7 - What's New - Chapter "Changements en profondeur" (French Version)
%
% @author	Michael Schams <schams.net>
% @license	Creative Commons BY-NC-SA 3.0
% @link		http://typo3.org/download/release-notes/whats-new/
% @language	French
% ------------------------------------------------------------------------------
% LTXE-CHAPTER-UID:		5ebcecbe-66abfa57-cf38bc00-aa637965
% LTXE-CHAPTER-NAME:	Changements en profondeur
% ------------------------------------------------------------------------------

\section{Changements en profondeur}
\begin{frame}[fragile]
	\frametitle{Changements en profondeur}

	\begin{center}\huge{Chapitre 2~:}\end{center}
	\begin{center}\huge{\color{typo3darkgrey}\textbf{Changements en profondeur}}\end{center}

\end{frame}

% ------------------------------------------------------------------------------
% LTXE-SLIDE-START
% LTXE-SLIDE-UID:		b6804abf-408fd09a-52ded7e1-312179ba
% LTXE-SLIDE-ORIGIN:	4af511c0-37c8bba2-c0e9fecc-290510c4 English
% LTXE-SLIDE-TITLE:		Feature: #79343 - Allow overriding PATH_site via environment variable
% ------------------------------------------------------------------------------

\begin{frame}[fragile]
	\frametitle{Changements en profondeur}
	\framesubtitle{Surcharge de \texttt{PATH\_site} par l'environnement}

	\begin{itemize}
		\item La constante \texttt{PATH\_site}, servant de base pour tous les points d'entrée
			des systèmes TYPO3, se surcharge à l'aide de la variable d'environnement
			\texttt{TYPO3\_PATH\_ROOT}.

		\item Cette variable est calculée automatiquement et définie pour toutes les installation
			TYPO3 via composer, rendant possible l'exécution de la ligne de commande TYPO3 depuis
			n'importe quel emplacement du système.

	\end{itemize}

\end{frame}


% ------------------------------------------------------------------------------
% LTXE-SLIDE-START
% LTXE-SLIDE-UID:		bd6cd956-9688a131-39474f3a-e5db45aa
% LTXE-SLIDE-ORIGIN:	276f9224-49ac8cf7-840470f3-8e32c0a1 English
% LTXE-SLIDE-TITLE:		Feature: #80126 maximum field length not set as attribute "maxlength"
% ------------------------------------------------------------------------------

\begin{frame}[fragile]
	\frametitle{Changements en profondeur}
	\framesubtitle{Longueur maximale des champs non définie par l'attribut \texttt{maxlength}}

	% decrease font size for code listing
	\lstset{basicstyle=\tiny\ttfamily}

	\begin{itemize}
		\item Si un élément de formulaire utilise la validation serveur \texttt{String length} à travers
			l'éditeur de formulaire, les propriétés de validation côté client \texttt{minlength} et
			\texttt{maxlength} seront mise en place.

			\begin{lstlisting}
				renderables:
				    -
				        type: <formElementType>
				        ...
				        properties:
				            fluidAdditionalAttributes:
				            minlength: 2
				            maxlength: 3
				            ...
				        validators:
				            -
				                identifier: StringLength
				                options:
				                    minimum: 2
				                    maximum: 3
			\end{lstlisting}
	\end{itemize}

\end{frame}

% ------------------------------------------------------------------------------
% LTXE-SLIDE-START
% LTXE-SLIDE-UID:		7802e0e3-ed593329-dc453c43-55727a93
% LTXE-SLIDE-ORIGIN:	0fcb1156-153ae841-291c74a1-cda5af0d English
% LTXE-SLIDE-TITLE:		Feature: #80196 - EXT:form - support multiple form elements per row
% ------------------------------------------------------------------------------

\begin{frame}[fragile]
	\frametitle{Changements en profondeur}
	\framesubtitle{EXT:form - support de plusieurs éléments par ligne}

	% decrease font size for code listing
	\lstset{basicstyle=\tiny\ttfamily}

	\begin{columns}[T]
		\begin{column}{0.5\textwidth}
			\begin{itemize}
				\item Deux nouveaux types d'élément sont ajoutés au framework de formulaire~: \texttt{GridContainer}
					et \texttt{GridRow}

				\item L'usage de ces éléments de formulaire "conteneur" permet de définir plusieurs éléments de
					contenu par ligne.
			\end{itemize}
		\end{column}
		\begin{column}{0.5\textwidth}
			\begin{lstlisting}
				type: Form
				identifier: example-form-gridcontainer
				label: 'Form Grid Container'
				prototypeName: standard
				renderables:
				    -
				        type: Page
				        identifier: page-1
				        label: Page
				        renderables:
				            -
				                type: GridContainer
				                identifier: gridcontainer-2
				                label: 'Grid: Container'
				                renderables:
				                    -
				                        type: GridRow
				                        identifier: gridrow-2
				                        label: 'Grid: Row'
				                        renderables:
				                        ...
			\end{lstlisting}
		\end{column}
	\end{columns}

\end{frame}


% ------------------------------------------------------------------------------
% LTXE-SLIDE-START
% LTXE-SLIDE-UID:		437b8530-ea448458-8ce964d2-fa79ac4a
% LTXE-SLIDE-ORIGIN:	5eaa97c2-1f92b65e-ab668373-6bf97c43 English
% LTXE-SLIDE-TITLE:		Feature: #80452 - Extbase CLI commands available via new CLI API
% ------------------------------------------------------------------------------
\begin{frame}[fragile]
	\frametitle{Changements en profondeur}
	\framesubtitle{Commandes CLI Extbase disponibles via la nouvelle API CLI}

	\begin{itemize}
		\item Tous les contrôleurs de commande Extbase sont accessibles depuis le point
		 	d'entrée de la Console Symfony en appelant
			\texttt{typo3/sysext/core/bin/typo3 controller:command}.

		\item L'usage du point d'entrée CLI existant via
			\texttt{typo3/cli\_dispatch.phpsh extbase controller:command} fonctionne toujours comme attendu.
	\end{itemize}

\end{frame}


% ------------------------------------------------------------------------------
% LTXE-SLIDE-START
% LTXE-SLIDE-UID:		4001b603-91a8b1a8-4d4abb27-54b9b309
% LTXE-SLIDE-ORIGIN:	bc5412de-1c7420ab-56b21fac-20d39822 English
% LTXE-SLIDE-TITLE:		Important: #80391 - Css Styled Content will not reset TypoScript Constants
% ------------------------------------------------------------------------------

\begin{frame}[fragile]
	\frametitle{Changements en profondeur}
	\framesubtitle{CSS Styled Content ne réinitialise pas les constantes TypoScript}

	\begin{itemize}
		\item Précédemment, les définitions TypoScript de CSS Styled Content réinitialisaient toutes
			les constantes définies avant l'inclusion du gabarit statique pour préserver l'espace de
			noms \texttt{styles.content}.

		\item Le besoin n'étant plus d'actualité, le comportement est retiré.
	\end{itemize}

\end{frame}

% ------------------------------------------------------------------------------

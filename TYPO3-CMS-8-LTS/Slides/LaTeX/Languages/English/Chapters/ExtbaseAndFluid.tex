% ------------------------------------------------------------------------------
% TYPO3 CMS 8 LTS - What's New - Chapter "Extbase & Fluid" (English Version)
%
% @author	Michael Schams <schams.net>
% @license	Creative Commons BY-NC-SA 3.0
% @link		http://typo3.org/download/release-notes/whats-new/
% @language	English
% ------------------------------------------------------------------------------
% LTXE-CHAPTER-UID:		846d40ce-66fdc2ec-750dcf95-33ce93e0
% LTXE-CHAPTER-NAME:	Extbase & Fluid
% ------------------------------------------------------------------------------

\section{Extbase \& Fluid}
\begin{frame}[fragile]
	\frametitle{Extbase \& Fluid}

	\begin{center}\huge{\color{typo3darkgrey}\textbf{Extbase \& Fluid}}\end{center}
	\begin{center}\large{\textit{}}\end{center}

\end{frame}

% ------------------------------------------------------------------------------
% LTXE-SLIDE-START
% LTXE-SLIDE-UID:		9413e7f2-dd08f22e-dd21ee20-d7a1b0bf
% LTXE-SLIDE-TITLE:		#69863: Standalone revised Fluid
% ------------------------------------------------------------------------------

\begin{frame}[fragile]
	\frametitle{Extbase \& Fluid}
	\framesubtitle{Standalone Fluid}

	% decrease font size for code listin
	\lstset{basicstyle=\tiny\ttfamily}

	\begin{itemize}

		\item The Fluid rendering engine of TYPO3 CMS is replaced by the standalone
			capable Fluid which is now included as a composer dependency

		\item The old Fluid extension is converted to a so-called \textit{Fluid adapter}
			which allows TYPO3 CMS to use standalone Fluid

		\item New features/capabilities have been added in nearly all areas of Fluid\newline
			\small
				(see \href{https://docs.typo3.org/typo3cms/extensions/core/8-dev/Changelog/8.0/Feature-69863-UseNewStandaloneFluidAsComposerDependency.html}{docs.typo3.org}
				for further details)
			\normalsize

		\item Most importantly: several of the Fluid components which were completely
			internal and impossible to replace in the past, are now easy to
			replace and have been fitted with a public API

		\item Standalone Fluid also improves backend performance

	\end{itemize}

\end{frame}

% ------------------------------------------------------------------------------
% LTXE-SLIDE-START
% LTXE-SLIDE-UID:		32daa81f-7e805b84-6f829395-94677911
% LTXE-SLIDE-TITLE:		#69863: Rendering Context (1)
% ------------------------------------------------------------------------------

\begin{frame}[fragile]
	\frametitle{Extbase \& Fluid}
	\framesubtitle{Rendering Context (1)}

	% decrease font size for code listing
	\lstset{basicstyle=\tiny\ttfamily}

	\begin{itemize}

		\item The most important new piece of public API is the RenderingContext

		\item The previously internal-only RenderingContext used by Fluid has been
			expanded to be responsible for a vital new Fluid feature:
			\textbf{implementation provisioning}

		\item This enables developers to change a range of classes, Fluid uses for
			parsing, resolving, caching etc.

		\item This can be achieved by either including a custom RenderingContext or
			manipulating the default RenderingContext by public methods.

	\end{itemize}

\end{frame}

% ------------------------------------------------------------------------------
% LTXE-SLIDE-START
% LTXE-SLIDE-UID:		3d44632f-c87d4adf-3328f596-2e5744bc
% LTXE-SLIDE-TITLE:		#69863: Rendering Context (2)
% ------------------------------------------------------------------------------

\begin{frame}[fragile]
	\frametitle{Extbase \& Fluid}
	\framesubtitle{Rendering Context (2)}

	% decrease font size for code listing
	%\lstset{basicstyle=\tiny\ttfamily}
	\lstset{basicstyle=\smaller\ttfamily}

	\begin{itemize}

		\item The features on the following slides can be controlled by manipulating
			the RenderingContext. By default, none of them are enabled - but calling
			a simple method (via your View instance) allows you to enable them:

		\begin{lstlisting}
			$view->getRenderingContext()->setLegacyMode(false);
		\end{lstlisting}

	\end{itemize}

\end{frame}

% ------------------------------------------------------------------------------
% LTXE-SLIDE-START
% LTXE-SLIDE-UID:		5f486f77-38e6694e-2fad0d6b-7b260ad1
% LTXE-SLIDE-TITLE:		#69863: ExpressionNodes (1)
% ------------------------------------------------------------------------------

\begin{frame}[fragile]
	\frametitle{Extbase \& Fluid}
	\framesubtitle{ExpressionNodes (1)}

	% decrease font size for code listing
	%\lstset{basicstyle=\tiny\ttfamily}
	\lstset{basicstyle=\smaller\ttfamily}

	\begin{itemize}

		\item ExpressionNodes are a new type of Fluid syntax structures which all
			share a common trait: they only work inside the curly braces

		\begin{lstlisting}
			$view->getRenderingContext()->setExpressionNodeTypes(array(
			   'Class\Number\One',
			   'Class\Number\Two'
			));
		\end{lstlisting}

		\item Developers can add their own additional ExpressionNode types

		\item Each one consists of a pattern to be matched and methods dictated
			by an interface to process the matches

		\item Any existing ExpressionNode type can be used as reference

	\end{itemize}

\end{frame}

% ------------------------------------------------------------------------------
% LTXE-SLIDE-START
% LTXE-SLIDE-UID:		090cec75-2751b839-6ad8eebc-0adfe66a
% LTXE-SLIDE-TITLE:		#69863: ExpressionNodes (2)
% ------------------------------------------------------------------------------

\begin{frame}[fragile]
	\frametitle{Extbase \& Fluid}
	\framesubtitle{ExpressionNodes (2)}

	ExpressionNodeTypes allow new syntaxes such as:

	\begin{itemize}

		\item \textbf{CastingExpressionNode}\newline
			\small
				allows casting a variable to certain types, for example to guarantee
				an integer or a boolean. It is used simply with an \texttt{as} keyword:
				\texttt{\{myStringVariable as boolean\}} or
				\texttt{\{myBooleanVariable as integer\}} and so on.
				Attempting to cast a variable to an incompatible type causes a standard
				Fluid error.
			\normalsize

		\item \textbf{MathExpressionNode}\newline
			\small
				allows basic mathematical operations on variables, for example
				\texttt{\{myNumber + 1\}}, \texttt{\{myPercent / 100\}} or
				\texttt{\{myNumber * 100\}} and so on.
				An impossible expression returns an empty output.
			\normalsize

	\end{itemize}

\end{frame}

% ------------------------------------------------------------------------------
% LTXE-SLIDE-START
% LTXE-SLIDE-UID:		c8935f68-0905db52-11932973-63a85a00
% LTXE-SLIDE-TITLE:		#69863: ExpressionNodes (3)
% ------------------------------------------------------------------------------

\begin{frame}[fragile]
	\frametitle{Extbase \& Fluid}
	\framesubtitle{ExpressionNodes (3)}

	ExpressionNodeTypes allow new syntaxes such as:

	\begin{itemize}

		\item \textbf{TernaryExpressionNode}\newline
			\small
				allows an inline ternary condition which only operates on variables.
				Typical use case is: "if this variable then use that variable else use
				another variable". It is used as:\newline
				\texttt{\{myToggleVariable ? myThenVariable : myElseVariable\}}\newline
				Note: does not support any nested expressions, inline ViewHelper
				syntaxes or similar inside it. It must be used only with standard
				variables as input.
			\normalsize

	\end{itemize}

\end{frame}

% ------------------------------------------------------------------------------
% LTXE-SLIDE-START
% LTXE-SLIDE-UID:		153b9b53-d7f97aef-dc1ba8ab-fc2a763e
% LTXE-SLIDE-TITLE:		#69863: Namespaces are extensible (1)
% ------------------------------------------------------------------------------

\begin{frame}[fragile]
	\frametitle{Extbase \& Fluid}
	\framesubtitle{Namespaces are extensible (1)}

	\begin{itemize}

		\item Fluid allows each namespace alias (for example \texttt{f:}) to be
			extended by adding an additional PHP namespace to it

		\item PHP namespaces are also checked for the presence of ViewHelper classes

		\item This also means that developers can override individual ViewHelpers
			with custom versions and have their ViewHelpers called when the
			\texttt{f:} namespace is used

		\item This change also implies that namespaces are no longer monadic.\newline
			When using
			\texttt{
				\{namespace f=My\textbackslash
				Extension\textbackslash
				ViewHelpers\textbackslash\}}\newline
				you will no longer receive an "namespace already registered" error.
				Fluid will add this PHP namespace instead and look for ViewHelpers
				there as well.

	\end{itemize}

\end{frame}

% ------------------------------------------------------------------------------
% LTXE-SLIDE-START
% LTXE-SLIDE-UID:		a2f0e799-fe523ffb-57634576-654a8efe
% LTXE-SLIDE-TITLE:		#69863: Namespaces are extensible (2)
% ------------------------------------------------------------------------------

\begin{frame}[fragile]
	\frametitle{Extbase \& Fluid}
	\framesubtitle{Namespaces are extensible (2)}

	\begin{itemize}

		\item Additional namespaces are checked from the bottom up, allowing the
			additional namespaces to override ViewHelper classes by placing them in
			the same scope

		\item For example: \texttt{f:format.nl2br} can be overridden by
			\texttt{
				My\textbackslash
				Extension\textbackslash
				ViewHelpers\textbackslash
				Format\textbackslash
				Nl2brViewHelper},

				given the namespace registration on previous slide

	\end{itemize}

\end{frame}


% ------------------------------------------------------------------------------
% LTXE-SLIDE-START
% LTXE-SLIDE-UID:		7c36b3a7-33860495-4efb65ef-f09ae14e
% LTXE-SLIDE-TITLE:		#69863: Complex conditional statements
% ------------------------------------------------------------------------------

\begin{frame}[fragile]
	\frametitle{Extbase \& Fluid}
	\framesubtitle{Complex conditional statements}

	% decrease font size for code listing
	\lstset{basicstyle=\tiny\ttfamily}

	\begin{itemize}

		\item Fluid now supports any degree of complex conditional statements with
			nesting and grouping:

			\begin{lstlisting}
				<f:if condition="({variableOne} && {variableTwo}) || {variableThree} || {variableFour}">
				  // Done if both variable one and two evaluate to true,
				  // or if either variable three or four do.
				</f:if>
			\end{lstlisting}

		\item In addition, \texttt{f:else} has been fitted with an "elseif"-like behavior:

			\begin{lstlisting}
				<f:if condition="{variableOne}">
				  <f:then>Do this</f:then>
				  <f:else if="{variableTwo}">
				    Do this instead if variable two evals true
				  </f:else>
				  <f:else if="{variableThree}">
				    Or do this if variable three evals true
				  </f:else>
				  <f:else>
				    Or do this if nothing above is true
				  </f:else>
				</f:if>
			\end{lstlisting}

	\end{itemize}

\end{frame}

% ------------------------------------------------------------------------------
% LTXE-SLIDE-START
% LTXE-SLIDE-UID:		d54cf918-48141072-8184ab81-cecc8e39
% LTXE-SLIDE-TITLE:		#69863: Dynamic variable name parts (1)
% ------------------------------------------------------------------------------

\begin{frame}[fragile]
	\frametitle{Extbase \& Fluid}
	\framesubtitle{Dynamic variable name parts (1)}

	% decrease font size for code listing
	\lstset{basicstyle=\tiny\ttfamily}

	\begin{itemize}

		\item Another new feature, likewise backwards compatible, is the added
			ability to use sub-variable references when accessing your variables.
			Consider the following Fluid template variables array:

			\begin{lstlisting}
				$mykey = 'foo'; // or 'bar', set by any source
				$view->assign('data', ['foo' => 1, 'bar' => 2]);
				$view->assign('key', $mykey);
			\end{lstlisting}

		\item With the following Fluid template:

			\begin{lstlisting}
				You chose: {data.{key}}.
				(output: "1" if key is "foo" or "2" if key is "bar")
			\end{lstlisting}

	\end{itemize}

\end{frame}

% ------------------------------------------------------------------------------
% LTXE-SLIDE-START
% LTXE-SLIDE-UID:		20385d89-9c0de3ac-9ef70ba6-310fdec3
% LTXE-SLIDE-TITLE:		#69863: Dynamic variable name parts (2)
% ------------------------------------------------------------------------------

\begin{frame}[fragile]
	\frametitle{Extbase \& Fluid}
	\framesubtitle{Dynamic variable name parts (2)}

	% decrease font size for code listing
	\lstset{basicstyle=\tiny\ttfamily}

	\begin{itemize}

		\item The same approach can also be used to generate dynamic parts of a string
			variable name:

			\begin{lstlisting}
				$mydynamicpart = 'First'; // or 'Second', set by any source
				$view->assign('myFirstVariable', 1);
				$view->assign('mySecondVariable', 2);
				$view->assign('which', $mydynamicpart);
			\end{lstlisting}

		\item With the following Fluid template:

			\begin{lstlisting}
				You chose: {my{which}Variable}.
				(output: "1" if which is "First" or "2" if which is "Second")
			\end{lstlisting}

	\end{itemize}

\end{frame}

% ------------------------------------------------------------------------------
% LTXE-SLIDE-START
% LTXE-SLIDE-UID:		15907f4f-d554b015-03be1fdc-4dc0fb25
% LTXE-SLIDE-TITLE:		#69863: New ViewHelpers
% ------------------------------------------------------------------------------

\begin{frame}[fragile]
	\frametitle{Extbase \& Fluid}
	\framesubtitle{New ViewHelpers}

	% decrease font size for code listing
	\lstset{basicstyle=\tiny\ttfamily}

	\begin{itemize}
		\item A few new ViewHelpers have been added as part of standalone Fluid
			and as such are also available in TYPO3 from now on:

			\begin{itemize}

				\item \textbf{\texttt{f:or}}\newline
					This is a shorter way of writing (chained) conditions.
					It supports the following syntax, which checks each variable
					and outputs the first one that is not empty:

					\begin{lstlisting}
						{variableOne -> f:or(alternative: variableTwo) -> f:or(alternative: variableThree)}
					\end{lstlisting}

				\item \textbf{\texttt{f:spaceless}}\newline
					This can be used in tag-mode around template code to
					eliminate redundant whitespace and blank lines for example
					caused by indenting ViewHelper usages

			\end{itemize}

	\end{itemize}

\end{frame}

% ------------------------------------------------------------------------------
% LTXE-SLIDE-START
% LTXE-SLIDE-UID:		06a9e027-1c0e09af-d41d1c4c-c0f07273
% LTXE-SLIDE-TITLE:		#76531: Add IconForRecordViewHelper
% ------------------------------------------------------------------------------
\begin{frame}[fragile]
	\frametitle{Extbase \& Fluid}
	\framesubtitle{Add IconForRecordViewHelper}

	% decrease font size for code listin
	\lstset{basicstyle=\tiny\ttfamily}

	\begin{itemize}

		\item A new ViewHelper to render icons for records has been added

			\begin{lstlisting}
				<core:iconForRecord table="sys_template" row="{templateRecord}" ></core:iconForRecord>

				// output:
				<span class="t3js-icon icon icon-size-small icon-state-default icon-mimetypes-x-content-template"
				  data-identifier="mimetypes-x-content-template">
				  <span class="icon-markup">
				    <img src="/typo3/sysext/core/Resources/Public/Icons/T3Icons/mimetypes/mimetypes-x-content-template.svg" width="16" height="16">
				  </span>
				</span>
			\end{lstlisting}

	\end{itemize}

\end{frame}

% ------------------------------------------------------------------------------
% LTXE-SLIDE-START
% LTXE-SLIDE-UID:		50143a1e-53a17c8c-83ffe7ea-5e88cba0
% LTXE-SLIDE-TITLE:		#29399: OptionViewHelper and OptgroupViewHelper
% ------------------------------------------------------------------------------

\begin{frame}[fragile]
	\frametitle{Extbase \& Fluid}
	\framesubtitle{Content for ViewHelper \texttt{f:form.select}}

	% decrease font size for code listing
	\lstset{basicstyle=\tiny\ttfamily}

	\begin{itemize}
		\item Introduced two new ViewHelpers allowing the manual definition of all options and
		    optgroups for the \texttt{f:form.select} as tag content of the select field

			\begin{itemize}
				\item \texttt{OptionViewHelper}
				\item \texttt{OptgroupViewHelper}
			\end{itemize}

		\item Example:

			\begin{lstlisting}
				<f:form.select name="myproperty">
				  <f:form.select.option value="1">Option one</f:form.select.option>
				  <f:form.select.option value="2">Option two</f:form.select.option>
				  <f:form.select.optgroup>
				    <f:form.select.option value="3">Grouped option one</f:form.select.option>
				    <f:form.select.option value="4">Grouped option twi</f:form.select.option>
				  </f:form.select.optgroup>
				</f:form.select>
			\end{lstlisting}

		\end{itemize}

\end{frame}

% ------------------------------------------------------------------------------
% LTXE-SLIDE-START
% LTXE-SLIDE-UID:		d36b9935-fc7d9e60-742a3f42-ef15e856
% LTXE-SLIDE-TITLE:		#78415: Global Fluid ViewHelper Namespace
% ------------------------------------------------------------------------------
\begin{frame}[fragile]
	\frametitle{Extbase \& Fluid}
	\framesubtitle{Global Fluid ViewHelper Namespace}

	\begin{itemize}
		\item Global Fluid ViewHelper namespaces are configurable now:\newline
			\smaller
				\texttt{\$GLOBALS['TYPO3\_CONF\_VARS']['SYS']['fluid']['namespaces']}
			\normalsize
		\item This allows the namespaces to be manipulated as part of the site configuration
		\item Benefits:

			\begin{itemize}
				\item Third party ViewHelper packages can manipulate the global Fluid namespace \texttt{f:}
				\item Third party ViewHelper packages are able to register new global namespaces as required
				\item Template developers can use such global namespaces without importing them first
					and can use them in all Fluid templates regardless of context
			\end{itemize}

	\end{itemize}

\end{frame}

% ------------------------------------------------------------------------------
% LTXE-SLIDE-START
% LTXE-SLIDE-UID:		9fed5ece-ae03648f-c3db2f41-42312f7f
% LTXE-SLIDE-TITLE:		#69863: ViewHelper namespaces can be extended also from PHP
% ------------------------------------------------------------------------------

\begin{frame}[fragile]
	\frametitle{Extbase \& Fluid}
	\framesubtitle{ViewHelper namespaces can be extended also from PHP}

	% decrease font size for code listing
	\lstset{basicstyle=\tiny\ttfamily}

	\begin{itemize}

		\item By accessing the ViewHelperResolver of the RenderingContext,
			developers can change the ViewHelper namespace inclusions on a global
			(read: per View instance) basis:

			\begin{lstlisting}
				$resolver = $view->getRenderingContext()->getViewHelperResolver();
				// equivalent of registering namespace in template(s):
				$resolver->registerNamespace('news', 'GeorgRinger\News\ViewHelpers');
				// adding additional PHP namespaces to check when resolving ViewHelpers:
				$resolver->extendNamespace('f', 'My\Extension\ViewHelpers');
				// setting all namespaces in advance, globally, before template parsing:
				$resolver->setNamespaces(array(
				  'f' => array(
				    'TYPO3Fluid\\Fluid\\ViewHelpers', 'TYPO3\\CMS\\Fluid\\ViewHelpers',
				    'My\\Extension\\ViewHelpers'
				  ),
				  'vhs' => array(
				    'FluidTYPO3\\Vhs\\ViewHelpers', 'My\\Extension\\ViewHelpers'
				  ),
				  'news' => array(
				    'GeorgRinger\\News\\ViewHelpers',
				  );
				));
			\end{lstlisting}
	\end{itemize}

\end{frame}

% ------------------------------------------------------------------------------
% LTXE-SLIDE-START
% LTXE-SLIDE-UID:		cbc1b52f-68132620-c5f08dbb-14723e1f
% LTXE-SLIDE-TITLE:		#78477: FlashMessageViewHelper refactored
% ------------------------------------------------------------------------------

\begin{frame}[fragile]
	\frametitle{Extbase \& Fluid}
	\framesubtitle{FlashMessageViewHelper}

	% decrease font size for code listing
	\lstset{basicstyle=\tiny\ttfamily}

	\begin{itemize}
		\item The \texttt{FlashMessageViewHelper} has been refactored and no longer inherits
			from the \texttt{TagBasedViewHelper}

		\item Remove the tag specific attributes and style the default output.
			If you need custom output use the possibility to render FlashMessages yourself, for example:

			\begin{lstlisting}
				<f:flashMessages as="flashMessages">
				  <dl class="messages">
				    <f:for each="{flashMessages}" as="flashMessage">
				      <dt>CODE!! {flashMessage.code}</dt>
				      <dd>MESSAGE:: {flashMessage.message}</dd>
				    </f:for>
				  </dl>
				</f:flashMessages>
			\end{lstlisting}

	\end{itemize}

\end{frame}

% ------------------------------------------------------------------------------
% LTXE-SLIDE-START
% LTXE-SLIDE-UID:		3e41fa42-1d526584-5b7d3a07-c8b347ab
% LTXE-SLIDE-TITLE:		#69863: ViewHelpers can accept arbitrary arguments (1)
% ------------------------------------------------------------------------------

\begin{frame}[fragile]
	\frametitle{Extbase \& Fluid}
	\framesubtitle{ViewHelpers can accept arbitrary arguments (1)}

	\begin{itemize}

		\item This feature allows your ViewHelper class to receive any number of
			additional arguments using any names you desire

		\item It works by separating the arguments that are passed to each ViewHelper
			into two groups: those that are declared using \texttt{registerArgument}
			(or render method arguments), and those that are not

		\item Those that are not declared, are passed to a special function
			\texttt{handleAdditionalArguments}
			on the ViewHelper class, which in the default implementation throws an
			error if additional arguments exist

	\end{itemize}

\end{frame}

% ------------------------------------------------------------------------------
% LTXE-SLIDE-START
% LTXE-SLIDE-UID:		2462a8ce-0a82a7f0-1b97a84f-02f0a4c2
% LTXE-SLIDE-TITLE:		#69863: ViewHelpers can accept arbitrary arguments (2)
% ------------------------------------------------------------------------------

\begin{frame}[fragile]
	\frametitle{Extbase \& Fluid}
	\framesubtitle{ViewHelpers can accept arbitrary arguments (2)}

	\begin{itemize}

		\item By overriding this method in your ViewHelper, you can change if and
			when the ViewHelper should throw an error on receiving unregistered
			arguments

		\item This feature is also the one allowing TagBasedViewHelpers to freely
			accept arbitrary \texttt{data-} prefixed arguments without failing

		\item on TagBasedViewHelpers, the \texttt{handleAdditionalArguments} method
			simply adds new attributes to the tag that gets generated and throws an
			error if any additional arguments which are neither registered nor
			prefixed with \texttt{data-} are given.

	\end{itemize}

\end{frame}

% ------------------------------------------------------------------------------
% LTXE-SLIDE-START
% LTXE-SLIDE-UID:		b5a66715-5681025e-cfb46644-9b9bad55
% LTXE-SLIDE-TITLE:		#67236: Added "allowedTags" argument to f:format.stripTags ViewHelper
% ------------------------------------------------------------------------------

\begin{frame}[fragile]
	\frametitle{Extbase \& Fluid}
	\framesubtitle{Argument "allowedTags" for f:format.stripTags}

	\begin{itemize}

		\item The argument \texttt{allowedTags} containing a list of HTML tags
			which will not be stripped can now be used on \texttt{f:format.stripTags}

		\item Tag list syntax is identical to second parameter of PHP function
			\texttt{strip\_tags} (see: \url{http://php.net/strip_tags})

	\end{itemize}

\end{frame}

% ------------------------------------------------------------------------------
% LTXE-SLIDE-START
% LTXE-SLIDE-UID:		a71d77d1-6a419c8d-0fbb35d0-e1ce57e4
% LTXE-SLIDE-TITLE:		Default Content Element Changed for Fluid Styled Content
% LTXE-SLIDE-REFERENCE:	!Breaking: #79622 - Default content element changed for Fluid Styled Content
% ------------------------------------------------------------------------------

\begin{frame}[fragile]
	\frametitle{Extbase \& Fluid}
	\framesubtitle{Default Content Element Changed for Fluid Styled Content}

	% decrease font size for code listing
	\lstset{basicstyle=\tiny\ttfamily}

	\begin{itemize}
		\item The default content element has been streamlined with CSS Styled Content and has been changed to "Text"
		\item To restore the configuration you need to set the default content element manually to your preferred choice.
			You can do this by simply overriding the configuration again in your
			\texttt{Configuration/TCA/Overrides/tt\_content.php} file.

			\begin{lstlisting}
				$GLOBALS['TCA']['tt_content']['columns']['CType']['config']['default'] = 'textmedia';
				$GLOBALS['TCA']['tt_content']['columns']['CType']['config']['default'] = 'header';
			\end{lstlisting}

	\end{itemize}

\end{frame}

% ------------------------------------------------------------------------------
% LTXE-SLIDE-START
% LTXE-SLIDE-UID:		b400c3b8-3e402480-cd7d6ff3-96cd93aa
% LTXE-SLIDE-TITLE:		#76085: Add fluid debug information to admin panel
% ------------------------------------------------------------------------------
\begin{frame}[fragile]
	\frametitle{Extbase \& Fluid}
	\framesubtitle{Fluid Debugging Using the Admin Panel}

	\begin{itemize}
		\item Admin Panel features a new setting to debug Fluid output:\newline
			\textbf{Preview -> Show fluid debug output}

		\item If enabled, the following details are shown in the frontend:

			\begin{itemize}
				\item path to the template file of a partial
				\item name of a section
			\end{itemize}

		\item This feature enables integrators to easily locate the correct template and section

	\end{itemize}

\end{frame}

% ------------------------------------------------------------------------------

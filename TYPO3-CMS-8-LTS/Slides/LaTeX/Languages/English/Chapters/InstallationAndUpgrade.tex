% ------------------------------------------------------------------------------
% TYPO3 CMS 8 LTS - What's New - Chapter "Installation and Upgrade" (English Version)
%
% @author	Michael Schams <schams.net>
% @license	Creative Commons BY-NC-SA 3.0
% @link		http://typo3.org/download/release-notes/whats-new/
% @language	English
% ------------------------------------------------------------------------------
% LTXE-CHAPTER-UID:		6dc8a329-7fd5f485-7c9c0ef0-dd2c6549
% LTXE-CHAPTER-NAME:	Installation and Upgrade
% ------------------------------------------------------------------------------

\section{Installation and Upgrade}
\begin{frame}[fragile]
	\frametitle{Installation and Upgrade}

	\begin{center}\huge{\color{typo3darkgrey}\textbf{Installation and Upgrade}}\end{center}
	\begin{center}\large{\textit{It's time to check out TYPO3 v8 LTS}}\end{center}

\end{frame}

% ------------------------------------------------------------------------------
% LTXE-SLIDE-START
% LTXE-SLIDE-UID:		2924aa0e-f575ecd8-c637f4c2-6b39b1c9
% LTXE-SLIDE-TITLE:		Installation
% ------------------------------------------------------------------------------
\begin{frame}[fragile]
	\frametitle{Installation and Upgrade}
	\framesubtitle{Traditional Installation Method}

	\begin{itemize}
		\item Official \textit{traditional} installation procedure under Linux/Mac OS X\newline
			(DocumentRoot for example \texttt{/var/www/site/htdocs}):
		\begin{lstlisting}
			$ cd /var/www/site/
			$ wget --content-disposition get.typo3.org/8.7
			$ tar xzf typo3_src-8.7.0.tar.gz
			$ cd htdocs
			$ ln -s ../typo3_src-8.7.0 typo3_src
			$ ln -s typo3_src/index.php
			$ ln -s typo3_src/typo3
			$ touch FIRST_INSTALL
		\end{lstlisting}

		\item Symbolic links under Microsoft Windows:

			\begin{itemize}
				\item Use \texttt{junction} under Windows XP/2000
				\item Use \texttt{mklink} under Windows Vista and Windows 7
			\end{itemize}

	\end{itemize}
\end{frame}

% ------------------------------------------------------------------------------
% LTXE-SLIDE-START
% LTXE-SLIDE-UID:		ed06d060-dc1f7c61-00bacc1f-6a2ccd9f
% LTXE-SLIDE-TITLE:		Installation using composer
% ------------------------------------------------------------------------------
\begin{frame}[fragile]
	\frametitle{Installation and Upgrade}
	\framesubtitle{Installation Using \texttt{composer}}

	\begin{itemize}
		\item Installation using \textit{composer} under Linux/Mac OS X

			\begin{lstlisting}
				$ cd /var/www/site/
				$ composer create-project typo3/cms-base-distribution
			\end{lstlisting}

		\item Alternatively, create your custom \texttt{composer.json} file and run:

			\begin{lstlisting}
				$ composer install
			\end{lstlisting}

			An example \texttt{composer.json} file can be downloaded at:\newline
			\small
				\href{https://git.typo3.org/TYPO3CMS/Distributions/Base.git/blob/HEAD:/composer.json}{git.typo3.org/TYPO3CMS/Distributions/Base.git/blob/HEAD:/composer.json}
			\normalsize


	\end{itemize}
\end{frame}

% ------------------------------------------------------------------------------
% LTXE-SLIDE-START
% LTXE-SLIDE-UID:		061ecffe-6aadad2d-6e64a67a-3c50a5cf
% LTXE-SLIDE-TITLE:		Upgrade to TYPO3 CMS 7
% ------------------------------------------------------------------------------
\begin{frame}[fragile]
	\frametitle{Installation and Upgrade}
	\framesubtitle{Upgrade to TYPO3 v8 LTS}

	\begin{itemize}
		\item Upgrades only possible from TYPO3 v7 LTS
		\item TYPO3 CMS < 7.6 LTS should be updated to TYPO3 version 7.6 first
	\end{itemize}

	\begin{itemize}

		\item Upgrade instructions:\newline
			\smaller\url{http://wiki.typo3.org/Upgrade#Upgrading_to_8.7}\normalsize
		\item Official TYPO3 guide "TYPO3 Installation and Upgrading":
			\smaller\url{http://docs.typo3.org/typo3cms/InstallationGuide}\normalsize
		\item General approach:
			\begin{itemize}
				\item Check minimum system requirements \small(PHP, MySQL, etc.)
				\item Review \textbf{deprecation\_*.log} in old TYPO3 instance
				\item Update all extensions to the latest version
				\item Deploy new sources and run Install Tool -> Upgrade Wizard
				\item Review startup module for backend users (optionally)
			\end{itemize}
	\end{itemize}

\end{frame}

% ------------------------------------------------------------------------------

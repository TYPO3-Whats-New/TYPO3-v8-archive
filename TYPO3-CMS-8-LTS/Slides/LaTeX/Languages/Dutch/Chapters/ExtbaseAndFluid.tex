% ------------------------------------------------------------------------------
% TYPO3 CMS 8 LTS - What's New - Chapter "Extbase & Fluid" (Dutch Version)
%
% @author	Michael Schams <schams.net>
% @license	Creative Commons BY-NC-SA 3.0
% @link		http://typo3.org/download/release-notes/whats-new/
% @language	English
% ------------------------------------------------------------------------------
% LTXE-CHAPTER-UID:		846d40ce-66fdc2ec-750dcf95-33ce93e0
% LTXE-CHAPTER-NAME:	Extbase & Fluid
% ------------------------------------------------------------------------------

\section{Extbase \& Fluid}
\begin{frame}[fragile]
	\frametitle{Extbase \& Fluid}

	\begin{center}\huge{\color{typo3darkgrey}\textbf{Extbase \& Fluid}}\end{center}
	\begin{center}\large{\textit{}}\end{center}

\end{frame}

% ------------------------------------------------------------------------------
% LTXE-SLIDE-START
% LTXE-SLIDE-UID:		a66ddb88-15975946-cde8cd82-45a8648c
% LTXE-SLIDE-ORIGIN:	9413e7f2-dd08f22e-dd21ee20-d7a1b0bf English
% LTXE-SLIDE-TITLE:		#69863: Standalone revised Fluid
% ------------------------------------------------------------------------------

\begin{frame}[fragile]
	\frametitle{Extbase \& Fluid}
	\framesubtitle{Standalone Fluid}

	% decrease font size for code listin
	\lstset{basicstyle=\tiny\ttfamily}

	\begin{itemize}

		\item De Fluid render-engine van TYPO3 CMS is vervangen door de standalone
			versie van Fluid die nu als een composer-afhankelijkheid opgenomen is

		\item De oude Fluid-extensie is omgezet in een zogenaamde \textit{Fluid adapter}
			waarmee TYPO3 CMS standalone Fluid kan gebruiken

		\item Nieuwe features/mogelijkheden zijn in bijna elk onderdeel van Fluid toegevoegd
			\small
				(Zie \href{https://docs.typo3.org/typo3cms/extensions/core/8-dev/Changelog/8.0/Feature-69863-UseNewStandaloneFluidAsComposerDependency.html}{docs.typo3.org}
				voor meer details)
			\normalsize

		\item Meest belangrijk: diverse onderdelen van Fluid die compleet intern waren en die
			onmogelijk overschreven konden worden, zijn nu simpel te vervangen en
			voorzien van een API

		\item Standalone Fluid verbetert ook de performance van de backend

	\end{itemize}

\end{frame}

% ------------------------------------------------------------------------------
% LTXE-SLIDE-START
% LTXE-SLIDE-UID:		a014e4e3-076b1978-5046e40f-6813bc28
% LTXE-SLIDE-ORIGIN:	32daa81f-7e805b84-6f829395-94677911 English
% LTXE-SLIDE-TITLE:		#69863: Rendering Context (1)
% ------------------------------------------------------------------------------

\begin{frame}[fragile]
	\frametitle{Extbase \& Fluid}
	\framesubtitle{RenderingContext (1)}

	% decrease font size for code listing
	\lstset{basicstyle=\tiny\ttfamily}

	\begin{itemize}

		\item Het belangrijkste onderdeel van de nieuwe API is de RenderingContext

		\item Voorheen was RenderingContext intern gebruikt door Fluid. Dit is nu uitgebreid
			en verantwoordelijk voor een nieuwe Fluid feature:
			\textbf{implementation provisioning}

		\item Hiermee kunnen ontwikkelaars een serie klassen wijzigen die Fluid gebruikt voor
			het parsen, oplossen, cachen, etc.

		\item Dit is te bereiken door ofwel een eigen RenderingContext of
			het bewerken van de standaard RenderingContext met publiek beschikbare functies.

	\end{itemize}

\end{frame}

% ------------------------------------------------------------------------------
% LTXE-SLIDE-START
% LTXE-SLIDE-UID:		e5501076-ea41ad8d-bf24e8a7-15bf9905
% LTXE-SLIDE-ORIGIN:	3d44632f-c87d4adf-3328f596-2e5744bc English
% LTXE-SLIDE-TITLE:		#69863: Rendering Context (2)
% ------------------------------------------------------------------------------

\begin{frame}[fragile]
	\frametitle{Extbase \& Fluid}
	\framesubtitle{Rendering Context (2)}

	% decrease font size for code listing
	%\lstset{basicstyle=\tiny\ttfamily}
	\lstset{basicstyle=\smaller\ttfamily}

	\begin{itemize}

		\item De features op de volgende pagina's worden aangestuurd door het wijzigen van de
			RenderingContext. Standaard zijn ze niet ingeschakeld, maar door het aanroepen van
			een simpele methode (via de View instantie) kunnen ze ingeschakeld worden:

		\begin{lstlisting}
			$view->getRenderingContext()->setLegacyMode(false);
		\end{lstlisting}

	\end{itemize}

\end{frame}

% ------------------------------------------------------------------------------
% LTXE-SLIDE-START
% LTXE-SLIDE-UID:		ac421a74-b4ac0554-9ded9f8e-16f532bd
% LTXE-SLIDE-ORIGIN:	5f486f77-38e6694e-2fad0d6b-7b260ad1 English
% LTXE-SLIDE-TITLE:		#69863: ExpressionNodes (1)
% ------------------------------------------------------------------------------

\begin{frame}[fragile]
	\frametitle{Extbase \& Fluid}
	\framesubtitle{ExpressionNodes (1)}

	% decrease font size for code listing
	%\lstset{basicstyle=\tiny\ttfamily}
	\lstset{basicstyle=\smaller\ttfamily}

	\begin{itemize}

		\item ExpressionNodes zijn een nieuw soort syntaxstructuren van Fluid die alle
			een eigenschap gemeen hebben: ze werken alleen binnen accolades

		\begin{lstlisting}
			$view->getRenderingContext()->setExpressionNodeTypes(array(
			   'Class\Number\One',
			   'Class\Number\Two'
			));
		\end{lstlisting}

		\item Ontwikkelaars kunnen hun eigen soorten ExpressionNode toevoegen

		\item Elk bestaat uit een patroon en methodes bepaald door een interface om de overeenkomsten
			af te handelen

		\item Een bestaand soort ExpressionNode kan gebruikt worden als referentie

	\end{itemize}

\end{frame}

% ------------------------------------------------------------------------------
% LTXE-SLIDE-START
% LTXE-SLIDE-UID:		50e1bc75-e4e94456-fbf3ebe0-79da03ff
% LTXE-SLIDE-ORIGIN:	090cec75-2751b839-6ad8eebc-0adfe66a English
% LTXE-SLIDE-TITLE:		#69863: ExpressionNodes (2)
% ------------------------------------------------------------------------------

\begin{frame}[fragile]
	\frametitle{Extbase \& Fluid}
	\framesubtitle{ExpressionNodes (2)}

	ExpressionNodeTypes maken nieuwe syntaxen mogelijk:

	\begin{itemize}

		\item \textbf{CastingExpressionNode}\newline
			\small
				maakt het omzetten van variabelen naar bepaalde types mogelijk; bijvoorbeeld
				om zeker te zijn van een geheel getal of een booleaan. Het wordt gebruikt met een
				\texttt{as} sleutelwoord:
				\texttt{\{myStringVariable as boolean\}} of
				\texttt{\{myBooleanVariable as integer\}} enzovoort.
				Pogingen om een variabele om te zetten naar een incompatibel type resulteren in een
				standaard Fluid-foutmelding.
			\normalsize

		\item \textbf{MathExpressionNode}\newline
			\small
				maakt basisberekeningen met variabelen mogelijk, bijvoorbeeld
				\texttt{\{myNumber + 1\}}, \texttt{\{myPercent / 100\}} of
				\texttt{\{myNumber * 100\}} enzovoort.
				Een onmogelijke uitdrukking geeft geen uitvoer.
			\normalsize

	\end{itemize}

\end{frame}

% ------------------------------------------------------------------------------
% LTXE-SLIDE-START
% LTXE-SLIDE-UID:		eceb3f8f-9014ef96-cea3ba48-27daeb7a
% LTXE-SLIDE-ORIGIN:	c8935f68-0905db52-11932973-63a85a00 English
% LTXE-SLIDE-TITLE:		#69863: ExpressionNodes (3)
% ------------------------------------------------------------------------------

\begin{frame}[fragile]
	\frametitle{Extbase \& Fluid}
	\framesubtitle{ExpressionNodes (3)}

	ExpressionNodeTypes maken nieuwe syntax mogelijk:

	\begin{itemize}

		\item \textbf{TernaryExpressionNode}\newline
			\small
				maakt een drieweg-voorwaarde mogelijk die werkt met variabelen.
				Typisch gebruik: "als deze variabele dan gebruik die variabele anders gebruik
				andere variabele". Het wordt gebruikt als:\newline
				\texttt{\{myToggleVariable ? myThenVariable : myElseVariable\}}\newline
				Let op: ondersteunt geen geneste uitdrukkingen, inline ViewHelper
				syntax of vergelijkbaar erbinnen. Het moet alleen met standaard variabelen
				gebruikt worden.
			\normalsize

	\end{itemize}

\end{frame}

% ------------------------------------------------------------------------------
% LTXE-SLIDE-START
% LTXE-SLIDE-UID:		2c14c4ca-962bcc0d-b3f2e44c-202a1bdd
% LTXE-SLIDE-ORIGIN:	153b9b53-d7f97aef-dc1ba8ab-fc2a763e English
% LTXE-SLIDE-TITLE:		#69863: Namespaces are extensible (1)
% ------------------------------------------------------------------------------

\begin{frame}[fragile]
	\frametitle{Extbase \& Fluid}
	\framesubtitle{Namespaces zijn uitbreidbaar (1)}

	\begin{itemize}

		\item Fluid maakt het mogelijk om een namespace alias (bijvoorbeeld \texttt{f:}) uit te
			breiden door door extra PHP namespaces eraan toe te voegen

		\item PHP namespaces worden ook gecontroleerd op de aanwezigheid van ViewHelper-klassen

		\item Dit betekent ook dat ontwikkelaars specifieke ViewHelpers kunnen overschrijven
			met eigen versies en eigen ViewHelpers laten aanroepen als de \texttt{f:} namespace
			gebruikt wordt

		\item Deze wijziging betekent ook dat namespaces niet langer eenduidig zijn.\newline
			Bij het gebruik van \texttt{
				\{namespace f=My\textbackslash
				Extension\textbackslash
				ViewHelpers\textbackslash\}}
				verschijnt er niet langer de foutmelding "namespace al geregistreerd".
				Fluid zal de PHP namespace toevoegen en ook daar zoeken naar ViewHelpers.

	\end{itemize}

\end{frame}

% ------------------------------------------------------------------------------
% LTXE-SLIDE-START
% LTXE-SLIDE-UID:		52be6b70-5ddf5c29-2ce070f9-b8c0c170
% LTXE-SLIDE-ORIGIN:	a2f0e799-fe523ffb-57634576-654a8efe English
% LTXE-SLIDE-TITLE:		#69863: Namespaces are extensible (2)
% ------------------------------------------------------------------------------

\begin{frame}[fragile]
	\frametitle{Extbase \& Fluid}
	\framesubtitle{Namespaces zijn uitbreidbaar (2)}

	\begin{itemize}

		\item Extra namespaces worden van onder af nagelopen waardoor extra namespaces ViewHelper-klassen
			kunnen overschrijven door ze in dezelfde scope te plaatsen.

		\item Voorbeeld: \texttt{f:format.nl2br} kan worden overschreven door
			\texttt{
				My\textbackslash
				Extension\textbackslash
				ViewHelpers\textbackslash
				Format\textbackslash
				Nl2brViewHelper},

				met de namespace-registratie van de vorige pagina

	\end{itemize}

\end{frame}


% ------------------------------------------------------------------------------
% LTXE-SLIDE-START
% LTXE-SLIDE-UID:		e9863444-d9fb4651-c3f812ae-eea122d7
% LTXE-SLIDE-ORIGIN:	7c36b3a7-33860495-4efb65ef-f09ae14e English
% LTXE-SLIDE-TITLE:		#69863: Complex conditional statements
% ------------------------------------------------------------------------------

\begin{frame}[fragile]
	\frametitle{Extbase \& Fluid}
	\framesubtitle{Complexe voorwaardelijke statements}

	% decrease font size for code listing
	\lstset{basicstyle=\tiny\ttfamily}

	\begin{itemize}

		\item Fluid ondersteunt elke mate van complexe voorwaardelijke statements zowel
			genest als gegroepeerd:

			\begin{lstlisting}
				<f:if condition="({variableOne} && {variableTwo}) || {variableThree} || {variableFour}">
				  // Uitgevoerd als zowel variabele een en twee 'waar' opleveren,
				  // of variabele drie of vier dat opleveren.
				</f:if>
			\end{lstlisting}

		\item Daarbij kent \texttt{f:else} nu een "elseif"-achtige functie:

			\begin{lstlisting}
				<f:if condition="{variableOne}">
				  <f:then>Doe dit</f:then>
				  <f:else if="{variableTwo}">
				    Doe dit als variabele twee 'waar' oplevert
				  </f:else>
				  <f:else if="{variableThree}">
				    Doe dit als variabele twee 'waar' oplevert
				  </f:else>
				  <f:else>
				    Of doe dit als niets waar is
				  </f:else>
				</f:if>
			\end{lstlisting}

	\end{itemize}

\end{frame}

% ------------------------------------------------------------------------------
% LTXE-SLIDE-START
% LTXE-SLIDE-UID:		f117349c-09b31069-09b279c6-09046ea6
% LTXE-SLIDE-ORIGIN:	d54cf918-48141072-8184ab81-cecc8e39 English
% LTXE-SLIDE-TITLE:		#69863: Dynamic variable name parts (1)
% ------------------------------------------------------------------------------

\begin{frame}[fragile]
	\frametitle{Extbase \& Fluid}
	\framesubtitle{Dynamische delen van namen van variabelen(1)}

	% decrease font size for code listing
	\lstset{basicstyle=\tiny\ttfamily}

	\begin{itemize}

		\item Een andere, nieuwe feature, ook backwards compatibel, is de mogelijkheid
			om sub-variabele referenties te gebruiken bij variabelen.
			Neem bijvoorbeeld de volgende Fluid variabelen-array:

			\begin{lstlisting}
				$mykey = 'foo'; // of 'bar', vanuit elke bron
				$view->assign('data', ['foo' => 1, 'bar' => 2]);
				$view->assign('key', $mykey);
			\end{lstlisting}

		\item Met het volgende Fluid-sjabloon:

			\begin{lstlisting}
				U koos: {data.{key}}.
				(uitvoer: "1" als de index "foo" is, "2" als de index "bar" is)
			\end{lstlisting}

	\end{itemize}

\end{frame}

% ------------------------------------------------------------------------------
% LTXE-SLIDE-START
% LTXE-SLIDE-UID:		85182399-b9763221-07a92897-3aa105de
% LTXE-SLIDE-ORIGIN:	20385d89-9c0de3ac-9ef70ba6-310fdec3 English
% LTXE-SLIDE-TITLE:		#69863: Dynamic variable name parts (2)
% ------------------------------------------------------------------------------

\begin{frame}[fragile]
	\frametitle{Extbase \& Fluid}
	\framesubtitle{Dynamische delen van namen van variabelen (2)}

	% decrease font size for code listing
	\lstset{basicstyle=\tiny\ttfamily}

	\begin{itemize}

		\item Dezelfde benadering kan ook gebruikt worden om dynamisch delen van de
		 	naam van een stringvariabele te genereren:

			\begin{lstlisting}
				$mijndynamischdeel = 'Eerste'; // of 'Tweede', vanuit elke bron
				$view->assign('mijnEersteVariabele', 1);
				$view->assign('mijnTweedeVariabele', 2);
				$view->assign('welke', $mijndynamischdeel);
			\end{lstlisting}

		\item Met dit Fluid-sjabloon:

			\begin{lstlisting}
				U koos: {mijn{welke}Variabele}.
				(uitvoer: "1" als 'welke' "Eerste" is, of "2" als 'welke' "Tweede" is)
			\end{lstlisting}

	\end{itemize}

\end{frame}

% ------------------------------------------------------------------------------
% LTXE-SLIDE-START
% LTXE-SLIDE-UID:		20a636c3-825d93b4-a7831932-37e51fb4
% LTXE-SLIDE-ORIGIN:	15907f4f-d554b015-03be1fdc-4dc0fb25 English
% LTXE-SLIDE-TITLE:		#69863: New ViewHelpers
% ------------------------------------------------------------------------------

\begin{frame}[fragile]
	\frametitle{Extbase \& Fluid}
	\framesubtitle{Nieuwe ViewHelpers}

	% decrease font size for code listing
	\lstset{basicstyle=\tiny\ttfamily}

	\begin{itemize}
		\item Er zijn een paar nieuwe ViewHelpers toegevoegd als onderdeel van standalone Fluid
			en daardoor ook beschikbaar in TYPO3:

			\begin{itemize}

				\item \textbf{\texttt{f:or}}\newline
					Dit is een korte vorm om (geschakelde) voorwaarden te schrijven.
					Het ondersteunt de volgende syntax die elke variabele naloopt en
					de eerste die niet leeg is als uitvoer gebruikt.

					\begin{lstlisting}
						{variableOne -> f:or(alternative: variableTwo) -> f:or(alternative: variableThree)}
					\end{lstlisting}

				\item \textbf{\texttt{f:spaceless}}\newline
					Dit kan gebruikt worden als tag rond sjablooncode om
					overtollige witruimte te verwijderen evenals lege regels
					die bijvoorbeeld ontstaan door het inspringen van ViewHelpers.

			\end{itemize}

	\end{itemize}

\end{frame}

% ------------------------------------------------------------------------------
% LTXE-SLIDE-START
% LTXE-SLIDE-UID:		15eb2206-efd4c8d5-dc95fb11-ee69034f
% LTXE-SLIDE-ORIGIN:	06a9e027-1c0e09af-d41d1c4c-c0f07273 English
% LTXE-SLIDE-TITLE:		#76531: Add IconForRecordViewHelper
% ------------------------------------------------------------------------------
\begin{frame}[fragile]
	\frametitle{Extbase \& Fluid}
	\framesubtitle{IconForRecordViewHelper toegevoegd}

	% decrease font size for code listin
	\lstset{basicstyle=\tiny\ttfamily}

	\begin{itemize}

		\item Er is een nieuwe ViewHelper toegevoegd om iconen voor records af te beelden

			\begin{lstlisting}
				<core:iconForRecord table="sys_template" row="{templateRecord}" ></core:iconForRecord>

				// uitvoer:
				<span class="t3js-icon icon icon-size-small icon-state-default icon-mimetypes-x-content-template"
				  data-identifier="mimetypes-x-content-template">
				  <span class="icon-markup">
				    <img src="/typo3/sysext/core/Resources/Public/Icons/T3Icons/mimetypes/mimetypes-x-content-template.svg" width="16" height="16">
				  </span>
				</span>
			\end{lstlisting}

	\end{itemize}

\end{frame}

% ------------------------------------------------------------------------------
% LTXE-SLIDE-START
% LTXE-SLIDE-UID:		2cd9fe63-224f0827-01b5b305-f7df6ddc
% LTXE-SLIDE-ORIGIN:	50143a1e-53a17c8c-83ffe7ea-5e88cba0 English
% LTXE-SLIDE-TITLE:		#29399: OptionViewHelper and OptgroupViewHelper
% ------------------------------------------------------------------------------

\begin{frame}[fragile]
	\frametitle{Extbase \& Fluid}
	\framesubtitle{Inhoud voor ViewHelper \texttt{f:form.select}}

	% decrease font size for code listing
	\lstset{basicstyle=\tiny\ttfamily}

	\begin{itemize}
		\item Er zijn twee nieuwe ViewHelpers voor het handmatig definiëren van alle opties en
		 	optgroups voor de \texttt{f:form.select} als inhoud van een select-veld

			\begin{itemize}
				\item \texttt{OptionViewHelper}
				\item \texttt{OptgroupViewHelper}
			\end{itemize}

		\item Voorbeeld:

			\begin{lstlisting}
				<f:form.select name="myproperty">
				  <f:form.select.option value="1">Optie eem</f:form.select.option>
				  <f:form.select.option value="2">Optie twee</f:form.select.option>
				  <f:form.select.optgroup>
				    <f:form.select.option value="3">Gegroepeerde optie een</f:form.select.option>
				    <f:form.select.option value="4">Gegroupeerde optie twee</f:form.select.option>
				  </f:form.select.optgroup>
				</f:form.select>
			\end{lstlisting}

		\end{itemize}

\end{frame}

% ------------------------------------------------------------------------------
% LTXE-SLIDE-START
% LTXE-SLIDE-UID:		9498160d-808ce251-529fb19a-3fc67761
% LTXE-SLIDE-ORIGIN:	d36b9935-fc7d9e60-742a3f42-ef15e856 English
% LTXE-SLIDE-TITLE:		#78415: Global Fluid ViewHelper Namespace
% ------------------------------------------------------------------------------
\begin{frame}[fragile]
	\frametitle{Extbase \& Fluid}
	\framesubtitle{Globale Fluid ViewHelper Namespace}

	\begin{itemize}
		\item Globale Fluid ViewHelper namespaces zijn nu instelbaar:\newline
			\smaller
				\texttt{\$GLOBALS['TYPO3\_CONF\_VARS']['SYS']['fluid']['namespaces']}
			\normalsize
		\item Hiermee kunnen de namespaces bewerkt worden als onderdeel van de configuratie van een site
		\item Voordelen:

			\begin{itemize}
				\item ViewHelper pakketten van derden kunnen de globale Fluid namespace \texttt{f:} aanpassen
				\item ViewHelper pakketten van derden kunnen nieuwe globale namespaces registreren
				\item Ontwikkelaars van sjablonen kunnen die globale namespaces gebruiken zonder ze eerst te importeren
					en kunnen ze gebruiken in alle Fluid-sjablonen ongeacht de context
			\end{itemize}

	\end{itemize}

\end{frame}

% ------------------------------------------------------------------------------
% LTXE-SLIDE-START
% LTXE-SLIDE-UID:		abfba9c9-c82f760a-d3822ce6-e36c7079
% LTXE-SLIDE-ORIGIN:	9fed5ece-ae03648f-c3db2f41-42312f7f English
% LTXE-SLIDE-TITLE:		#69863: ViewHelper namespaces can be extended also from PHP
% ------------------------------------------------------------------------------

\begin{frame}[fragile]
	\frametitle{Extbase \& Fluid}
	\framesubtitle{ViewHelper-namespaces kunnen ook vanuit PHP uitgebreid worden}

	% decrease font size for code listing
	\lstset{basicstyle=\tiny\ttfamily}

	\begin{itemize}

		\item Via de ViewHelperResolver van de RenderingContext kunnen ontwikkelaars
			de ViewHelper-namespaceinsluitingen wijzigen op globale (lees: per
			View-instantie) schaal:

			\begin{lstlisting}
				$resolver = $view->getRenderingContext()->getViewHelperResolver();
				// equivalent van registratie van namespace in sjablo(o)n(en):
				$resolver->registerNamespace('news', 'GeorgRinger\News\ViewHelpers');
				// extra PHP namespace toevoegen om te controleren bij het opzoeken van ViewHelpers:
				$resolver->extendNamespace('f', 'My\Extension\ViewHelpers');
				// alle namespaces vooraf, globaal, voor het verwerken van de sjabloon instellen:
				$resolver->setNamespaces(array(
				  'f' => array(
				    'TYPO3Fluid\\Fluid\\ViewHelpers', 'TYPO3\\CMS\\Fluid\\ViewHelpers',
				    'My\\Extension\\ViewHelpers'
				  ),
				  'vhs' => array(
				    'FluidTYPO3\\Vhs\\ViewHelpers', 'My\\Extension\\ViewHelpers'
				  ),
				  'news' => array(
				    'GeorgRinger\\News\\ViewHelpers',
				  );
				));
			\end{lstlisting}
	\end{itemize}

\end{frame}

% ------------------------------------------------------------------------------
% LTXE-SLIDE-START
% LTXE-SLIDE-UID:		e77f943c-8cf97b1e-088b9c2f-3ab7fe82
% LTXE-SLIDE-ORIGIN:	cbc1b52f-68132620-c5f08dbb-14723e1f English
% LTXE-SLIDE-TITLE:		#78477: FlashMessageViewHelper refactored
% ------------------------------------------------------------------------------

\begin{frame}[fragile]
	\frametitle{Extbase \& Fluid}
	\framesubtitle{FlashMessageViewHelper}

	% decrease font size for code listing
	\lstset{basicstyle=\tiny\ttfamily}

	\begin{itemize}
		\item De \texttt{FlashMessageViewHelper} is aangepast en overerft niet langer van de \texttt{TagBasedViewHelper}

		\item Verwijder de tag-specifieke attributen en stijl de standaard uitvoer. Als maatwerk uitvoer nodig is, gebruik
			dan de mogelijkheid om zelf FlashMessages af te beelden. Voorbeeld:

			\begin{lstlisting}
				<f:flashMessages as="flashMessages">
				  <dl class="messages">
				    <f:for each="{flashMessages}" as="flashMessage">
				      <dt>CODE!! {flashMessage.code}</dt>
				      <dd>MESSAGE:: {flashMessage.message}</dd>
				    </f:for>
				  </dl>
				</f:flashMessages>
			\end{lstlisting}

	\end{itemize}

\end{frame}

% ------------------------------------------------------------------------------
% LTXE-SLIDE-START
% LTXE-SLIDE-UID:		8277357a-25fb913d-4f02b92c-0715a0c4
% LTXE-SLIDE-ORIGIN:	3e41fa42-1d526584-5b7d3a07-c8b347ab English
% LTXE-SLIDE-TITLE:		#69863: ViewHelpers can accept arbitrary arguments (1)
% ------------------------------------------------------------------------------

\begin{frame}[fragile]
	\frametitle{Extbase \& Fluid}
	\framesubtitle{ViewHelpers kunnen allerlei argumenten accepteren (1)}

	\begin{itemize}

		\item Hiermee kan een ViewHelper-klasse elk aantal extra argumenten ontvangen
			met elke gewenste naam

		\item Het werkt door het scheiden van de argumenten die doorgegeven worden naar
			elke ViewHelper in twee groepen: diegene die gedeclareerd zijn via \texttt{registerArgument}
			(of de argumenten van de rendermethode) en diegene die dat niet zijn

		\item De niet-gedeclareerde argumenten worden doorgegeven aan de speciale functie
			\texttt{handleAdditionalArguments}
			in de ViewHelper-klasse, die in de standaardimplementatie een foutmelding geeft als
			er extra argumenten zijn

	\end{itemize}

\end{frame}

% ------------------------------------------------------------------------------
% LTXE-SLIDE-START
% LTXE-SLIDE-UID:		4e5b615c-a7088304-80ca4626-efa34576
% LTXE-SLIDE-ORIGIN:	2462a8ce-0a82a7f0-1b97a84f-02f0a4c2 English
% LTXE-SLIDE-TITLE:		#69863: ViewHelpers can accept arbitrary arguments (2)
% ------------------------------------------------------------------------------

\begin{frame}[fragile]
	\frametitle{Extbase \& Fluid}
	\framesubtitle{ViewHelpers kunnen allerlei argumenten accepteren (2)}

	\begin{itemize}

		\item Door het overschrijven van deze methode in de ViewHelper, is te wijzigen
		 	of en wanneer er een foutmelding moet komen als er ongeregistreerde
		 	argumenten zijn

		\item Hiermee kunnen TagBasedViewHelpers ook willekeurige argumenten met een
			\texttt{data-}prefix accepteren zonder foutmelding

		\item bij TagBasedViewHelpers voegt de methode \texttt{handleAdditionalArguments}
			eenvoudigweg de nieuwe argumenten toe aan de tag die opgebouwd wordt en geeft
			een foutmelding als er extra argumenten zijn die noch geregistreerd zijn,
			noch een \texttt{data-}prefix hebben.

	\end{itemize}

\end{frame}

% ------------------------------------------------------------------------------
% LTXE-SLIDE-START
% LTXE-SLIDE-UID:		3ece9bfe-47aefaa5-68767297-8d143213
% LTXE-SLIDE-ORIGIN:	b5a66715-5681025e-cfb46644-9b9bad55 English
% LTXE-SLIDE-TITLE:		#67236: Added "allowedTags" argument to f:format.stripTags ViewHelper
% ------------------------------------------------------------------------------

\begin{frame}[fragile]
	\frametitle{Extbase \& Fluid}
	\framesubtitle{Argument "allowedTags" voor f:format.stripTags}

	\begin{itemize}

		\item Het argument \texttt{allowedTags} bevat een lijst met HTML tags
		 	die niet verwijderd worden bij \texttt{f:format.stripTags}

		\item De syntax voor de lijst met tags is identiek aan de tweede parameter
		 	van de PHP functie \texttt{strip\_tags} (zie: \url{http://php.net/strip_tags})

	\end{itemize}

\end{frame}

% ------------------------------------------------------------------------------
% LTXE-SLIDE-START
% LTXE-SLIDE-UID:		0cf44713-2a9df6d4-1f2bcdf4-e67b8974
% LTXE-SLIDE-ORIGIN:	a71d77d1-6a419c8d-0fbb35d0-e1ce57e4 English
% LTXE-SLIDE-TITLE:		Default Content Element Changed for Fluid Styled Content
% LTXE-SLIDE-REFERENCE:	!Breaking: #79622 - Default content element changed for Fluid Styled Content
% ------------------------------------------------------------------------------

\begin{frame}[fragile]
	\frametitle{Extbase \& Fluid}
	\framesubtitle{Standaard inhoudselement gewijzigd voor Fluid Styled Content}

	% decrease font size for code listing
	\lstset{basicstyle=\tiny\ttfamily}

	\begin{itemize}
		\item Het standaard inhoudselement voor CSS Styled Content is gewijzigd in "Tekst"
		\item Om de configuratie terug te zetten moet het standaard inhoudselement handmatig op de voorkeur ingesteld worden.
		 	Dit kan gedaan worden door het overschrijven van de configuratie in een
		 	\texttt{Configuration/TCA/Overrides/tt\_content.php} bestand.

			\begin{lstlisting}
				$GLOBALS['TCA']['tt_content']['columns']['CType']['config']['default'] = 'textmedia';
				$GLOBALS['TCA']['tt_content']['columns']['CType']['config']['default'] = 'header';
			\end{lstlisting}

	\end{itemize}

\end{frame}

% ------------------------------------------------------------------------------
% LTXE-SLIDE-START
% LTXE-SLIDE-UID:		8271f1ec-fbf07b10-ff6f1c6f-e5890119
% LTXE-SLIDE-ORIGIN:	b400c3b8-3e402480-cd7d6ff3-96cd93aa English
% LTXE-SLIDE-TITLE:		#76085: Add fluid debug information to admin panel
% ------------------------------------------------------------------------------
\begin{frame}[fragile]
	\frametitle{Extbase \& Fluid}
	\framesubtitle{Admin Panel}

	\begin{itemize}
		\item Admin Panel heeft een nieuwe optie om Fluid uitvoer te debuggen:\newline
			\textbf{Preview -> Show fluid debug output}
		\item Indien ingeschakeld worden de volgende details getoond:

			\begin{itemize}
				\item pad naar het sjabloonbestand van een partial
				\item naam van de section
			\end{itemize}

		\item Hiermee kunnen integrators eenvoudig de juiste sjabloon en section vinden

	\end{itemize}

\end{frame}

% ------------------------------------------------------------------------------

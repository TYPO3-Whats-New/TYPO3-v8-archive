% ------------------------------------------------------------------------------
% TYPO3 CMS 8 LTS - What's New - Chapter "Doctrine" (Dutch Version)
%
% @author	Michael Schams <schams.net>
% @license	Creative Commons BY-NC-SA 3.0
% @link		http://typo3.org/download/release-notes/whats-new/
% @language	English
% ------------------------------------------------------------------------------
% LTXE-CHAPTER-UID:		60d29a083c46763520c15ac80b028f76
% LTXE-CHAPTER-NAME:	Doctrine DBAL
% ------------------------------------------------------------------------------

\section{Doctrine DBAL}
\begin{frame}[fragile]
	\frametitle{Doctrine DBAL}

	\begin{center}\huge{\color{typo3darkgrey}\textbf{Doctrine DBAL}}\end{center}
	\begin{center}\large{\textit{State-of-the-art database-abstractielaag}}\end{center}

\end{frame}



% ------------------------------------------------------------------------------
% LTXE-SLIDE-START
% LTXE-SLIDE-UID:		ba4ee976-109c9b7e-80468f33-9f6d2f79
% LTXE-SLIDE-ORIGIN:	861205b5-d3f4a55e-69279ab1-e18dcb15 English
% LTXE-SLIDE-TITLE:		Feature: #75454 - PHP Library Doctrine DBAL (1)
% LTXE-SLIDE-REFERENCE:	Feature-75454-DoctrineDBALForDatabaseConnections.rst
% ------------------------------------------------------------------------------
\begin{frame}[fragile]
	\frametitle{Doctrine DBAL}
	\framesubtitle{PHP Bibliotheek "Doctrine DBAL" (1)}

	\begin{itemize}

		\item De PHP-bibliotheek
			"\href{http://www.doctrine-project.org}{Doctrine DBAL}"
			is toegevoegd als composer-afhankelijkheid
			om als een krachtige database-abstractielaag te werken met vele opties
			voor database-abstractie, schema-analyse en schemabeheer binnen
			TYPO3 CMS

		\item Een TYPO3-specifieke PHP-klasse
			\texttt{TYPO3\textbackslash
				CMS\textbackslash
				Core\textbackslash
				Database\textbackslash
				ConnectionPool}\newline
			is toegevoegd voor het beheer van databaseconnecties

		\item Alle connecties geconfigureerd onder
			\texttt{\$GLOBALS['TYPO3\_CONF\_VARS']['DB']['Connections']}\newline
			zijn toegankelijk via deze beheerder waardoor parallel gebruik van
			verschillende databasesystemen mogelijk is

	\end{itemize}

\end{frame}

% ------------------------------------------------------------------------------
% LTXE-SLIDE-START
% LTXE-SLIDE-UID:		b520183c-580c1557-2a9b61f0-12bbe5f2
% LTXE-SLIDE-ORIGIN:	988bc43f-175df2eb-d7158156-f8f36703 English
% LTXE-SLIDE-TITLE:		Feature: #75454 - PHP Library Doctrine DBAL (2)
% LTXE-SLIDE-REFERENCE:	Feature-75454-DoctrineDBALForDatabaseConnections.rst
% ------------------------------------------------------------------------------
\begin{frame}[fragile]
	\frametitle{Doctrine DBAL}
	\framesubtitle{PHP Bibliotheek "Doctrine DBAL" (2)}

	\begin{itemize}

		\item Met het gebruik van de database-abstractiemogelijkheden en de QueryBuilder
		 	worden de opgebouwde SQL-opdrachten correct gecodeerd en zijn ze automatisch
		 	zoveel mogelijk compatibel met de verschillende DBMS'en

		\item Bestaande opties in \texttt{\$GLOBALS['TYPO3\_CONF\_VARS']['DB']} zijn
			verwijderd en/of gemigreerd naar de nieuwe Doctrine-opties

		\item De \texttt{Connection} klasse biedt handige functies voor
			\texttt{insert}, \texttt{select}, \texttt{update}, \texttt{delete} en
			\texttt{truncate} opdrachten

		\item Voor \texttt{select}, \texttt{update} en \texttt{delete} worden alleen simpele
			vergelijkingen (zoals \texttt{WHERE "aField" = 'aValue'}) ondersteund.
			Voor complexe opdrachten is de \texttt{QueryBuilder} nodig.

	\end{itemize}

\end{frame}

% ------------------------------------------------------------------------------
% LTXE-SLIDE-START
% LTXE-SLIDE-UID:		64eb0394-df21af2f-d8c172c8-70951606
% LTXE-SLIDE-ORIGIN:	4e6e640e-d37fcca8-bd0ca330-a70baaab English
% LTXE-SLIDE-TITLE:		Feature: #75454 - PHP Library Doctrine DBAL (3)
% LTXE-SLIDE-REFERENCE:	!Feature-75454-DoctrineDBALForDatabaseConnections.rst
% ------------------------------------------------------------------------------
\begin{frame}[fragile]
	\frametitle{Doctrine DBAL}
	\framesubtitle{PHP Bibliotheek "Doctrine DBAL" (3)}

	% decrease font size for code listing
	\lstset{basicstyle=\tiny\ttfamily}

	\begin{itemize}

		\item De \texttt{ConnectionPool} klasse kan zo gebruikt worden:
			\begin{lstlisting}
				// Haal verbinding voor meerdere bewerkingen
				/** @var \TYPO3\CMS\Core\Database\Connecction $conn */
				$conn = GeneralUtility::makeInstance(ConnectionPool::class)->getConnectionForTable('aTable');
				$affectedRows = $conn->insert(
				  'aTable',
				  $fields, // Array met kolom/waarde paren, automatisch gecodeerd
				);

				// Haal QueryBuilder (voor eenmalig gebruik)
				$query = GeneralUtility::makeInstance(ConnectionPool::class)->getQueryBuilderForTable('aTable);
				$query->select('*')
				  ->from('aTable)
				  ->where($query->expr()->eq('aField', $query->createNamedParameter($aValue)))
				  ->andWhere(
					$query->expr()->lte(
					  'anotherField',
					  $query->createNamedParameter($anotherValue)
					)
				  )
				$rows = $query->execute()->fetchAll();
			\end{lstlisting}
	\end{itemize}

\end{frame}


% ------------------------------------------------------------------------------
% LTXE-SLIDE-START
% LTXE-SLIDE-UID:		b5cc034a-bf6b57f6-699cb151-a705a474
% LTXE-SLIDE-ORIGIN:	c5986329-fb36f473-a5430d2e-6ba49fb1 English
% LTXE-SLIDE-TITLE:		#78116: Extbase support for Doctrine's native DBAL Statement and QueryBuilder
% ------------------------------------------------------------------------------
\begin{frame}[fragile]
	\frametitle{Doctrine DBAL}
	\framesubtitle{Doctrine in Extbase}

	% decrease font size for code listing
	\lstset{basicstyle=\tiny\ttfamily}

	\begin{itemize}
		\item Als extensies de Extbase standaard gebruiken is het niet nodig code bij te werken

		\item Directe SQL-query-functionaliteit ondersteunt ook QueryBuilder-objecten en instanties van
			\texttt{\textbackslash
			Doctrine\textbackslash
			DBAL\textbackslash
			Statement} als prepared statements
		\item Het volgende voorbeeld werkt in elke Extbase repository met pure Doctrine DBAL statements:

			\begin{lstlisting}
				$connectie = $this->objectManager->get(ConnectionPool::class)->getConnectionForTable('mijntabel');
				$statement = $this->objectManager->get(
				  \Doctrine\DBAL\Statement::class,
				  'SELECT * FROM mijtabel WHERE uid=? OR titel=?',
				  $connection
				);

				$query = $this->createQuery();
				$query->statement($statement, [$uid, $titel]);
			\end{lstlisting}
	\end{itemize}

\end{frame}

% ------------------------------------------------------------------------------
% LTXE-SLIDE-START
% LTXE-SLIDE-UID:		b6ef8184-24e7e019-cfe3c248-7b725922
% LTXE-SLIDE-ORIGIN:	82c70aee-3b375252-6cac0ad6-b385b95f English
% LTXE-SLIDE-TITLE:		Miscellaneous (1)
% ------------------------------------------------------------------------------
\begin{frame}[fragile]
	\frametitle{Doctrine DBAL}
	\framesubtitle{Diversen (1)}

	% decrease font size for code listing
	\lstset{basicstyle=\tiny\ttfamily}

	\begin{itemize}

		\item Met de migratie naar Doctrine is de hook \texttt{buildQueryParameters} toegevoegd in de klasse
			\texttt{DatabaseRecordList}.

		\item Deze vervangt de hook \texttt{makeQueryArray} uit de verouderde methode
			\texttt{AbstractDatabaseRecordList::makeQueryArray}.

		\item Met de nieuwe hook kunnen de parameters gewijzigd worden die gebruikt worden voor de databasequery
			voor het afbeelden van de lijst met records.

		\item De hook kan geregistreerd worden in \texttt{ext\_localconf.php}:

			\begin{lstlisting}
				$GLOBALS['TYPO3_CONF_VARS']['SC_OPTIONS']
				  [\TYPO3\CMS\Recordlist\RecordList\DatabaseRecordList::class]['buildQueryParameters'][]
			\end{lstlisting}

		\item ...en implementeert de publieke methode \texttt{buildQueryParametersPostProcess}

	\end{itemize}

\end{frame}

% ------------------------------------------------------------------------------
% LTXE-SLIDE-START
% LTXE-SLIDE-UID:		167511ec-4e8e5ffe-6a69829f-001f4e53
% LTXE-SLIDE-ORIGIN:	11a9285f-18bdbfac-88ece0c7-b3f2276a English
% LTXE-SLIDE-TITLE:		Miscellaneous (2)
% ------------------------------------------------------------------------------

\begin{frame}[fragile]
	\frametitle{Doctrine DBAL}
	\framesubtitle{Diversen (2)}

	\begin{itemize}
		\item De databaselaag in Extbase is nu ook complete gebaseerd op de Doctrine DBAL QueryBuilder
		\item \texttt{EXT:dbal} en \texttt{EXT:adodb} zijn verwijderd uit de TYPO3 core\newline
			\smaller
				Als extensies de oude \texttt{TYPO3\_DB} API gebruiken om andere database tabellen dan MySQL te
				bevragen kunnen deze twee extensies uit TER gehaald worden.
			\normalsize

		\item \texttt{TYPO3\_DB} verkorte functionaliteit is verwijderd voor het grootste deel van de TYPO3 core PHP klassen\newline
			\smaller
				(gebruik van \texttt{\$GLOBALS[TYPO3\_DB]} is nog mogelijk maar wordt afgeraden)
			\normalsize

	\end{itemize}

\end{frame}

% ------------------------------------------------------------------------------

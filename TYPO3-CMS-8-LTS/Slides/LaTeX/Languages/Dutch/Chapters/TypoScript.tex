% ------------------------------------------------------------------------------
% TYPO3 CMS 8 LTS - What's New - Chapter "TypoScript" (Dutch Version)
%
% @author	Michael Schams <schams.net>
% @license	Creative Commons BY-NC-SA 3.0
% @link		http://typo3.org/download/release-notes/whats-new/
% @language	English
% ------------------------------------------------------------------------------
% LTXE-CHAPTER-UID:		3a9852ea-e2360d9d-1ff5eec1-a7de3f9f
% LTXE-CHAPTER-NAME:	TypoScript
% ------------------------------------------------------------------------------

\section{TSconfig \& TypoScript}
\begin{frame}[fragile]
	\frametitle{TSconfig \& TypoScript}

	\begin{center}\huge{\color{typo3darkgrey}\textbf{TSconfig \& TypoScript}}\end{center}
	\begin{center}\large{\textit{Nieuwe configuratieopties, nieuwe functies, onbegrensde mogelijkheden}}\end{center}

\end{frame}

% ------------------------------------------------------------------------------
% LTXE-SLIDE-START
% LTXE-SLIDE-UID:		6e557da9-68146b5a-6e72a0e8-a8f6bc17
% LTXE-SLIDE-ORIGIN:	a7d30bb0-c3eeb0dc-efa59a5c-fae9a471 English
% LTXE-SLIDE-TITLE:		#39597: Multiple locale names for TypoScript config.locale_all
% ------------------------------------------------------------------------------
\begin{frame}[fragile]
	\frametitle{TSconfig \& TypoScript}
	\framesubtitle{Meerdere Locale-namen voor TypoScript \texttt{config.locale\_all}}

	% decrease font size for code listing
	\lstset{basicstyle=\small\ttfamily}

	\begin{itemize}

		\item TypoScript optie \texttt{config.locale\_all} kan nu een set mogelijke
			locales krijgen als komma-gescheiden lijst, net zoals de PHP functie
			\texttt{setlocale()} ondersteunt:

			\begin{lstlisting}
				config.locale_all = de_AT@euro, de_AT, de_DE, deu_deu
			\end{lstlisting}

			Zie \url{http://php.net/setlocale}

	\end{itemize}

\end{frame}

% ------------------------------------------------------------------------------
% LTXE-SLIDE-START
% LTXE-SLIDE-UID:		36d9ef83-31f2053e-3f116b89-234ca719
% LTXE-SLIDE-ORIGIN:	cd70399f-ccfb78e5-15d95ae4-c96d579f English
% LTXE-SLIDE-TITLE:		#18586: Configurable width & height for editpanel in feedit
% ------------------------------------------------------------------------------
\begin{frame}[fragile]
	\frametitle{TSconfig \& TypoScript}
	\framesubtitle{Configureerbare breedte en hoogte van het Editpanel in \texttt{EXT:feedit}}

	% decrease font size for code listing
	\lstset{basicstyle=\small\ttfamily}

	\begin{itemize}

		\item Het is nu mogelijk om de breedte en hoogte van de popup die gebruikt wordt in het Editpanel van \texttt{EXT:feedit}
		    te wijzigen door gebruik te maken van User TSconfig:
			\begin{lstlisting}
				options.feedit.popupHeight = 700
				options.feedit.popupWidth = 900
			\end{lstlisting}

	\end{itemize}

\end{frame}

% ------------------------------------------------------------------------------
% LTXE-SLIDE-START
% LTXE-SLIDE-UID:		0172ba6c-7279a89b-082e6492-9cad0b93
% LTXE-SLIDE-ORIGIN:	b287e11b-2386594e-3ad7e012-cc1cdef2 English
% LTXE-SLIDE-TITLE:		#17309: Access FlexForm values
% ------------------------------------------------------------------------------
\begin{frame}[fragile]
	\frametitle{TSconfig \& TypoScript}
	\framesubtitle{Toegang tot FlexForm-waarden}

	% decrease font size for code listing
	\lstset{basicstyle=\tiny\ttfamily}

	\begin{itemize}

		\item Het is nu mogelijk om eigenschappen van een FlexForm-veld op te vragen

		\begin{lstlisting}
			lib.flexformContent = CONTENT
			lib.flexformContent {
			  table = tt_content
			  select {
			    pidInList = this
			  }

			  renderObj = COA
			  renderObj {
			    10 = TEXT
			    10 {
			      data = flexform: pi_flexform:settings.categories
			    }
			  }
			}
		\end{lstlisting}

	\end{itemize}

\end{frame}

% ------------------------------------------------------------------------------
% LTXE-SLIDE-START
% LTXE-SLIDE-UID:		c3f75c69-d6fa8884-a77e18b9-b80b5828
% LTXE-SLIDE-ORIGIN:	aee7c947-a271d451-cb640e43-101a5543 English
% LTXE-SLIDE-TITLE:		#78549: Override New Page Creation Wizard via page TSconfig
% ------------------------------------------------------------------------------

\begin{frame}[fragile]
	\frametitle{TSconfig \& TypoScript}
	\framesubtitle{Nieuwe assistent pagina aanmaken}

	\begin{itemize}
		\item In vorige versies van TYPO3 CMS was het mogelijk om de "Assistent nieuwe pagina aanmaken"
		 	te overschrijven met eigen scripts:\newline
			\small
				\texttt{mod.web\_list.newPageWiz.overrideWithExtension = myextension}
			\normalsize
		\item De nieuwe manier van het afhandelen van ingangen en eigen scripts is gebouwd met modules/routes
		 	en de bovenstaande optie is verwijderd
		\item De volgende TSconfig-optie kan in plaats hiervan gebruikt worden:
			\small
				\texttt{mod.newPageWizard.override = my\_custom\_module}
			\normalsize

		\item In plaats van het instellen van een extensie-key moet er nu een module of
		 	route gespecificeerd worden

	\end{itemize}

\end{frame}

% ------------------------------------------------------------------------------
% LTXE-SLIDE-START
% LTXE-SLIDE-UID:		1fe4e756-c2023dbe-b99d245c-32fffa28
% LTXE-SLIDE-ORIGIN:	c56cb2e8-f50d7eed-e93dcb9e-ab7be017 English
% LTXE-SLIDE-TITLE:		#73626: Number of search results is configurable now
% ------------------------------------------------------------------------------
\begin{frame}[fragile]
	\frametitle{TSconfig \& TypoScript}
	\framesubtitle{Aantal zoekresultaten}

	\begin{itemize}
		\item Het maximale aantal zoekresultaten kan in TypoScript ingesteld worden:\newline
			\texttt{plugin.tx\_indexedsearch.settings.blind.numberOfResults}
		\item Deze instelling slaat een lijst met waardes op
		\item Als het aantal resultaten wordt vermeld in het request en overeenkomt met een van
			de ingestelde waarden dan wordt het gebruikt
		\item Als het niet wordt meegestuurd of niet overeenkomt met de ingestelde waarden dan
		 	wordt de eerste van de lijst gebruikt
		\item Om compatibel te zijn met vorige versies is de standaard:\newline
			\texttt{10}, \texttt{25}, \texttt{50} and \texttt{100}
	\end{itemize}

\end{frame}

% ------------------------------------------------------------------------------
% LTXE-SLIDE-START
% LTXE-SLIDE-UID:		62981d89-c849c214-5aebc9b1-63880904
% LTXE-SLIDE-ORIGIN:	d9b9e73b-db575400-8276f4dd-810f4f4e English
% LTXE-SLIDE-TITLE:		#78672: Fluid data processor for menus (1)
% ------------------------------------------------------------------------------
\begin{frame}[fragile]
	\frametitle{TSconfig \& TypoScript}
	\framesubtitle{Fluid Data Processor voor menu's (1)}

	\begin{itemize}
		\item Menu processor gebruikt \texttt{HMENU} om de JSON-gecodeerde menu-tekenreeks te maken
			die gedecodeerd wordt en toegewezen aan \texttt{FLUIDTEMPLATE}
		\item Extra DataProcessing wordt ondersteund en op elk record toegepast
		\item Ondersteunde opties: \texttt{as}, \texttt{levels}, \texttt{expandAll}, \texttt{includeSpacer},
			\texttt{titleField}
			(zie \href{https://docs.typo3.org/typo3cms/TyposcriptReference/ContentObjects/Hmenu/Index.html}{TyposcriptReference} voor meer opties)
	\end{itemize}

\end{frame}

% ------------------------------------------------------------------------------
% LTXE-SLIDE-START
% LTXE-SLIDE-UID:		3d9aab4d-ffdcd16d-7f0f2b5a-39b49709
% LTXE-SLIDE-ORIGIN:	03e756a0-07740056-a2009898-d6159f4b English
% LTXE-SLIDE-TITLE:		#78672: Fluid data processor for menus (2)
% ------------------------------------------------------------------------------
\begin{frame}[fragile]
	\frametitle{TSconfig \& TypoScript}
	\framesubtitle{Fluid Data Processor voor menu's (2)}

	% decrease font size for code listing
	\lstset{basicstyle=\tiny\ttfamily}

	\begin{itemize}
		\item Voorbeeld TypoScript-configuratie:

			\begin{lstlisting}
				10 = TYPO3\CMS\Frontend\DataProcessing\MenuProcessor
				10 {
				  special = list
				  special.value.field = pages
				  levels = 7
				  as = menu
				  expandAll = 1
				  includeSpacer = 1
				  titleField = nav_title // title
				  dataProcessing {
				    10 = TYPO3\CMS\Frontend\DataProcessing\FilesProcessor
				    10 {
				      references.fieldName = media
				    }
				  }
				}
			\end{lstlisting}

	\end{itemize}

\end{frame}

% ------------------------------------------------------------------------------
% LTXE-SLIDE-START
% LTXE-SLIDE-UID:		0f68cd3f-18cc962b-5b309d16-b98ec783
% LTXE-SLIDE-ORIGIN:	f523cc8b-11af7890-81973e17-40dc35bf English
% LTXE-SLIDE-TITLE:		#79622 - Section Frame for CSS Styled Content Replaced with Frame Class
% ------------------------------------------------------------------------------

\begin{frame}[fragile]
	\frametitle{TSconfig \& TypoScript}
	\framesubtitle{Sectieframe voor CSS Styled Content vervangen door Frame Class}

	% decrease font size for code listing
	\lstset{basicstyle=\tiny\ttfamily}

	\begin{itemize}
		\item De functionaliteit van de \texttt{Section Frame} is verbeterd met Fluid Styled Content en is nu beschikbaar als \texttt{Frame Class}.
		\item De TypoScript sleutels gebruiken nu het juiste deel van de CSS-klasse \texttt{csc-frame-} in plaats van getallen
		\item Voor:

			\begin{lstlisting}
				tt_content.stdWrap.innerWrap.cObject.key.field = section_frame
				tt_content.stdWrap.innerWrap.cObject.5 =< tt_content.stdWrap.innerWrap.cObject.default
				tt_content.stdWrap.innerWrap.cObject.5.20.10.value = csc-frame csc-frame-ruler-before
			\end{lstlisting}

		\item Na:

			\begin{lstlisting}
				tt_content.stdWrap.innerWrap.cObject.key.field = frame_class
				tt_content.stdWrap.innerWrap.cObject.ruler-before =< tt_content.stdWrap.innerWrap.cObject.default
				tt_content.stdWrap.innerWrap.cObject.ruler-before.20.10.value = csc-frame csc-frame-ruler-before
			\end{lstlisting}

	\end{itemize}

\end{frame}

% ------------------------------------------------------------------------------

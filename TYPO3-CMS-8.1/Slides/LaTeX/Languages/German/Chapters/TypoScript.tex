% ------------------------------------------------------------------------------
% TYPO3 CMS 8.1 - What's New - Chapter "TypoScript" (English Version)
%
% @author	Patrick Lobacher <patrick@lobacher.de> and Michael Schams <schams.net>
% @license	Creative Commons BY-NC-SA 3.0
% @link		http://typo3.org/download/release-notes/whats-new/
% @language	English
% ------------------------------------------------------------------------------
% LTXE-CHAPTER-UID:		4c33fc9c-f89434c5-a7b6238d-851057cd
% LTXE-CHAPTER-NAME:	TypoScript
% ------------------------------------------------------------------------------

\section{TSconfig \& TypoScript}
\begin{frame}[fragile]
	\frametitle{TSconfig \& TypoScript}

	\begin{center}\huge{Kapitel 2:}\end{center}
	\begin{center}\huge{\color{typo3darkgrey}\textbf{TSconfig \& TypoScript}}\end{center}

\end{frame}

% ------------------------------------------------------------------------------
% LTXE-SLIDE-START
% LTXE-SLIDE-UID:		37f1a892-067b99ee-11cd61d1-c99ab374
% LTXE-SLIDE-ORIGIN:	331bddc9-39173400-45bf043c-ee04ad88 English
% LTXE-SLIDE-TITLE:		Feature: #27471 - Allow asterisk for hideTables
% LTXE-SLIDE-REFERENCE:	!Feature-27471-AllowAsteriskForHideTables.rst
% ------------------------------------------------------------------------------
\begin{frame}[fragile]
	\frametitle{TSconfig \& TypoScript}
	\framesubtitle{Zeichen für "alle" in \texttt{hideTables}}

	% decrease font size for code listing
	\lstset{basicstyle=\tiny\ttfamily}

	\begin{itemize}
		\item Es ist nun möglich, alle Tabellen gleichzeitig im List-View via PageTS-Config anzusprechen

		\item Um eine einzelne Tabelle anzuzeigen, kann man alle verstecken und nur die eine anzeigen lassen:

			\begin{lstlisting}
				mod.web_list {
				  hideTables = *
				  table.tx_cal_event.hideTable = 0
				}
			\end{lstlisting}

	\end{itemize}

\end{frame}

% ------------------------------------------------------------------------------
% LTXE-SLIDE-START
% LTXE-SLIDE-UID:		b38ef93b-c4378ca5-4b5c9caf-548a70c2
% LTXE-SLIDE-ORIGIN:	a7d30bb0-c3eeb0dc-efa59a5c-fae9a471 English
% LTXE-SLIDE-TITLE:		Feature: #39597 - Multiple Locale Names for TypoScript config.locale_all
% LTXE-SLIDE-REFERENCE:	!Feature-39597-MultipleLocaleNamesForTypoScriptConfiglocale_all.rst
% ------------------------------------------------------------------------------
\begin{frame}[fragile]
	\frametitle{TSconfig \& TypoScript}
	\framesubtitle{Mehrere Locale-Namen in der TypoScript-Config \texttt{config.locale\_all}}

	% decrease font size for code listing
	\lstset{basicstyle=\small\ttfamily}

	\begin{itemize}

		\item Die TypoScript Option \texttt{config.locale\_all} erlaubt es nun, Fallbacks für Locales als Kommaseparierte Liste	(wie die PHP-Funktion \texttt{setlocale()}) zu setzen:

			\begin{lstlisting}
				config.locale_all = de_AT@euro, de_AT, de_DE, deu_deu
			\end{lstlisting}

			Siehe \url{http://php.net/setlocale}

	\end{itemize}

\end{frame}

% ------------------------------------------------------------------------------
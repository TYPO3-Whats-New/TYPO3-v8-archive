% ------------------------------------------------------------------------------
% TYPO3 CMS 8.1 - What's New - Chapter "In-Depth Changes" (English Version)
%
% @author	Patrick Lobacher <patrick@lobacher.de> and Michael Schams <schams.net>
% @license	Creative Commons BY-NC-SA 3.0
% @link		http://typo3.org/download/release-notes/whats-new/
% @language	English
% ------------------------------------------------------------------------------
% LTXE-CHAPTER-UID:		e2881b46-7caaff58-912a10b9-7d517f6e
% LTXE-CHAPTER-NAME:	In-Depth Changes
% ------------------------------------------------------------------------------

\section{In-Depth Changes}
\begin{frame}[fragile]
	\frametitle{Änderungen im System}

	\begin{center}\huge{Kapitel 3:}\end{center}
	\begin{center}\huge{\color{typo3darkgrey}\textbf{In-Depth Changes}}\end{center}

\end{frame}


% ------------------------------------------------------------------------------
% LTXE-SLIDE-START
% LTXE-SLIDE-UID:		2dc345dc-9c015d5e-4594ecd2-6092896b
% LTXE-SLIDE-ORIGIN:	861205b5-d3f4a55e-69279ab1-e18dcb15 English
% LTXE-SLIDE-TITLE:		Feature: #75454 - PHP Library Doctrine DBAL (1)
% LTXE-SLIDE-REFERENCE:	Feature-75454-DoctrineDBALForDatabaseConnections.rst
% ------------------------------------------------------------------------------
\begin{frame}[fragile]
	\frametitle{Änderungen im System}
	\framesubtitle{PHP-Bibliothek "Doctrine DBAL" (1)}

	\begin{itemize}

		\item Es wurde die PHP-Bibliothek
			"\href{http://www.doctrine-project.org}{Doctrine DBAL}"
			als Composer Abhängigkeit hinzugefügt, um eine leistungsfähige Datenbank-Abstraktionsschicht
			innerhalb von TYPO3 zur Verfügung zu stellen. Doctrine besitzt viele leistungsfähig Features
			wie Datenbank-Abstaktion, Schema-Introspection und Schema-Management

		\item Es wurde zudem eine TYPO3-spezifische PHP-Klasse
			\smaller
				\texttt{TYPO3\textbackslash
					CMS\textbackslash
					Core\textbackslash
					Database\textbackslash
					ConnectionPool}\normalsize\newline
			als Manager der Datenbank-Verbindung zugefügt

		\item Alle Verbindungen, die per
			\smaller
				\texttt{\$GLOBALS['TYPO3\_CONF\_VARS']['DB']['Connections']}
			\normalsize\newline
			konfiguriert wurden, sind über den Manager erreichbar - damit können auch mehrere Datenbanken
			gleichzeitig angesprochen werden

	\end{itemize}

\end{frame}

% ------------------------------------------------------------------------------
% LTXE-SLIDE-START
% LTXE-SLIDE-UID:		97a2f761-66e83a78-5993138a-278df7a2
% LTXE-SLIDE-ORIGIN:	988bc43f-175df2eb-d7158156-f8f36703 English
% LTXE-SLIDE-TITLE:		Feature: #75454 - PHP Library Doctrine DBAL (2)
% LTXE-SLIDE-REFERENCE:	Feature-75454-DoctrineDBALForDatabaseConnections.rst
% ------------------------------------------------------------------------------
\begin{frame}[fragile]
	\frametitle{In-Depth Changes}
	\framesubtitle{PHP-Bibliothek "Doctrine DBAL" (2)}

	\begin{itemize}

		\item Durch Verwendung der Optionen der Datenbank Abstraktion und den Query-Builders sind die SQL Statements out-of-the-box kompatibel mit verschiedensten RBMS.

		\item Die Optionen unter \texttt{\$GLOBALS['TYPO3\_CONF\_VARS']['DB']} wurden entfernt und/eingeführt wurden

		\item Die \texttt{Connection} Klasse stellt bequeme
			\texttt{insert}, \texttt{select}, \texttt{update}, \texttt{delete} und
			\texttt{truncate} Statements zur Verfügung

		\item Für \texttt{select}, \texttt{update} und \texttt{delete} gibt es bislang nur einfache Vergleiche  (wie \texttt{WHERE "aField" = 'aValue'}).
			Für komplexere Statements muss man den \texttt{QueryBuilder} verwenden.

	\end{itemize}

\end{frame}

% ------------------------------------------------------------------------------
% LTXE-SLIDE-START
% LTXE-SLIDE-UID:		bdb8c402-b6f86d88-c7471ab1-37412bca
% LTXE-SLIDE-ORIGIN:	4e6e640e-d37fcca8-bd0ca330-a70baaab English
% LTXE-SLIDE-TITLE:		Feature: #75454 - PHP Library Doctrine DBAL (3)
% LTXE-SLIDE-REFERENCE:	!Feature-75454-DoctrineDBALForDatabaseConnections.rst
% ------------------------------------------------------------------------------
\begin{frame}[fragile]
	\frametitle{In-Depth Changes}
	\framesubtitle{PHP-Bibliothek "Doctrine DBAL" (3)}

	% decrease font size for code listing
	\lstset{basicstyle=\tiny\ttfamily}

	\begin{itemize}

		\item Die \texttt{ConnectionPool} Klasse kann wie folgt verwendet werden:
			\begin{lstlisting}
				// Get a connection which can be used for muliple operations
				/** @var \TYPO3\CMS\Core\Database\Connecction $conn */
				$conn = GeneralUtility::makeInstance(ConnectionPool::class)->getConnectionForTable('aTable');
				$affectedRows = $conn->insert(
				  'aTable',
				  $fields, // Associative array of column/value pairs, automatically quoted & escaped
				);

				// Get a QueryBuilder, which should only be used a single time
				$query = GeneralUtility::makeInstance(ConnectionPool::class)->getQueryBuilderForTable('aTable);
				$query->select('*')
				  ->from('aTable)
				  ->where($query->expr()->eq('aField', $query->createNamedParameter($aValue)))
				  ->andWhere(
					$query->expr()->lte(
					  'anotherField',
					  $query->createNamedParameter($anotherValue)
					)
				  )
				$rows = $query->execute()->fetchAll();
			\end{lstlisting}
	\end{itemize}

\end{frame}

% ------------------------------------------------------------------------------
% LTXE-SLIDE-START
% LTXE-SLIDE-UID:		ed7dd0d3-8c72f0d8-84d7bac2-ce765529
% LTXE-SLIDE-ORIGIN:	443c416c-63c0f980-719e38f1-1d117e50 English
% LTXE-SLIDE-TITLE:		Feature: #69439 - Enhance SQL query reduction in page tree in workspaces
% LTXE-SLIDE-REFERENCE:	!Feature-69439-EnhanceSQLQueryReductionInPageTreeInWorkspaces.rst
% ------------------------------------------------------------------------------
\begin{frame}[fragile]
	\frametitle{In-Depth Changes}
	\framesubtitle{Neue Hooks im Workspacemodul}

	% decrease font size for code listing
	\lstset{basicstyle=\tiny\ttfamily}

	\begin{itemize}

		\item Der Prozess, um festzustellen, ob eine Seite Versionen im Workspace besitzt, wurde nun mit Hooks zur Erweiterung ausgestattet

		\item Dadurch können beliebige Versionen mit Hooks erweitert werden

		\item Der Hook kann beispielsweise wie folgt angesprochen werden:

			\begin{lstlisting}
				$GLOBALS['TYPO3_CONF_VARS']['SC_OPTIONS']...
				  ...['TYPO3\\CMS\\Workspaces\\Service\\WorkspaceService']['hasPageRecordVersions'];

				$GLOBALS['TYPO3_CONF_VARS']['SC_OPTIONS']...
				  ...['TYPO3\\CMS\\Workspaces\\Service\\WorkspaceService']['fetchPagesWithVersionsInTable']
			\end{lstlisting}

	\end{itemize}

\end{frame}


% ------------------------------------------------------------------------------
% LTXE-SLIDE-START
% LTXE-SLIDE-UID:		11c697d5-a821be8b-1756e6fd-cd7e8dd3
% LTXE-SLIDE-ORIGIN:	630de169-c2699a8a-ee44dde5-cf39c2ca English
% LTXE-SLIDE-TITLE:		Feature: #70056 - PHP Library "Guzzle" (1)
% LTXE-SLIDE-REFERENCE:	!Feature-70056-GuzzleForHttpRequests.rst
% ------------------------------------------------------------------------------
\begin{frame}[fragile]
	\frametitle{In-Depth Changes}
	\framesubtitle{PHP-Bibliothek "Guzzle" (1)}

	\begin{itemize}

		\item Die PHP-Bibliothek
			"\href{http://docs.guzzlephp.org}{Guzzle}"
			wurde per Composer-Abhängigkeit zugefügt, um als umfangreiche Lösung für HTTP-Requests (basierend auf PSR-7 zu dienen

		\item Guzzle erkannt automatisch ob es im System Adapter dafür gibt (z.B. cURL oder Stream Wrapper) und wählt die beste Lösung für das System aus

		\item Es wurde zumde eine TYPO3-spezifische PHP Klasse mit dem Namen 
			\texttt{TYPO3\textbackslash
				CMS\textbackslash
				Core\textbackslash
				Http\textbackslash
				RequestFactory}\newline
			zugefügt um einen simplifizierten Wrapper für Guzzle-Clients zu haben.

	\end{itemize}

\end{frame}


% ------------------------------------------------------------------------------
% LTXE-SLIDE-START
% LTXE-SLIDE-UID:		eaa33a39-947734cc-c78b2436-4c075870
% LTXE-SLIDE-ORIGIN:	db2b31c1-47321ffe-3ed51ebc-5dd524a4 English
% LTXE-SLIDE-TITLE:		Feature: #70056 - PHP Library "Guzzle" (2)
% LTXE-SLIDE-REFERENCE:	!Feature-70056-GuzzleForHttpRequests.rst
% ------------------------------------------------------------------------------
\begin{frame}[fragile]
	\frametitle{In-Depth Changes}
	\framesubtitle{PHP-Bibliothek "Guzzle" (2)}

	% decrease font size for code listing
	\lstset{basicstyle=\tiny\ttfamily}

	\begin{itemize}

		\item Die \texttt{RequestFactory} Klasse kann wie folgt verwendet werden:

			\begin{lstlisting}
				// Initiate RequestFactory

				/** @var \TYPO3\CMS\Core\Http\RequestFactory $requestFactory */
				$requestFactory = GeneralUtility::makeInstance(
				  \TYPO3\CMS\Core\Http\RequestFactory\RequestFactory::class);

				$uri = $additionalOptions = [
				  // additional headers for this specific request
				  'headers' => ['Cache-Control' => 'no-cache'],
				  'allow_redirects' => false,
				  'cookies' => true
				];

				// return a PSR-7 compliant response object
				$response = $requestFactory->request($url, 'GET', $additionalOptions);

				// get the content as a string on a successful request
				if ($response->getStatusCode() === 200) {
				  if ($response->getHeader('Content-Type') === 'text/html') {
				    $content = $response->getBody()->getContents();
				  }
				}
			\end{lstlisting}

	\end{itemize}

\end{frame}

% ------------------------------------------------------------------------------

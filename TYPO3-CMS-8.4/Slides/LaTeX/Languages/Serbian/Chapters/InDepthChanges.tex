% ------------------------------------------------------------------------------
% TYPO3 CMS 8.4 - What's New - Chapter "In-Depth Changes" (Serbian Version)
%
% @author	Michael Schams <schams.net>
% @license	Creative Commons BY-NC-SA 3.0
% @link		http://typo3.org/download/release-notes/whats-new/
% @language	English
% ------------------------------------------------------------------------------
% LTXE-CHAPTER-UID:		5ebcecbe-66abfa57-cf38bc00-aa637965
% LTXE-CHAPTER-NAME:	In-Depth Changes
% ------------------------------------------------------------------------------

\section{In-Depth Changes}
\begin{frame}[fragile]
	\frametitle{In-Depth Changes}

	\begin{center}\huge{Chapter 3:}\end{center}
	\begin{center}\huge{\color{typo3darkgrey}\textbf{In-Depth Changes}}\end{center}

\end{frame}

% ------------------------------------------------------------------------------
% LTXE-SLIDE-START
% LTXE-SLIDE-UID:		b30a8386-3b92d45d-d7d42685-d0c14da4
% LTXE-SLIDE-ORIGIN:	bde270e6-ffef8544-ea472ed5-89ba8c3d English
% LTXE-SLIDE-TITLE:		#52877: Remove ExtJS Viewport
% ------------------------------------------------------------------------------

\begin{frame}[fragile]
	\frametitle{In-Depth Changes}
	\framesubtitle{ExtJS Removal}

	\begin{itemize}
		\item ExtJS component \texttt{TYPO3.Viewport} has been removed
		\item \texttt{Ext.layout} and \texttt{Ext.Viewport} are no longer used in the backend viewport
		\item Functionality has been re-implemented with a native JavaScript, jQuery, CSS solution
		\item ExtJS notifications components \texttt{TYPO3.Window} and \texttt{TYPO3.Dialog} have been removed
		\item Remaining parts/tasks of the complete ExtJS removal:

		\begin{itemize}
			\item page tree
			\item form extension drag'n drop functionality
			\item ExtDirect functionality
		\end{itemize}

	\end{itemize}

\end{frame}

% ------------------------------------------------------------------------------
% LTXE-SLIDE-START
% LTXE-SLIDE-UID:		3728237a-3052e1f3-1a904615-c939b519
% LTXE-SLIDE-ORIGIN:	11a9285f-18bdbfac-88ece0c7-b3f2276a English
% LTXE-SLIDE-TITLE:		Doctrine DBAL
% ------------------------------------------------------------------------------

\begin{frame}[fragile]
	\frametitle{In-Depth Changes}
	\framesubtitle{Doctrine DBAL}

	\begin{itemize}
		\item Further progress has been made in migrating all database calls of the TYPO3 core to Doctrine DBAL
		\item Extbase's persistence is now also built completely on Doctrine DBAL's QueryBuilder
		\item \texttt{EXT:dbal} and \texttt{EXT:adodb} have been removed from the TYPO3 core\newline
			\smaller
				If third party extensions use the old \texttt{TYPO3\_DB} API to query non-MySQL database tables,
				these two extension can be installed from TER.
			\normalsize

		\item \texttt{TYPO3\_DB} shorthand functionality has been removed for most of the TYPO3 core PHP classes\newline
			\smaller
				(using \texttt{\$GLOBALS[TYPO3\_DB]} is still possible but discouraged)
			\normalsize

	\end{itemize}

\end{frame}

% ------------------------------------------------------------------------------
% LTXE-SLIDE-START
% LTXE-SLIDE-UID:		48dd5927-c34355ac-6ed7457f-237139a8
% LTXE-SLIDE-ORIGIN:	bb5c02d9-8dddd379-e73500a9-9534588b English
% LTXE-SLIDE-TITLE:		#77900: TYPO3 CMS supports TypeScript (1)
% ------------------------------------------------------------------------------

\begin{frame}[fragile]
	\frametitle{In-Depth Changes}
	\framesubtitle{TypeScript Support (1)}

	\begin{itemize}
		\item \textbf{TypeScript} has been introduced to the TYPO3 core for the internal JavaScript handling
		\item TypeScript is a free and open source programming language developed and maintained by Microsoft
		\item It is a strict superset of JavaScript, which can compile JavaScript
		\item More details at: \url{https://www.typescriptlang.org}
		\item A grunt task compiles each TypeScript file (.ts) to a JavaScript file (.js) and produces an AMD module
	\end{itemize}

	\small
		Note: all AMD modules currently in TYPO3 CMS must be ported to TypeScript to ensure a future proof concept of JavaScript handling.
		The goal is to migrate all AMD modules to TypeScript before CMS 8 LTS is released.
	\normalsize

\end{frame}

% ------------------------------------------------------------------------------
% LTXE-SLIDE-START
% LTXE-SLIDE-UID:		1239409d-ba5e0fb9-7494135c-2ced3bde
% LTXE-SLIDE-ORIGIN:	d467c514-583b44e5-03e4927f-38ba0ad1 English
% LTXE-SLIDE-TITLE:		#77900: TYPO3 CMS supports TypeScript (2)
% ------------------------------------------------------------------------------

\begin{frame}[fragile]
	\frametitle{In-Depth Changes}
	\framesubtitle{TypeScript Support (2)}

	\begin{itemize}
		\item The most important rules for TypeScript are defined in rulesets which are checked by the TypeScript Linter:

			\begin{itemize}
				\item Always define types and return types, even if TypeScript provides a default type
				\item Variable scoping: prefer \texttt{let} instead of \texttt{var}
				\item Optional properties in interfaces are not allowed for the core
				\item An interface will never extend a class
				\item Iterables: ise \texttt{for (i of list)} instead of \texttt{for (i in list)}
				\item Use keyword \texttt{implements}, even if TypeScript does not require it
				\item Any class or interface must be declared with "export" to ensure re-use or export an instance of the object for existing code which can't be updated now.
			\end{itemize}

			\small(not all rules can be checked by the Linter yet)\normalsize

	\end{itemize}

\end{frame}

% ------------------------------------------------------------------------------
% LTXE-SLIDE-START
% LTXE-SLIDE-UID:		5a047152-c35fefd9-654e9143-0b60c39d
% LTXE-SLIDE-ORIGIN:	2e0d0d9e-44771297-efe747f4-89c8b960 English
% LTXE-SLIDE-TITLE:		#38496: Shortcuts take all URL parameters into account
% ------------------------------------------------------------------------------

\begin{frame}[fragile]
	\frametitle{In-Depth Changes}
	\framesubtitle{URL Parameters in Shortcuts}

	\begin{itemize}
		\item Shortcuts take all URL parameters into account now.
		\item Example:

			\begin{itemize}
				\item Page UID 2 is a shortcut to page UID 1
				\item TypoScript configuration set: \texttt{config.linkVars = L}
			\end{itemize}

		\item \underline{Old} behaviour:\newline
			\smaller
				\tabto{0.5cm}\texttt{http://example.com?id=2\&L=1\&customparam=X}\newline
				redirects to:\newline
				\tabto{0.5cm}\texttt{http://example.com?id=1\&L=1}
			\normalsize

		\item \underline{New} behaviour:\newline
			\smaller
				\tabto{0.5cm}\texttt{http://example.com?id=2\&L=1\&customparam=X}\newline
				redirects to:\newline
				\tabto{0.5cm}\texttt{http://example.com?id=1\&L=1\&customparam=X}
			\normalsize

	\end{itemize}

\end{frame}

% ------------------------------------------------------------------------------
% LTXE-SLIDE-START
% LTXE-SLIDE-UID:		cb204fc7-e09a4646-8d689723-52dcae93
% LTXE-SLIDE-ORIGIN:	3b3fcb91-081940b0-2ec33de4-ef56f577 English
% LTXE-SLIDE-TITLE:		#75031 and #75032: Fluidification of TypoScriptTemplate...ModuleFunctionController
% ------------------------------------------------------------------------------

\begin{frame}[fragile]
	\frametitle{In-Depth Changes}
	\framesubtitle{Fluidification}

	\begin{itemize}
		\item HTML code has been migrated from PHP code to a Fluid template
		\item Affected methods:\newline

			\smaller\texttt{TypoScriptTemplateInformationModuleFunctionController\newline
				->tableRow()}\normalsize\newline
			\smaller\texttt{TypoScriptTemplateConstantEditorModuleFunctionController\newline
				->displayExample()}\normalsize

		\item Calling these methods results in a fatal error now

	\end{itemize}

\end{frame}

% ------------------------------------------------------------------------------
% LTXE-SLIDE-START
% LTXE-SLIDE-UID:		77025ded-afaebb08-329018a2-663ebaf5
% LTXE-SLIDE-ORIGIN:	5f2bab62-fc48a5b2-0f70e9b0-834e47ce English
% LTXE-SLIDE-TITLE:		PageRenderer and ResourceCompressor support EXT: syntax
% LTXE-SLIDE-REFERENCE:	#77589: EXT: syntax in PageRenderer and Compressor
% ------------------------------------------------------------------------------

\begin{frame}[fragile]
	\frametitle{In-Depth Changes}
	\framesubtitle{PageRenderer and Compressor}

	% decrease font size for code listing
	\lstset{basicstyle=\smaller\ttfamily}

	\begin{itemize}

		\item PageRenderer and ResourceCompressor PHP classes now support the
			\texttt{EXT:} syntax for referencing JS and CSS files inside extension
			directories.\newline
			\textbf{Previously:}

			\begin{lstlisting}
				$this->pageRenderer->addJsFile(
				  ExtensionManagementUtility::extRelPath('myextension') .
				  'Resources/Public/JavaScript/example.js'
				);
			\end{lstlisting}

			\textbf{Now possible:}

			\begin{lstlisting}
				$this->pageRenderer->addJsFile(
				  'EXT:myextension/Resources/Public/JavaScript/example.js'
				);
			\end{lstlisting}

	\end{itemize}

\end{frame}

% ------------------------------------------------------------------------------
% LTXE-SLIDE-START
% LTXE-SLIDE-UID:		91006579-bb73be3b-4430662b-5c07d4ff
% LTXE-SLIDE-ORIGIN:	081940b0-ef56f577-3b3fcb91-0b008194 English
% LTXE-SLIDE-TITLE:		Miscellaneous (1) (#77700, #77750, #77814 and #78222)
% LTXE-SLIDE-REFERENCE:	#77700: EXT:indexed_search_mysql merged into EXT:indexed_search
% LTXE-SLIDE-REFERENCE:	#77750: Return value of ContentObjectRenderer::exec_Query changed
% LTXE-SLIDE-REFERENCE:	#77814: Remove feature subsearch from indexed search
% LTXE-SLIDE-REFERENCE:	#78222: Extension autoload information is now in typo3conf/autoload
% ------------------------------------------------------------------------------

\begin{frame}[fragile]
	\frametitle{In-Depth Changes}
	\framesubtitle{Miscellaneous (1)}

	\begin{itemize}

		\item EXT:indexed\_search\_mysql merged into EXT:indexed\_search

		\item Feature "subsearch" has been removed from EXT:indexed\_search\_mysql\newline
			\smaller
				(TypoScript option \texttt{plugin.tx\_indexedsearch.clearSearchBox} removed, too)
			\normalsize

		\item Return type of \texttt{ContentObjectRenderer::exec\_Query()} changed\newline
			\smaller
				(return value is always
					\texttt{\textbackslash
						Doctrine\textbackslash
						DBAL\textbackslash
						Driver\textbackslash
						Statement}
					now)
			\normalsize

		\item To make it clear that autoload information is not a cache, the files
			have been moved from \texttt{typo3temp/} to \texttt{typo3conf/}\newline
			\smaller
				Note: TYPO3 deployments, which do not take advantage of composer, possibly
				need some adjustments to take the new location into account.
			\normalsize

	\end{itemize}

\end{frame}

% ------------------------------------------------------------------------------

% ------------------------------------------------------------------------------
% TYPO3 CMS 8.4 - What's New - Chapter "In-Depth Changes" (German Version)
%
% @author	Michael Schams and Patrick Lobacher
% @license	Creative Commons BY-NC-SA 3.0
% @link		http://typo3.org/download/release-notes/whats-new/
% @language	German
% ------------------------------------------------------------------------------
% LTXE-CHAPTER-UID:		5ebcecbe-66abfa57-cf38bc00-aa637965
% LTXE-CHAPTER-NAME:	In-Depth Changes
% ------------------------------------------------------------------------------

\section{Änderungen im System}
\begin{frame}[fragile]
	\frametitle{Änderungen im System}

	\begin{center}\huge{Kapitel 3:}\end{center}
	\begin{center}\huge{\color{typo3darkgrey}\textbf{Änderungen im System}}\end{center}

\end{frame}

% ------------------------------------------------------------------------------
% LTXE-SLIDE-START
% LTXE-SLIDE-UID:		f43ae4a7-e150f2de-ac8d21f6-5c9df7f1
% LTXE-SLIDE-ORIGIN:	bde270e6-ffef8544-ea472ed5-89ba8c3d English
% LTXE-SLIDE-TITLE:		#52877: Remove ExtJS Viewport
% ------------------------------------------------------------------------------

\begin{frame}[fragile]
	\frametitle{Änderungen im System}
	\framesubtitle{ExtJS-Entfernung}

	\begin{itemize}
		\item Die ExtJS-Komponente \texttt{TYPO3.Viewport} wurde entfernt.
		\item \texttt{Ext.layout} und \texttt{Ext.Viewport} werden nicht mehr im Backend-Viewport verwendet.
		\item Die bisherige Funktionalität wurde mittels nativem JavaScript, jQuery und CSS reimplementiert.
		\item Die ExtJS-Notification-Komponenten \texttt{TYPO3.Window} und \texttt{TYPO3.Dialog} wurden entfernt.
		\item Verbleibende Teile/Aufgaben einer kompletten ExtJS-Entfernung:

		\begin{itemize}
			\item Seitenbaum
			\item Drag'n Drop Funktionalität der Form Engine
			\item ExtDirect Funktionalität
		\end{itemize}

	\end{itemize}

\end{frame}

% ------------------------------------------------------------------------------
% LTXE-SLIDE-START
% LTXE-SLIDE-UID:		93531561-92c37f47-1c6137d7-762abae4
% LTXE-SLIDE-ORIGIN:	11a9285f-18bdbfac-88ece0c7-b3f2276a English
% LTXE-SLIDE-TITLE:		Doctrine DBAL
% ------------------------------------------------------------------------------

\begin{frame}[fragile]
	\frametitle{Änderungen im System}
	\framesubtitle{Doctrine DBAL}

	\begin{itemize}
		\item Es wurden weitere Fortschritte damit gemacht, alle Datenbank-Aufrufe des TYPO3-Kerns durch Doctrine DBAL realisieren zu lassen
		\item Die Extbase Persistence-Schicht verwendet im QueryBuilder nun schon  Doctrine DBAL
		\item \texttt{EXT:dbal} und \texttt{EXT:adodb} wurden aus dem TYPO3-Kern entfernt\newline
			\smaller
				Sollte eine 3rd-Party Extension die alte \texttt{TYPO3\_DB} API verwenden wollen, um auf Nicht-MySQL-Datenbanktabellen zuzugreifen, können die obigen Extensions aus dem TER installiert werden.
			\normalsize

		\item Der Zugriff mittels \texttt{TYPO3\_DB} wurde aus den meisten TYPO3-Kernklassen entfernt\newline
			\smaller
				(der Zugriff per \texttt{\$GLOBALS[TYPO3\_DB]} ist zwar möglich, wird aber nicht empfohlen)
			\normalsize

	\end{itemize}

\end{frame}

% ------------------------------------------------------------------------------
% LTXE-SLIDE-START
% LTXE-SLIDE-UID:		7408cd34-8a83f4bf-4a95bdae-2563ab06
% LTXE-SLIDE-ORIGIN:	bb5c02d9-8dddd379-e73500a9-9534588b English
% LTXE-SLIDE-TITLE:		#77900: TYPO3 CMS supports TypeScript (1)
% ------------------------------------------------------------------------------

\begin{frame}[fragile]
	\frametitle{Änderungen im System}
	\framesubtitle{TypeScript Support (1)}

	\begin{itemize}
		\item \textbf{TypeScript} wurde im TYPO3 Kern für das interne JavaScript-Handling zugefügt.
		\item TypeScript ist eine freie und unter Open Source Lizenz stehende Programmiersprache von Microsoft.
		\item Es ist ein Superset von JavaScript, welches in der Lage ist, JavaScript zu kompilieren.
		\item Ein Grunt-Task kompiliert jede TypeScript-Datei (.ts) in eine JavaScript-Datei (.js) und produziert ein AMD-Module
	\end{itemize}

\end{frame}
% ------------------------------------------------------------------------------
% LTXE-SLIDE-START
% LTXE-SLIDE-UID:		a4aef27f-ad43dac0-e3c0fee1-9afce1db
% LTXE-SLIDE-ORIGIN:	bb5c02d9-8dddd379-e73500a9-9534588b English
% LTXE-SLIDE-TITLE:		#77900: TYPO3 CMS supports TypeScript (2)
% ------------------------------------------------------------------------------

\begin{frame}[fragile]
	\frametitle{Änderungen im System}
	\framesubtitle{TypeScript Support (2)}

	\begin{itemize}
		\item Alle in TYPO3 CMS enthaltenen AMD-Module müssen zu TypeScript portiert werden um die zukunftsfähigkeit des JavaScript Handlings sicher zu stellen
		\item Ziel ist es, alle AMD-Module noch von der Veröffentlichung von CMS 8 LTS in TypeScript zu konvertieren
		\item Mehr Details hierzu gibt es unter: \url{https://www.typescriptlang.org}
	\end{itemize}

\end{frame}


% ------------------------------------------------------------------------------
% LTXE-SLIDE-START
% LTXE-SLIDE-UID:		6198f54c-d73166b2-a3d8af28-14019df7
% LTXE-SLIDE-ORIGIN:	d467c514-583b44e5-03e4927f-38ba0ad1 English
% LTXE-SLIDE-TITLE:		#77900: TYPO3 CMS supports TypeScript (3)
% ------------------------------------------------------------------------------

\begin{frame}[fragile]
	\frametitle{Änderungen im System}
	\framesubtitle{TypeScript Support (3)}

	\begin{itemize}
		\item Die wichtigsten TypeScript-Regeln werden in einem Regelwerk definiert, welches der TypeScript-Linter überprüft:

			\begin{itemize}
				\item Definiere immer Typen und Rückgabe-Typen, selbst wenn TypeScript einen Default-Typen zur Verfügung stellt.
				\item Variable Scoping: bevorzugt \texttt{let} anstelle von \texttt{var}.
				\item Optionale Eigenschaften in Interfaces sind im Core nicht erlaubt.
				\item Ein Interface erweitert niemals eine Klasse
				\item Iterables: bevorzugt \texttt{for (i of list)} anstelle von \texttt{for (i in list)}.
				\item Verwende das Keyword \texttt{implements}, selbst wenn TypeScript dies nicht benötigt.
				\item Jede Klasse bzw. jedes Interface muss mit "export" deklariert werden, um sicherzustellen, dass dieses wiederverwendet bzw. exportiert werden kann.
			\end{itemize}

			\small(es können noch nicht alle Regeln durch den Linter überprüft werden)\normalsize

	\end{itemize}

\end{frame}

% ------------------------------------------------------------------------------
% LTXE-SLIDE-START
% LTXE-SLIDE-UID:		900606b7-ab9749cc-d21bc5bb-84af6cef
% LTXE-SLIDE-ORIGIN:	2e0d0d9e-44771297-efe747f4-89c8b960 English
% LTXE-SLIDE-TITLE:		#38496: Shortcuts take all URL parameters into account
% ------------------------------------------------------------------------------

\begin{frame}[fragile]
	\frametitle{Änderungen im System}
	\framesubtitle{URL Parameter in Shortcuts}

	\begin{itemize}
		\item Shortcuts beachten nun alle URL-Parameter.
		\item Beispiel:

			\begin{itemize}
				\item Page UID 2 ist ein Shortcut zur Seite UID 1
				\item TypoScript Konfiguration: \texttt{config.linkVars = L}
			\end{itemize}

		\item \underline{Altes} Verhalten:\newline
			\smaller
				\tabto{0.5cm}\texttt{http://example.com?id=2\&L=1\&customparam=X}\newline
				Weiterleitung zu:\newline
				\tabto{0.5cm}\texttt{http://example.com?id=1\&L=1}
			\normalsize

		\item \underline{Neues} Verhalten:\newline
			\smaller
				\tabto{0.5cm}\texttt{http://example.com?id=2\&L=1\&customparam=X}\newline
				Weiterleitung zu:\newline
				\tabto{0.5cm}\texttt{http://example.com?id=1\&L=1\&customparam=X}
			\normalsize

	\end{itemize}

\end{frame}

% ------------------------------------------------------------------------------
% LTXE-SLIDE-START
% LTXE-SLIDE-UID:		e9acc1a7-87652206-d66e27d9-7d7f4aee
% LTXE-SLIDE-ORIGIN:	3b3fcb91-081940b0-2ec33de4-ef56f577 English
% LTXE-SLIDE-TITLE:		#75031 and #75032: Fluidification of TypoScriptTemplate...ModuleFunctionController
% ------------------------------------------------------------------------------

\begin{frame}[fragile]
	\frametitle{Änderungen im System}
	\framesubtitle{Fluidification}

	\begin{itemize}
		\item HTML-Code wurde von PHP zu Fluid migriert.
		\item Betroffene Methoden:\newline

			\smaller\texttt{TypoScriptTemplateInformationModuleFunctionController\newline
				->tableRow()}\normalsize\newline
			\smaller\texttt{TypoScriptTemplateConstantEditorModuleFunctionController\newline
				->displayExample()}\normalsize

		\item Der Aufruf dieser Methoden resultiert in einem Fatal-Error.

	\end{itemize}

\end{frame}

% ------------------------------------------------------------------------------
% LTXE-SLIDE-START
% LTXE-SLIDE-UID:		94d5e64b-6833a908-d317e2ad-58b6c96b
% LTXE-SLIDE-ORIGIN:	5f2bab62-fc48a5b2-0f70e9b0-834e47ce English
% LTXE-SLIDE-TITLE:		PageRenderer and ResourceCompressor support EXT: syntax
% LTXE-SLIDE-REFERENCE:	#77589: EXT: syntax in PageRenderer and Compressor
% ------------------------------------------------------------------------------

\begin{frame}[fragile]
	\frametitle{Änderungen im System}
	\framesubtitle{PageRenderer und Compressor}

	% decrease font size for code listing
	\lstset{basicstyle=\smaller\ttfamily}

	\begin{itemize}

		\item Die \texttt{PageRenderer} und \texttt{ResourceCompressor} PHP-Klassem unterstützen nun die
			\texttt{EXT:} Syntax, um auf JS- und CSS-Dateien innerhalb von Extension-Verzeichnissen zu referenzieren.\newline
			\textbf{Vorher:}

			\begin{lstlisting}
				$this->pageRenderer->addJsFile(
				  ExtensionManagementUtility::extRelPath('myextension') .
				  'Resources/Public/JavaScript/example.js'
				);
			\end{lstlisting}

			\textbf{Nun möglich:}

			\begin{lstlisting}
				$this->pageRenderer->addJsFile(
				  'EXT:myextension/Resources/Public/JavaScript/example.js'
				);
			\end{lstlisting}

	\end{itemize}

\end{frame}

% ------------------------------------------------------------------------------
% LTXE-SLIDE-START
% LTXE-SLIDE-UID:		03b9e718-4b7e9531-848c4a3f-ada3825f
% LTXE-SLIDE-ORIGIN:	081940b0-ef56f577-3b3fcb91-0b008194 English
% LTXE-SLIDE-TITLE:		Miscellaneous (1) (#77700, #77750, #77814 and #78222)
% LTXE-SLIDE-REFERENCE:	#77700: EXT:indexed_search_mysql merged into EXT:indexed_search
% LTXE-SLIDE-REFERENCE:	#77750: Return value of ContentObjectRenderer::exec_Query changed
% LTXE-SLIDE-REFERENCE:	#77814: Remove feature subsearch from indexed search
% LTXE-SLIDE-REFERENCE:	#78222: Extension autoload information is now in typo3conf/autoload
% ------------------------------------------------------------------------------

\begin{frame}[fragile]
	\frametitle{Änderungen im System}
	\framesubtitle{Miscellaneous (1)}

	\begin{itemize}

		\item Die Extension EXT:indexed\_search\_mysql wurde in die Extension EXT:indexed\_search integriert

		\item Das Feature "subsearch" wurde von EXT:indexed\_search\_mysql entfert\newline
			\smaller
				(Die TypoScript-Option \texttt{plugin.tx\_indexedsearch.clearSearchBox} wurde ebenfalls entfernt)
			\normalsize

		\item Der Rückgabe-Typ von \texttt{ContentObjectRenderer::exec\_Query()} wurde geändert\newline
			\smaller
				(der Rückgabe werde ist nun immer vom Typ
					\texttt{\textbackslash
						Doctrine\textbackslash
						DBAL\textbackslash
						Driver\textbackslash
						Statement}
					)
			\normalsize

	\end{itemize}

\end{frame}
% ------------------------------------------------------------------------------
% LTXE-SLIDE-START
% LTXE-SLIDE-UID:		07bcdbdd-0248b7c1-ff40e727-e5a01dfd
% LTXE-SLIDE-ORIGIN:	081940b0-ef56f577-3b3fcb91-0b008194 English
% LTXE-SLIDE-TITLE:		Miscellaneous (2) (#77700, #77750, #77814 and #78222)
% LTXE-SLIDE-REFERENCE:	#77700: EXT:indexed_search_mysql merged into EXT:indexed_search
% LTXE-SLIDE-REFERENCE:	#77750: Return value of ContentObjectRenderer::exec_Query changed
% LTXE-SLIDE-REFERENCE:	#77814: Remove feature subsearch from indexed search
% LTXE-SLIDE-REFERENCE:	#78222: Extension autoload information is now in typo3conf/autoload
% ------------------------------------------------------------------------------

\begin{frame}[fragile]
	\frametitle{Änderungen im System}
	\framesubtitle{Miscellaneous (2)}

	\begin{itemize}

		\item Um zu verdeutlichen, dass Autoload-Information keinen Cache darstellen, wurde die zugehörige
			Datei von \texttt{typo3temp/} nach \texttt{typo3conf/} verschoben.\newline
			\smaller
				Achtung: TYPO3 Deployments, welche nicht auf Composer basieren, müssen ggf. manuell angepasst werden
			\normalsize

	\end{itemize}

\end{frame}


% ------------------------------------------------------------------------------

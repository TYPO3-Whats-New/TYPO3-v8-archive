% ------------------------------------------------------------------------------
% TYPO3 CMS 8.4 - What's New - Chapter "In-Depth Changes" (Dutch Version)
%
% @author	Michael Schams <schams.net>
% @license	Creative Commons BY-NC-SA 3.0
% @link		http://typo3.org/download/release-notes/whats-new/
% @language	English
% ------------------------------------------------------------------------------
% LTXE-CHAPTER-UID:		5ebcecbe-66abfa57-cf38bc00-aa637965
% LTXE-CHAPTER-NAME:	In-Depth Changes
% ------------------------------------------------------------------------------

\section{Systeemwijzigingen}
\begin{frame}[fragile]
	\frametitle{Systeemwijzigingen}

	\begin{center}\huge{Hoofdstuk 3:}\end{center}
	\begin{center}\huge{\color{typo3darkgrey}\textbf{Systeemwijzigingen}}\end{center}

\end{frame}

% ------------------------------------------------------------------------------
% LTXE-SLIDE-START
% LTXE-SLIDE-UID:		eb3eb340-5fa1d291-bbdb05e6-5bdf76d1
% LTXE-SLIDE-ORIGIN:	bde270e6-ffef8544-ea472ed5-89ba8c3d English
% LTXE-SLIDE-TITLE:		#52877: Remove ExtJS Viewport
% ------------------------------------------------------------------------------

\begin{frame}[fragile]
	\frametitle{Systeemwijzigingen}
	\framesubtitle{Verwijdering van ExtJS}

	\begin{itemize}
		\item ExtJS-component \texttt{TYPO3.Viewport} is verwijderd
		\item \texttt{Ext.layout} en \texttt{Ext.Viewport} worden niet meer gebruikt in de backend
		\item De functionaliteit is opnieuw gebouwd met eigen JavaScript, jQuery en CSS
		\item ExtJS-componenten \texttt{TYPO3.Window} en \texttt{TYPO3.Dialog} (voor meldingen) zijn verwijderd
		\item Resterende delen/taken voor het volledig verwijderen van ExtJS:

		\begin{itemize}
			\item paginaboom
			\item sleepfunctionaliteit van de form-extensie
			\item ExtDirect-functionaliteiten
		\end{itemize}

	\end{itemize}

\end{frame}

% ------------------------------------------------------------------------------
% LTXE-SLIDE-START
% LTXE-SLIDE-UID:		4c58ca1c-07abc547-30f08edc-b3c75e99
% LTXE-SLIDE-ORIGIN:	11a9285f-18bdbfac-88ece0c7-b3f2276a English
% LTXE-SLIDE-TITLE:		Doctrine DBAL
% ------------------------------------------------------------------------------

\begin{frame}[fragile]
	\frametitle{Systeemwijzigingen}
	\framesubtitle{Doctrine DBAL}

	\begin{itemize}
		\item Er zijn vorderingen gemaakt met de migratie van alle aanroepen naar de database vanuit 
			de TYPO3-core naar Doctrine DBAL
		\item De persistentie van Extbase is nu volledig gebaseerd op Doctrine DBAL QueryBuilder
		\item \texttt{EXT:dbal} en \texttt{EXT:adodb} zijn uit de TYPO3-core verwijderd\newline
			\smaller
				Als extensies van derden de oude \texttt{TYPO3\_DB}-API gebruiken voor query's naar
				niet-MySQL-databasetabellen, dan kunnen deze twee extensies via de TER worden geïnstalleerd.
			\normalsize

		\item \texttt De {TYPO3\_DB}-functionaliteit is grotendeels uit de PHP-klassen van de core verwijderd\newline
			\smaller
				(het gebruik van \texttt{\$GLOBALS[TYPO3\_DB]} is nog steeds mogelijk, maar wordt afgeraden)
			\normalsize

	\end{itemize}

\end{frame}

% ------------------------------------------------------------------------------
% LTXE-SLIDE-START
% LTXE-SLIDE-UID:		6e6ef973-ea0707ce-f0eb9338-53d6bd7c
% LTXE-SLIDE-ORIGIN:	bb5c02d9-8dddd379-e73500a9-9534588b English
% LTXE-SLIDE-TITLE:		#77900: TYPO3 CMS supports TypeScript (1)
% ------------------------------------------------------------------------------

\begin{frame}[fragile]
	\frametitle{Systeemwijzigingen}
	\framesubtitle{TypeScript-ondersteuning (1)}

	\begin{itemize}
		\item Voor de interne afhandeling van JavaScript in de TYPO3-core is \textbf{TypeScript} geïntroduceerd 
		\item TypeScript is een gratis en open source programmeertaal die is ontwikkeld en wordt onderhouden door Microsoft. 
		\item Het is een stricte superset van JavaScript, die JavaScript kan compileren
		\item Kijk voor meer details op: \url{https://www.typescriptlang.org}
		\item Een grunt-taak compileert elk TypeScriptbestand (.ts) in een JavaScriptbestand (.js) 
				en produceert een AMD-module
	\end{itemize}

	\small
		Opmerking: alle bestaande AMD-modulen in TYPO3 CMS moeten aan TypeScript worden aangepast om 
		een toekomstbestendige afhandeling van JavaScript te garanderen.
		Het doel is om voor de release van CMS 8 LTS alle AMD-modulen naar TypeScript te migreren.
	\normalsize

\end{frame}

% ------------------------------------------------------------------------------
% LTXE-SLIDE-START
% LTXE-SLIDE-UID:		edb7e24c-4ff332ce-551fb7ef-dc2126de
% LTXE-SLIDE-ORIGIN:	d467c514-583b44e5-03e4927f-38ba0ad1 English
% LTXE-SLIDE-TITLE:		#77900: TYPO3 CMS supports TypeScript (2)
% ------------------------------------------------------------------------------

\begin{frame}[fragile]
	\frametitle{Systeemwijzigingen}
	\framesubtitle{TypeScript-ondersteuning (2)}

	\begin{itemize}
		\item De belangrijkste regels voor TypeScript zijn gedefinieerd in een verzameling regels 
			waarvan een deel door TypeScript Linter worden gecontroleerd:

			\begin{itemize}
				\item Definieer altijd alle typen en terugkeertypen, zelfs als er in TypeScript een 
					standaardtype voorhanden is
				\item Bereik van variabelen: prefereer \texttt{let} in plaats van \texttt{var}
				\item Optionele eigenschappen in de interface zijn niet toegestaan in de core.
				\item Een interface is nooit een uitbreiding op een klasse
				\item Iterators: gebruik \texttt{for(i of list)} in plaats van \texttt{for(i in list)}
				\item Gebruik het keyword \texttt{implements}, zelfs als TypeScript dat niet vereist
				\item Elke klasse of interface moet met "export" worden gedeclareerd om hergebruik en 
					export van instanties van een object mogelijk te maken
			\end{itemize}

	\end{itemize}

\end{frame}

% ------------------------------------------------------------------------------
% LTXE-SLIDE-START
% LTXE-SLIDE-UID:		10f9cc96-1b52da9f-fb04d1f6-c84a69d0
% LTXE-SLIDE-ORIGIN:	2e0d0d9e-44771297-efe747f4-89c8b960 English
% LTXE-SLIDE-TITLE:		#38496: Shortcuts take all URL parameters into account
% ------------------------------------------------------------------------------

\begin{frame}[fragile]
	\frametitle{Systeemwijzigingen}
	\framesubtitle{URL-parameters in snelkoppelingen}

	\begin{itemize}
		\item Snelkoppelingen houden nu rekening met alle parameters in een URL.
		\item Bijvoorbeeld:

			\begin{itemize}
				\item Pagina met UID 2 is een snelkoppeling naar pagina met UID 1
				\item TypoScript-configuratie: \texttt{config.linkVars = L}
			\end{itemize}

		\item \underline{Oud} gedrag:\newline
			\smaller
				\tabto{0.5cm}\texttt{http://example.com?id=2\&L=1\&customparam=X}\newline
					redirect naar:\newline
				\tabto{0.5cm}\texttt{http://example.com?id=1\&L=1}
			\normalsize

		\item \underline{Nieuw} gedrag:\newline
			\smaller
				\tabto{0.5cm}\texttt{http://example.com?id=2\&L=1\&customparam=X}\newline
					redirect naar:\newline
				\tabto{0.5cm}\texttt{http://example.com?id=1\&L=1\&customparam=X}
			\normalsize

	\end{itemize}

\end{frame}

% ------------------------------------------------------------------------------
% LTXE-SLIDE-START
% LTXE-SLIDE-UID:		8bd39959-39f7a174-cce50cd7-2f4574aa
% LTXE-SLIDE-ORIGIN:	3b3fcb91-081940b0-2ec33de4-ef56f577 English
% LTXE-SLIDE-TITLE:		#75031 and #75032: Fluidification of TypoScriptTemplate...ModuleFunctionController
% ------------------------------------------------------------------------------

\begin{frame}[fragile]
	\frametitle{Systeemwijzigingen}
	\framesubtitle{Fluidificatie}

	\begin{itemize}
		\item HTML-code is gemigreerd van PHP-code naar een Fluid-sjabloon
		\item Getroffen methoden:\newline

			\smaller\texttt{TypoScriptTemplateInformationModuleFunctionController\newline
				->tableRow()}\normalsize\newline
			\smaller\texttt{TypoScriptTemplateConstantEditorModuleFunctionController\newline
				->displayExample()}\normalsize

		\item Het aanroepen van deze methoden leidt nu tot een fatal error

	\end{itemize}

\end{frame}

% ------------------------------------------------------------------------------
% LTXE-SLIDE-START
% LTXE-SLIDE-UID:		396184e8-246ebc26-0cbdc68e-627aea23
% LTXE-SLIDE-ORIGIN:	5f2bab62-fc48a5b2-0f70e9b0-834e47ce English
% LTXE-SLIDE-TITLE:		PageRenderer and ResourceCompressor support EXT: syntax
% LTXE-SLIDE-REFERENCE:	#77589: EXT: syntax in PageRenderer and Compressor
% ------------------------------------------------------------------------------

\begin{frame}[fragile]
	\frametitle{Systeemwijzigingen}
	\framesubtitle{PageRenderer en Compressor}

	% decrease font size for code listing
	\lstset{basicstyle=\smaller\ttfamily}

	\begin{itemize}

		\item De PHP-klassen PageRenderer en ResourceCompressor ondersteunen nu de \texttt{EXT:}-syntax 
			voor het verwijzen naar naar JS- en CSS-bestanden binnen een extensie.\newline
			\textbf{Vroeger:}

			\begin{lstlisting}
				$this->pageRenderer->addJsFile(
				  ExtensionManagementUtility::extRelPath('myextension') .
				  'Resources/Public/JavaScript/example.js'
				);
			\end{lstlisting}

			\textbf{Nu mogelijk:}

			\begin{lstlisting}
				$this->pageRenderer->addJsFile(
				  'EXT:myextension/Resources/Public/JavaScript/example.js'
				);
			\end{lstlisting}

	\end{itemize}

\end{frame}

% ------------------------------------------------------------------------------
% LTXE-SLIDE-START
% LTXE-SLIDE-UID:		b9b8da21-9715069f-d545f69b-40a575b2
% LTXE-SLIDE-ORIGIN:	081940b0-ef56f577-3b3fcb91-0b008194 English
% LTXE-SLIDE-TITLE:		Miscellaneous (1) (#77700, #77750, #77814 and #78222)
% LTXE-SLIDE-REFERENCE:	#77700: EXT:indexed_search_mysql merged into EXT:indexed_search
% LTXE-SLIDE-REFERENCE:	#77750: Return value of ContentObjectRenderer::exec_Query changed
% LTXE-SLIDE-REFERENCE:	#77814: Remove feature subsearch from indexed search
% LTXE-SLIDE-REFERENCE:	#78222: Extension autoload information is now in typo3conf/autoload
% ------------------------------------------------------------------------------

\begin{frame}[fragile]
	\frametitle{Systeemwijzigingen}
	\framesubtitle{Divers (1)}

	\begin{itemize}

		\item EXT:indexed\_search\_mysql is samengevoegd met EXT:indexed\_search

		\item Kenmerk "subsearch" is uit EXT:indexed\_search\_mysql verwijderd\newline
			\smaller
				(ook de TypoScript-optie \texttt{plugin.tx\_indexedsearch.clearSearchBox} is verwijderd)
			\normalsize

		\item Het resultaattype van \texttt{ContentObjectRenderer::exec\_Query()} is gewijzigd\newline
			\smaller
				(het resultaat is nu altijd
					\texttt{\textbackslash
						Doctrine\textbackslash
						DBAL\textbackslash
						Driver\textbackslash
						Statement})
			\normalsize

		\item Om duidelijk te maken dat autoload-informatie geen cache is, 
			zijn de bestanden verplaatst van \texttt{typo3temp/} naar \texttt{typo3conf/}\newline
			\smaller
				Opmerking: TYPO3-implementaties die geen gebruik maken van Composer, 
					moeten mogelijk een aantal aanpassingen doorvoeren om rekening te houden met de nieuwe locatie.
			\normalsize

	\end{itemize}

\end{frame}

% ------------------------------------------------------------------------------

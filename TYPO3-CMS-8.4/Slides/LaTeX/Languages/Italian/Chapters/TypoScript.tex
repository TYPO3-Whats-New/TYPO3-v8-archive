% ------------------------------------------------------------------------------
% TYPO3 CMS 8.4 - What's New - Chapter "TypoScript" (Italian Version)
%
% @author	Michael Schams <schams.net>
% @license	Creative Commons BY-NC-SA 3.0
% @link		http://typo3.org/download/release-notes/whats-new/
% @language	English
% ------------------------------------------------------------------------------
% LTXE-CHAPTER-UID:		3a9852ea-e2360d9d-1ff5eec1-a7de3f9f
% LTXE-CHAPTER-NAME:	TypoScript
% ------------------------------------------------------------------------------

\section{TSconfig \& TypoScript}
\begin{frame}[fragile]
	\frametitle{TSconfig \& TypoScript}

	\begin{center}\huge{Capitolo 2:}\end{center}
	\begin{center}\huge{\color{typo3darkgrey}\textbf{TSconfig \& TypoScript}}\end{center}

\end{frame}

% ------------------------------------------------------------------------------
% LTXE-SLIDE-START
% LTXE-SLIDE-UID:		e8cf8f2e-3d77ad12-d46ec402-ff75c834
% LTXE-SLIDE-ORIGIN:	ef0aacb4-ab905cfb-43240134-014bdb32 English
% LTXE-SLIDE-TITLE:		#77592 and #76748
% LTXE-SLIDE-REFERENCE:	#77592: Dropped TCA option showIfRTE in type=check
% LTXE-SLIDE-REFERENCE:	#76748: Configure the availability of the elementbrowser
% ------------------------------------------------------------------------------

\begin{frame}[fragile]
	\frametitle{TSconfig \& TypoScript}
	\framesubtitle{Opzione TCA \texttt{showIfRTE}}

	% decrease font size for code listing
	\lstset{basicstyle=\tiny\ttfamily}

	\begin{itemize}
		\item L'opzione TCA \texttt{showIfRTE} per \texttt{type=check} è stata rimossa dal TCA di tutti i campi
		\item La disponibilità dell'Element Browser può essere configurata sulla base del singolo utente\newline
			\smaller
				Per disabilitare il bottone puoi usare le impostazioni TCA:
			\normalsize

			\begin{lstlisting}
				[table_name]['columns'][field_name]['config']['appearance']['elementBrowserEnabled'] = false;
			\end{lstlisting}

			\smaller
				Per disabilitare il bottone puoi usare le impostazioni pageTs:
			\normalsize

			\begin{lstlisting}
				TCEFORM.table_name.field_name.config.appearance.elementBrowserEnabled = 0
			\end{lstlisting}

			\smaller
				Per disabilitare il bottone puoi usare le impostazioni userTs:
			\normalsize

			\begin{lstlisting}
				page.TCEFORM.table_name.field_name.config.appearance.elementBrowserEnabled = 0
			\end{lstlisting}

	\end{itemize}

\end{frame}

% ------------------------------------------------------------------------------
% LTXE-SLIDE-START
% LTXE-SLIDE-UID:		76210e97-e7680df1-0032b468-2bf86812
% LTXE-SLIDE-ORIGIN:	b287e11b-2386594e-3ad7e012-cc1cdef2 English
% LTXE-SLIDE-TITLE:		#17309: Access FlexForm values
% ------------------------------------------------------------------------------

\begin{frame}[fragile]
	\frametitle{TSconfig \& TypoScript}
	\framesubtitle{Accesso variabili FlexForm}

	% decrease font size for code listing
	\lstset{basicstyle=\tiny\ttfamily}

	\begin{itemize}

		\item Adesso è possibile accedere alle variabili dei campi di un FlexForm:

		\begin{lstlisting}
			lib.flexformContent = CONTENT
			lib.flexformContent {
			  table = tt_content
			  select {
			    pidInList = this
			  }

			  renderObj = COA
			  renderObj {
			    10 = TEXT
			    10 {
			      data = flexform: pi_flexform:settings.categories
			    }
			  }
			}
		\end{lstlisting}

	\end{itemize}

\end{frame}

% ------------------------------------------------------------------------------

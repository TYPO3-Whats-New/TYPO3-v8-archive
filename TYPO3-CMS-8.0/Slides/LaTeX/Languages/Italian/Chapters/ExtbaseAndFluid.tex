% ------------------------------------------------------------------------------
% TYPO3 CMS 7.6 - What's New - Chapter "Extbase & Fluid" (English Version)
%
% @author	Patrick Lobacher <patrick@lobacher.de> and Michael Schams <schams.net>
% @license	Creative Commons BY-NC-SA 3.0
% @link		http://typo3.org/download/release-notes/whats-new/
% @language	English
% ------------------------------------------------------------------------------
% LTXE-CHAPTER-UID:		cfcf5377-f3e96f39-d1ba01e9-2f341646
% LTXE-CHAPTER-NAME:	Extbase & Fluid
% ------------------------------------------------------------------------------

\section{Extbase \& Fluid}
\begin{frame}[fragile]
	\frametitle{Extbase \& Fluid}

	\begin{center}\huge{Capitolo 4:}\end{center}
	\begin{center}\huge{\color{typo3darkgrey}\textbf{Extbase \& Fluid}}\end{center}

\end{frame}

% ------------------------------------------------------------------------------
% LTXE-SLIDE-START
% LTXE-SLIDE-UID:		4ff82ec5-19851cf7-43da82ad-4191fd26
% LTXE-SLIDE-ORIGIN:	9413e7f2-dd08f22e-dd21ee20-d7a1b0bf English
% LTXE-SLIDE-TITLE:		Standalone revised Fluid
% LTXE-SLIDE-REFERENCE:	!Feature-69863-UseNewStandaloneFluidAsComposerDependency.rst
% ------------------------------------------------------------------------------

\begin{frame}[fragile]
	\frametitle{Extbase \& Fluid}
	\framesubtitle{Standalone revised Fluid}

	% decrease font size for code listin
	\lstset{basicstyle=\tiny\ttfamily}

	\begin{itemize}

		\item Il motore di rendering Fluid di TYPO3 CMS è sostituito dal pacchetto
			Fluid standalone che è incluso come una dipendenza di composer

		\item La vecchia estensione Fluid è convertita con il così chiamato \textit{Fluid adapter}
			che permette a TYPO3 CMS di usare Fluid standalone

		\item Nuove caratteristiche/funzionalità sono state aggiunte in tutte le aree di Fluid

		\item Molto importante: molti dei componenti di Fluid che erano completamente interni
			ed impossibili da sostituire in passato, sono ora facilmente sostituibili
			e sono stati dotati di un API pubblica.

	\end{itemize}

\end{frame}

% ------------------------------------------------------------------------------
% LTXE-SLIDE-START
% LTXE-SLIDE-UID:		f105cfa5-e76e12f4-19b98cf7-1ef162e1
% LTXE-SLIDE-ORIGIN:	32daa81f-7e805b84-6f829395-94677911 English
% LTXE-SLIDE-TITLE:		Rendering Context (1)
% LTXE-SLIDE-REFERENCE:	!Feature-69863-UseNewStandaloneFluidAsComposerDependency.rst
% ------------------------------------------------------------------------------

\begin{frame}[fragile]
	\frametitle{Extbase \& Fluid}
	\framesubtitle{RenderingContext (1)}

	% decrease font size for code listing
	\lstset{basicstyle=\tiny\ttfamily}

	\begin{itemize}

		\item La novità più importante delle API pubbliche è il RenderingContext

		\item Il precedente, solo interno, RenderingContext usato da Fluid è stato 
			ampliato per essere responsabile di nuove funzionalità vitali di Fluid:
			\textbf{implementation provisioning}

		\item Questo permette agli sviluppatori di cambiare un range di classi, che Fluid usa per
			il parser, la conversione, il caching ecc.

		\item Questo può essere ottenuto sia includendo un RenderingContext personalizzato o
			intervenendo con i metodi pubblici sul RenderingContext di default.

	\end{itemize}

\end{frame}

% ------------------------------------------------------------------------------
% LTXE-SLIDE-START
% LTXE-SLIDE-UID:		48c26a1e-7effcdd0-57776618-31d36640
% LTXE-SLIDE-ORIGIN:	3d44632f-c87d4adf-3328f596-2e5744bc English
% LTXE-SLIDE-TITLE:		Rendering Context (2)
% LTXE-SLIDE-REFERENCE:	!Feature-69863-UseNewStandaloneFluidAsComposerDependency.rst
% ------------------------------------------------------------------------------

\begin{frame}[fragile]
	\frametitle{Extbase \& Fluid}
	\framesubtitle{Rendering Context (2)}

	% decrease font size for code listing
	%\lstset{basicstyle=\tiny\ttfamily}
	\lstset{basicstyle=\smaller\ttfamily}

	\begin{itemize}

		\item Le funzionalità delle seguenti slide possono essere controllate intervenendo
			sul RenderingContext. Di default, nessuna di esse è abilitata - ma chiamando
			un metodo semplice (attraverso un istanza View) permette di abilitarle:

		\begin{lstlisting}
			$view->getRenderingContext()->setLegacyMode(false);
		\end{lstlisting}

	\end{itemize}

\end{frame}

% ------------------------------------------------------------------------------
% LTXE-SLIDE-START
% LTXE-SLIDE-UID:		c04028b7-50451ff4-b89d8f24-df40d928
% LTXE-SLIDE-ORIGIN:	5f486f77-38e6694e-2fad0d6b-7b260ad1 English
% LTXE-SLIDE-TITLE:		ExpressionNodes (1)
% LTXE-SLIDE-REFERENCE:	!Feature-69863-UseNewStandaloneFluidAsComposerDependency.rst
% ------------------------------------------------------------------------------

\begin{frame}[fragile]
	\frametitle{Extbase \& Fluid}
	\framesubtitle{ExpressionNodes (1)}

	% decrease font size for code listing
	%\lstset{basicstyle=\tiny\ttfamily}
	\lstset{basicstyle=\smaller\ttfamily}

	\begin{itemize}

		\item Le ExpressionNodes sono un nuovo tipo di sintassi della struttura Fluid che condividono
			tutte un comune tratto: funzionano solo all'interno delle parentesi grafe

		\begin{lstlisting}
			$view->getRenderingContext()->setExpressionNodeTypes(array(
			   'Class\Number\One',
			   'Class\Number\Two'
			));
		\end{lstlisting}

		\item Gli sviluppatori possono aggiungere le loro ExpressionNode

		\item Ciascuna di esse consiste in un modello da abbinare e dei metodi dettati da
			un interfaccia di processo

		\item Ogni tipo esistente di ExpressionNode può essere usato come riferimento

	\end{itemize}

\end{frame}

% ------------------------------------------------------------------------------
% LTXE-SLIDE-START
% LTXE-SLIDE-UID:		21b6bf40-820b4673-9d96ed71-3e60bbf7
% LTXE-SLIDE-ORIGIN:	090cec75-2751b839-6ad8eebc-0adfe66a English
% LTXE-SLIDE-TITLE:		ExpressionNodes (2)
% LTXE-SLIDE-REFERENCE:	!Feature-69863-UseNewStandaloneFluidAsComposerDependency.rst
% ------------------------------------------------------------------------------

\begin{frame}[fragile]
	\frametitle{Extbase \& Fluid}
	\framesubtitle{ExpressionNodes (2)}

	ExpressionNodeTypes permettono nuove sintassi come ad esempio:

	\begin{itemize}

		\item \textbf{CastingExpressionNode}\newline
			\small
				permette il casting di una variabile ad un certo tipo, per esempio per garantire
				un integer o un boolean. E' usato semplicemente con la chiave \texttt{as}:
				\texttt{\{myStringVariable as boolean\}} o
				\texttt{\{myBooleanVariable as integer\}} e così via.
				Il tentativo di cast di una variabile ad un tipo incompatibile, causa un errore
				Fluid standard.
			\normalsize

		\item \textbf{MathExpressionNode}\newline
			\small
				permette operazioni matematiche di base sulle variabili, per esempio
				\texttt{\{myNumber + 1\}}, \texttt{\{myPercent / 100\}} o
				\texttt{\{myNumber * 100\}} e così via.
				Un espressione impossibile ritorna un output vuoto.
			\normalsize

	\end{itemize}

\end{frame}

% ------------------------------------------------------------------------------
% LTXE-SLIDE-START
% LTXE-SLIDE-UID:		0696f038-d46d33d4-d9399fd8-d7e55f59
% LTXE-SLIDE-ORIGIN:	c8935f68-0905db52-11932973-63a85a00 English
% LTXE-SLIDE-TITLE:		ExpressionNodes (3)
% LTXE-SLIDE-REFERENCE:	!Feature-69863-UseNewStandaloneFluidAsComposerDependency.rst
% ------------------------------------------------------------------------------

\begin{frame}[fragile]
	\frametitle{Extbase \& Fluid}
	\framesubtitle{ExpressionNodes (3)}

	ExpressionNodeTypes permettono nuove sintassi come ad esempio:

	\begin{itemize}

		\item \textbf{TernaryExpressionNode}\newline
			\small
				permette una condizione ternaria che opera solo su variabili.
				Un caso di uso tipico è: "se questa variabile allora usa quella variabile altrimenti usa
				un'altra variabile". E' usato come:\newline
				\texttt{\{myToggleVariable ? myThenVariable : myElseVariable\}}\newline
				Nota: non supporta espressioni nidificate, sintassi di inline ViewHelper
				o simili al suo interno. Deve essere utilizzata solo con variabili standard
				come input.
			\normalsize

	\end{itemize}

\end{frame}

% ------------------------------------------------------------------------------
% LTXE-SLIDE-START
% LTXE-SLIDE-UID:		bfecf675-2a559511-d5305769-ff4653f4
% LTXE-SLIDE-ORIGIN:	153b9b53-d7f97aef-dc1ba8ab-fc2a763e English
% LTXE-SLIDE-TITLE:		Namespaces are extensible (1)
% LTXE-SLIDE-REFERENCE:	!Feature-69863-UseNewStandaloneFluidAsComposerDependency.rst
% ------------------------------------------------------------------------------

\begin{frame}[fragile]
	\frametitle{Extbase \& Fluid}
	\framesubtitle{I Namespace sono estensibili (1)}

	\begin{itemize}

		\item Fluid permette che ogni alias di namespace (ad esempio \texttt{f:}) possa
			essere esteso con l'aggiunta di un ulteriore namespaces

		\item Anche i namespace PHP sono verificati per la presenza di classi ViewHelper

		\item Questo significa che gli sviluppatori possono sovrascrivere singoli ViewHelper
			con versioni personalizzate ed avere i loro ViewHelper chiamati quando si usa
			il namespace \texttt{f:} 

		\item Questo cambiamento implica anche che i namespace non siano più monadic.
			Usando
			\texttt{
				\{namespace f=My\textbackslash
				Extension\textbackslash
				ViewHelpers\textbackslash\}}\newline
				non si riceverà più un errore "namespace already registered".
				Fluid aggiungerà invece questo namespace PHP e cercherà i ViewHelper
				anche lì.

	\end{itemize}

\end{frame}

% ------------------------------------------------------------------------------
% LTXE-SLIDE-START
% LTXE-SLIDE-UID:		b1d76282-98fd3d1d-b0330e62-45d9f4e8
% LTXE-SLIDE-ORIGIN:	a2f0e799-fe523ffb-57634576-654a8efe English
% LTXE-SLIDE-TITLE:		Namespaces are extensible (2)
% LTXE-SLIDE-REFERENCE:	!Feature-69863-UseNewStandaloneFluidAsComposerDependency.rst
% ------------------------------------------------------------------------------

\begin{frame}[fragile]
	\frametitle{Extbase \& Fluid}
	\framesubtitle{I Namespaces sono estensibili (2)}

	\begin{itemize}

		\item I namespace aggiunti sono verificati dal basso verso l'alto, permettendo
			ai namespace aggiunti di sostituire le classi ViewHelper mettendole 
			alla stessa portata

		\item Per esempio: \texttt{f:format.nl2br} può essere sostituito in
			\texttt{
				My\textbackslash
				Extension\textbackslash
				ViewHelpers\textbackslash
				Format\textbackslash
				Nl2brViewHelper},

				con la registrazione del namespace alla precedente slide

	\end{itemize}

\end{frame}

% ------------------------------------------------------------------------------
% LTXE-SLIDE-START
% LTXE-SLIDE-UID:		1d1ee87a-9a8a72d9-634f2c85-7d63d113
% LTXE-SLIDE-ORIGIN:	2a98636b-3a4db59b-588597f9-0aca826e English
% LTXE-SLIDE-TITLE:		Rendering using f:render (1)
% LTXE-SLIDE-REFERENCE:	!Feature-69863-UseNewStandaloneFluidAsComposerDependency.rst
% ------------------------------------------------------------------------------

\begin{frame}[fragile]
	\frametitle{Extbase \& Fluid}
	\framesubtitle{Uso del rendering \texttt{f:render} (1)}

	Consente contenuto predefinito come opzionale \texttt{f:render}:

	\begin{itemize}

		\item Ogniqualvolta \texttt{f:render} è usato con flag \texttt{optional = TRUE}
			impostato, la visualizzazione di una sezione mancante da come risultato un output vuoto.

		\item Invece di visualizzare un output vuoto, un nuovo attributo di \texttt{default}
			(mixed) è aggiunto e può essere compilato con un valore di default di tipo fallback.

		\item In alternativa, il contenuto tag può essere usato per definire questo valore predefinito
			come tanti altri contenuti/attributi flessibili dei ViewHelper

	\end{itemize}

\end{frame}

% ------------------------------------------------------------------------------
% LTXE-SLIDE-START
% LTXE-SLIDE-UID:		f0d46114-d9c9d723-dbc5e622-fd959aae
% LTXE-SLIDE-ORIGIN:	78998a07-75b3fe30-431ca5e2-f7b816c4 English
% LTXE-SLIDE-TITLE:		Rendering using f:render (2)
% LTXE-SLIDE-REFERENCE:	!Feature-69863-UseNewStandaloneFluidAsComposerDependency.rst
% ------------------------------------------------------------------------------

\begin{frame}[fragile]
	\frametitle{Extbase \& Fluid}
	\framesubtitle{Uso del rendering \texttt{f:render} (2)}

	% decrease font size for code listing
	\lstset{basicstyle=\tiny\ttfamily}

	Passaggio di contenuti con tag dal \texttt{f:render} al partial/section:

	\begin{itemize}

		\item Permette un nuovo approcio per strutturare la composizione dei template Fluid

		\item Partial e section possono essere utilizzati come "wrapper" per parti arbitrarie di
			codice.

		\item Esempio:

			\begin{lstlisting}
				<f:section name="MyWrap">
				  <div>
				    <!-- more HTML, using variables if desired -->
				    <!-- tag content of f:render output: -->
				    {contentVariable -> f:format.raw()}
				  </div>
				</f:section>

				<f:render section="MyWrap" contentAs="contentVariable">
				  This content will be wrapped. Any Fluid code can go here.
				</f:render>
			\end{lstlisting}

	\end{itemize}

\end{frame}

% ------------------------------------------------------------------------------
% LTXE-SLIDE-START
% LTXE-SLIDE-UID:		3a725ebf-9072ebab-21861f49-a9f2b4b4
% LTXE-SLIDE-ORIGIN:	7c36b3a7-33860495-4efb65ef-f09ae14e English
% LTXE-SLIDE-TITLE:		Complex conditional statements
% LTXE-SLIDE-REFERENCE:	!Feature-69863-UseNewStandaloneFluidAsComposerDependency.rst
% ------------------------------------------------------------------------------

\begin{frame}[fragile]
	\frametitle{Extbase \& Fluid}
	\framesubtitle{Istruzioni condizionali complesse}

	% decrease font size for code listing
	\lstset{basicstyle=\tiny\ttfamily}

	\begin{itemize}

		\item Ora Fluid supporta qualsiasi tipo di istruzione complessa di condizioni con nidificazioni e
			raggruppamenti:

			\begin{lstlisting}
				<f:if condition="({variableOne} && {variableTwo}) || {variableThree} || {variableFour}">
				  // Esegue se le variabili One e Two valgono true ,
				  // o se lo sono le variabili Three o Four.
				</f:if>
			\end{lstlisting}

		\item In più, \texttt{f:else} è stato dotato del comportamento tipo "elseif":

			\begin{lstlisting}
				<f:if condition="{variableOne}">
				  <f:then>Fa questo se la variabile One vale true</f:then>
				  <f:else if="{variableTwo}">
				    Oppure fa questo se invece la variabile Two vale true
				  </f:else>
				  <f:else if="{variableThree}">
				    Oppure fa questo se la variabile Three vale true
				  </f:else>
				  <f:else>
				    Oppure fa questo se nessuno dei precedenti vale true
				  </f:else>
				</f:if>
			\end{lstlisting}

	\end{itemize}

\end{frame}

% ------------------------------------------------------------------------------
% LTXE-SLIDE-START
% LTXE-SLIDE-UID:		dd3159b5-384b2e5f-33611c71-9034c190
% LTXE-SLIDE-ORIGIN:	d54cf918-48141072-8184ab81-cecc8e39 English
% LTXE-SLIDE-TITLE:		Dynamic variable name parts (1)
% LTXE-SLIDE-REFERENCE:	!Feature-69863-UseNewStandaloneFluidAsComposerDependency.rst
% ------------------------------------------------------------------------------

\begin{frame}[fragile]
	\frametitle{Extbase \& Fluid}
	\framesubtitle{Parti dei nomi di variabili dinamiche (1)}

	% decrease font size for code listing
	\lstset{basicstyle=\tiny\ttfamily}

	\begin{itemize}

		\item Una nuova caratteristica è la possibilità di utilizzare dei riferimenti 
			dinamici con variabili quando si accede ad una variabile.
			Si consideri il seguente template Fluid con array di variabili:

			\begin{lstlisting}
				$mykey = 'foo'; // o 'bar', impostato in un codice sorgente
				$view->assign('data', ['foo' => 1, 'bar' => 2]);
				$view->assign('key', $mykey);
			\end{lstlisting}

		\item Con il seguente template Fluid:

			\begin{lstlisting}
				Scegliendo: {data.{key}}.
				(output: "1" con la chiave "foo" o "2" con "bar")
			\end{lstlisting}

	\end{itemize}

\end{frame}

% ------------------------------------------------------------------------------
% LTXE-SLIDE-START
% LTXE-SLIDE-UID:		b7eaa593-561af711-840817d7-65e2e9a1
% LTXE-SLIDE-ORIGIN:	20385d89-9c0de3ac-9ef70ba6-310fdec3 English
% LTXE-SLIDE-TITLE:		Dynamic variable name parts (2)
% LTXE-SLIDE-REFERENCE:	!Feature-69863-UseNewStandaloneFluidAsComposerDependency.rst
% ------------------------------------------------------------------------------

\begin{frame}[fragile]
	\frametitle{Extbase \& Fluid}
	\framesubtitle{Parti dei nomi di variabili dinamiche (2)}

	% decrease font size for code listing
	\lstset{basicstyle=\tiny\ttfamily}

	\begin{itemize}

		\item Lo stesso approcio può essere utilizzato per generare parti dinamiche del nome di variabile:

			\begin{lstlisting}
				$mydynamicpart = 'First'; // o 'Second', impostato in un codice sorgente
				$view->assign('myFirstVariable', 1);
				$view->assign('mySecondVariable', 2);
				$view->assign('which', $mydynamicpart);
			\end{lstlisting}

		\item Con il seguente template Fluid:

			\begin{lstlisting}
				Scegliendo: {my{which}Variable}.
				(output: "1" se which vale "First" o "2" se which vale "Second")
			\end{lstlisting}

	\end{itemize}

\end{frame}

% ------------------------------------------------------------------------------
% LTXE-SLIDE-START
% LTXE-SLIDE-UID:		d424eec6-c2e69e90-8ee2ae31-90914168
% LTXE-SLIDE-ORIGIN:	15907f4f-d554b015-03be1fdc-4dc0fb25 English
% LTXE-SLIDE-TITLE:		New ViewHelpers
% LTXE-SLIDE-REFERENCE:	!Feature-69863-UseNewStandaloneFluidAsComposerDependency.rst
% ------------------------------------------------------------------------------

\begin{frame}[fragile]
	\frametitle{Extbase \& Fluid}
	\framesubtitle{Nuovi ViewHelper}

	% decrease font size for code listing
	\lstset{basicstyle=\tiny\ttfamily}

	\begin{itemize}
		\item Alcuni nuovi ViewHelper sono stati aggiunti come parti standalone di Fluid
			e come tali sono disponibili da ora in TYPO3:

			\begin{itemize}

				\item \textbf{\texttt{f:or}}\newline
					Questo è un modo più breve di scrivere condizioni concatenate.
					Esso supporta la seguente sintassi, che verifica ogni variabile
					restituendo la prima che non è vuota:

					\begin{lstlisting}
						{variableOne -> f:or(alternative: variableTwo) -> f:or(alternative: variableThree)}
					\end{lstlisting}

				\item \textbf{\texttt{f:spaceless}}\newline
					Può essere utilizzato nei tag nel codice dei template per eliminare
					gli spazi e righe vuote per esempio creati dall'uso di indentazione
					dei ViewHelper
			\end{itemize}

	\end{itemize}

\end{frame}

% ------------------------------------------------------------------------------
% LTXE-SLIDE-START
% LTXE-SLIDE-UID:		aacee63a-79e519a6-ff40fa2b-2b47751b
% LTXE-SLIDE-ORIGIN:	9fed5ece-ae03648f-c3db2f41-42312f7f English
% LTXE-SLIDE-TITLE:		ViewHelper namespaces can be extended also from PHP
% LTXE-SLIDE-REFERENCE:	!Feature-69863-UseNewStandaloneFluidAsComposerDependency.rst
% ------------------------------------------------------------------------------

\begin{frame}[fragile]
	\frametitle{Extbase \& Fluid}
	\framesubtitle{I namespaces dei ViewHelper possono essere estesi da PHP}

	% decrease font size for code listing
	\lstset{basicstyle=\tiny\ttfamily}

	\begin{itemize}

		\item Accedendo al ViewHelperResolver del RenderingContext,
			gli sviluppatori possono cambiare le inclusioni del namespace di base del ViewHelper a livello globale
			(leggi: istanza View):

			\begin{lstlisting}
				$resolver = $view->getRenderingContext()->getViewHelperResolver();
				// equivalent of registering namespace in template(s):
				$resolver->registerNamespace('news', 'GeorgRinger\News\ViewHelpers');
				// adding additional PHP namespaces to check when resolving ViewHelpers:
				$resolver->extendNamespace('f', 'My\Extension\ViewHelpers');
				// setting all namespaces in advance, globally, before template parsing:
				$resolver->setNamespaces(array(
				  'f' => array(
				    'TYPO3Fluid\\Fluid\\ViewHelpers', 'TYPO3\\CMS\\Fluid\\ViewHelpers',
				    'My\\Extension\\ViewHelpers'
				  ),
				  'vhs' => array(
				    'FluidTYPO3\\Vhs\\ViewHelpers', 'My\\Extension\\ViewHelpers'
				  ),
				  'news' => array(
				    'GeorgRinger\\News\\ViewHelpers',
				  );
				));
			\end{lstlisting}
	\end{itemize}

\end{frame}

% ------------------------------------------------------------------------------
% LTXE-SLIDE-START
% LTXE-SLIDE-UID:		f3bb31e2-8db113ee-fee52f59-4b727b9e
% LTXE-SLIDE-ORIGIN:	3e41fa42-1d526584-5b7d3a07-c8b347ab English
% LTXE-SLIDE-TITLE:		ViewHelpers can accept arbitrary arguments (1)
% LTXE-SLIDE-REFERENCE:	!Feature-69863-UseNewStandaloneFluidAsComposerDependency.rst
% ------------------------------------------------------------------------------

\begin{frame}[fragile]
	\frametitle{Extbase \& Fluid}
	\framesubtitle{I ViewHelper possono accettare argomenti opzionali (1)}

	\begin{itemize}

		\item Questa funzionalità permette alla classe ViewHelper un numero qualsiasi di parametri
			aggiuntivi utilizzando qualsiasi variabile si desideri

		\item Funziona separando i parametri che vengono passati ad ogni ViewHelper in due
			gruppi: quelli che vengono dichiarati utilizzando \texttt{registerArgument}
			(o come parametri del metodo) e quelli che non lo sono

		\item I parametri non dichiarati, sono passati ad una funziona speciale
			\texttt{handleAdditionalArguments}
			nella classe del ViewHelper, che nell'implementazione predefinita genera un errore
			se esistono parametri aggiuntivi.

	\end{itemize}

\end{frame}

% ------------------------------------------------------------------------------
% LTXE-SLIDE-START
% LTXE-SLIDE-UID:		4e136a6b-1f0ffd0f-b163d696-1529683e
% LTXE-SLIDE-ORIGIN:	2462a8ce-0a82a7f0-1b97a84f-02f0a4c2 English
% LTXE-SLIDE-TITLE:		ViewHelpers can accept arbitrary arguments (2)
% LTXE-SLIDE-REFERENCE:	!Feature-69863-UseNewStandaloneFluidAsComposerDependency.rst
% ------------------------------------------------------------------------------

\begin{frame}[fragile]
	\frametitle{Extbase \& Fluid}
	\framesubtitle{I ViewHelper possono accettare argomenti opzionali (2)}

	\begin{itemize}

		\item Con l'ovveride di questo metodo nel ViewHelper, si può modificare se e quando
			il ViewHelper dovrebbe registrare un errore alla ricezione di parametri non registrati

		\item Questa funzionalità è quella che permette ai TagBasedViewHelpers di accettare liberamente
			parametri con prefisso \texttt{data-} senza andare in errore

		\item In TagBasedViewHelpers, il metodo \texttt{handleAdditionalArguments} 
			aggiunge semplicemente nuovi attributi al tag generato e crea un errore
			se sono presenti parametri aggiuntivi che non sono registrati o 
			con il prefisso \texttt{data-}.

	\end{itemize}

\end{frame}

% ------------------------------------------------------------------------------
% LTXE-SLIDE-START
% LTXE-SLIDE-UID:		515d6c27-37efbb5c-29a19a03-741a301c
% LTXE-SLIDE-ORIGIN:	b5a66715-5681025e-cfb46644-9b9bad55 English
% LTXE-SLIDE-TITLE:		Added "allowedTags" argument to f:format.stripTags ViewHelper
% LTXE-SLIDE-REFERENCE:	!Feature-67236-AddedAllowedTagsArgumentToFformatstripTagsViewHelper.rst
% ------------------------------------------------------------------------------

\begin{frame}[fragile]
	\frametitle{Extbase \& Fluid}
	\framesubtitle{Parametri "allowedTags" per f:format.stripTags}

	\begin{itemize}

		\item Il parametro \texttt{allowedTags} contiene una lista di tag HTML 
			che non devono essere tolti usando \texttt{f:format.stripTags}

		\item La sintassi della lista di tag è identica al secondo parametro del comando PHP
			\texttt{strip\_tags} (vedi: \url{http://php.net/strip_tags})

	\end{itemize}

\end{frame}

% ------------------------------------------------------------------------------
% LTXE-SLIDE-START
% LTXE-SLIDE-UID:		a5c83c6d-26b97d82-ccec012e-40e9af6b
% LTXE-SLIDE-ORIGIN:	12b7dd4a-73b912f7-e5d1cefa-e3c8386d English
% LTXE-SLIDE-TITLE:		Allow accessing ObjectStorage as array in Fluid
% LTXE-SLIDE-REFERENCE:	!Feature-73752-AllowAccessingObjectStorageAsArrayInFluidAndOtherPlaces.rst
% ------------------------------------------------------------------------------

\begin{frame}[fragile]
	\frametitle{Extbase \& Fluid}
	\framesubtitle{Consentire l'accesso a ObjectStorage come array in Fluid}

	% decrease font size for code listing
	\lstset{basicstyle=\tiny\ttfamily}

	\begin{itemize}

		\item Creando un alias di \texttt{toArray()} si permette al metodo di essere
			chiamato come \texttt{getArray()} che a sua volta permette al metodo di
			essere chiamato da \texttt{ObjectAccess::getPropertyPath},
			che consente l'accesso a Fluid e altre parti

		\item Con la creazione di un semplice alias di \texttt{toArray()} su
			ObjectStorage, permette di essere chiamato come \texttt{getArray()}

		\item Esempio: ottenere il quarto elemento

			\begin{lstlisting}
				// in PHP:
				ObjectAccess::getPropertyPath($subject, 'objectstorageproperty.array.4')

				// in Fluid:
				{myObject.objectstorageproperty.array.4}
				{myObject.objectstorageproperty.array.{dynamicIndex}}
			\end{lstlisting}

	\end{itemize}

\end{frame}

% ------------------------------------------------------------------------------

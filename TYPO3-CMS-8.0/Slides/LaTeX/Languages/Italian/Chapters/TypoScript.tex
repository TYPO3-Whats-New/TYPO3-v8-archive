% ------------------------------------------------------------------------------
% TYPO3 CMS 7.6 - What's New - Chapter "TypoScript" (English Version)
%
% @author	Patrick Lobacher <patrick@lobacher.de> and Michael Schams <schams.net>
% @license	Creative Commons BY-NC-SA 3.0
% @link		http://typo3.org/download/release-notes/whats-new/
% @language	English
% ------------------------------------------------------------------------------
% LTXE-CHAPTER-UID:		b8fd0ec6-eeb4e126-fcdd7604-74ef6c57
% LTXE-CHAPTER-NAME:	TypoScript
% ------------------------------------------------------------------------------

\section{TSconfig \& TypoScript}
\begin{frame}[fragile]
	\frametitle{TSconfig \& TypoScript}

	\begin{center}\huge{Capitolo 2:}\end{center}
	\begin{center}\huge{\color{typo3darkgrey}\textbf{TSconfig \& TypoScript}}\end{center}

\end{frame}

% ------------------------------------------------------------------------------
% LTXE-SLIDE-START
% LTXE-SLIDE-UID:		8dab4677-bbd2da33-aab59d76-8321e140
% LTXE-SLIDE-ORIGIN:	0956a9a6-bb0ea776-0327261b-7722ac16 English
% LTXE-SLIDE-TITLE:		Make new content element wizard tab sort order configurable
% LTXE-SLIDE-REFERENCE:	!Feature-71876-MakeNewContentElementWizardTabSortOrderConfigurable.rst
% ------------------------------------------------------------------------------
\begin{frame}[fragile]
	\frametitle{TSconfig \& TypoScript}
	\framesubtitle{Ordinamento di nuove schede di dati nei contenuti}

	% decrease font size for code listing
	\lstset{basicstyle=\tiny\ttfamily}

	\begin{itemize}
		\item E' possibile configurare l'ordinamento delle schede di dati negli elementi di contenuto
			impostando i valori \texttt{before} e \texttt{after} nel TSconfig di pagina:

			\begin{lstlisting}
				mod.wizards.newContentElement.wizardItems.special.before = common
				mod.wizards.newContentElement.wizardItems.forms.after = common,special
			\end{lstlisting}

	\end{itemize}

\end{frame}

% ------------------------------------------------------------------------------
% LTXE-SLIDE-START
% LTXE-SLIDE-UID:		d58005a7-b0c63414-c90bec7f-40604c72
% LTXE-SLIDE-ORIGIN:	330b8895-6cecc482-88177047-8d65c88b English
% LTXE-SLIDE-TITLE:		HTMLparser.stripEmptyTags.keepTags
% LTXE-SLIDE-REFERENCE:	!Feature-72045-KeepTagsInHtmlParserWhenStrippingEmptyTags.rst
% ------------------------------------------------------------------------------
\begin{frame}[fragile]
	\frametitle{TSconfig \& TypoScript}
	\framesubtitle{\texttt{HTMLparser.stripEmptyTags.keepTags}}

	% decrease font size for code listing
	\lstset{basicstyle=\tiny\ttfamily}

	\begin{itemize}

		\item La nuova opzione per la configurazione di \texttt{HTMLparser.stripEmptyTags} è stata aggiunta,
			essa permette di mantenere liste di tag configurabili.
		\item Prima di questa modifica, solo la lista di tag forniti venivano rimossi.
		\item L'esempio seguente toglie tutti i tag vuoti \textbf{eccetto} i tag \texttt{tr} e \texttt{td}:

			\begin{lstlisting}
				HTMLparser.stripEmptyTags = 1
				HTMLparser.stripEmptyTags.keepTags = tr,td
			\end{lstlisting}

	\end{itemize}

	\underline{Importante:} se è utilizzata questa configurazione, la configurazione \texttt{stripEmptyTags.tags}
		non ha più effetto. Si può usare una sola opzione alla volta.

\end{frame}

% ------------------------------------------------------------------------------
% LTXE-SLIDE-START
% LTXE-SLIDE-UID:		ffe717b2-fff11a0d-40115a7b-42506247
% LTXE-SLIDE-ORIGIN:	ec7822d8-5f3eeeb3-06a2c047-723568f7 English
% LTXE-SLIDE-TITLE:		EXT:form - integration of predefined forms (1)
% LTXE-SLIDE-REFERENCE:	!Feature-72309-EXTform-AllowIntegrationOfPredefinedForms.rst
% ------------------------------------------------------------------------------
\begin{frame}[fragile]
	\frametitle{TSconfig \& TypoScript}
	\framesubtitle{\texttt{EXT:form} - integrazione di form predefiniti (1)}

	% decrease font size for code listing
	\lstset{basicstyle=\tiny\ttfamily}

	\begin{itemize}

		\item L'elemento di contenuto della \texttt{EXT:form} permette l'integrazione di form
			predefiniti.

		\item Un integratore può definire dei form (es. all'interno di un pacchetto sito) usando
			\texttt{plugin.tx\_form.predefinedForms}

		\item Un editore può aggiungere un nuovo elemento di contenuto \texttt{mailform} ad una pagina e
			scegliere un form da una lista di elementi predefiniti

		\item L'integratore può costruire i suoi form con TypoScript, che permette molte più opzioni che
			crearlo con il wizard dei form. Es. l'integratore può usare la funzionalità
			\texttt{stdWrap}, che non è disponibile quando si usa il wizard dei form (per ragioni di sicurezza)

	\end{itemize}

\end{frame}

% ------------------------------------------------------------------------------
% LTXE-SLIDE-START
% LTXE-SLIDE-UID:		fe206b00-09cea6f6-f7a2fbac-87b3f5af
% LTXE-SLIDE-ORIGIN:	04706a2c-eeb35f3e-22d8ec78-68f77235 English
% LTXE-SLIDE-TITLE:		EXT:form - integration of predefined forms (2)
% LTXE-SLIDE-REFERENCE:	!Feature-72309-EXTform-AllowIntegrationOfPredefinedForms.rst
% ------------------------------------------------------------------------------
\begin{frame}[fragile]
	\frametitle{TSconfig \& TypoScript}
	\framesubtitle{\texttt{EXT:form} - integrazione di form predefiniti (2)}

	% decrease font size for code listing
	\lstset{basicstyle=\tiny\ttfamily}

	\begin{itemize}

		\item Non è più necessario che un editore usi il wizard dei form.
			Gli editori possono predefinire dei form che sono ottimizzati nel layout.

		\item I form possono essere riutilizzati nell'intera installazione

		\item I form possono essere salvati fuori dal DB e versionati

		\item Per poter selezionare i form predefiniti nel backend,
			essi devono essere salvati utilizzando PageTS:

		\begin{lstlisting}
			TCEFORM.tt_content.tx_form_predefinedform.addItems.contactForm =
			  LLL:EXT:my_theme/Resources/Private/Language/locallang.xlf:contactForm
		\end{lstlisting}

	\end{itemize}

\end{frame}

% ------------------------------------------------------------------------------
% LTXE-SLIDE-START
% LTXE-SLIDE-UID:		1cf546dd-2f2ff79e-7b0a24c6-746b5467
% LTXE-SLIDE-ORIGIN:	5f3eeeb3-06a2c047-723568f7-ec7822d8 English
% LTXE-SLIDE-ORIGIN:	3076f305-06ba38cb-646d578c-b617df02 German
% LTXE-SLIDE-TITLE:		EXT:form - integration of predefined forms (3)
% LTXE-SLIDE-REFERENCE:	!Feature-72309-EXTform-AllowIntegrationOfPredefinedForms.rst
% ------------------------------------------------------------------------------
\begin{frame}[fragile]
	\frametitle{TSconfig \& TypoScript}
	\framesubtitle{\texttt{EXT:form}: integrazione di form predefiniti (3)}

	% decrease font size for code listing
	\lstset{basicstyle=\tiny\ttfamily}

	\begin{itemize}

		\item Esempio di form:

		\begin{lstlisting}
			plugin.tx_form.predefinedForms.contactForm = FORM
			plugin.tx_form.predefinedForms.contactForm {
			  enctype = multipart/form-data
			  method = post
			  prefix = contact
			  confirmation = 1
			  postProcessor {
			    1 = mail
			    1 {
			      recipientEmail = test@example.com
			      senderEmail = test@example.com
			      subject {
			        value = Contact form
			        lang.de = Kontakt Formular
			      }
			    }
			  }
			  10 = TEXTLINE
			  10 {
			    name = name
			...
		\end{lstlisting}

	\end{itemize}

\end{frame}

% ------------------------------------------------------------------------------

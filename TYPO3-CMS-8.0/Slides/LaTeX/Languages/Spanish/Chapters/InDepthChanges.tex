% ------------------------------------------------------------------------------
% TYPO3 CMS 7.6 - What's New - Chapter "In-Depth Changes" (English Version)
%
% @author	Patrick Lobacher <patrick@lobacher.de> and Michael Schams <schams.net>
% @license	Creative Commons BY-NC-SA 3.0
% @link		http://typo3.org/download/release-notes/whats-new/
% @language	English
% ------------------------------------------------------------------------------
% LTXE-CHAPTER-UID:		7229f1b9-b481e9bc-09c46183-a86b6a7e
% LTXE-CHAPTER-NAME:	In-Depth Changes
% ------------------------------------------------------------------------------

\section{Cambios en Profundidad}
\begin{frame}[fragile]
	\frametitle{Cambios en Profundidad}

	\begin{center}\huge{Capítulo 3:}\end{center}
	\begin{center}\huge{\color{typo3darkgrey}\textbf{Cambios en Profundidad}}\end{center}

\end{frame}


% ------------------------------------------------------------------------------
% LTXE-SLIDE-START
% LTXE-SLIDE-UID:		b9b259e4-9d35ace7-edff1014-7390d1d1
% LTXE-SLIDE-TITLE:		Support PECL-memcached in MemcachedBackend
% LTXE-SLIDE-REFERENCE:	!Feature-69794-SupportPecl-memcachedInMemcachedBackend.rst
% ------------------------------------------------------------------------------
\begin{frame}[fragile]
	\frametitle{Cambios en Profundidad}
	\framesubtitle{Soporte de PECL-memcached en MemcachedBackend}

	% decrease font size for code listing
	\lstset{basicstyle=\tiny\ttfamily}

	\begin{itemize}

		\item Se ha añadido soporte para el módulo PECL "memcached" al
			MemcachedBackend del Framework de Cacheo

		\item Si ambos, "memcache" y "memcached" están instalados, se usa "memcache" para evitar que sea un cambio en profundidad.

		\item Un integrador puede configurar la opción \texttt{peclModule} para usar el
			módulo PECL preferido:

			\begin{lstlisting}
				$GLOBALS['TYPO3_CONF_VARS']['SYS']['caching']['cacheConfigurations']['my_memcached'] = [
				  'frontend' => \TYPO3\CMS\Core\Cache\Frontend\VariableFrontend::class
				  'backend' => \TYPO3\CMS\Core\Cache\Backend\MemcachedBackend::class,
				  'options' => [
				    'peclModule' => 'memcached',
				    'servers' => [
				      'localhost',
				      'server2:port'
				    ]
				  ]
				];
			\end{lstlisting}

	\end{itemize}

\end{frame}


% ------------------------------------------------------------------------------
% LTXE-SLIDE-START
% LTXE-SLIDE-UID:		ef8c124e-f902453a-d48770b9-4cf86545
% LTXE-SLIDE-TITLE:		Native support for Symfony Console (1)
% LTXE-SLIDE-REFERENCE:	!Feature-73042-IntroduceNativeSupportForSymfonyConsole.rst
% ------------------------------------------------------------------------------
\begin{frame}[fragile]
	\frametitle{Cambios en Profundidad}
	\framesubtitle{Soporte nativo para la Consola de Symfony (1)}

	% decrease font size for code listing
	\lstset{basicstyle=\tiny\ttfamily}

	\begin{itemize}

		\item TYPO3 soporta ahora el componente de Consola de Symfony listo para usar proporcionando
			un nuevo script de línea de comandos ubicado en \texttt{typo3/sysext/core/bin/typo3}.
			En instancias TYPO3 instaladas a través de Composer, el binario está enlazado dentro del
			directorio \texttt{bin}, p.e. \texttt{bin/typo3}.

		\item El nuevo binario todavía soporta los argumentos existentes de la línea de comandos al
			si no se encuentra por defecto un comando adecuado de Consola de Symfony.

	\end{itemize}

\end{frame}

% ------------------------------------------------------------------------------
% LTXE-SLIDE-START
% LTXE-SLIDE-UID:		ef8c124e-f902453a-d48770b9-4cf86545
% LTXE-SLIDE-TITLE:		Native support for Symfony Console (1)
% LTXE-SLIDE-REFERENCE:	!Feature-73042-IntroduceNativeSupportForSymfonyConsole.rst
% ------------------------------------------------------------------------------
\begin{frame}[fragile]
	\frametitle{Cambios en Profundidad}
	\framesubtitle{Soporte nativo para la Consola de Symfony (2)}

	% decrease font size for code listing
	\lstset{basicstyle=\tiny\ttfamily}

	\begin{itemize}

		\item Registrar un comando para estar disponible a través de la herramienta de línea de comandos \texttt{typo3}
			funciona poniendo un fichero \texttt{Configuration/Commands.php} en una
			extensión instalada. Esto lista las clases Symfony/Console/Command classes para ser
			ejecutadas por \texttt{typo3} es un vector asociativo. La clave es el nombre del
			comando a ser llamado como el primer argumento para \texttt{typo3}.

	\end{itemize}

\end{frame}

% ------------------------------------------------------------------------------
% LTXE-SLIDE-START
% LTXE-SLIDE-UID:		28b018bb-21804fa1-c578f0ff-fc16cbb1
% LTXE-SLIDE-TITLE:		Native support for Symfony Console (2)
% LTXE-SLIDE-REFERENCE:	!Feature-73042-IntroduceNativeSupportForSymfonyConsole.rst
% ------------------------------------------------------------------------------
\begin{frame}[fragile]
	\frametitle{Cambios en Profundidad}
	\framesubtitle{Soporte nativo para la Consola de Symfony (3)}

	% decrease font size for code listing
	\lstset{basicstyle=\tiny\ttfamily}

	\begin{itemize}

		\item Un parámetro requerido al registrar un comando es la propiedad \texttt{class}.
			Opcionalmente el parámetro \texttt{user} puede ser configurado para que un usuario del backend esté autenticado
			al llamar al comando.

		\item Un \texttt{Configuration/Commands.php} podría parecerse a:

			\begin{lstlisting}
				return [
				  'backend:lock' => [
				    'class' => \TYPO3\CMS\Backend\Command\LockBackendCommand::class
				  ],
				  'referenceindex:update' => [
				    'class' => \TYPO3\CMS\Backend\Command\ReferenceIndexUpdateCommand::class,
				    'user' => '_cli_lowlevel'
				  ]
				];
			\end{lstlisting}

	\end{itemize}

\end{frame}

% ------------------------------------------------------------------------------
% LTXE-SLIDE-START
% LTXE-SLIDE-UID:		fc16cbb1-21804fa1-c578f0ff-28b018bb
% LTXE-SLIDE-TITLE:		Native support for Symfony Console (3)
% LTXE-SLIDE-REFERENCE:	!Feature-73042-IntroduceNativeSupportForSymfonyConsole.rst
% ------------------------------------------------------------------------------
\begin{frame}[fragile]
	\frametitle{Cambios en Profundidad}
	\framesubtitle{Soporte nativo para la Consola de Symfony (3)}

	% decrease font size for code listing
	\lstset{basicstyle=\tiny\ttfamily}

	\begin{itemize}

		\item Una llamada de ejemplo podría parecerse a:
			\begin{lstlisting}
				bin/typo3 backend:lock http://example.com/maintenance.html
			\end{lstlisting}

		\item Para una instalación no-Composer:
			\begin{lstlisting}
				typo3/sysext/core/bin/typo3 backend:lock http://example.com/maintenance.html
			\end{lstlisting}

	\end{itemize}

\end{frame}

% ------------------------------------------------------------------------------
% LTXE-SLIDE-START
% LTXE-SLIDE-UID:		1626edc6-8d93ff84-f64178bf-8e7650df
% LTXE-SLIDE-TITLE:		Cryptographically secure pseudorandom number generator
% LTXE-SLIDE-REFERENCE:	!Feature-73050-AddACSPRNGAPI.rst
% ------------------------------------------------------------------------------
\begin{frame}[fragile]
	\frametitle{Cambios en Profundidad}
	\framesubtitle{Generador de números pseudoaleatorio criptográficamente seguro}

	% decrease font size for code listing
	\lstset{basicstyle=\tiny\ttfamily}

	\begin{itemize}

		\item Un nuevo generador de números pseudo-aleatorios criptográficamente seguros (CSPRNG) ha sido
			implementado en el núcleo de TYPO3.\newline
			Se beneficia de las nuevas funciones CSPRNG en PHP 7.

		\item El API reside en la clase
			\texttt{\textbackslash
				TYPO3\textbackslash
				CMS\textbackslash
				Core\textbackslash
				Crypto\textbackslash
				Random}

		\item Ejemplo:

			\begin{lstlisting}
				use \TYPO3\CMS\Core\Crypto\Random;
				use \TYPO3\CMS\Core\Utility\GeneralUtility;

				// Retrieving random bytes
				$someRandomString = GeneralUtility::makeInstance(Random::class)->generateRandomBytes(64);

				// Rolling the dice..
				$tossedValue = GeneralUtility::makeInstance(Random::class)->generateRandomInteger(1, 6);
			\end{lstlisting}

	\end{itemize}

\end{frame}

% ------------------------------------------------------------------------------
% LTXE-SLIDE-START
% LTXE-SLIDE-UID:		f098bb23-81eee73f-d9eb0d80-50bbe58b
% LTXE-SLIDE-TITLE:		Wizard component (1)
% LTXE-SLIDE-REFERENCE:	!Feature-73429-WizardComponentHasBeenAdded.rst
% ------------------------------------------------------------------------------
\begin{frame}[fragile]
	\frametitle{Cambios en Profundidad}
	\framesubtitle{Componente asistente (1)}

	% decrease font size for code listing
	\lstset{basicstyle=\tiny\ttfamily}

	\begin{itemize}

		\item Un nuevo componente asistente ha sido añadido. Este componente puede ser usar para interacciones guiadas de usuario

		\item El módulo RequireJS puede ser usado incluyendo \texttt{TYPO3/CMS/Backend/Wizard}

		\item El asistente soporta acciones sencillas sólo\newline
			(junctions no son posibles todavía)

		\item El API reside en la clase
			\texttt{\textbackslash
				TYPO3\textbackslash
				CMS\textbackslash
				Core\textbackslash
				Crypto\textbackslash
				Random}

		\item El componente asistente tiene los siguientes métodos públicos:

			\begin{lstlisting}
				addSlide(identifier, title, content, severity, callback)
				addFinalProcessingSlide(callback)
				set(key, value)
				show()
				dismiss()
				getComponent()
				lockNextStep()
				unlockNextStep()
			\end{lstlisting}

	\end{itemize}

\end{frame}

% ------------------------------------------------------------------------------
% LTXE-SLIDE-START
% LTXE-SLIDE-UID:		f098bb23-81eee73f-d9eb0d80-50bbe58b
% LTXE-SLIDE-TITLE:		Wizard component (2)
% LTXE-SLIDE-REFERENCE:	!Feature-73429-WizardComponentHasBeenAdded.rst
% ------------------------------------------------------------------------------
\begin{frame}[fragile]
	\frametitle{Cambios en Profundidad}
	\framesubtitle{Componente asistente (2)}

	% decrease font size for code listing
	\lstset{basicstyle=\tiny\ttfamily}

	\begin{itemize}

		\item El evento \texttt{wizard-visible} es lanzado cuando el renderizado del asistente ha finalizado

		\item Los asistentes pueden ser cerrados al lanzar el evento \texttt{wizard-dismiss}

		\item Los asistentes lanzan el evento \texttt{wizard-dismissed} si el asistente es cerrado

		\item Puedes integrar tu propio manejador usando \texttt{Wizard.getComponent()}

	\end{itemize}

\end{frame}

% ------------------------------------------------------------------------------
% LTXE-SLIDE-START
% LTXE-SLIDE-UID:		5f0d3565-8b6ad76b-44ec4d62-247401f7
% LTXE-SLIDE-TITLE:		Generated asset files moved
% LTXE-SLIDE-REFERENCE:	!Important-72580-PubliclyAccessibleGeneratedAssetFilesMovedToTypo3tempassets.rst
% ------------------------------------------------------------------------------
\begin{frame}[fragile]
	\frametitle{Cambios en Profundidad}
	\framesubtitle{Movidos ficheros asset generados}

	% decrease font size for code listing
	\lstset{basicstyle=\tiny\ttfamily}

	\begin{itemize}

		\item La estructura de carpeta dentro de \texttt{typo3temp} cambió a assets separados
			que necesitan ser accedidos por el cliente desde los ficheros para los cuales son temporalmente
			creados (p.e. para propósitos de cacheo o cierre y requiere sólo acceso
			al lado del servidor).

		\item Estos assets fueron movidos desde las carpetas:\newline
			\texttt{\_processed\_}, \texttt{compressor}, \texttt{GB}, \texttt{temp},
			\texttt{Language}, \texttt{pics}\newline
			y reorganizados en:

			\begin{itemize}
				\item \texttt{typo3temp/assets/js/}
				\item \texttt{typo3temp/assets/css/},
				\item \texttt{typo3temp/assets/compressed/}
				\item \texttt{typo3temp/assets/images/}
			\end{itemize}

	\end{itemize}

\end{frame}

% ------------------------------------------------------------------------------
% LTXE-SLIDE-START
% LTXE-SLIDE-UID:		a4d7bec0-6c19d8b2-6a6be951-5d5aa0c0
% LTXE-SLIDE-TITLE:		ImageMagick/GraphicsMagick changes (1)
% LTXE-SLIDE-REFERENCE:	!Breaking-43085-RenamedGraphicsProcessorSettings.rst
% ------------------------------------------------------------------------------
\begin{frame}[fragile]
	\frametitle{Cambios en Profundidad}
	\framesubtitle{Cambios en ImageMagick/GraphicsMagick (1)}

	% decrease font size for code listing
	\lstset{basicstyle=\tiny\ttfamily}

	\begin{itemize}

		\item Los ajustes del procesador de gráficos para Image- y GraphicsMagick han sido renombrados
			(fichero: \texttt{LocalConfiguration.php}).\newline
			OLD:\tabto{1.0cm} \texttt{im\_}\newline
			NEW:\tabto{1.0cm} \texttt{processor\_}

		\item Nombramientos negativos tales como \texttt{noScaleUp} han sido cambiados a equivalentes positivos.
			Durante la conversión, los valores de configuración previos son negados para reflejar los
			cambios en semántica de estas opciones.

		\item Adicionalmente, las referencias a versiones de ImageMagick/GraphicsMagick han sido
			eliminadas de los nombres y valores de configuración.

	\end{itemize}

\end{frame}

% ------------------------------------------------------------------------------
% LTXE-SLIDE-START
% LTXE-SLIDE-UID:		d48d2af2-66b218a6-4a54e402-097ad85e
% LTXE-SLIDE-TITLE:		ImageMagick/GraphicsMagick changes (2)
% LTXE-SLIDE-REFERENCE:	!Breaking-43085-RenamedGraphicsProcessorSettings.rst
% ------------------------------------------------------------------------------
\begin{frame}[fragile]
	\frametitle{Cambios en Profundidad}
	\framesubtitle{Cambios ImageMagick/GraphicsMagick (2)}

	% decrease font size for code listing
	\lstset{basicstyle=\tiny\ttfamily}

	\begin{itemize}

		\item La opción de configuración no empleada \texttt{image\_processing} ha sido eliminada sin
			reemplazo

		\item La opción de configuración de procesador específico \texttt{colorspace} se ha transformado a \textit{espacio de nombres}
			bajo la jerarquía \texttt{processor\_}

	\end{itemize}

\end{frame}

% ------------------------------------------------------------------------------
% LTXE-SLIDE-START
% LTXE-SLIDE-UID:		74e4b871-9f2ded00-f54bc3c1-85d22167
% LTXE-SLIDE-TITLE:		Hooks and Signals (1): post-processing of the preview URL
% LTXE-SLIDE-REFERENCE:	!Feature-54887-Post-processingOfThePreviewUrl.rst
% ------------------------------------------------------------------------------
\begin{frame}[fragile]
	\frametitle{Cambios en Profundidad}
	\framesubtitle{Hooks y Señales (1)}

	% decrease font size for code listing
	\lstset{basicstyle=\tiny\ttfamily}

	\begin{itemize}

		\item Un hook adicional ha sido añadido al método \texttt{BackendUtility::viewOnClick()}
			para post-procesar la URL de previsualización

		\item Registre una clase hook que implementa el método con el nombre \texttt{postProcess}:

			\begin{lstlisting}
				$GLOBALS['TYPO3_CONF_VARS']['SC_OPTIONS']['t3lib/class.t3lib_befunc.php']['viewOnClickClass'][] =
				  \VENDOR\MyExt\Hooks\BackendUtilityHook::class;
			\end{lstlisting}

	\end{itemize}

\end{frame}

% ------------------------------------------------------------------------------
% LTXE-SLIDE-START
% LTXE-SLIDE-UID:		9c150d50-ee3ba19b-120ef653-e714135c
% LTXE-SLIDE-TITLE:		Hooks and Signals (2): hook to override a record overlay
% LTXE-SLIDE-REFERENCE:	!Feature-72505-IntroduceHookToOverrideARecordOverlay.rst
% ------------------------------------------------------------------------------
\begin{frame}[fragile]
	\frametitle{Cambios en Profundidad}
	\framesubtitle{Hooks y Señales (2)}

	% decrease font size for code listing
	\lstset{basicstyle=\tiny\ttfamily}

	\begin{itemize}

		\item Previamente a TYPO3 CMS 7.6, era posible reemplazar una superposición de registro en \texttt{Web -> List}.
			Un nuevo hook en TYPO3 CMS 8.0 proporciona la antigua funcionalidad.

		\item El hook es llamado con la siguiente signatura:
			\begin{lstlisting}
				/**
				 * @param string $table
				 * @param array $row
				 * @param array $status
				 * @param string $iconName
				 * @return string the new (or given) $iconName
				 */
				function postOverlayPriorityLookup($table, array $row, array $status, $iconName) { ... }
			\end{lstlisting}

		\item Registre la clase hook que implementa el método con el nombre \texttt{postOverlayPriorityLookup}:

			\begin{lstlisting}
				$GLOBALS['TYPO3_CONF_VARS']['SC_OPTIONS'][IconFactory::class]['overrideIconOverlay'][] =
				  \VENDOR\MyExt\Hooks\IconFactoryHook::class;
			\end{lstlisting}

	\end{itemize}

\end{frame}

% ------------------------------------------------------------------------------
% LTXE-SLIDE-START
% LTXE-SLIDE-UID:		37ba9f7c-fb51932d-cca9a99f-6bb7a961
% LTXE-SLIDE-TITLE:		Hooks and Signals (3): preProcessStorage signal
% LTXE-SLIDE-REFERENCE:	!Feature-72904-AddPreProcessStorageSignalToResourceFactory.rst
% ------------------------------------------------------------------------------
\begin{frame}[fragile]
	\frametitle{Cambios en Profundidad}
	\framesubtitle{Hooks y Señales (3)}

	% decrease font size for code listing
	\lstset{basicstyle=\tiny\ttfamily}

	\begin{itemize}

		\item Una nueva señal ha sido implementada antes de que sea inicializado un almacenamiento de recursos.

		\item Registre la clase que implementa su lógica en \texttt{ext\_localconf.php}:

			\begin{lstlisting}
				$dispatcher = \TYPO3\CMS\Core\Utility\GeneralUtility::makeInstance(
				  \TYPO3\CMS\Extbase\SignalSlot\Dispatcher::class);
				$dispatcher->connect(
				  \TYPO3\CMS\Core\Resource\ResourceFactory::class,
				  ResourceFactoryInterface::SIGNAL_PreProcessStorage,
				  \MY\ExtKey\Slots\ResourceFactorySlot::class,
				  'preProcessStorage'
				);
			\end{lstlisting}

		\item El método es llamado con los siguientes argumentos:

			\begin{itemize}
				\item int \texttt{\$uid} el uid del registro
				\item array \texttt{\$recordData} todos los datos de registro como vector
				\item string \texttt{\$fileIdentifier} el identificador de fichero
			\end{itemize}

	\end{itemize}

\end{frame}

% ------------------------------------------------------------------------------
% LTXE-SLIDE-START
% LTXE-SLIDE-UID:		2c656d9e-38b9be7a-d590cb6e-21eccd4d
% LTXE-SLIDE-TITLE:		Password hashing algorithm: PBKDF2
% LTXE-SLIDE-REFERENCE:	!Feature-28230-AddSupportForPBKDF2ToSaltedpasswords.rst
% ------------------------------------------------------------------------------
\begin{frame}[fragile]
	\frametitle{Cambios en Profundidad}
	\framesubtitle{Algoritmo de hashing para contraseñas: PBKDF2}

	\begin{itemize}

		\item Un nuevo algoritmo "PBKDF2" de hashing para contraseñas ha sido añadido a la extensión del sistema "saltedpasswords"

		\item PBKDF2 se refiere a: Password-Based Key Derivation Function 2

		\item El algoritmo está diseñado para ser computacionalmente caro al resistir el crackeo de fuerza bruta

	\end{itemize}

\end{frame}

% ------------------------------------------------------------------------------
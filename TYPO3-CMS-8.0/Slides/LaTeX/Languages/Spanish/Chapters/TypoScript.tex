% ------------------------------------------------------------------------------
% TYPO3 CMS 7.6 - What's New - Chapter "TypoScript" (English Version)
%
% @author	Patrick Lobacher <patrick@lobacher.de> and Michael Schams <schams.net>
% @license	Creative Commons BY-NC-SA 3.0
% @link		http://typo3.org/download/release-notes/whats-new/
% @language	English
% ------------------------------------------------------------------------------
% LTXE-CHAPTER-UID:		b8fd0ec6-eeb4e126-fcdd7604-74ef6c57
% LTXE-CHAPTER-NAME:	TypoScript
% ------------------------------------------------------------------------------

\section{TSconfig \& TypoScript}
\begin{frame}[fragile]
	\frametitle{TSconfig \& TypoScript}

	\begin{center}\huge{Capítulo 2:}\end{center}
	\begin{center}\huge{\color{typo3darkgrey}\textbf{TSconfig \& TypoScript}}\end{center}

\end{frame}

% ------------------------------------------------------------------------------
% LTXE-SLIDE-START
% LTXE-SLIDE-UID:		b38c80ea-1a1e886e-0349d72c-17ed66fa
% LTXE-SLIDE-ORIGIN:	0956a9a6-bb0ea776-0327261b-7722ac16 English
% LTXE-SLIDE-TITLE:		Make new content element wizard tab sort order configurable
% LTXE-SLIDE-REFERENCE:	!Feature-71876-MakeNewContentElementWizardTabSortOrderConfigurable.rst
% ------------------------------------------------------------------------------
\begin{frame}[fragile]
	\frametitle{TSconfig \& TypoScript}
	\framesubtitle{Clasifique orden de tabs del asistente de nuevo elemento de contenido}

	% decrease font size for code listing
	\lstset{basicstyle=\tiny\ttfamily}

	\begin{itemize}
		\item Es posible configurar el orden de los tabs en el asistente de nuevo elemento de
			contenido configurando los valores \texttt{before} y \texttt{after} en Page TSconfig:

			\begin{lstlisting}
				mod.wizards.newContentElement.wizardItems.special.before = common
				mod.wizards.newContentElement.wizardItems.forms.after = common,special
			\end{lstlisting}

	\end{itemize}

\end{frame}

% ------------------------------------------------------------------------------
% LTXE-SLIDE-START
% LTXE-SLIDE-UID:		2e70b261-175a7291-115a4a20-661ad8e2
% LTXE-SLIDE-ORIGIN:	330b8895-6cecc482-88177047-8d65c88b English
% LTXE-SLIDE-TITLE:		HTMLparser.stripEmptyTags.keepTags
% LTXE-SLIDE-REFERENCE:	!Feature-72045-KeepTagsInHtmlParserWhenStrippingEmptyTags.rst
% ------------------------------------------------------------------------------
\begin{frame}[fragile]
	\frametitle{TSconfig \& TypoScript}
	\framesubtitle{\texttt{HTMLparser.stripEmptyTags.keepTags}}

	% decrease font size for code listing
	\lstset{basicstyle=\tiny\ttfamily}

	\begin{itemize}

		\item Ha sido añadida una nueva opción para la configuración de \texttt{HTMLparser.stripEmptyTags},
			que permite mantener etiquetas configuradas
		\item Antes de este cambio, sólo una lista de etiquetas podía ser proporcionada para ser eliminada
		\item El siguiente ejemplo elimina todas las etiquetas vacías \textbf{excepto} las etiquetas \texttt{tr} y \texttt{td}:

			\begin{lstlisting}
				HTMLparser.stripEmptyTags = 1
				HTMLparser.stripEmptyTags.keepTags = tr,td
			\end{lstlisting}

	\end{itemize}

	\underline{Importante:} si se usa este parámetro, la configuración \texttt{stripEmptyTags.tags}
		no tiene más efecto. Sólo puede usar una opción a la vez.

\end{frame}

% ------------------------------------------------------------------------------
% LTXE-SLIDE-START
% LTXE-SLIDE-UID:		1ff1453d-6a2f08db-78019620-3b52e2db
% LTXE-SLIDE-ORIGIN:	ec7822d8-5f3eeeb3-06a2c047-723568f7 English
% LTXE-SLIDE-TITLE:		EXT:form - integration of predefined forms (1)
% LTXE-SLIDE-REFERENCE:	!Feature-72309-EXTform-AllowIntegrationOfPredefinedForms.rst
% ------------------------------------------------------------------------------
\begin{frame}[fragile]
	\frametitle{TSconfig \& TypoScript}
	\framesubtitle{\texttt{EXT:form} - integración de formularios predefinidos (1)}

	% decrease font size for code listing
	\lstset{basicstyle=\tiny\ttfamily}

	\begin{itemize}

		\item El elemento de contenido de \texttt{EXT:form} ahora permite la integración de
			formularios predefinidos.

		\item Un integrador puede definir formularios (p.e. dentro de un paquete de sitio web) usando
			\texttt{plugin.tx\_form.predefinedForms}

		\item Un editor puede añadir un nuevo elemento de contenido \texttt{mailform} a una página y
			elegir un formulario de una lista de elementos predefinidos

		\item Los integradores pueden construir sus formularios con TypoScript, que proporciona muchas más
			opciones que al hacerlo dentro del asistente de formulario (p.e. integradores pueden
			usar la funcionalidad \texttt{stdWrap}, que no está disponible al usar el
			asistente de formulario (por razones de seguridad)

	\end{itemize}

\end{frame}

% ------------------------------------------------------------------------------
% LTXE-SLIDE-START
% LTXE-SLIDE-UID:		6c924846-658874f4-6c59f113-272ab247
% LTXE-SLIDE-ORIGIN:	04706a2c-eeb35f3e-22d8ec78-68f77235 English
% LTXE-SLIDE-TITLE:		EXT:form - integration of predefined forms (2)
% LTXE-SLIDE-REFERENCE:	!Feature-72309-EXTform-AllowIntegrationOfPredefinedForms.rst
% ------------------------------------------------------------------------------
\begin{frame}[fragile]
	\frametitle{TSconfig \& TypoScript}
	\framesubtitle{\texttt{EXT:form} - integración de formularios predefinidos (2)}

	% decrease font size for code listing
	\lstset{basicstyle=\tiny\ttfamily}

	\begin{itemize}

		\item No hay necesidad para los editores de usar el asistente de formulario nunca más.
			Los editores pueden elegir los formularios predefinidos que están optimizados en base al diseño.

		\item Los formularios pueden ser reutilizados a través de la instalación al completo

		\item Los formularios pueden almacenarse fuera de la BBDD y versionarse

		\item Para ser capaces de seleccionar el formulario predefinido en el backend,
			el formulario tiene que registrarse usando PageTS:

		\begin{lstlisting}
			TCEFORM.tt_content.tx_form_predefinedform.addItems.contactForm =
			  LLL:EXT:my_theme/Resources/Private/Language/locallang.xlf:contactForm
		\end{lstlisting}

	\end{itemize}

\end{frame}

% ------------------------------------------------------------------------------
% LTXE-SLIDE-START
% LTXE-SLIDE-UID:		71606144-47c32cbc-020a4c57-bb16e06f
% LTXE-SLIDE-ORIGIN:	5f3eeeb3-06a2c047-723568f7-ec7822d8 English
% LTXE-SLIDE-ORIGIN:	3076f305-06ba38cb-646d578c-b617df02 German
% LTXE-SLIDE-TITLE:		EXT:form - integration of predefined forms (3)
% LTXE-SLIDE-REFERENCE:	!Feature-72309-EXTform-AllowIntegrationOfPredefinedForms.rst
% ------------------------------------------------------------------------------
\begin{frame}[fragile]
	\frametitle{TSconfig \& TypoScript}
	\framesubtitle{\texttt{EXT:form}: integración de formularios predefinidos (3)}

	% decrease font size for code listing
	\lstset{basicstyle=\tiny\ttfamily}

	\begin{itemize}

		\item Formulario de ejemplo:

		\begin{lstlisting}
			plugin.tx_form.predefinedForms.contactForm = FORM
			plugin.tx_form.predefinedForms.contactForm {
			  enctype = multipart/form-data
			  method = post
			  prefix = contact
			  confirmation = 1
			  postProcessor {
			    1 = mail
			    1 {
			      recipientEmail = test@example.com
			      senderEmail = test@example.com
			      subject {
			        value = Contact form
			        lang.de = Kontakt Formular
			      }
			    }
			  }
			  10 = TEXTLINE
			  10 {
			    name = name
			...
		\end{lstlisting}

	\end{itemize}

\end{frame}

% ------------------------------------------------------------------------------

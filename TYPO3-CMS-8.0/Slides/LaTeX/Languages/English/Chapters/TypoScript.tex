% ------------------------------------------------------------------------------
% TYPO3 CMS 7.6 - What's New - Chapter "TypoScript" (English Version)
%
% @author	Patrick Lobacher <patrick@lobacher.de> and Michael Schams <schams.net>
% @license	Creative Commons BY-NC-SA 3.0
% @link		http://typo3.org/download/release-notes/whats-new/
% @language	English
% ------------------------------------------------------------------------------
% LTXE-CHAPTER-UID:		b8fd0ec6-eeb4e126-fcdd7604-74ef6c57
% LTXE-CHAPTER-NAME:	TypoScript
% ------------------------------------------------------------------------------

\section{TSconfig \& TypoScript}
\begin{frame}[fragile]
	\frametitle{TSconfig \& TypoScript}

	\begin{center}\huge{Chapter 2:}\end{center}
	\begin{center}\huge{\color{typo3darkgrey}\textbf{TSconfig \& TypoScript}}\end{center}

\end{frame}

% ------------------------------------------------------------------------------
% LTXE-SLIDE-START
% LTXE-SLIDE-UID:		0956a9a6-bb0ea776-0327261b-7722ac16
% LTXE-SLIDE-TITLE:		Make new content element wizard tab sort order configurable
% LTXE-SLIDE-REFERENCE:	!Feature-71876-MakeNewContentElementWizardTabSortOrderConfigurable.rst
% ------------------------------------------------------------------------------
\begin{frame}[fragile]
	\frametitle{TSconfig \& TypoScript}
	\framesubtitle{Sort order of new content element wizard tabs}

	% decrease font size for code listing
	\lstset{basicstyle=\tiny\ttfamily}

	\begin{itemize}
		\item It is possible to configure the order of the tabs in the new content element
			wizard by setting \texttt{before} and \texttt{after} values in Page TSconfig:

			\begin{lstlisting}
				mod.wizards.newContentElement.wizardItems.special.before = common
				mod.wizards.newContentElement.wizardItems.forms.after = common,special
			\end{lstlisting}

	\end{itemize}

\end{frame}

% ------------------------------------------------------------------------------
% LTXE-SLIDE-START
% LTXE-SLIDE-UID:		330b8895-6cecc482-88177047-8d65c88b
% LTXE-SLIDE-TITLE:		HTMLparser.stripEmptyTags.keepTags
% LTXE-SLIDE-REFERENCE:	!Feature-72045-KeepTagsInHtmlParserWhenStrippingEmptyTags.rst
% ------------------------------------------------------------------------------
\begin{frame}[fragile]
	\frametitle{TSconfig \& TypoScript}
	\framesubtitle{\texttt{HTMLparser.stripEmptyTags.keepTags}}

	% decrease font size for code listing
	\lstset{basicstyle=\tiny\ttfamily}

	\begin{itemize}

		\item New option for the \texttt{HTMLparser.stripEmptyTags} configuration has been added,
			which allows for keeping configured tags

		\item Before this change, only a list of tags that should be removed could be configured

		\item The following example strips all empty tags \textbf{except} \texttt{tr} and
			\texttt{td} tags:

			\begin{lstlisting}
				HTMLparser.stripEmptyTags = 1
				HTMLparser.stripEmptyTags.keepTags = tr,td
			\end{lstlisting}

	\end{itemize}

	\underline{Important:} if this setting is used, configuration \texttt{stripEmptyTags.tags}
		has no effect anymore. You can only use one option at a time.

\end{frame}

% ------------------------------------------------------------------------------
% LTXE-SLIDE-START
% LTXE-SLIDE-UID:		ec7822d8-5f3eeeb3-06a2c047-723568f7
% LTXE-SLIDE-TITLE:		EXT:form - integration of predefined forms (1)
% LTXE-SLIDE-REFERENCE:	!Feature-72309-EXTform-AllowIntegrationOfPredefinedForms.rst
% ------------------------------------------------------------------------------
\begin{frame}[fragile]
	\frametitle{TSconfig \& TypoScript}
	\framesubtitle{\texttt{EXT:form} - integration of predefined forms (1)}

	% decrease font size for code listing
	\lstset{basicstyle=\tiny\ttfamily}

	\begin{itemize}

		\item The content element of \texttt{EXT:form} now allows the integration of
			predefined forms.

		\item An integrator can define forms (e.g. within a site package) using
			\texttt{plugin.tx\_form.predefinedForms}

		\item An editor can add a new \texttt{mailform} content element to a page and
			choose a form from a list of predefined elements

		\item Integrators can build their forms with TypoScript, which provide much more
			options than doing it within the form wizard (e.g. integrators can
			use \texttt{stdWrap} functionality, which is not available when using the
			form wizard (for security reasons)

	\end{itemize}

\end{frame}

% ------------------------------------------------------------------------------
% LTXE-SLIDE-START
% LTXE-SLIDE-UID:		04706a2c-eeb35f3e-22d8ec78-68f77235
% LTXE-SLIDE-TITLE:		EXT:form - integration of predefined forms (2)
% LTXE-SLIDE-REFERENCE:	!Feature-72309-EXTform-AllowIntegrationOfPredefinedForms.rst
% ------------------------------------------------------------------------------
\begin{frame}[fragile]
	\frametitle{TSconfig \& TypoScript}
	\framesubtitle{\texttt{EXT:form} - integration of predefined forms (2)}

	% decrease font size for code listing
	\lstset{basicstyle=\tiny\ttfamily}

	\begin{itemize}

		\item There is no need for editors to use the form wizard anymore.
			Editors can choose the predefined forms which are optimized layout-wise.

		\item Forms can be re-used throughout the whole installation

		\item Forms can be stored outside the DB and versioned

		\item In order to be able to select the pre-defined form in the backend,
			the form has to be registered using PageTS:

		\begin{lstlisting}
			TCEFORM.tt_content.tx_form_predefinedform.addItems.contactForm =
			  LLL:EXT:my_theme/Resources/Private/Language/locallang.xlf:contactForm
		\end{lstlisting}

	\end{itemize}

\end{frame}

% ------------------------------------------------------------------------------
% LTXE-SLIDE-START
% LTXE-SLIDE-UID:		5f3eeeb3-06a2c047-723568f7-ec7822d8
% LTXE-SLIDE-ORIGIN:	3076f305-06ba38cb-646d578c-b617df02 German
% LTXE-SLIDE-TITLE:		EXT:form - integration of predefined forms (3)
% LTXE-SLIDE-REFERENCE:	!Feature-72309-EXTform-AllowIntegrationOfPredefinedForms.rst
% ------------------------------------------------------------------------------
\begin{frame}[fragile]
	\frametitle{TSconfig \& TypoScript}
	\framesubtitle{\texttt{EXT:form}: integration of predefined forms (3)}

	% decrease font size for code listing
	\lstset{basicstyle=\tiny\ttfamily}

	\begin{itemize}

		\item Example form:

		\begin{lstlisting}
			plugin.tx_form.predefinedForms.contactForm = FORM
			plugin.tx_form.predefinedForms.contactForm {
			  enctype = multipart/form-data
			  method = post
			  prefix = contact
			  confirmation = 1
			  postProcessor {
			    1 = mail
			    1 {
			      recipientEmail = test@example.com
			      senderEmail = test@example.com
			      subject {
			        value = Contact form
			        lang.de = Kontakt Formular
			      }
			    }
			  }
			  10 = TEXTLINE
			  10 {
			    name = name
			...
		\end{lstlisting}

	\end{itemize}

\end{frame}

% ------------------------------------------------------------------------------

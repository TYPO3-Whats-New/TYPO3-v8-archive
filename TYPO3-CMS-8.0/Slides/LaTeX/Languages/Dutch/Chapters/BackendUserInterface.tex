% ------------------------------------------------------------------------------
% TYPO3 CMS 8.0 - What's New (English Version)
%
% @author	Patrick Lobacher <patrick@lobacher.de> and Michael Schams <schams.net>
% @license	Creative Commons BY-NC-SA 3.0
% @link		http://typo3.org/download/release-notes/whats-new/
% @language	English
% ------------------------------------------------------------------------------
% LTXE-CHAPTER-UID:		d71b4b88-f79b2343-d2b356bf-452cef81
% LTXE-CHAPTER-NAME:	Backend User Interface
% ------------------------------------------------------------------------------

\section{Gebruikersinterface backend}
\begin{frame}[fragile]
	\frametitle{Gebruikersinterface backend}

	\begin{center}\huge{Hoofdstuk 1:}\end{center}
	\begin{center}\huge{\color{typo3darkgrey}\textbf{Gebruikersinterface backend}}\end{center}

\end{frame}

% ------------------------------------------------------------------------------
% LTXE-SLIDE-START
% LTXE-SLIDE-UID:		4f210947-8cc2dee1-1eeefd63-da860791
% LTXE-SLIDE-ORIGIN:	d8599a04-11b4fa8c-9be5812a-715820e1 English
% LTXE-SLIDE-ORIGIN:	ef1d6d8c-0db67396-dedb5814-1767169f German
% LTXE-SLIDE-TITLE:		Recover pages recursively to top of rootline
% LTXE-SLIDE-REFERENCE:	!Feature-1835-RecoverPagesRecursivelyToTop.rst
% ------------------------------------------------------------------------------
\begin{frame}[fragile]
	\frametitle{Gebruikersinterface backend}
	\framesubtitle{Herstel pagina's en alle onderliggende pagina's}

	De Prullenbak ondersteunt het herstellen van verwijderde pagina's en alle onderliggende
	pagina's.
	Dit is wegens interne rechtenbeperkingen alleen beschikbaar voor admin-gebruikers.

	\begin{figure}
		\includegraphics[width=0.70\linewidth]{BackendUserInterface/1835.png}
	\end{figure}

\end{frame}

% ------------------------------------------------------------------------------
% LTXE-SLIDE-START
% LTXE-SLIDE-UID:		0ff34ac6-181f0ad7-b9a79c5f-a05ce437
% LTXE-SLIDE-ORIGIN:	04855567-dac16e24-f5274c64-f1d49c36 English
% LTXE-SLIDE-TITLE:		EXT:form - Directly load form wizard as inline wizard
% LTXE-SLIDE-REFERENCE:	!Feature-69394-EXTform-DirectlyLoadFormWizardAsInlineWizard.rst
% ------------------------------------------------------------------------------
\begin{frame}[fragile]
	\frametitle{Gebruikersinterface backend}
	\framesubtitle{Gebruik formulierassistent als interne assistent}

	De assistent van EXT:form wordt direct als inline assistent geladen.
	Het is niet nodig om het nieuwe inhoudselement op te slaan en te herladen
	om de assistent te kunnen openen. Dit is een grote verbetering in het gebruik.

	\begin{figure}
		\includegraphics[width=0.70\linewidth]{BackendUserInterface/69394.png}
	\end{figure}

\end{frame}

% ------------------------------------------------------------------------------
% LTXE-SLIDE-START
% LTXE-SLIDE-UID:		d8423e0b-c6a1bd58-4253f498-7a427409
% LTXE-SLIDE-ORIGIN:	fd6d762a-b268caf0-cb6f9195-f553e035 English
% LTXE-SLIDE-TITLE:		Set the alternative backend logo via Extension Manager
% LTXE-SLIDE-REFERENCE:	!Feature-74109-SetTheAlternativeBackendLogoViaExtensionManager.rst
% ------------------------------------------------------------------------------
\begin{frame}[fragile]
	\frametitle{Gebruikersinterface backend}
	\framesubtitle{Alternatief logo in de backend}

	Het logo in de backend in de linkerbovenhoek kan nu via de configuratie van
	EXT:backend in het Extensiebeheer ingesteld worden.\newline
	Configuratieopties zijn:

	\begin{itemize}
		\item relatief pad binnen de TYPO3-installatie\newline
			\smaller
				bijv. "\texttt{fileadmin/images/my-background.jpg}"
			\normalsize

		\item pad naar een extensie\newline
			\smaller
				bijv. "\texttt{EXT:my\_theme/Resources/Public/Images/my-background.jpg}"
			\normalsize

		\item extern bestand\newline
			\smaller
				bijv. "\texttt{//example.com/my-background.png}"
			\normalsize

	\end{itemize}

	\begin{figure}
		\includegraphics[width=0.7\linewidth]{BackendUserInterface/74109.png}
	\end{figure}

\end{frame}

% ------------------------------------------------------------------------------
% LTXE-SLIDE-START
% LTXE-SLIDE-UID:		3921de05-ff8620d0-e13868b5-74c9d039
% LTXE-SLIDE-ORIGIN:	ab5ef36b-c670fbea-fd6d762a-cb6f9195 English
% LTXE-SLIDE-TITLE:		Page module: drag and drop supports copying now
% LTXE-SLIDE-REFERENCE:	!Feature-74179-PageModuleDragDropCanDoCopiesViaCTRLKeyNow.rst
% ------------------------------------------------------------------------------
\begin{frame}[fragile]
	\frametitle{Gebruikersinterface backend}
	\framesubtitle{Pagina's kopiëren via slepen}

	Naast het gebruikelijke slepen in de paginamodule (waarbij inhoudselementen \textit{verplaatst} worden),
	kunnen nu kopieën gemaakt worden: houd de CTRL-toets ingedrukt om een kopie van het gesleepte element
	te maken. Na het loslaten wordt de pagina herladen om zeker te zijn dat het nieuwe element wordt aangemaakt
	met alle benodigde informatie.

	\begin{figure}
		\includegraphics[width=0.7\linewidth]{BackendUserInterface/74179.png}
	\end{figure}

\end{frame}

% ------------------------------------------------------------------------------

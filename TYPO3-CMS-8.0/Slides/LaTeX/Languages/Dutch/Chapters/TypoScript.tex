% ------------------------------------------------------------------------------
% TYPO3 CMS 7.6 - What's New - Chapter "TypoScript" (English Version)
%
% @author	Patrick Lobacher <patrick@lobacher.de> and Michael Schams <schams.net>
% @license	Creative Commons BY-NC-SA 3.0
% @link		http://typo3.org/download/release-notes/whats-new/
% @language	English
% ------------------------------------------------------------------------------
% LTXE-CHAPTER-UID:		b8fd0ec6-eeb4e126-fcdd7604-74ef6c57
% LTXE-CHAPTER-NAME:	TypoScript
% ------------------------------------------------------------------------------

\section{TSconfig \& TypoScript}
\begin{frame}[fragile]
	\frametitle{TSconfig \& TypoScript}

	\begin{center}\huge{Hoofdstuk 2:}\end{center}
	\begin{center}\huge{\color{typo3darkgrey}\textbf{TSconfig \& TypoScript}}\end{center}

\end{frame}

% ------------------------------------------------------------------------------
% LTXE-SLIDE-START
% LTXE-SLIDE-UID:		5f1b30f8-262be3df-fff3eed0-e0fd6c6d
% LTXE-SLIDE-ORIGIN:	0956a9a6-bb0ea776-0327261b-7722ac16 English
% LTXE-SLIDE-TITLE:		Make new content element wizard tab sort order configurable
% LTXE-SLIDE-REFERENCE:	!Feature-71876-MakeNewContentElementWizardTabSortOrderConfigurable.rst
% ------------------------------------------------------------------------------
\begin{frame}[fragile]
	\frametitle{TSconfig \& TypoScript}
	\framesubtitle{Sorteervolgorde van tabs van assistent nieuw inhoudselement}

	% decrease font size for code listing
	\lstset{basicstyle=\tiny\ttfamily}

	\begin{itemize}
		\item Het is mogelijk om de volgorde van de tabbladen van de assistent voor nieuwe inhoudselementen
			in te stellen met \texttt{before}- en \texttt{after}-waardes in de page-TSconfig:

			\begin{lstlisting}
				mod.wizards.newContentElement.wizardItems.special.before = common
				mod.wizards.newContentElement.wizardItems.forms.after = common,special
			\end{lstlisting}

	\end{itemize}

\end{frame}

% ------------------------------------------------------------------------------
% LTXE-SLIDE-START
% LTXE-SLIDE-UID:		1f49924c-0de722e9-669530c6-e8ed9ca2
% LTXE-SLIDE-ORIGIN:	330b8895-6cecc482-88177047-8d65c88b English
% LTXE-SLIDE-TITLE:		HTMLparser.stripEmptyTags.keepTags
% LTXE-SLIDE-REFERENCE:	!Feature-72045-KeepTagsInHtmlParserWhenStrippingEmptyTags.rst
% ------------------------------------------------------------------------------
\begin{frame}[fragile]
	\frametitle{TSconfig \& TypoScript}
	\framesubtitle{\texttt{HTMLparser.stripEmptyTags.keepTags}}

	% decrease font size for code listing
	\lstset{basicstyle=\tiny\ttfamily}

	\begin{itemize}

		\item Nieuwe optie voor de configuratie van \texttt{HTMLparser.stripEmptyTags} is toegevoegd,
			voor het behouden van aangegeven tags
		\item Eerder kon alleen een serie tags verwijderd worden
		\item Het volgende voorbeeld verwijdert alle lege tags \textbf{behalve} \texttt{tr} en \texttt{td} :

			\begin{lstlisting}
				HTMLparser.stripEmptyTags = 1
				HTMLparser.stripEmptyTags.keepTags = tr,td
			\end{lstlisting}

	\end{itemize}

	\underline{Belangrijk:} als deze instelling gebruikt wordt, heeft \texttt{stripEmptyTags.tags}
		geen effect meer. Er kan maar één optie tegelijkertijd gebruikt worden.

\end{frame}

% ------------------------------------------------------------------------------
% LTXE-SLIDE-START
% LTXE-SLIDE-UID:		1a180912-20d51dac-53e69157-97a570fd
% LTXE-SLIDE-ORIGIN:	ec7822d8-5f3eeeb3-06a2c047-723568f7 English
% LTXE-SLIDE-TITLE:		EXT:form - integration of predefined forms (1)
% LTXE-SLIDE-REFERENCE:	!Feature-72309-EXTform-AllowIntegrationOfPredefinedForms.rst
% ------------------------------------------------------------------------------
\begin{frame}[fragile]
	\frametitle{TSconfig \& TypoScript}
	\framesubtitle{\texttt{EXT:form} - integratie van voorgedefinieerde formulieren (1)}

	% decrease font size for code listing
	\lstset{basicstyle=\tiny\ttfamily}

	\begin{itemize}

		\item Het inhoudselement van \texttt{EXT:form} ondersteunt nu het opnemen van
			voorgedefinieerde formulieren.

		\item Een integrator kan formulieren definiëren (bijv. in een site package) via
			\texttt{plugin.tx\_form.predefinedForms}

		\item Een redacteur kan een nieuw \texttt{mailform}-inhoudselement op een pagina zetten en
			een formulier uit de voorgedefinieerde lijst kiezen

		\item Integrators kunnen formulieren maken met TypoScript, waarbij meer opties beschikbaar
			zijn dan via de assistent (integrators kunnen bijv. \texttt{stdWrap} functionaliteit gebruiken
			die via de assistent (uit veiligheidsoverwegingen) niet beschikbaar zijn.

	\end{itemize}

\end{frame}

% ------------------------------------------------------------------------------
% LTXE-SLIDE-START
% LTXE-SLIDE-UID:		a980d081-8d4dfacd-640f17eb-b73f6473
% LTXE-SLIDE-ORIGIN:	04706a2c-eeb35f3e-22d8ec78-68f77235 English
% LTXE-SLIDE-TITLE:		EXT:form - integration of predefined forms (2)
% LTXE-SLIDE-REFERENCE:	!Feature-72309-EXTform-AllowIntegrationOfPredefinedForms.rst
% ------------------------------------------------------------------------------
\begin{frame}[fragile]
	\frametitle{TSconfig \& TypoScript}
	\framesubtitle{\texttt{EXT:form} - integratie van voorgedefinieerde formulieren (2)}

	% decrease font size for code listing
	\lstset{basicstyle=\tiny\ttfamily}

	\begin{itemize}

		\item Het is voor redacteuren niet meer nodig om de formulierassistent te gebruiken.
			Redacteuren kunnen kiezen uit voorgedefinieerde formulier die qua lay-out geoptimaliseerd zijn.

		\item Formulieren kunnen hergebruikt worden in de hele installatie

		\item Formulieren kunnen buiten de database opgeslagen worden onder versiebeheer

		\item Om voorgedefinieerde formulier in de backend te kunnen kiezen, moet het formulier via
			PageTS geregistreerd worden:

		\begin{lstlisting}
			TCEFORM.tt_content.tx_form_predefinedform.addItems.contactForm =
			  LLL:EXT:my_theme/Resources/Private/Language/locallang.xlf:contactForm
		\end{lstlisting}

	\end{itemize}

\end{frame}

% ------------------------------------------------------------------------------
% LTXE-SLIDE-START
% LTXE-SLIDE-UID:		14bf86b1-2e3b9e4a-11c2899c-a4aafb61
% LTXE-SLIDE-ORIGIN:	5f3eeeb3-06a2c047-723568f7-ec7822d8 English
% LTXE-SLIDE-ORIGIN:	3076f305-06ba38cb-646d578c-b617df02 German
% LTXE-SLIDE-TITLE:		EXT:form - integration of predefined forms (3)
% LTXE-SLIDE-REFERENCE:	!Feature-72309-EXTform-AllowIntegrationOfPredefinedForms.rst
% ------------------------------------------------------------------------------
\begin{frame}[fragile]
	\frametitle{TSconfig \& TypoScript}
	\framesubtitle{\texttt{EXT:form}: integratie van voorgedefinieerde formulieren (3)}

	% decrease font size for code listing
	\lstset{basicstyle=\tiny\ttfamily}

	\begin{itemize}

		\item Voorbeeld:

		\begin{lstlisting}
			plugin.tx_form.predefinedForms.contactForm = FORM
			plugin.tx_form.predefinedForms.contactForm {
			  enctype = multipart/form-data
			  method = post
			  prefix = contact
			  confirmation = 1
			  postProcessor {
			    1 = mail
			    1 {
			      recipientEmail = test@example.com
			      senderEmail = test@example.com
			      subject {
			        value = Contact form
			        lang.nl = Contactformulier
			      }
			    }
			  }
			  10 = TEXTLINE
			  10 {
			    name = name
			...
		\end{lstlisting}

	\end{itemize}

\end{frame}

% ------------------------------------------------------------------------------

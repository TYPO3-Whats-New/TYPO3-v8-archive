% ------------------------------------------------------------------------------
% TYPO3 CMS 7.6 - What's New - Chapter "In-Depth Changes" (English Version)
%
% @author	Patrick Lobacher <patrick@lobacher.de> and Michael Schams <schams.net>
% @license	Creative Commons BY-NC-SA 3.0
% @link		http://typo3.org/download/release-notes/whats-new/
% @language	English
% ------------------------------------------------------------------------------
% LTXE-CHAPTER-UID:		7229f1b9-b481e9bc-09c46183-a86b6a7e
% LTXE-CHAPTER-NAME:	In-Depth Changes
% ------------------------------------------------------------------------------

\section{Systeemwijzigingen}
\begin{frame}[fragile]
	\frametitle{Systeemwijzigingen}

	\begin{center}\huge{Hoofdstuk 3:}\end{center}
	\begin{center}\huge{\color{typo3darkgrey}\textbf{Systeemwijzigingen}}\end{center}

\end{frame}


% ------------------------------------------------------------------------------
% LTXE-SLIDE-START
% LTXE-SLIDE-UID:		358d7f01-6bcb94b7-c19463ad-19bf6151
% LTXE-SLIDE-ORIGIN:	b9b259e4-9d35ace7-edff1014-7390d1d1 English
% LTXE-SLIDE-TITLE:		Support PECL-memcached in MemcachedBackend
% LTXE-SLIDE-REFERENCE:	!Feature-69794-SupportPecl-memcachedInMemcachedBackend.rst
% ------------------------------------------------------------------------------
\begin{frame}[fragile]
	\frametitle{Systeemwijzigingen}
	\framesubtitle{Ondersteuning PECL-memcached in MemcachedBackend}

	% decrease font size for code listing
	\lstset{basicstyle=\tiny\ttfamily}

	\begin{itemize}

		\item Ondersteuning voor de PECL-module "memcached" is toegevoegd aan de
			MemcachedBackend van het Caching Framework

		\item Als zowel "memcache" als "memcached" zijn geïnstalleerd, wordt "memcache"
			gebruikt om geen brekende wijziging te zijn.

		\item Een integrator kan de optie \texttt{peclModule} gebruiken om de voorkeur
			in te stellen:

			\begin{lstlisting}
				$GLOBALS['TYPO3_CONF_VARS']['SYS']['caching']['cacheConfigurations']['my_memcached'] = [
				  'frontend' => \TYPO3\CMS\Core\Cache\Frontend\VariableFrontend::class
				  'backend' => \TYPO3\CMS\Core\Cache\Backend\MemcachedBackend::class,
				  'options' => [
				    'peclModule' => 'memcached',
				    'servers' => [
				      'localhost',
				      'server2:port'
				    ]
				  ]
				];
			\end{lstlisting}

	\end{itemize}

\end{frame}


% ------------------------------------------------------------------------------
% LTXE-SLIDE-START
% LTXE-SLIDE-UID:		7917813f-fa07a854-76aa5f92-80638d21
% LTXE-SLIDE-ORIGIN:	ef8c124e-f902453a-d48770b9-4cf86545 English
% LTXE-SLIDE-TITLE:		Native support for Symfony Console (1)
% LTXE-SLIDE-REFERENCE:	!Feature-73042-IntroduceNativeSupportForSymfonyConsole.rst
% ------------------------------------------------------------------------------
\begin{frame}[fragile]
	\frametitle{Systeemwijzigingen}
	\framesubtitle{Ondersteuning voor Symfony Console (1)}

	% decrease font size for code listing
	\lstset{basicstyle=\tiny\ttfamily}

	\begin{itemize}

		\item TYPO3 ondersteunt de Symfony Console automatisch via een nieuw commandoregel
			script in \texttt{typo3/sysext/core/bin/typo3}.
			Bij TYPO3-installaties die via Composer zijn gemaakt wordt het uitvoerbare bestand
			gekoppeld aan de \texttt{bin}-folder, bijv. \texttt{bin/typo3}.

		\item Het nieuwe script ondersteunt nog steeds de bestaande argumenten als er geen
			correct Symfony Console-commando wordt gevonden.

		\item Een commando dat via de TYPO3-commandoregel beschikbaar moet zijn, kan
		 	geregistreerd worden door het plaatsen van een \texttt{Configuration/Commands.php}-bestand
		 	in een geïnstalleerde extensie. De lijst Symfony/Console/Command klassen die uitgevoerd
		 	kunnen worden door TYPO3, is een hash-tabel. De index is de naam van het commando
		 	dat als eerste argument voor TYPO3 gebruikt wordt.

	\end{itemize}

\end{frame}

% ------------------------------------------------------------------------------
% LTXE-SLIDE-START
% LTXE-SLIDE-UID:		efd2a9a8-383469bd-7ec7f2f0-28e272d6
% LTXE-SLIDE-ORIGIN:	28b018bb-21804fa1-c578f0ff-fc16cbb1 English
% LTXE-SLIDE-TITLE:		Native support for Symfony Console (2)
% LTXE-SLIDE-REFERENCE:	!Feature-73042-IntroduceNativeSupportForSymfonyConsole.rst
% ------------------------------------------------------------------------------
\begin{frame}[fragile]
	\frametitle{Systeemwijzigingen}
	\framesubtitle{Ondersteuning voor Symfony Console (2)}

	% decrease font size for code listing
	\lstset{basicstyle=\tiny\ttfamily}

	\begin{itemize}

		\item Een vereiste parameter bij het registreren van een commando is de \texttt{class}-eigenschap.
			Optioneel kan de \texttt{user}-parameter ingesteld worden zodat een backend gebruiker ingelogd
			is bij het aanroepen van het commando.

		\item Een \texttt{Configuration/Commands.php} kan er zo uitzien:

			\begin{lstlisting}
				return [
				  'backend:lock' => [
				    'class' => \TYPO3\CMS\Backend\Command\LockBackendCommand::class
				  ],
				  'referenceindex:update' => [
				    'class' => \TYPO3\CMS\Backend\Command\ReferenceIndexUpdateCommand::class,
				    'user' => '_cli_lowlevel'
				  ]
				];
			\end{lstlisting}

	\end{itemize}

\end{frame}

% ------------------------------------------------------------------------------
% LTXE-SLIDE-START
% LTXE-SLIDE-UID:		a01a317d-ebef09d0-416a1b1f-b986b8de
% LTXE-SLIDE-ORIGIN:	fc16cbb1-21804fa1-c578f0ff-28b018bb English
% LTXE-SLIDE-TITLE:		Native support for Symfony Console (3)
% LTXE-SLIDE-REFERENCE:	!Feature-73042-IntroduceNativeSupportForSymfonyConsole.rst
% ------------------------------------------------------------------------------
\begin{frame}[fragile]
	\frametitle{Systeemwijzigingen}
	\framesubtitle{Ondersteuning voor Symfony Console (3)}

	% decrease font size for code listing
	\lstset{basicstyle=\tiny\ttfamily}

	\begin{itemize}

		\item Een aanroep kan er zo uitzien:
			\begin{lstlisting}
				bin/typo3 backend:lock http://example.com/maintenance.html
			\end{lstlisting}

		\item Voor een niet-Composer-installatie:
			\begin{lstlisting}
				typo3/sysext/core/bin/typo3 backend:lock http://example.com/maintenance.html
			\end{lstlisting}

	\end{itemize}

\end{frame}

% ------------------------------------------------------------------------------
% LTXE-SLIDE-START
% LTXE-SLIDE-UID:		31010378-9a0c1ea5-4960335c-70a674f4
% LTXE-SLIDE-ORIGIN:	1626edc6-8d93ff84-f64178bf-8e7650df English
% LTXE-SLIDE-TITLE:		Cryptographically secure pseudorandom number generator
% LTXE-SLIDE-REFERENCE:	!Feature-73050-AddACSPRNGAPI.rst
% ------------------------------------------------------------------------------
\begin{frame}[fragile]
	\frametitle{Systeemwijzigingen}
	\framesubtitle{Cryptografisch veilige pseudotoevalsgenerator}

	% decrease font size for code listing
	\lstset{basicstyle=\tiny\ttfamily}

	\begin{itemize}

		\item Een nieuwe cryptografisch veilige pseudotoevalsgenerator (CSPRNG) is
			geïmplementeerd in de TYPO3 core.\newline
			Het gebruikt de nieuwe CSPRNG-functies in PHP 7.

		\item De API zit in de klasse
			\texttt{\textbackslash
				TYPO3\textbackslash
				CMS\textbackslash
				Core\textbackslash
				Crypto\textbackslash
				Random}

		\item Voorbeeld:

			\begin{lstlisting}
				use \TYPO3\CMS\Core\Crypto\Random;
				use \TYPO3\CMS\Core\Utility\GeneralUtility;

				// Willekeurige bytes ophalen
				$someRandomString = GeneralUtility::makeInstance(Random::class)->generateRandomBytes(64);

				// Dobbelsteen rollen...
				$tossedValue = GeneralUtility::makeInstance(Random::class)->generateRandomInteger(1, 6);
			\end{lstlisting}

	\end{itemize}

\end{frame}

% ------------------------------------------------------------------------------
% LTXE-SLIDE-START
% LTXE-SLIDE-UID:		46daf6a6-4411ad12-d896d620-44625b5b
% LTXE-SLIDE-ORIGIN:	f098bb23-81eee73f-d9eb0d80-50bbe58b English
% LTXE-SLIDE-TITLE:		Wizard component (1)
% LTXE-SLIDE-REFERENCE:	!Feature-73429-WizardComponentHasBeenAdded.rst
% ------------------------------------------------------------------------------
\begin{frame}[fragile]
	\frametitle{Systeemwijzigingen}
	\framesubtitle{Assistentcomponent (1)}

	% decrease font size for code listing
	\lstset{basicstyle=\tiny\ttfamily}

	\begin{itemize}

		\item Een nieuwe assistentcomponent is toegevoegd. Dit component kan gebruikt worden voor begeleide interacties

		\item De RequireJS module kan gebruikt worden door het invoegen van \texttt{TYPO3/CMS/Backend/Wizard}

		\item De assistent ondersteunt alleen simpele acties\newline
			(vertakkingen zijn nog niet mogelijk)

		\item De API zit in de klasse
			\texttt{\textbackslash
				TYPO3\textbackslash
				CMS\textbackslash
				Core\textbackslash
				Crypto\textbackslash
				Random}

		\item De assistentcomponent heeft de volgende publieke functies:

			\begin{lstlisting}
				addSlide(identifier, title, content, severity, callback)
				addFinalProcessingSlide(callback)
				set(key, value)
				show()
				dismiss()
				getComponent()
				lockNextStep()
				unlockNextStep()
			\end{lstlisting}

	\end{itemize}

\end{frame}

% ------------------------------------------------------------------------------
% LTXE-SLIDE-START
% LTXE-SLIDE-UID:		f94d8d18-a3c59876-9bad2add-4e73fa68
% LTXE-SLIDE-ORIGIN:	f098bb23-81eee73f-d9eb0d80-50bbe58b English
% LTXE-SLIDE-TITLE:		Wizard component (2)
% LTXE-SLIDE-REFERENCE:	!Feature-73429-WizardComponentHasBeenAdded.rst
% ------------------------------------------------------------------------------
\begin{frame}[fragile]
	\frametitle{Systeemwijzigingen}
	\framesubtitle{Assistentcomponent (2)}

	% decrease font size for code listing
	\lstset{basicstyle=\tiny\ttfamily}

	\begin{itemize}

		\item De "event" \texttt{wizard-visible} wordt afgevuurd als de assistent klaar is met renderen

		\item Assistenten worden gesloten door het afvuren van \texttt{wizard-dismiss}

		\item Assistenten vuren het \texttt{wizard-dismissed} "event" af als de asisstent wordt gesloten

		\item Een eigen listener kan geregistreerd worden met \texttt{Wizard.getComponent()}

	\end{itemize}

\end{frame}

% ------------------------------------------------------------------------------
% LTXE-SLIDE-START
% LTXE-SLIDE-UID:		0ce93836-0a06676d-3e78ecd8-82b5d51d
% LTXE-SLIDE-ORIGIN:	5f0d3565-8b6ad76b-44ec4d62-247401f7 English
% LTXE-SLIDE-TITLE:		Generated asset files moved
% LTXE-SLIDE-REFERENCE:	!Important-72580-PubliclyAccessibleGeneratedAssetFilesMovedToTypo3tempassets.rst
% ------------------------------------------------------------------------------
\begin{frame}[fragile]
	\frametitle{Systeemwijzigingen}
	\framesubtitle{Gegenereerde assetbestanden verplaatst}

	% decrease font size for code listing
	\lstset{basicstyle=\tiny\ttfamily}

	\begin{itemize}

		\item De mappenstructuur binnen \texttt{typo3temp} is gewijzigd om de assets
			die publiekelijk beschikbaar zijn, te scheiden van de tijdelijke bestanden (bijv.
			voor caching of blokkeringen die alleen voor de server bedoeld zijn).

		\item Deze assets zijn verplaatst van de mappen:\newline
			\texttt{\_processed\_}, \texttt{compressor}, \texttt{GB}, \texttt{temp},
			\texttt{Language}, \texttt{pics}\newline
			naar:

			\begin{itemize}
				\item \texttt{typo3temp/assets/js/}
				\item \texttt{typo3temp/assets/css/},
				\item \texttt{typo3temp/assets/compressed/}
				\item \texttt{typo3temp/assets/images/}
			\end{itemize}

	\end{itemize}

\end{frame}

% ------------------------------------------------------------------------------
% LTXE-SLIDE-START
% LTXE-SLIDE-UID:		367178e7-0eb9cc9d-42ecd81a-db357229
% LTXE-SLIDE-ORIGIN:	a4d7bec0-6c19d8b2-6a6be951-5d5aa0c0 English
% LTXE-SLIDE-TITLE:		ImageMagick/GraphicsMagick changes (1)
% LTXE-SLIDE-REFERENCE:	!Breaking-43085-RenamedGraphicsProcessorSettings.rst
% ------------------------------------------------------------------------------
\begin{frame}[fragile]
	\frametitle{Systeemwijzigingen}
	\framesubtitle{ImageMagick/GraphicsMagick wijzigingen (1)}

	% decrease font size for code listing
	\lstset{basicstyle=\tiny\ttfamily}

	\begin{itemize}

		\item Instellingen voor grafische acties van Image- of GraphicsMagick zijn hernoemd
			(bestand: \texttt{LocalConfiguration.php}).\newline
			OUD:\tabto{1.2cm} \texttt{im\_}\newline
			NIEUW:\tabto{1.2cm} \texttt{processor\_}

		\item Negatieve naamgeving zoals \texttt{noScaleUp} is omgezet in positieve varianten.
			Tijdens de conversie worden de vorige waarden omgekeerd overeenkomstig de naamgeving
			van deze opties.

		\item Daarnaast zijn verwijzingen naar specifieke versies van ImageMagick/GraphicsMagick
		 	verwijderd uit de namen en waarden van de opties.

	\end{itemize}

\end{frame}

% ------------------------------------------------------------------------------
% LTXE-SLIDE-START
% LTXE-SLIDE-UID:		ee82c7a4-67a9e37d-ddbc041a-b76a3112
% LTXE-SLIDE-ORIGIN:	d48d2af2-66b218a6-4a54e402-097ad85e English
% LTXE-SLIDE-TITLE:		ImageMagick/GraphicsMagick changes (2)
% LTXE-SLIDE-REFERENCE:	!Breaking-43085-RenamedGraphicsProcessorSettings.rst
% ------------------------------------------------------------------------------
\begin{frame}[fragile]
	\frametitle{Systeemwijzigingen}
	\framesubtitle{ImageMagick/GraphicsMagick wijzigingen (2)}

	% decrease font size for code listing
	\lstset{basicstyle=\tiny\ttfamily}

	\begin{itemize}

		\item De niet gebruikte configuratieoptie \texttt{image\_processing} is verwijderd zonder vervanging

		\item De processor-specifieke optie \texttt{colorspace} is voorzien van een \textit{namespace}
			onder de \texttt{processor\_} boom

	\end{itemize}

\end{frame}

% ------------------------------------------------------------------------------
% LTXE-SLIDE-START
% LTXE-SLIDE-UID:		722b8be7-151c7a2f-21928960-17502a0a
% LTXE-SLIDE-ORIGIN:	74e4b871-9f2ded00-f54bc3c1-85d22167 English
% LTXE-SLIDE-TITLE:		Hooks and Signals (1): post-processing of the preview URL
% LTXE-SLIDE-REFERENCE:	!Feature-54887-Post-processingOfThePreviewUrl.rst
% ------------------------------------------------------------------------------
\begin{frame}[fragile]
	\frametitle{Systeemwijzigingen}
	\framesubtitle{Hooks en Signals (1)}

	% decrease font size for code listing
	\lstset{basicstyle=\tiny\ttfamily}

	\begin{itemize}

		\item Een extra hook is toegevoegd aan de methode \texttt{BackendUtility::viewOnClick()}
			voor het nabewerken van de voorvertoningsURL

		\item Registreer een hook-klasse die de methode \texttt{postProcess} bevat:

			\begin{lstlisting}
				$GLOBALS['TYPO3_CONF_VARS']['SC_OPTIONS']['t3lib/class.t3lib_befunc.php']['viewOnClickClass'][] =
				  \VENDOR\MyExt\Hooks\BackendUtilityHook::class;
			\end{lstlisting}

	\end{itemize}

\end{frame}

% ------------------------------------------------------------------------------
% LTXE-SLIDE-START
% LTXE-SLIDE-UID:		fd062553-e94fc1b3-dd3a6889-57cafffc
% LTXE-SLIDE-ORIGIN:	9c150d50-ee3ba19b-120ef653-e714135c English
% LTXE-SLIDE-TITLE:		Hooks and Signals (2): hook to override a record overlay
% LTXE-SLIDE-REFERENCE:	!Feature-72505-IntroduceHookToOverrideARecordOverlay.rst
% ------------------------------------------------------------------------------
\begin{frame}[fragile]
	\frametitle{Systeemwijzigingen}
	\framesubtitle{Hooks en Signals (2)}

	% decrease font size for code listing
	\lstset{basicstyle=\tiny\ttfamily}

	\begin{itemize}

		\item Vóór TYPO3 CMS 7.6 was het mogelijk om een overlapping van een record te overschrijven in \texttt{Web -> List}.
			Een nieuwe hook in TYPO3 CMS 8.0 biedt de oude functionaliteit.

		\item De hook wordt aangeroepen met de volgende argumenten:
			\begin{lstlisting}
				/**
				 * @param string $table
				 * @param array $row
				 * @param array $status
				 * @param string $iconName
				 * @return string the new (or given) $iconName
				 */
				function postOverlayPriorityLookup($table, array $row, array $status, $iconName) { ... }
			\end{lstlisting}

		\item Registreer de hook-klasse die dit implementeert met de naam \texttt{postOverlayPriorityLookup}:

			\begin{lstlisting}
				$GLOBALS['TYPO3_CONF_VARS']['SC_OPTIONS'][IconFactory::class]['overrideIconOverlay'][] =
				  \VENDOR\MyExt\Hooks\IconFactoryHook::class;
			\end{lstlisting}

	\end{itemize}

\end{frame}

% ------------------------------------------------------------------------------
% LTXE-SLIDE-START
% LTXE-SLIDE-UID:		db5170d1-ee6d35af-5c7d2b3d-f7733e35
% LTXE-SLIDE-ORIGIN:	37ba9f7c-fb51932d-cca9a99f-6bb7a961 English
% LTXE-SLIDE-TITLE:		Hooks and Signals (3): preProcessStorage signal
% LTXE-SLIDE-REFERENCE:	!Feature-72904-AddPreProcessStorageSignalToResourceFactory.rst
% ------------------------------------------------------------------------------
\begin{frame}[fragile]
	\frametitle{Systeemwijzigingen}
	\framesubtitle{Hooks en Signals (3)}

	% decrease font size for code listing
	\lstset{basicstyle=\tiny\ttfamily}

	\begin{itemize}

		\item Een nieuw signal is geïmplementeerd voordat een opslag voor bronnen is geïnitialiseerd.

		\item Registreer de klasse met de logica in \texttt{ext\_localconf.php}:

			\begin{lstlisting}
				$dispatcher = \TYPO3\CMS\Core\Utility\GeneralUtility::makeInstance(
				  \TYPO3\CMS\Extbase\SignalSlot\Dispatcher::class);
				$dispatcher->connect(
				  \TYPO3\CMS\Core\Resource\ResourceFactory::class,
				  ResourceFactoryInterface::SIGNAL_PreProcessStorage,
				  \MY\ExtKey\Slots\ResourceFactorySlot::class,
				  'preProcessStorage'
				);
			\end{lstlisting}

		\item De methode wordt aangeroepen met de volgende parameters:

			\begin{itemize}
				\item int \texttt{\$uid} de uid van het record
				\item array \texttt{\$recordData} alle record-gegevens als array
				\item string \texttt{\$fileIdentifier} de bestandsidentifier
			\end{itemize}

	\end{itemize}

\end{frame}

% ------------------------------------------------------------------------------
% LTXE-SLIDE-START
% LTXE-SLIDE-UID:		dedc29ab-fa6e6fa7-a925cea9-bc136d1c
% LTXE-SLIDE-ORIGIN:	2c656d9e-38b9be7a-d590cb6e-21eccd4d English
% LTXE-SLIDE-TITLE:		Password hashing algorithm: PBKDF2
% LTXE-SLIDE-REFERENCE:	!Feature-28230-AddSupportForPBKDF2ToSaltedpasswords.rst
% ------------------------------------------------------------------------------
\begin{frame}[fragile]
	\frametitle{Systeemwijzigingen}
	\framesubtitle{Algoritme voor wachtwoordhash: PBKDF2}

	\begin{itemize}

		\item Een nieuw hash-algoritme voor wachwoorden, "PBKDF2", is toegevoegd aan de systeemextensie "saltedpasswords"

		\item PBKDF2 betekent: Password-Based Key Derivation Function 2

		\item Het algoritme kost veel rekenkracht om brute force password cracking te weerstaan

	\end{itemize}

\end{frame}

% ------------------------------------------------------------------------------
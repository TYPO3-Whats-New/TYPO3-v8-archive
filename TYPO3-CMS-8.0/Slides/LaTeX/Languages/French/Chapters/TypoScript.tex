% ------------------------------------------------------------------------------
% TYPO3 CMS 8.0 - What's New - Chapter "TypoScript" (French Version)
%
% @author	Patrick Lobacher <patrick@lobacher.de> and Michael Schams <schams.net>
% @license	Creative Commons BY-NC-SA 3.0
% @link		http://typo3.org/download/release-notes/whats-new/
% @language	French
% ------------------------------------------------------------------------------
% LTXE-CHAPTER-UID:		b8fd0ec6-eeb4e126-fcdd7604-74ef6c57
% LTXE-CHAPTER-NAME:	TypoScript
% ------------------------------------------------------------------------------

\section{TSconfig \& TypoScript}
\begin{frame}[fragile]
	\frametitle{TSconfig \& TypoScript}

	\begin{center}\huge{Chapitre 2~:}\end{center}
	\begin{center}\huge{\color{typo3darkgrey}\textbf{TSconfig \& TypoScript}}\end{center}

\end{frame}

% ------------------------------------------------------------------------------
% LTXE-SLIDE-START
% LTXE-SLIDE-UID:		13223b3c-b3eaa963-1505fb01-79ac3d5e
% LTXE-SLIDE-ORIGIN:	0956a9a6-bb0ea776-0327261b-7722ac16 English
% LTXE-SLIDE-TITLE:		Make new content element wizard tab sort order configurable
% LTXE-SLIDE-REFERENCE:	!Feature-71876-MakeNewContentElementWizardTabSortOrderConfigurable.rst
% ------------------------------------------------------------------------------
\begin{frame}[fragile]
	\frametitle{TSconfig \& TypoScript}
	\framesubtitle{Ordre de tri des onglets de l'assistant nouveau contenu}

	% decrease font size for code listing
	\lstset{basicstyle=\tiny\ttfamily}

	\begin{itemize}
		\item Il est possible de configurer l'ordre des onglets dans l'assistant nouvel
			élément de contenu avec les options \texttt{before} et \texttt{after} dans
			le TSconfig de page~:

			\begin{lstlisting}
				mod.wizards.newContentElement.wizardItems.special.before = common
				mod.wizards.newContentElement.wizardItems.forms.after = common,special
			\end{lstlisting}

	\end{itemize}

\end{frame}

% ------------------------------------------------------------------------------
% LTXE-SLIDE-START
% LTXE-SLIDE-UID:		de85fa88-5c21d0a8-46fda064-08c68009
% LTXE-SLIDE-ORIGIN:	330b8895-6cecc482-88177047-8d65c88b English
% LTXE-SLIDE-TITLE:		HTMLparser.stripEmptyTags.keepTags
% LTXE-SLIDE-REFERENCE:	!Feature-72045-KeepTagsInHtmlParserWhenStrippingEmptyTags.rst
% ------------------------------------------------------------------------------
\begin{frame}[fragile]
	\frametitle{TSconfig \& TypoScript}
	\framesubtitle{\texttt{HTMLparser.stripEmptyTags.keepTags}}

	% decrease font size for code listing
	\lstset{basicstyle=\tiny\ttfamily}

	\begin{itemize}

		\item De nouvelles options pour \texttt{HTMLparser.stripEmptyTags} sont ajoutées,
			permettant de garder les balises configurées
		\item Avant le changement, seul une liste de balises devant être retirées pouvait être fournie
		\item L'exemple suivant retire toutes les balises vides \textbf{à l'exception} des balises
		 	\texttt{tr} et \texttt{td}~:

			\begin{lstlisting}
				HTMLparser.stripEmptyTags = 1
				HTMLparser.stripEmptyTags.keepTags = tr,td
			\end{lstlisting}

	\end{itemize}

	\underline{Important~:} si l'option est utilisée, la configuration \texttt{stripEmptyTags.tags}
		n'a plus d'effet. Vous ne pouvez utiliser que l'une des deux options à la fois.

\end{frame}

% ------------------------------------------------------------------------------
% LTXE-SLIDE-START
% LTXE-SLIDE-UID:		742f3490-afa200e1-19572e49-4f1a981a
% LTXE-SLIDE-ORIGIN:	ec7822d8-5f3eeeb3-06a2c047-723568f7 English
% LTXE-SLIDE-TITLE:		EXT:form - integration of predefined forms (1)
% LTXE-SLIDE-REFERENCE:	!Feature-72309-EXTform-AllowIntegrationOfPredefinedForms.rst
% ------------------------------------------------------------------------------
\begin{frame}[fragile]
	\frametitle{TSconfig \& TypoScript}
	\framesubtitle{\texttt{EXT:form} - intégration de formulaires prédéfinis (1)}

	% decrease font size for code listing
	\lstset{basicstyle=\tiny\ttfamily}

	\begin{itemize}

		\item L'élément de contenu de \texttt{EXT:form} permet l'intégration de
			formulaires prédéfinis.

		\item Un intégrateur peut définir des formulaires (par ex. dans une distribution)
			en utilisant \texttt{plugin.tx\_form.predefinedForms}

		\item Un éditeur peut ajouter un nouvel élément de contenu \texttt{mailform} à une
			page et choisir un formulaire dans la liste des éléments prédéfinis

		\item Les intégrateurs peuvent construire leur formulaire en TypoScript, fournissant
			plus d'options qu'avec l'assistant formulaire (par ex. la fonctionnalité
			\texttt{stdWrap}, non disponible avec l'assistant (pour des raisons de sécurité))

	\end{itemize}

\end{frame}

% ------------------------------------------------------------------------------
% LTXE-SLIDE-START
% LTXE-SLIDE-UID:		e4206f3c-19d7fc59-9a9201bb-5c4d78f5
% LTXE-SLIDE-ORIGIN:	04706a2c-eeb35f3e-22d8ec78-68f77235 English
% LTXE-SLIDE-TITLE:		EXT:form - integration of predefined forms (2)
% LTXE-SLIDE-REFERENCE:	!Feature-72309-EXTform-AllowIntegrationOfPredefinedForms.rst
% ------------------------------------------------------------------------------
\begin{frame}[fragile]
	\frametitle{TSconfig \& TypoScript}
	\framesubtitle{\texttt{EXT:form} - intégration de formulaires prédéfinis (2)}

	% decrease font size for code listing
	\lstset{basicstyle=\tiny\ttfamily}

	\begin{itemize}

		\item Les éditeurs n'ont plus besoin d'utiliser l'assistant formulaire.
			Ils peuvent utiliser les formulaires prédéfinis qui sont optimisés.

		\item Les formulaires peuvent être réutilisés dans toute l'instance

		\item Les formulaires peuvent être enregistrés en dehors de la base et versionnés

		\item Afin de pouvoir sélectionner un formulaire prédéfini en backend,
			le formulaire doit être inscrit en utilisant le PageTS~:

		\begin{lstlisting}
			TCEFORM.tt_content.tx_form_predefinedform.addItems.contactForm =
			  LLL:EXT:my_theme/Resources/Private/Language/locallang.xlf:contactForm
		\end{lstlisting}

	\end{itemize}

\end{frame}

% ------------------------------------------------------------------------------
% LTXE-SLIDE-START
% LTXE-SLIDE-UID:		0b612df1-ea827c02-e115053e-77e52858
% LTXE-SLIDE-ORIGIN:	5f3eeeb3-06a2c047-723568f7-ec7822d8 English
% LTXE-SLIDE-ORIGIN:	3076f305-06ba38cb-646d578c-b617df02 German
% LTXE-SLIDE-TITLE:		EXT:form - integration of predefined forms (3)
% LTXE-SLIDE-REFERENCE:	!Feature-72309-EXTform-AllowIntegrationOfPredefinedForms.rst
% ------------------------------------------------------------------------------
\begin{frame}[fragile]
	\frametitle{TSconfig \& TypoScript}
	\framesubtitle{\texttt{EXT:form}: intégration de formulaires prédéfinis (3)}

	% decrease font size for code listing
	\lstset{basicstyle=\tiny\ttfamily}

	\begin{itemize}

		\item Formulaire d'exemple~:

		\begin{lstlisting}
			plugin.tx_form.predefinedForms.contactForm = FORM
			plugin.tx_form.predefinedForms.contactForm {
			  enctype = multipart/form-data
			  method = post
			  prefix = contact
			  confirmation = 1
			  postProcessor {
			    1 = mail
			    1 {
			      recipientEmail = test@example.com
			      senderEmail = test@example.com
			      subject {
			        value = Contact form
			        lang.de = Kontakt Formular
			      }
			    }
			  }
			  10 = TEXTLINE
			  10 {
			    name = name
			...
		\end{lstlisting}

	\end{itemize}

\end{frame}

% ------------------------------------------------------------------------------

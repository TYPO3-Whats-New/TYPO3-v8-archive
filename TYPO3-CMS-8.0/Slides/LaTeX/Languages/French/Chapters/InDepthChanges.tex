% ------------------------------------------------------------------------------
% TYPO3 CMS 8.0 - What's New - Chapter "In-Depth Changes" (French Version)
%
% @author	Patrick Lobacher <patrick@lobacher.de> and Michael Schams <schams.net>
% @license	Creative Commons BY-NC-SA 3.0
% @link		http://typo3.org/download/release-notes/whats-new/
% @language	French
% ------------------------------------------------------------------------------
% LTXE-CHAPTER-UID:		7229f1b9-b481e9bc-09c46183-a86b6a7e
% LTXE-CHAPTER-NAME:	In-Depth Changes
% ------------------------------------------------------------------------------

\section{Changements en profondeur}
\begin{frame}[fragile]
	\frametitle{Changements en profondeur}

	\begin{center}\huge{Chapitre 3~:}\end{center}
	\begin{center}\huge{\color{typo3darkgrey}\textbf{Changements en profondeur}}\end{center}

\end{frame}


% ------------------------------------------------------------------------------
% LTXE-SLIDE-START
% LTXE-SLIDE-UID:		8bab6e90-a06496e0-18d0321a-4db3e755
% LTXE-SLIDE-ORIGIN:	b9b259e4-9d35ace7-edff1014-7390d1d1 English
% LTXE-SLIDE-TITLE:		Support PECL-memcached in MemcachedBackend
% LTXE-SLIDE-REFERENCE:	!Feature-69794-SupportPecl-memcachedInMemcachedBackend.rst
% ------------------------------------------------------------------------------
\begin{frame}[fragile]
	\frametitle{Changements en profondeur}
	\framesubtitle{Support de PECL-memcached dans MemcachedBackend}

	% decrease font size for code listing
	\lstset{basicstyle=\tiny\ttfamily}

	\begin{itemize}

		\item Le support du module PECL «~memcached~» est ajouté au
			MemcachedBackend du Caching Framework

		\item Si les deux extensions «~memcache~» et «~memcached~» sont installées,
			alors «~memcached~» est utilisé afin de ne pas provoquer d'anomalie.

		\item Un intégrateur peut définir l'option \texttt{peclModule} pour préférer
			utiliser le module PECL~:

			\begin{lstlisting}
				$GLOBALS['TYPO3_CONF_VARS']['SYS']['caching']['cacheConfigurations']['my_memcached'] = [
				  'frontend' => \TYPO3\CMS\Core\Cache\Frontend\VariableFrontend::class
				  'backend' => \TYPO3\CMS\Core\Cache\Backend\MemcachedBackend::class,
				  'options' => [
				    'peclModule' => 'memcached',
				    'servers' => [
				      'localhost',
				      'server2:port'
				    ]
				  ]
				];
			\end{lstlisting}

	\end{itemize}

\end{frame}


% ------------------------------------------------------------------------------
% LTXE-SLIDE-START
% LTXE-SLIDE-UID:		f1e7026b-89f2170f-b8d1e483-2667df6f
% LTXE-SLIDE-ORIGIN:	ef8c124e-f902453a-d48770b9-4cf86545 English
% LTXE-SLIDE-TITLE:		Native support for Symfony Console (1)
% LTXE-SLIDE-REFERENCE:	!Feature-73042-IntroduceNativeSupportForSymfonyConsole.rst
% ------------------------------------------------------------------------------
\begin{frame}[fragile]
	\frametitle{Changements en profondeur}
	\framesubtitle{Support natif de la Console Symfony (1)}

	% decrease font size for code listing
	\lstset{basicstyle=\tiny\ttfamily}

	\begin{itemize}

		\item TYPO3 supporte directement le composant Console de Symfony en fournissant un
			nouveau script de ligne de commande placé dans \texttt{typo3/sysext/core/bin/typo3}.
			Pour les installations TYPO3 via Composer, un lien symbolique est créé dans le dossier
			\texttt{bin}, par ex. \texttt{bin/typo3}.

		\item Inscrire une commande pour qu'elle soit disponible via la commande
			\texttt{typo3} s'effectue par la création du fichier \texttt{Configuration/Commands.php}
			dans une extension installée. Il liste les classes Symfony/Console/Command
			à exécuter par \texttt{typo3} dans un tableau associatif. La clé est le nom de
			la commande à appeler, premier argument de \texttt{typo3}.

	\end{itemize}

\end{frame}

% ------------------------------------------------------------------------------
% LTXE-SLIDE-START
% LTXE-SLIDE-UID:		34db1d1a-0ef75f9d-0e4efd3f-7a0c08a4
% LTXE-SLIDE-ORIGIN:	28b018bb-21804fa1-c578f0ff-fc16cbb1 English
% LTXE-SLIDE-TITLE:		Native support for Symfony Console (2)
% LTXE-SLIDE-REFERENCE:	!Feature-73042-IntroduceNativeSupportForSymfonyConsole.rst
% ------------------------------------------------------------------------------
\begin{frame}[fragile]
	\frametitle{Changements en profondeur}
	\framesubtitle{Support natif de la Console Symfony (2)}

	% decrease font size for code listing
	\lstset{basicstyle=\tiny\ttfamily}

	\begin{itemize}

		\item Le nouveau script supporte toujours les anciennes commandes lorsqu'aucune
			commande Symfony Console n'est trouvée pour la compatibilité

		\item Lors de l'inscription d'une commande, la propriété \texttt{class} est requise.
			La propriété optionnelle \texttt{user} est à utiliser pour qu'un utilisateur
			backend soit authentifié lors de l'appel.

		\item Un fichier \texttt{Configuration/Commands.php} peut ressembler à ceci~:

			\begin{lstlisting}
				return [
				  'backend:lock' => [
				    'class' => \TYPO3\CMS\Backend\Command\LockBackendCommand::class
				  ],
				  'referenceindex:update' => [
				    'class' => \TYPO3\CMS\Backend\Command\ReferenceIndexUpdateCommand::class,
				    'user' => '_cli_lowlevel'
				  ]
				];
			\end{lstlisting}

	\end{itemize}

\end{frame}

% ------------------------------------------------------------------------------
% LTXE-SLIDE-START
% LTXE-SLIDE-UID:		8759751d-de0b238a-d0456a68-b3fb0ed6
% LTXE-SLIDE-ORIGIN:	fc16cbb1-21804fa1-c578f0ff-28b018bb English
% LTXE-SLIDE-TITLE:		Native support for Symfony Console (3)
% LTXE-SLIDE-REFERENCE:	!Feature-73042-IntroduceNativeSupportForSymfonyConsole.rst
% ------------------------------------------------------------------------------
\begin{frame}[fragile]
	\frametitle{Changements en profondeur}
	\framesubtitle{Support natif de la Console Symfony (3)}

	% decrease font size for code listing
	\lstset{basicstyle=\tiny\ttfamily}

	\begin{itemize}

		\item En appel d'exemple peut être~:
			\begin{lstlisting}
				bin/typo3 backend:lock http://example.com/maintenance.html
			\end{lstlisting}

		\item Pour une installation non Composer~:
			\begin{lstlisting}
				typo3/sysext/core/bin/typo3 backend:lock http://example.com/maintenance.html
			\end{lstlisting}

	\end{itemize}

\end{frame}

% ------------------------------------------------------------------------------
% LTXE-SLIDE-START
% LTXE-SLIDE-UID:		e1de10cf-2746eb50-da30c1ca-72b3a462
% LTXE-SLIDE-ORIGIN:	1626edc6-8d93ff84-f64178bf-8e7650df English
% LTXE-SLIDE-TITLE:		Cryptographically secure pseudorandom number generator
% LTXE-SLIDE-REFERENCE:	!Feature-73050-AddACSPRNGAPI.rst
% ------------------------------------------------------------------------------
\begin{frame}[fragile]
	\frametitle{Changements en profondeur}
	\framesubtitle{Cryptographically secure pseudorandom number generator}

	% decrease font size for code listing
	\lstset{basicstyle=\tiny\ttfamily}

	\begin{itemize}

		\item Un nouveau «~cryptographically secure pseudo-random number generator~» (CSPRNG) est
			implémenté dans le noyau de TYPO3.\newline
			Il utilise les nouvelles fonctions CSPRNG de PHP 7.

		\item L'API est dans la classe
			\texttt{\textbackslash
				TYPO3\textbackslash
				CMS\textbackslash
				Core\textbackslash
				Crypto\textbackslash
				Random}

		\item Exemple~:

			\begin{lstlisting}
				use \TYPO3\CMS\Core\Crypto\Random;
				use \TYPO3\CMS\Core\Utility\GeneralUtility;

				// Retrieving random bytes
				$someRandomString = GeneralUtility::makeInstance(Random::class)->generateRandomBytes(64);

				// Rolling the dice..
				$tossedValue = GeneralUtility::makeInstance(Random::class)->generateRandomInteger(1, 6);
			\end{lstlisting}

	\end{itemize}

\end{frame}

% ------------------------------------------------------------------------------
% LTXE-SLIDE-START
% LTXE-SLIDE-UID:		cb196cd9-d0c9c728-a608111d-286320ec
% LTXE-SLIDE-ORIGIN:	f098bb23-81eee73f-d9eb0d80-50bbe58b English
% LTXE-SLIDE-TITLE:		Wizard component (1)
% LTXE-SLIDE-REFERENCE:	!Feature-73429-WizardComponentHasBeenAdded.rst
% ------------------------------------------------------------------------------
\begin{frame}[fragile]
	\frametitle{Changements en profondeur}
	\framesubtitle{Composant Assistant (1)}

	% decrease font size for code listing
	\lstset{basicstyle=\tiny\ttfamily}

	\begin{itemize}

		\item Un nouveau composant d'assistant (wizard) est ajouté.
			Ce composant s'utilise pour des interactions guidées

		\item Le module RequireJS s'utilise en incluant \texttt{TYPO3/CMS/Backend/Wizard}

		\item Seul des actions simples sont supportées\newline
			(pas de branchement pour le moment)

		\item Le composant assistant possède les méthodes publiques suivantes~:

			\begin{lstlisting}
				addSlide(identifier, title, content, severity, callback)
				addFinalProcessingSlide(callback)
				set(key, value)
				show()
				dismiss()
				getComponent()
				lockNextStep()
				unlockNextStep()
			\end{lstlisting}

	\end{itemize}

\end{frame}

% ------------------------------------------------------------------------------
% LTXE-SLIDE-START
% LTXE-SLIDE-UID:		7a3986e3-31f34cb5-7ef6b005-5be25b66
% LTXE-SLIDE-ORIGIN:	f098bb23-81eee73f-d9eb0d80-50bbe58b English
% LTXE-SLIDE-TITLE:		Wizard component (2)
% LTXE-SLIDE-REFERENCE:	!Feature-73429-WizardComponentHasBeenAdded.rst
% ------------------------------------------------------------------------------
\begin{frame}[fragile]
	\frametitle{Changements en profondeur}
	\framesubtitle{Composant Assistant (2)}

	% decrease font size for code listing
	\lstset{basicstyle=\tiny\ttfamily}

	\begin{itemize}

		\item L'événement \texttt{wizard-visible} est déclenché lorsque le rendu de l'assistant est terminé

		\item L'assistant se ferme lorsque l'événement \texttt{wizard-dismiss} est déclenché

		\item L'assistant déclenche l'événement \texttt{wizard-dismissed} s'il est fermé

		\item Vous pouvez intégrer vos propre gestionnaires en utilisant \texttt{Wizard.getComponent()}

	\end{itemize}

\end{frame}

% ------------------------------------------------------------------------------
% LTXE-SLIDE-START
% LTXE-SLIDE-UID:		c3e97b40-b0d84171-98d2011d-19a838a1
% LTXE-SLIDE-ORIGIN:	5f0d3565-8b6ad76b-44ec4d62-247401f7 English
% LTXE-SLIDE-TITLE:		Generated asset files moved
% LTXE-SLIDE-REFERENCE:	!Important-72580-PubliclyAccessibleGeneratedAssetFilesMovedToTypo3tempassets.rst
% ------------------------------------------------------------------------------
\begin{frame}[fragile]
	\frametitle{Changements en profondeur}
	\framesubtitle{Déplacement des fichiers générés}

	% decrease font size for code listing
	\lstset{basicstyle=\tiny\ttfamily}

	\begin{itemize}

		\item La structure des dossiers sous \texttt{typo3temp} est changée pour séparer
			les ressources accessibles aux clients et les fichiers créés temporairement
			(par ex. pour le cache ou le verrouillage et ne nécessitant que l'accès par le serveur).

		\item Ces éléments ont été déplacés des dossiers~:\newline
			\texttt{\_processed\_}, \texttt{compressor}, \texttt{GB}, \texttt{temp},
			\texttt{Language}, \texttt{pics}\newline
			et réorganisés sous~:

			\begin{itemize}
				\item \texttt{typo3temp/assets/js/}
				\item \texttt{typo3temp/assets/css/},
				\item \texttt{typo3temp/assets/compressed/}
				\item \texttt{typo3temp/assets/images/}
			\end{itemize}

	\end{itemize}

\end{frame}

% ------------------------------------------------------------------------------
% LTXE-SLIDE-START
% LTXE-SLIDE-UID:		145db732-7365dfd6-430b9fcb-0b62380a
% LTXE-SLIDE-ORIGIN:	a4d7bec0-6c19d8b2-6a6be951-5d5aa0c0 English
% LTXE-SLIDE-TITLE:		ImageMagick/GraphicsMagick changes (1)
% LTXE-SLIDE-REFERENCE:	!Breaking-43085-RenamedGraphicsProcessorSettings.rst
% ------------------------------------------------------------------------------
\begin{frame}[fragile]
	\frametitle{Changements en profondeur}
	\framesubtitle{Changements ImageMagick/GraphicsMagick (1)}

	% decrease font size for code listing
	\lstset{basicstyle=\tiny\ttfamily}

	\begin{itemize}

		\item Les options des transformateurs graphique Image- ou GraphicsMagick sont renommées
			(fichier: \texttt{LocalConfiguration.php}).\newline
			Ancien~:\tabto{1.6cm} \texttt{im\_}\newline
			Nouveau~:\tabto{1.6cm} \texttt{processor\_}

		\item Les noms négatifs comme \texttt{noScaleUp} sont changés vers leur correspondance positive.
			Pendant la conversion, les valeurs des précédentes configurations sont inversées pour
			refléter le changement de sens des options.

		\item De plus, les références à des versions spécifiques de ImageMagick/GraphicsMagick ont
			été retirées des options.

	\end{itemize}

\end{frame}

% ------------------------------------------------------------------------------
% LTXE-SLIDE-START
% LTXE-SLIDE-UID:		f5b692f0-7760e064-501a569a-c99c3c2b
% LTXE-SLIDE-ORIGIN:	d48d2af2-66b218a6-4a54e402-097ad85e English
% LTXE-SLIDE-TITLE:		ImageMagick/GraphicsMagick changes (2)
% LTXE-SLIDE-REFERENCE:	!Breaking-43085-RenamedGraphicsProcessorSettings.rst
% ------------------------------------------------------------------------------
\begin{frame}[fragile]
	\frametitle{Changements en profondeur}
	\framesubtitle{Changements ImageMagick/GraphicsMagick (2)}

	% decrease font size for code listing
	\lstset{basicstyle=\tiny\ttfamily}

	\begin{itemize}

		\item L'option de configuration inutilisée \texttt{image\_processing} est retirée sans remplacement

		\item L'option spécifique des transformateurs \texttt{colorspace} est mise en \textit{espace de nom}
			sous la hiérarchie \texttt{processor\_}

	\end{itemize}

\end{frame}

% ------------------------------------------------------------------------------
% LTXE-SLIDE-START
% LTXE-SLIDE-UID:		da63c49f-4f3b9eb7-ce880b6f-ccdfbbe5
% LTXE-SLIDE-ORIGIN:	74e4b871-9f2ded00-f54bc3c1-85d22167 English
% LTXE-SLIDE-TITLE:		Hooks and Signals (1): post-processing of the preview URL
% LTXE-SLIDE-REFERENCE:	!Feature-54887-Post-processingOfThePreviewUrl.rst
% ------------------------------------------------------------------------------
\begin{frame}[fragile]
	\frametitle{Changements en profondeur}
	\framesubtitle{Hooks et Signals (1)}

	% decrease font size for code listing
	\lstset{basicstyle=\tiny\ttfamily}

	\begin{itemize}

		\item Un hook est ajouté à la méthode \texttt{BackendUtility::viewOnClick()}
			pour le post-traitement de l'URL de prévisualisation

		\item Inscription d'une classe de hook implémentant la méthode \texttt{postProcess}~:

			\begin{lstlisting}
				$GLOBALS['TYPO3_CONF_VARS']['SC_OPTIONS']['t3lib/class.t3lib_befunc.php']['viewOnClickClass'][] =
				  \VENDOR\MyExt\Hooks\BackendUtilityHook::class;
			\end{lstlisting}

	\end{itemize}

\end{frame}

% ------------------------------------------------------------------------------
% LTXE-SLIDE-START
% LTXE-SLIDE-UID:		33d08c4f-e94765f5-10593e19-85461b66
% LTXE-SLIDE-ORIGIN:	9c150d50-ee3ba19b-120ef653-e714135c English
% LTXE-SLIDE-TITLE:		Hooks and Signals (2): hook to override a record overlay
% LTXE-SLIDE-REFERENCE:	!Feature-72505-IntroduceHookToOverrideARecordOverlay.rst
% ------------------------------------------------------------------------------
\begin{frame}[fragile]
	\frametitle{Changements en profondeur}
	\framesubtitle{Hooks et Signals (2)}

	% decrease font size for code listing
	\lstset{basicstyle=\tiny\ttfamily}

	\begin{itemize}

		\item Avant TYPO3 CMS 7.6, il était possible de surcharger le recouvrement (overlay) dans \texttt{Web -> List}.
			Un nouveau hook dans TYPO3 CMS 8.0 fourni l'ancienne fonctionnalité.

		\item Le hook est appelé avec la signature suivante~:
			\begin{lstlisting}
				/**
				 * @param string $table
				 * @param array $row
				 * @param array $status
				 * @param string $iconName
				 * @return string the new (or given) $iconName
				 */
				function postOverlayPriorityLookup($table, array $row, array $status, $iconName) { ... }
			\end{lstlisting}

		\item Inscrire la classe de hook implémentant la méthode \texttt{postOverlayPriorityLookup}~:

			\begin{lstlisting}
				$GLOBALS['TYPO3_CONF_VARS']['SC_OPTIONS'][IconFactory::class]['overrideIconOverlay'][] =
				  \VENDOR\MyExt\Hooks\IconFactoryHook::class;
			\end{lstlisting}

	\end{itemize}

\end{frame}

% ------------------------------------------------------------------------------
% LTXE-SLIDE-START
% LTXE-SLIDE-UID:		b6716a21-348ae53b-4e008177-e5dcec4b
% LTXE-SLIDE-ORIGIN:	37ba9f7c-fb51932d-cca9a99f-6bb7a961 English
% LTXE-SLIDE-TITLE:		Hooks and Signals (3): preProcessStorage signal
% LTXE-SLIDE-REFERENCE:	!Feature-72904-AddPreProcessStorageSignalToResourceFactory.rst
% ------------------------------------------------------------------------------
\begin{frame}[fragile]
	\frametitle{Changements en profondeur}
	\framesubtitle{Hooks et Signals (3)}

	% decrease font size for code listing
	\lstset{basicstyle=\tiny\ttfamily}

	\begin{itemize}

		\item Un nouveau signal est implémenté avant l'initialisation d'un stockage de ressources.

		\item Inscrire la classe dans \texttt{ext\_localconf.php}~:

			\begin{lstlisting}
				$dispatcher = \TYPO3\CMS\Core\Utility\GeneralUtility::makeInstance(
				  \TYPO3\CMS\Extbase\SignalSlot\Dispatcher::class);
				$dispatcher->connect(
				  \TYPO3\CMS\Core\Resource\ResourceFactory::class,
				  ResourceFactoryInterface::SIGNAL_PreProcessStorage,
				  \MY\ExtKey\Slots\ResourceFactorySlot::class,
				  'preProcessStorage'
				);
			\end{lstlisting}

		\item La méthode est appelée avec les paramètres suivants~:

			\begin{itemize}
				\item int \texttt{\$uid} the uid of the record
				\item array \texttt{\$recordData} all record data as array
				\item string \texttt{\$fileIdentifier} the file identifier
			\end{itemize}

	\end{itemize}

\end{frame}

% ------------------------------------------------------------------------------
% LTXE-SLIDE-START
% LTXE-SLIDE-UID:		3efc0333-2b52ffde-3de17625-09ff99dd
% LTXE-SLIDE-ORIGIN:	2c656d9e-38b9be7a-d590cb6e-21eccd4d English
% LTXE-SLIDE-TITLE:		Password hashing algorithm: PBKDF2
% LTXE-SLIDE-REFERENCE:	!Feature-28230-AddSupportForPBKDF2ToSaltedpasswords.rst
% ------------------------------------------------------------------------------
\begin{frame}[fragile]
	\frametitle{Changements en profondeur}
	\framesubtitle{Algorithme de hachage des mots de passe~: PBKDF2}

	\begin{itemize}

		\item L'algorithme de hachage des mots de passe «~PBKDF2~» est ajouté à l'extension système
			«~saltedpasswords~»

		\item PBKDF2 est l'abréviation de~: Password-Based Key Derivation Function~2

		\item L'algorithme est conçu pour être coûteux en temps machine pour résister aux
			tentatives de force brute

	\end{itemize}

\end{frame}

% ------------------------------------------------------------------------------

% ------------------------------------------------------------------------------
% TYPO3 CMS 8.0 - What's New - Chapter "Extbase & Fluid" (French Version)
%
% @author	Patrick Lobacher <patrick@lobacher.de> and Michael Schams <schams.net>
% @license	Creative Commons BY-NC-SA 3.0
% @link		http://typo3.org/download/release-notes/whats-new/
% @language	French
% ------------------------------------------------------------------------------
% LTXE-CHAPTER-UID:		cfcf5377-f3e96f39-d1ba01e9-2f341646
% LTXE-CHAPTER-NAME:	Extbase & Fluid
% ------------------------------------------------------------------------------

\section{Extbase \& Fluid}
\begin{frame}[fragile]
	\frametitle{Extbase \& Fluid}

	\begin{center}\huge{Chapitre 4~:}\end{center}
	\begin{center}\huge{\color{typo3darkgrey}\textbf{Extbase \& Fluid}}\end{center}

\end{frame}

% ------------------------------------------------------------------------------
% LTXE-SLIDE-START
% LTXE-SLIDE-UID:		f14b7c0d-cbb723c4-52f427ef-d390199b
% LTXE-SLIDE-ORIGIN:	9413e7f2-dd08f22e-dd21ee20-d7a1b0bf English
% LTXE-SLIDE-TITLE:		Standalone revised Fluid
% LTXE-SLIDE-REFERENCE:	!Feature-69863-UseNewStandaloneFluidAsComposerDependency.rst
% ------------------------------------------------------------------------------

\begin{frame}[fragile]
	\frametitle{Extbase \& Fluid}
	\framesubtitle{Fluid indépendant révisé}

	% decrease font size for code listin
	\lstset{basicstyle=\tiny\ttfamily}

	\begin{itemize}

		\item Le moteur de rendu Fluid de TYPO3 CMS est remplacé par la version
			indépendante de Fluid inclue comme une dépendance Composer

		\item L'ancienne extension Fluid est convertie en \textit{adaptateur Fluid}
			permettant à TYPO3 CMS d'utiliser le Fluid indépendant

		\item De nouvelles fonctionnalités sont ajoutés dans la plupart des domaines de Fluid

		\item Plus important~: de nombreux composants Fluid qui étaient entièrement internes et
			impossible à remplacer par le passé sont facile à remplacer et sont intégrés dans
			une API publique

	\end{itemize}

\end{frame}

% ------------------------------------------------------------------------------
% LTXE-SLIDE-START
% LTXE-SLIDE-UID:		c46ff86c-43f0495f-617e6d9c-51522d51
% LTXE-SLIDE-ORIGIN:	32daa81f-7e805b84-6f829395-94677911 English
% LTXE-SLIDE-TITLE:		Rendering Context (1)
% LTXE-SLIDE-REFERENCE:	!Feature-69863-UseNewStandaloneFluidAsComposerDependency.rst
% ------------------------------------------------------------------------------

\begin{frame}[fragile]
	\frametitle{Extbase \& Fluid}
	\framesubtitle{RenderingContext (1)}

	% decrease font size for code listing
	\lstset{basicstyle=\tiny\ttfamily}

	\begin{itemize}

		\item La plus importante des nouvelles pièces de l'API publique est le RenderingContext
			(contexte de rendu)

		\item Le contexte de rendu précédemment interne seulement utilisé par Fluid
			est étendu pour être responsable de la nouvelle fonctionnalité Fluid~:
			\textbf{provisionnement de l'implémentation}

		\item Permet aux développeurs de changer un ensemble de classes que Fluid utilise
			pour le traitement, la résolution, le cache, etc.

		\item L'utilisation s'effectue soit par l'implémentation d'un contexte de rendu
			personnalisé ou en manipulant celui par défaut avec les méthodes publiques.

	\end{itemize}

\end{frame}

% ------------------------------------------------------------------------------
% LTXE-SLIDE-START
% LTXE-SLIDE-UID:		21f25e22-135212bb-e21f4ba5-a2a846d5
% LTXE-SLIDE-ORIGIN:	3d44632f-c87d4adf-3328f596-2e5744bc English
% LTXE-SLIDE-TITLE:		Rendering Context (2)
% LTXE-SLIDE-REFERENCE:	!Feature-69863-UseNewStandaloneFluidAsComposerDependency.rst
% ------------------------------------------------------------------------------

\begin{frame}[fragile]
	\frametitle{Extbase \& Fluid}
	\framesubtitle{RenderingContext (2)}

	% decrease font size for code listing
	%\lstset{basicstyle=\tiny\ttfamily}
	\lstset{basicstyle=\smaller\ttfamily}

	\begin{itemize}

		\item Les comportements suivants peuvent tous être contrôlés en manipulant
			le contexte de rendu. Par défaut, aucun n'est activé. Mais un seul appel
			(via l'instance de la vue) permet de les activer~:

		\begin{lstlisting}
			$view->getRenderingContext()->setLegacyMode(false);
		\end{lstlisting}

	\end{itemize}

\end{frame}

% ------------------------------------------------------------------------------
% LTXE-SLIDE-START
% LTXE-SLIDE-UID:		f15dd77e-54757b42-9ca20d2b-e2db1628
% LTXE-SLIDE-ORIGIN:	5f486f77-38e6694e-2fad0d6b-7b260ad1 English
% LTXE-SLIDE-TITLE:		ExpressionNodes (1)
% LTXE-SLIDE-REFERENCE:	!Feature-69863-UseNewStandaloneFluidAsComposerDependency.rst
% ------------------------------------------------------------------------------

\begin{frame}[fragile]
	\frametitle{Extbase \& Fluid}
	\framesubtitle{ExpressionNodes (1)}

	% decrease font size for code listing
	%\lstset{basicstyle=\tiny\ttfamily}
	\lstset{basicstyle=\smaller\ttfamily}

	\begin{itemize}

		\item Les ExpressionNodes (nœuds d'expression) sont un nouveau type
			de structure syntaxique de Fluid partageant une caractéristique~:
			ils ne peuvent être utilisés qu'entre accolades

		\begin{lstlisting}
			$view->getRenderingContext()->setExpressionNodeTypes(array(
			   'Class\Number\One',
			   'Class\Number\Two'
			));
		\end{lstlisting}

		\item Les développeurs peuvent ajouter leurs propres types d'expressions

		\item Chacun des types consiste en un motif à faire correspondre et des
			méthodes de l'interface pour traiter la correspondance

		\item Les nœuds d'expression existants servent de référence

	\end{itemize}

\end{frame}

% ------------------------------------------------------------------------------
% LTXE-SLIDE-START
% LTXE-SLIDE-UID:		8e4eddda-9b6bbc2e-c93a1d7d-73c692e0
% LTXE-SLIDE-ORIGIN:	090cec75-2751b839-6ad8eebc-0adfe66a English
% LTXE-SLIDE-TITLE:		ExpressionNodes (2)
% LTXE-SLIDE-REFERENCE:	!Feature-69863-UseNewStandaloneFluidAsComposerDependency.rst
% ------------------------------------------------------------------------------

\begin{frame}[fragile]
	\frametitle{Extbase \& Fluid}
	\framesubtitle{ExpressionNodes (2)}

	Les nœuds d'expression permettent de nouvelles syntaxes comme~:

	\begin{itemize}

		\item \textbf{CastingExpressionNode}\newline
			\small
				Permet de convertir une variable à un type, pour par exemple garantir
				un entier ou un booléen. S'utilise à l'aide du mot clé \texttt{as}~:
				\texttt{\{myStringVariable as boolean\}} ou
				\texttt{\{myBooleanVariable as integer\}} et ainsi de suite.
				Tenter de convertir une variable dans un type non compatible déclenche
				une erreur Fluid standard.
			\normalsize

		\item \textbf{MathExpressionNode}\newline
			\small
				Permet des opérations mathématiques basiques sur les variables, par exemple
				\texttt{\{myNumber + 1\}}, \texttt{\{myPercent / 100\}} ou
				\texttt{\{myNumber * 100\}} et ainsi de suite.
				Une expression impossible retourne une sortie vide.
			\normalsize

	\end{itemize}

\end{frame}

% ------------------------------------------------------------------------------
% LTXE-SLIDE-START
% LTXE-SLIDE-UID:		f062383a-2b537432-04ada137-50803801
% LTXE-SLIDE-ORIGIN:	c8935f68-0905db52-11932973-63a85a00 English
% LTXE-SLIDE-TITLE:		ExpressionNodes (3)
% LTXE-SLIDE-REFERENCE:	!Feature-69863-UseNewStandaloneFluidAsComposerDependency.rst
% ------------------------------------------------------------------------------

\begin{frame}[fragile]
	\frametitle{Extbase \& Fluid}
	\framesubtitle{ExpressionNodes (3)}

	Les nœuds d'expression permettent de nouvelles syntaxes comme~:

	\begin{itemize}

		\item \textbf{TernaryExpressionNode}\newline
			\small
				Permet une condition ternaire opérant uniquement sur les variables.
				Cas d'usage typique~: "si cette variable alors utilise cette variable
				sinon une autre variable". Utilisé comme~:\newline
				\texttt{\{myToggleVariable ? myThenVariable : myElseVariable\}}\newline
				Note~: ne supporte pas les expressions imbriquées, les ViewHelper ou similaire
				à l'intérieur. S'utilise uniquement avec de simples variables.
			\normalsize

	\end{itemize}

\end{frame}

% ------------------------------------------------------------------------------
% LTXE-SLIDE-START
% LTXE-SLIDE-UID:		71fd76dc-6d963f01-123db7fa-154c0df2
% LTXE-SLIDE-ORIGIN:	153b9b53-d7f97aef-dc1ba8ab-fc2a763e English
% LTXE-SLIDE-TITLE:		Namespaces are extensible (1)
% LTXE-SLIDE-REFERENCE:	!Feature-69863-UseNewStandaloneFluidAsComposerDependency.rst
% ------------------------------------------------------------------------------

\begin{frame}[fragile]
	\frametitle{Extbase \& Fluid}
	\framesubtitle{Espaces de nom extensibles (1)}

	\begin{itemize}

		\item Fluid permet aux alias d'espace de nom (par exemple \texttt{f:})
			d'être étendu en y ajoutant un espace de nom PHP supplémentaire

		\item Les espaces de nom PHP sont aussi consultés pour la présence de classes
			ViewHelper

		\item Ceci implique que les développeurs peuvent surcharger un ViewHelper avec
			une version personnalisée et avoir leur ViewHelper appelé lorsque l'espace
			de nom \texttt{f:} est utilisé

		\item Ce changement implique que les espaces de nom ne sont plus monadiques.
			Lors de l'utilisation de\newline
			\texttt{
				\{namespace f=My\textbackslash
				Extension\textbackslash
				ViewHelpers\textbackslash\}}
				l'erreur «~espace de nom déjà inscrit~» n'est plus déclenchée.
				Fluid ajoutera cet espace de nom PHP à la place et y recherchera
				aussi des ViewHelper.

	\end{itemize}

\end{frame}

% ------------------------------------------------------------------------------
% LTXE-SLIDE-START
% LTXE-SLIDE-UID:		ce64b37e-737d9821-46f1f08d-9d6fc024
% LTXE-SLIDE-ORIGIN:	a2f0e799-fe523ffb-57634576-654a8efe English
% LTXE-SLIDE-TITLE:		Namespaces are extensible (2)
% LTXE-SLIDE-REFERENCE:	!Feature-69863-UseNewStandaloneFluidAsComposerDependency.rst
% ------------------------------------------------------------------------------

\begin{frame}[fragile]
	\frametitle{Extbase \& Fluid}
	\framesubtitle{Espaces de nom extensibles (2)}

	\begin{itemize}

		\item Les espaces additionnels sont consultés du bas vers le haut,
			permettant à ceux-ci de surcharger les classe ViewHelper en les
			plaçant dans la même portée

		\item Par exemple~: \texttt{f:format.nl2br} peut être surchargé par
			\texttt{
				My\textbackslash
				Extension\textbackslash
				ViewHelpers\textbackslash
				Format\textbackslash
				Nl2brViewHelper},

				avec l'inscription de l'espace de nom de la slide précédente

	\end{itemize}

\end{frame}

% ------------------------------------------------------------------------------
% LTXE-SLIDE-START
% LTXE-SLIDE-UID:		51d02821-c630e455-000a4a7d-e1115900
% LTXE-SLIDE-ORIGIN:	2a98636b-3a4db59b-588597f9-0aca826e English
% LTXE-SLIDE-TITLE:		Rendering using f:render (1)
% LTXE-SLIDE-REFERENCE:	!Feature-69863-UseNewStandaloneFluidAsComposerDependency.rst
% ------------------------------------------------------------------------------

\begin{frame}[fragile]
	\frametitle{Extbase \& Fluid}
	\framesubtitle{Rendu utilisant \texttt{f:render} (1)}

	Contenu par défaut d'un \texttt{f:render} optionnel~:

	\begin{itemize}

		\item Lorsque \texttt{f:render} est utilisé et l'indicateur \texttt{optional = TRUE}
			est défini, le rendu d'une section manquante retourne vide.

		\item Au lieu de cette sortie vide, un nouvel attribut \texttt{default}
			(mixed) est ajouté et remplie avec une valeur par défaut de type repli.

		\item En alternative, le contenu de la balise s'utilise pour la valeur par défaut
			comme beaucoup d'autres ViewHelper avec contenu et attributs flexibles

	\end{itemize}

\end{frame}

% ------------------------------------------------------------------------------
% LTXE-SLIDE-START
% LTXE-SLIDE-UID:		87a11024-e3b2044c-ba99dd3e-2e5f4149
% LTXE-SLIDE-ORIGIN:	78998a07-75b3fe30-431ca5e2-f7b816c4 English
% LTXE-SLIDE-TITLE:		Rendering using f:render (2)
% LTXE-SLIDE-REFERENCE:	!Feature-69863-UseNewStandaloneFluidAsComposerDependency.rst
% ------------------------------------------------------------------------------

\begin{frame}[fragile]
	\frametitle{Extbase \& Fluid}
	\framesubtitle{Rendu utilisant \texttt{f:render} (2)}

	% decrease font size for code listing
	\lstset{basicstyle=\tiny\ttfamily}

	Passage du contenu de la balise \texttt{f:render} aux partial/section~:

	\begin{itemize}

		\item Permet une nouvelle approche à la structuration du rendu du modèle Fluid

		\item Les partials et sections peuvent s'utiliser comme habillage de
			partie arbitraire de code de modèle

		\item Exemple~:

			\begin{lstlisting}
				<f:section name="MyWrap">
				  <div>
				    <!-- more HTML, using variables if desired -->
				    <!-- tag content of f:render output: -->
				    {contentVariable -> f:format.raw()}
				  </div>
				</f:section>

				<f:render section="MyWrap" contentAs="contentVariable">
				  This content will be wrapped. Any Fluid code can go here.
				</f:render>
			\end{lstlisting}

	\end{itemize}

\end{frame}

% ------------------------------------------------------------------------------
% LTXE-SLIDE-START
% LTXE-SLIDE-UID:		2c9abe59-cbdc9612-87a9df69-6d9c6235
% LTXE-SLIDE-ORIGIN:	7c36b3a7-33860495-4efb65ef-f09ae14e English
% LTXE-SLIDE-TITLE:		Complex conditional statements
% LTXE-SLIDE-REFERENCE:	!Feature-69863-UseNewStandaloneFluidAsComposerDependency.rst
% ------------------------------------------------------------------------------

\begin{frame}[fragile]
	\frametitle{Extbase \& Fluid}
	\framesubtitle{Expressions conditionnelles complexes}

	% decrease font size for code listing
	\lstset{basicstyle=\tiny\ttfamily}

	\begin{itemize}

		\item Fluid supporte tout degré de complexité d'expression conditionnelle avec
			imbrication et groupement~:

			\begin{lstlisting}
				<f:if condition="({variableOne} && {variableTwo}) || {variableThree} || {variableFour}">
				  // Done if both variable one and two evaluate to true,
				  // or if either variable three or four do.
				</f:if>
			\end{lstlisting}

		\item En plus, le comportement type «~sinon si~» est ajouté à \texttt{f:else}~:

			\begin{lstlisting}
				<f:if condition="{variableOne}">
				  <f:then>Do this</f:then>
				  <f:else if="{variableTwo}">
				    Do this instead if variable two evals true
				  </f:else>
				  <f:else if="{variableThree}">
				    Or do this if variable three evals true
				  </f:else>
				  <f:else>
				    Or do this if nothing above is true
				  </f:else>
				</f:if>
			\end{lstlisting}

	\end{itemize}

\end{frame}

% ------------------------------------------------------------------------------
% LTXE-SLIDE-START
% LTXE-SLIDE-UID:		a878b832-4f65f15a-d8511f51-4bb6a1ed
% LTXE-SLIDE-ORIGIN:	d54cf918-48141072-8184ab81-cecc8e39 English
% LTXE-SLIDE-TITLE:		Dynamic variable name parts (1)
% LTXE-SLIDE-REFERENCE:	!Feature-69863-UseNewStandaloneFluidAsComposerDependency.rst
% ------------------------------------------------------------------------------

\begin{frame}[fragile]
	\frametitle{Extbase \& Fluid}
	\framesubtitle{Partie de nom de variable dynamique (1)}

	% decrease font size for code listing
	\lstset{basicstyle=\tiny\ttfamily}

	\begin{itemize}

		\item Une nouvelle fonctionnalité forcée, aussi compatible arrière, est
			l'ajout de la capacité à utiliser des sous-références de variable
			lors de l'accès à une variable. Considérer l'initialisation de la
			vue suivante~:

			\begin{lstlisting}
				$mykey = 'foo'; // or 'bar', set by any source
				$view->assign('data', ['foo' => 1, 'bar' => 2]);
				$view->assign('key', $mykey);
			\end{lstlisting}

		\item Avec le modèle Fluid suivant~:

			\begin{lstlisting}
				You chose: {data.{key}}.
				(output: "1" if key is "foo" or "2" if key is "bar")
			\end{lstlisting}

	\end{itemize}

\end{frame}

% ------------------------------------------------------------------------------
% LTXE-SLIDE-START
% LTXE-SLIDE-UID:		1f526d40-bdfc3f0c-4d5ff31a-92053a62
% LTXE-SLIDE-ORIGIN:	20385d89-9c0de3ac-9ef70ba6-310fdec3 English
% LTXE-SLIDE-TITLE:		Dynamic variable name parts (2)
% LTXE-SLIDE-REFERENCE:	!Feature-69863-UseNewStandaloneFluidAsComposerDependency.rst
% ------------------------------------------------------------------------------

\begin{frame}[fragile]
	\frametitle{Extbase \& Fluid}
	\framesubtitle{Partie de nom de variable dynamique (2)}

	% decrease font size for code listing
	\lstset{basicstyle=\tiny\ttfamily}

	\begin{itemize}

		\item La même approche s'utilise pour générer dynamiquement des
			parties d'un nom de variable~:

			\begin{lstlisting}
				$mydynamicpart = 'First'; // or 'Second', set by any source
				$view->assign('myFirstVariable', 1);
				$view->assign('mySecondVariable', 2);
				$view->assign('which', $mydynamicpart);
			\end{lstlisting}

		\item Avec le modèle Fluid suivant~:

			\begin{lstlisting}
				You chose: {my{which}Variable}.
				(output: "1" if which is "First" or "2" if which is "Second")
			\end{lstlisting}

	\end{itemize}

\end{frame}

% ------------------------------------------------------------------------------
% LTXE-SLIDE-START
% LTXE-SLIDE-UID:		a98e133e-b3ed70fc-e7e12167-6ad44c6d
% LTXE-SLIDE-ORIGIN:	15907f4f-d554b015-03be1fdc-4dc0fb25 English
% LTXE-SLIDE-TITLE:		New ViewHelpers
% LTXE-SLIDE-REFERENCE:	!Feature-69863-UseNewStandaloneFluidAsComposerDependency.rst
% ------------------------------------------------------------------------------

\begin{frame}[fragile]
	\frametitle{Extbase \& Fluid}
	\framesubtitle{Nouveaux ViewHelper}

	% decrease font size for code listing
	\lstset{basicstyle=\tiny\ttfamily}

	\begin{itemize}
		\item Quelques nouveaux ViewHelper sont ajoutés par l'utilisation du Fluid
			indépendant et sont donc disponibles pour TYPO3~:

			\begin{itemize}

				\item \textbf{\texttt{f:or}}\newline
					C'est une manière raccourcis d'écrire des conditions (chainées).
					Supporte la syntaxe suivante qui vérifie chaque variable et retourne
					le contenu de la première non vide~:

					\begin{lstlisting}
						{variableOne -> f:or(alternative: variableTwo) -> f:or(alternative: variableThree)}
					\end{lstlisting}

				\item \textbf{\texttt{f:spaceless}}\newline
					S'utilise en balise autour du code de modèle pour éliminer les
					espaces redondants et lignes vides causés par l'usage indenté des
					ViewHelper

			\end{itemize}

	\end{itemize}

\end{frame}

% ------------------------------------------------------------------------------
% LTXE-SLIDE-START
% LTXE-SLIDE-UID:		35505a07-5f938dc4-72fef502-6e3ce2b2
% LTXE-SLIDE-ORIGIN:	9fed5ece-ae03648f-c3db2f41-42312f7f English
% LTXE-SLIDE-TITLE:		ViewHelper namespaces can be extended also from PHP
% LTXE-SLIDE-REFERENCE:	!Feature-69863-UseNewStandaloneFluidAsComposerDependency.rst
% ------------------------------------------------------------------------------

\begin{frame}[fragile]
	\frametitle{Extbase \& Fluid}
	\framesubtitle{Extension des espaces de nom par PHP}

	% decrease font size for code listing
	\lstset{basicstyle=\tiny\ttfamily}

	\begin{itemize}

		\item En accédant au ViewHelperResolver du contexte de rendu,
			les développeurs peuvent changer les espaces de nom des ViewHelper
			sur une base globale (lire~: par instance de vue)~:

			\begin{lstlisting}
				$resolver = $view->getRenderingContext()->getViewHelperResolver();
				// equivalent of registering namespace in template(s):
				$resolver->registerNamespace('news', 'GeorgRinger\News\ViewHelpers');
				// adding additional PHP namespaces to check when resolving ViewHelpers:
				$resolver->extendNamespace('f', 'My\Extension\ViewHelpers');
				// setting all namespaces in advance, globally, before template parsing:
				$resolver->setNamespaces(array(
				  'f' => array(
				    'TYPO3Fluid\\Fluid\\ViewHelpers', 'TYPO3\\CMS\\Fluid\\ViewHelpers',
				    'My\\Extension\\ViewHelpers'
				  ),
				  'vhs' => array(
				    'FluidTYPO3\\Vhs\\ViewHelpers', 'My\\Extension\\ViewHelpers'
				  ),
				  'news' => array(
				    'GeorgRinger\\News\\ViewHelpers',
				  );
				));
			\end{lstlisting}
	\end{itemize}

\end{frame}

% ------------------------------------------------------------------------------
% LTXE-SLIDE-START
% LTXE-SLIDE-UID:		3ac11216-6db2f0cc-eeb73aec-7b9ee63a
% LTXE-SLIDE-ORIGIN:	3e41fa42-1d526584-5b7d3a07-c8b347ab English
% LTXE-SLIDE-TITLE:		ViewHelpers can accept arbitrary arguments (1)
% LTXE-SLIDE-REFERENCE:	!Feature-69863-UseNewStandaloneFluidAsComposerDependency.rst
% ------------------------------------------------------------------------------

\begin{frame}[fragile]
	\frametitle{Extbase \& Fluid}
	\framesubtitle{Arguments arbitraires de ViewHelper (1)}

	\begin{itemize}

		\item Permet au ViewHelper de recevoir des arguments additionnels utilisant
			des noms personnalisés

		\item Fonctionne en séparant les arguments passés à chaque ViewHelper en
			deux groupes~: ceux déclarés en utilisant \texttt{registerArgument}
			(ou les arguments de la méthode render) et ceux qui ne le sont pas

		\item Ces derniers sont passés à la méthode spéciale
			\texttt{handleAdditionalArguments}
			de la classe ViewHelper, qui dans l'implémentation par défaut déclenche
			une erreur si des arguments supplémentaires existent

	\end{itemize}

\end{frame}

% ------------------------------------------------------------------------------
% LTXE-SLIDE-START
% LTXE-SLIDE-UID:		e3132cd6-6be1cf85-d4f917cb-5105d096
% LTXE-SLIDE-ORIGIN:	2462a8ce-0a82a7f0-1b97a84f-02f0a4c2 English
% LTXE-SLIDE-TITLE:		ViewHelpers can accept arbitrary arguments (2)
% LTXE-SLIDE-REFERENCE:	!Feature-69863-UseNewStandaloneFluidAsComposerDependency.rst
% ------------------------------------------------------------------------------

\begin{frame}[fragile]
	\frametitle{Extbase \& Fluid}
	\framesubtitle{Arguments arbitraires de ViewHelper (2)}

	\begin{itemize}

		\item En surchargeant cette méthode dans un Viewhelper, il est possible de
			changer si et quand celui-ci devrait déclencher une erreur lors de la
			réception d'arguments non déclarés

		\item Cette fonctionnalité est celle permettant au TagBasedViewHelper
		 	d'accepter librement des arguments préfixés de \texttt{data-} sans échouer

		\item sur les TagBasedViewHelper, la méthode \texttt{handleAdditionalArguments}
			ajoute simplement les nouveaux attributs à la balise générée et déclenche une
			erreur si d'autres attributs ni déclarés ni préfixés de \texttt{data-} sont fournis

	\end{itemize}

\end{frame}

% ------------------------------------------------------------------------------
% LTXE-SLIDE-START
% LTXE-SLIDE-UID:		e66697aa-de0db186-f1ce1ab7-c4a2bd11
% LTXE-SLIDE-ORIGIN:	b5a66715-5681025e-cfb46644-9b9bad55 English
% LTXE-SLIDE-TITLE:		Added "allowedTags" argument to f:format.stripTags ViewHelper
% LTXE-SLIDE-REFERENCE:	!Feature-67236-AddedAllowedTagsArgumentToFformatstripTagsViewHelper.rst
% ------------------------------------------------------------------------------

\begin{frame}[fragile]
	\frametitle{Extbase \& Fluid}
	\framesubtitle{Argument «~allowedTags~» de f:format.stripTags}

	\begin{itemize}

		\item L'argument \texttt{allowedTags} contenant la liste des balises HTML ne
			devant pas être retirés s'utilise sur \texttt{f:format.stripTags}

		\item La syntaxe de la liste est la même que pour le second paramètre de la
			fonction PHP \texttt{strip\_tags} (voir~: \url{http://php.net/strip_tags})

	\end{itemize}

\end{frame}

% ------------------------------------------------------------------------------
% LTXE-SLIDE-START
% LTXE-SLIDE-UID:		66f49b79-4de7bb91-25472149-88b0761f
% LTXE-SLIDE-ORIGIN:	12b7dd4a-73b912f7-e5d1cefa-e3c8386d English
% LTXE-SLIDE-TITLE:		Allow accessing ObjectStorage as array in Fluid
% LTXE-SLIDE-REFERENCE:	!Feature-73752-AllowAccessingObjectStorageAsArrayInFluidAndOtherPlaces.rst
% ------------------------------------------------------------------------------

\begin{frame}[fragile]
	\frametitle{Extbase \& Fluid}
	\framesubtitle{Accès à ObjectStorage en tableau par Fluid}

	% decrease font size for code listing
	\lstset{basicstyle=\tiny\ttfamily}

	\begin{itemize}

		\item L'alias \texttt{getArray()} de la méthode \texttt{toArray()} est créé

		\item Cet alias permet la récupération des éléments par la propriété virtuelle
			\texttt{array}

		\item Cette propriété est accédée par Fluid de manière transparente à l'aide de
			\texttt{ObjectAccess::getPropertyPath}

		\item Cette capacité n'est pas limité à Fluid

		\item Exemple~: récupérer le 5ième élément (les indices commencent à zéro)

			\begin{lstlisting}
				// in PHP:
				ObjectAccess::getPropertyPath($subject, 'objectstorageproperty.array.4')

				// in Fluid:
				{myObject.objectstorageproperty.array.4}
				{myObject.objectstorageproperty.array.{dynamicIndex}}
			\end{lstlisting}

	\end{itemize}

\end{frame}

% ------------------------------------------------------------------------------

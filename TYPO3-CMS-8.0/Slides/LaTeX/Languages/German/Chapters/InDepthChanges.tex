% ------------------------------------------------------------------------------
% TYPO3 CMS 7.6 - What's New - Chapter "In-Depth Changes" (English Version)
%
% @author	Patrick Lobacher <patrick@lobacher.de> and Michael Schams <schams.net>
% @license	Creative Commons BY-NC-SA 3.0
% @link		http://typo3.org/download/release-notes/whats-new/
% @language	English
% ------------------------------------------------------------------------------
% LTXE-CHAPTER-UID:		7229f1b9-b481e9bc-09c46183-a86b6a7e
% LTXE-CHAPTER-NAME:	In-Depth Changes
% ------------------------------------------------------------------------------

\section{Änderungen im System}
\begin{frame}[fragile]
	\frametitle{Änderungen im System}

	\begin{center}\huge{Kapitel 3:}\end{center}
	\begin{center}\huge{\color{typo3darkgrey}\textbf{Änderungen im System}}\end{center}

\end{frame}


% ------------------------------------------------------------------------------
% LTXE-SLIDE-START
% LTXE-SLIDE-UID:		b9b259e4-9d35ace7-edff1014-7390d1d1
% LTXE-SLIDE-TITLE:		Support PECL-memcached in MemcachedBackend
% LTXE-SLIDE-REFERENCE:	!Feature-69794-SupportPecl-memcachedInMemcachedBackend.rst
% ------------------------------------------------------------------------------
\begin{frame}[fragile]
	\frametitle{Änderungen im System}
	\framesubtitle{Unterstützung von PECL-memcached im MemcachedBackend}

	% decrease font size for code listing
	\lstset{basicstyle=\tiny\ttfamily}

	\begin{itemize}

		\item Es wurde eine Unterstützung für das PECL Modul "memcached" zum MemcachedBackend des Caching-Frameworks hinzugefügt

		\item Wenn beide ("memcache" und "memcached") installiert sind, wird "memcache"
			verwendet, um einen Breaking Change zu vermeiden

		\item Als Integrator kann man die Option \texttt{peclModule} setzen, um das bevorzugte PECL-Modul auszuwählen:

			\begin{lstlisting}
				$GLOBALS['TYPO3_CONF_VARS']['SYS']['caching']['cacheConfigurations']['my_memcached'] = [
				  'frontend' => \TYPO3\CMS\Core\Cache\Frontend\VariableFrontend::class
				  'backend' => \TYPO3\CMS\Core\Cache\Backend\MemcachedBackend::class,
				  'options' => [
				    'peclModule' => 'memcached',
				    'servers' => [
				      'localhost',
				      'server2:port'
				    ]
				  ]
				];
			\end{lstlisting}

	\end{itemize}

\end{frame}


% ------------------------------------------------------------------------------
% LTXE-SLIDE-START
% LTXE-SLIDE-UID:		ef8c124e-f902453a-d48770b9-4cf86545
% LTXE-SLIDE-TITLE:		Native support for Symfony Console (1)
% LTXE-SLIDE-REFERENCE:	!Feature-73042-IntroduceNativeSupportForSymfonyConsole.rst
% ------------------------------------------------------------------------------
\begin{frame}[fragile]
	\frametitle{Änderungen im System}
	\framesubtitle{Native Unterstützung der Symfony-Konsole (1)}

	% decrease font size for code listing
	\lstset{basicstyle=\tiny\ttfamily}

	\begin{itemize}

		\item TYPO3 unterstützt nun out-of-the-box die Symfony-Konsole in dem es ein CLI-Skript in \texttt{typo3/sysext/core/bin/typo3} zur Verfügung stellt. Auf Instanzen, die mittels Composer installiert wurden, wird das Binary in das \texttt{bin}-Verzeichnis verlinkt, z.B. \texttt{bin/typo3}.

		\item Das neue Binary unterstützt die existierenden CLI-Kommandos als Fallback, wenn kein entsprechender Symfony Konsolen-Befehl gefunden wurde

		\item Um einen Befehl zu registrieren, der via \texttt{typo3} CLI aufgerufen werden soll, muss man eine entsprechende Datei \texttt{Configuration/Commands.php} innerhalb der eigenen Extension anlegen

	\end{itemize}

\end{frame}

% ------------------------------------------------------------------------------
% LTXE-SLIDE-START
% LTXE-SLIDE-UID:		28b018bb-21804fa1-c578f0ff-fc16cbb1
% LTXE-SLIDE-TITLE:		Native support for Symfony Console (2)
% LTXE-SLIDE-REFERENCE:	!Feature-73042-IntroduceNativeSupportForSymfonyConsole.rst
% ------------------------------------------------------------------------------
\begin{frame}[fragile]
	\frametitle{Änderungen im System}
	\framesubtitle{Native Unterstützung der Symfony-Konsole (2)}

	% decrease font size for code listing
	\lstset{basicstyle=\tiny\ttfamily}

	\begin{itemize}

		\item Sobald man ein Kommando regirstriert, muss man zwingend die Eigenschaft \texttt{class} angeben.
			Optional kann der Parameter \texttt{user} angegeben werden, mit dem angegeben wird, als welcher User das Kommando im Backend ausgeführt wird

		\item Eine beispielhafte Datei \texttt{Configuration/Commands.php} könnte so aussehen:

			\begin{lstlisting}
				return [
				  'backend:lock' => [
				    'class' => \TYPO3\CMS\Backend\Command\LockBackendCommand::class
				  ],
				  'referenceindex:update' => [
				    'class' => \TYPO3\CMS\Backend\Command\ReferenceIndexUpdateCommand::class,
				    'user' => '_cli_lowlevel'
				  ]
				];
			\end{lstlisting}

	\end{itemize}

\end{frame}

% ------------------------------------------------------------------------------
% LTXE-SLIDE-START
% LTXE-SLIDE-UID:		fc16cbb1-21804fa1-c578f0ff-28b018bb
% LTXE-SLIDE-TITLE:		Native support for Symfony Console (3)
% LTXE-SLIDE-REFERENCE:	!Feature-73042-IntroduceNativeSupportForSymfonyConsole.rst
% ------------------------------------------------------------------------------
\begin{frame}[fragile]
	\frametitle{Änderungen im System}
	\framesubtitle{Native Unterstützung der Symfony-Konsole (3)}

	% decrease font size for code listing
	\lstset{basicstyle=\tiny\ttfamily}

	\begin{itemize}

		\item Ein exemplarischer Aufruf könnte wie folgt lauten:
			\begin{lstlisting}
				bin/typo3 backend:lock http://example.com/maintenance.html
			\end{lstlisting}

		\item Bei Nicht-Composer Installationen:
			\begin{lstlisting}
				typo3/sysext/core/bin/typo3 backend:lock http://example.com/maintenance.html
			\end{lstlisting}

	\end{itemize}

\end{frame}

% ------------------------------------------------------------------------------
% LTXE-SLIDE-START
% LTXE-SLIDE-UID:		1626edc6-8d93ff84-f64178bf-8e7650df
% LTXE-SLIDE-TITLE:		Cryptographically secure pseudorandom number generator
% LTXE-SLIDE-REFERENCE:	!Feature-73050-AddACSPRNGAPI.rst
% ------------------------------------------------------------------------------
\begin{frame}[fragile]
	\frametitle{Änderungen im System}
	\framesubtitle{Kryptografisch sicherer Pseudo-Zufallszahlen-Generator}

	% decrease font size for code listing
	\lstset{basicstyle=\tiny\ttfamily}

	\begin{itemize}

		\item Es wurde ein neuer kryptografisch sicherer Pseudo-Zufallszahlen-Generator (CSPRNG) in den Core integriert.\newline
			Dieser verwendet die neue CSPRNG-Funktion in PHP 7.

		\item Die API hierzu befindet sich in der Klasse
			\texttt{\textbackslash
				TYPO3\textbackslash
				CMS\textbackslash
				Core\textbackslash
				Crypto\textbackslash
				Random}

		\item Beispiel:

			\begin{lstlisting}
				use \TYPO3\CMS\Core\Crypto\Random;
				use \TYPO3\CMS\Core\Utility\GeneralUtility;

				// Retrieving random bytes
				$someRandomString = GeneralUtility::makeInstance(Random::class)->generateRandomBytes(64);

				// Rolling the dice..
				$tossedValue = GeneralUtility::makeInstance(Random::class)->generateRandomInteger(1, 6);
			\end{lstlisting}

	\end{itemize}

\end{frame}

% ------------------------------------------------------------------------------
% LTXE-SLIDE-START
% LTXE-SLIDE-UID:		f098bb23-81eee73f-d9eb0d80-50bbe58b
% LTXE-SLIDE-TITLE:		Wizard component (1)
% LTXE-SLIDE-REFERENCE:	!Feature-73429-WizardComponentHasBeenAdded.rst
% ------------------------------------------------------------------------------
\begin{frame}[fragile]
	\frametitle{Änderungen im System}
	\framesubtitle{Wizard Komponente (1)}

	% decrease font size for code listing
	\lstset{basicstyle=\tiny\ttfamily}

	\begin{itemize}

		\item Es wurde eine neue Wizard Komponente (Assistent) zugefügt. Dieser Assistent kann für User-unterstützte Interaktionen verwendet werden

		\item Das RequireJS-Module kann durch die Inklusion von \texttt{TYPO3/CMS/Backend/Wizard} verwendet werde

		\item Der Assistent unterstützt bislang nur direkte Actions\newline
			(und keine Verbindungen untereinander)

		\item Die Wizard-Komponente besitzt die folgenden öffentlichen Methoden:

			\begin{lstlisting}
				addSlide(identifier, title, content, severity, callback)
				addFinalProcessingSlide(callback)
				set(key, value)
				show()
				dismiss()
				getComponent()
				lockNextStep()
				unlockNextStep()
			\end{lstlisting}

	\end{itemize}

\end{frame}

% ------------------------------------------------------------------------------
% LTXE-SLIDE-START
% LTXE-SLIDE-UID:		f098bb23-81eee73f-d9eb0d80-50bbe58b
% LTXE-SLIDE-TITLE:		Wizard component (2)
% LTXE-SLIDE-REFERENCE:	!Feature-73429-WizardComponentHasBeenAdded.rst
% ------------------------------------------------------------------------------
\begin{frame}[fragile]
	\frametitle{Änderungen im System}
	\framesubtitle{Wizard Komponente (2)}

	% decrease font size for code listing
	\lstset{basicstyle=\tiny\ttfamily}

	\begin{itemize}

		\item Der Event \texttt{wizard-visible} wird abgefeuert, wenn das Rendering des Wizards fertig gestellt wurde

		\item Wizards können über das Abfeuern des \texttt{wizard-dismiss} Events geschlossen werden

		\item Wizards feuern den \texttt{wizard-dismissed} Event, wenn diese geschlossen wurden

		\item Eigene Listener können über \texttt{Wizard.getComponent()} integriert werden

	\end{itemize}

\end{frame}

% ------------------------------------------------------------------------------
% LTXE-SLIDE-START
% LTXE-SLIDE-UID:		5f0d3565-8b6ad76b-44ec4d62-247401f7
% LTXE-SLIDE-TITLE:		Generated asset files moved
% LTXE-SLIDE-REFERENCE:	!Important-72580-PubliclyAccessibleGeneratedAssetFilesMovedToTypo3tempassets.rst
% ------------------------------------------------------------------------------
\begin{frame}[fragile]
	\frametitle{Änderungen im System}
	\framesubtitle{Generierte Asset-Dateien wurden verschoben}

	% decrease font size for code listing
	\lstset{basicstyle=\tiny\ttfamily}

	\begin{itemize}

		\item Die Ordner-Struktur innerhalb von \texttt{typo3temp} wurde verändert, um Assets zu separieren, die von temporär erzeugten Dateien öffentlichen Zugriff benötigen (so benötigt z.B. zum Zwecke des Cachings oder Locking lediglich der Server Zugriff auf deren Dateien)

		\item Diese Assets werden nun nicht mehr in den folgenden Ordnern gespeichert:\newline
			\texttt{\_processed\_}, \texttt{compressor}, \texttt{GB}, \texttt{temp},
			\texttt{Language}, \texttt{pics}\newline
			sondern in Zukunft in den nachfolgenden:

			\begin{itemize}
				\item \texttt{typo3temp/assets/js/}
				\item \texttt{typo3temp/assets/css/},
				\item \texttt{typo3temp/assets/compressed/}
				\item \texttt{typo3temp/assets/images/}
			\end{itemize}

	\end{itemize}

\end{frame}

% ------------------------------------------------------------------------------
% LTXE-SLIDE-START
% LTXE-SLIDE-UID:		a4d7bec0-6c19d8b2-6a6be951-5d5aa0c0
% LTXE-SLIDE-TITLE:		ImageMagick/GraphicsMagick changes (1)
% LTXE-SLIDE-REFERENCE:	!Breaking-43085-RenamedGraphicsProcessorSettings.rst
% ------------------------------------------------------------------------------
\begin{frame}[fragile]
	\frametitle{Änderungen im System}
	\framesubtitle{ImageMagick/GraphicsMagick Änderungen (1)}

	% decrease font size for code listing
	\lstset{basicstyle=\tiny\ttfamily}

	\begin{itemize}

		\item Die Settings für den Graphic-Prozessor (Image- oder GraphicsMagick) wurden umbenannt
			(Datei: \texttt{LocalConfiguration.php}).\newline
			ALT:\tabto{1.0cm} \texttt{im\_}\newline
			NEU:\tabto{1.0cm} \texttt{processor\_}

		\item Negative Benennung - wie \texttt{noScaleUp} - wurden in positive Entsprechungen geändert

		\item Zusätzlich wurden Referenzen zu spezifischen Versionen von ImageMagick/GraphicsMagick entfernt

	\end{itemize}

\end{frame}

% ------------------------------------------------------------------------------
% LTXE-SLIDE-START
% LTXE-SLIDE-UID:		d48d2af2-66b218a6-4a54e402-097ad85e
% LTXE-SLIDE-TITLE:		ImageMagick/GraphicsMagick changes (2)
% LTXE-SLIDE-REFERENCE:	!Breaking-43085-RenamedGraphicsProcessorSettings.rst
% ------------------------------------------------------------------------------
\begin{frame}[fragile]
	\frametitle{Änderungen im System}
	\framesubtitle{ImageMagick/GraphicsMagick Änderungen (2)}

	% decrease font size for code listing
	\lstset{basicstyle=\tiny\ttfamily}

	\begin{itemize}

		\item Die nicht benutzte Konfigurationsoption \texttt{image\_processing} wurde ohne Ersatz entfernt

		\item Die Prozessor-spezifische Konfigurationsoption \texttt{colorspace} wurde mit einem  \textit{Namespace} unterhalb der \texttt{processor\_} Hierarchie eingeordnet

	\end{itemize}

\end{frame}

% ------------------------------------------------------------------------------
% LTXE-SLIDE-START
% LTXE-SLIDE-UID:		74e4b871-9f2ded00-f54bc3c1-85d22167
% LTXE-SLIDE-TITLE:		Hooks and Signals (1): post-processing of the preview URL
% LTXE-SLIDE-REFERENCE:	!Feature-54887-Post-processingOfThePreviewUrl.rst
% ------------------------------------------------------------------------------
\begin{frame}[fragile]
	\frametitle{Änderungen im System}
	\framesubtitle{Hooks und Signals (1)}

	% decrease font size for code listing
	\lstset{basicstyle=\tiny\ttfamily}

	\begin{itemize}

		\item Ein zusätzlicher Hook wurde zur Methode \texttt{BackendUtility::viewOnClick()}
			zugefügt, um die Preview-URL zu verarbeiten

		\item Die Registrierung einer Hook-Klasse implementiert eine Methode \texttt{postProcess}:

			\begin{lstlisting}
				$GLOBALS['TYPO3_CONF_VARS']['SC_OPTIONS']['t3lib/class.t3lib_befunc.php']['viewOnClickClass'][] =
				  \VENDOR\MyExt\Hooks\BackendUtilityHook::class;
			\end{lstlisting}

	\end{itemize}

\end{frame}

% ------------------------------------------------------------------------------
% LTXE-SLIDE-START
% LTXE-SLIDE-UID:		9c150d50-ee3ba19b-120ef653-e714135c
% LTXE-SLIDE-TITLE:		Hooks and Signals (2): hook to override a record overlay
% LTXE-SLIDE-REFERENCE:	!Feature-72505-IntroduceHookToOverrideARecordOverlay.rst
% ------------------------------------------------------------------------------
\begin{frame}[fragile]
	\frametitle{Änderungen im System}
	\framesubtitle{Hooks und Signals (2)}

	% decrease font size for code listing
	\lstset{basicstyle=\tiny\ttfamily}

	\begin{itemize}

		\item Vor TYPO3 CMS 7.6, war es möglich, ein Record-Overlay innerhalb von \texttt{Web -> List} zu überschreiben. Ein neuer Hook in TYPO3 CMS 8.0 stellt die alte Funktionalität wieder zur Verfügung

		\item Der Hook wird mit der folgenden Signatur aufgerufen:
			\begin{lstlisting}
				/**
				 * @param string $table
				 * @param array $row
				 * @param array $status
				 * @param string $iconName
				 * @return string the new (or given) $iconName
				 */
				function postOverlayPriorityLookup($table, array $row, array $status, $iconName) { ... }
			\end{lstlisting}

		\item Die Registrierung einer Hook-Klasse implementiert eine Methode \texttt{postOverlayPriorityLookup}:

			\begin{lstlisting}
				$GLOBALS['TYPO3_CONF_VARS']['SC_OPTIONS'][IconFactory::class]['overrideIconOverlay'][] =
				  \VENDOR\MyExt\Hooks\IconFactoryHook::class;
			\end{lstlisting}

	\end{itemize}

\end{frame}

% ------------------------------------------------------------------------------
% LTXE-SLIDE-START
% LTXE-SLIDE-UID:		37ba9f7c-fb51932d-cca9a99f-6bb7a961
% LTXE-SLIDE-TITLE:		Hooks and Signals (3): preProcessStorage signal
% LTXE-SLIDE-REFERENCE:	!Feature-72904-AddPreProcessStorageSignalToResourceFactory.rst
% ------------------------------------------------------------------------------
\begin{frame}[fragile]
	\frametitle{Änderungen im System}
	\framesubtitle{Hooks und Signals (3)}

	% decrease font size for code listing
	\lstset{basicstyle=\tiny\ttfamily}

	\begin{itemize}

		\item Es wurde ein neues Signal implementiert, welches vor der Initialisierung eines Resource Storage ausgesendet wird

		\item Die Registrierung der Klasse erfolgt in der Datei \texttt{ext\_localconf.php}:

			\begin{lstlisting}
				$dispatcher = \TYPO3\CMS\Core\Utility\GeneralUtility::makeInstance(
				  \TYPO3\CMS\Extbase\SignalSlot\Dispatcher::class);
				$dispatcher->connect(
				  \TYPO3\CMS\Core\Resource\ResourceFactory::class,
				  ResourceFactoryInterface::SIGNAL_PreProcessStorage,
				  \MY\ExtKey\Slots\ResourceFactorySlot::class,
				  'preProcessStorage'
				);
			\end{lstlisting}

		\item Die Methode wird mit den folgenden Argumenten aufgerufen:

			\begin{itemize}
				\item int \texttt{\$uid} - UID des Datensatzes
				\item array \texttt{\$recordData} - alle Daten als Array
				\item string \texttt{\$fileIdentifier} - Datei-Identifier
			\end{itemize}

	\end{itemize}

\end{frame}

% ------------------------------------------------------------------------------
% LTXE-SLIDE-START
% LTXE-SLIDE-UID:		2c656d9e-38b9be7a-d590cb6e-21eccd4d
% LTXE-SLIDE-TITLE:		Password hashing algorithm: PBKDF2
% LTXE-SLIDE-REFERENCE:	!Feature-28230-AddSupportForPBKDF2ToSaltedpasswords.rst
% ------------------------------------------------------------------------------
\begin{frame}[fragile]
	\frametitle{Änderungen im System}
	\framesubtitle{Password Hashing Algorithmus: PBKDF2}

	\begin{itemize}

		\item Es wurde ein neuer Password Hashing Algorithmus "PBKDF2" zur System-Extension "saltedpasswords" hinzugefügt

		\item PBKDF2 steht für: Password-Based Key Derivation Function 2

		\item Der Algorithmus wurde entworfen, um Brute Force Passwort-Cracken deutlich zu erschweren

	\end{itemize}

\end{frame}

% ------------------------------------------------------------------------------
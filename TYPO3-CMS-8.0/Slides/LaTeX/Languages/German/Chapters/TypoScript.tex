% ------------------------------------------------------------------------------
% TYPO3 CMS 7.6 - What's New - Chapter "TypoScript" (English Version)
%
% @author	Patrick Lobacher <patrick@lobacher.de> and Michael Schams <schams.net>
% @license	Creative Commons BY-NC-SA 3.0
% @link		http://typo3.org/download/release-notes/whats-new/
% @language	English
% ------------------------------------------------------------------------------
% LTXE-CHAPTER-UID:		b8fd0ec6-eeb4e126-fcdd7604-74ef6c57
% LTXE-CHAPTER-NAME:	TypoScript
% ------------------------------------------------------------------------------

\section{TSconfig \& TypoScript}
\begin{frame}[fragile]
	\frametitle{TSconfig \& TypoScript}

	\begin{center}\huge{Kapitel 2:}\end{center}
	\begin{center}\huge{\color{typo3darkgrey}\textbf{TSconfig \& TypoScript}}\end{center}

\end{frame}

% ------------------------------------------------------------------------------
% LTXE-SLIDE-START
% LTXE-SLIDE-UID:		0956a9a6-bb0ea776-0327261b-7722ac16
% LTXE-SLIDE-TITLE:		Make new content element wizard tab sort order configurable
% LTXE-SLIDE-REFERENCE:	!Feature-71876-MakeNewContentElementWizardTabSortOrderConfigurable.rst
% ------------------------------------------------------------------------------
\begin{frame}[fragile]
	\frametitle{TSconfig \& TypoScript}
	\framesubtitle{Sortierung der Tabs im "New content element wizard"}

	% decrease font size for code listing
	\lstset{basicstyle=\tiny\ttfamily}

	\begin{itemize}
		\item Es ist nun möglich die Sortierung der Tabs im "New content elementwizard"  zu verändern, indem die Optionen \texttt{before} und \texttt{after} im Page TSconfig gesetzt werden:

			\begin{lstlisting}
				mod.wizards.newContentElement.wizardItems.special.before = common
				mod.wizards.newContentElement.wizardItems.forms.after = common,special
			\end{lstlisting}

	\end{itemize}

\end{frame}

% ------------------------------------------------------------------------------
% LTXE-SLIDE-START
% LTXE-SLIDE-UID:		330b8895-6cecc482-88177047-8d65c88b
% LTXE-SLIDE-TITLE:		HTMLparser.stripEmptyTags.keepTags
% LTXE-SLIDE-REFERENCE:	!Feature-72045-KeepTagsInHtmlParserWhenStrippingEmptyTags.rst
% ------------------------------------------------------------------------------
\begin{frame}[fragile]
	\frametitle{TSconfig \& TypoScript}
	\framesubtitle{\texttt{HTMLparser.stripEmptyTags.keepTags}}

	% decrease font size for code listing
	\lstset{basicstyle=\tiny\ttfamily}

	\begin{itemize}

		\item Es wurde eine neue Option für die \texttt{HTMLparser.stripEmptyTags} Konfiguration hinzugefügt, welche es ermöglicht, die Tags anzugeben, die behalten werden sollen
		\item Vorher war es nur möglich, anzugeben, welche Tags entfernt werden sollen
		\item Das folgende Beispiel entfernt alle leeren Tags, \textbf{außer} \texttt{tr} und \texttt{td} Tags:

			\begin{lstlisting}
				HTMLparser.stripEmptyTags = 1
				HTMLparser.stripEmptyTags.keepTags = tr,td
			\end{lstlisting}

	\end{itemize}

	\underline{Wichtig:} wenn diese Einstellung verwendet wird, hat die Konfiguration \texttt{stripEmptyTags.tags} keinen Effekt mehr. Man kann nur eine Option zur selben Zeit verwenden.

\end{frame}

% ------------------------------------------------------------------------------
% LTXE-SLIDE-START
% LTXE-SLIDE-UID:		ec7822d8-5f3eeeb3-06a2c047-723568f7
% LTXE-SLIDE-TITLE:		EXT:form - integration of predefined forms (1)
% LTXE-SLIDE-REFERENCE:	!Feature-72309-EXTform-AllowIntegrationOfPredefinedForms.rst
% ------------------------------------------------------------------------------
\begin{frame}[fragile]
	\frametitle{TSconfig \& TypoScript}
	\framesubtitle{\texttt{EXT:form} - Integration von vordefinierten Formularen (1)}

	% decrease font size for code listing
	\lstset{basicstyle=\tiny\ttfamily}

	\begin{itemize}

		\item Das Content-Element von \texttt{EXT:form} erlaubt nun die Integration von vordefinierten Formularen

		\item Der Integrator kann Formulare definieren (z.B. innerhalb eines Site Packages), indem der Schlüssel \texttt{plugin.tx\_form.predefinedForms} verwendet wird

		\item Der Redakteur kann das neue Content-Element \texttt{mailform} auf einer Seite platzieren und dort aus einer Liste vordefinierter Formulare wählen

		\item Integratoren können eigene Formulare über TypoScript erstellen, wodurch mehr Optionen zur Verfügung stehen, als es im Form-Wizard möglich wäre (z.B. durch die Verwendung von \texttt{stdWrap})

	\end{itemize}

\end{frame}

% ------------------------------------------------------------------------------
% LTXE-SLIDE-START
% LTXE-SLIDE-UID:		04706a2c-eeb35f3e-22d8ec78-68f77235
% LTXE-SLIDE-TITLE:		EXT:form - integration of predefined forms (2)
% LTXE-SLIDE-REFERENCE:	!Feature-72309-EXTform-AllowIntegrationOfPredefinedForms.rst
% ------------------------------------------------------------------------------
\begin{frame}[fragile]
	\frametitle{TSconfig \& TypoScript}
	\framesubtitle{\texttt{EXT:form} - Integration von vordefinierten Formularen (2)}

	% decrease font size for code listing
	\lstset{basicstyle=\tiny\ttfamily}

	\begin{itemize}

		\item Es gibt für Redakteure keine Notwendigkeit mehr, den Form-Wizard zu verwenden - diese können aus vordefinierten Formularen wählen, welche Layout-technisch optimiert sind

		\item Formulare können überall wiederverwendet werden

		\item Formulare können außerhalb der Datenbank gespeichert und somit versioniert werden

		\item Damit man die Formulare im Backend auswählen kann, müssen diese über PageTS registriert werden:

		\begin{lstlisting}
			TCEFORM.tt_content.tx_form_predefinedform.addItems.contactForm =
			  LLL:EXT:my_theme/Resources/Private/Language/locallang.xlf:contactForm
		\end{lstlisting}

	\end{itemize}

\end{frame}

% ------------------------------------------------------------------------------
% LTXE-SLIDE-START
% LTXE-SLIDE-UID:		5f3eeeb3-06a2c047-723568f7-ec7822d8
% LTXE-SLIDE-ORIGIN:	3076f305-06ba38cb-646d578c-b617df02 German
% LTXE-SLIDE-TITLE:		EXT:form - integration of predefined forms (3)
% LTXE-SLIDE-REFERENCE:	!Feature-72309-EXTform-AllowIntegrationOfPredefinedForms.rst
% ------------------------------------------------------------------------------
\begin{frame}[fragile]
	\frametitle{TSconfig \& TypoScript}
	\framesubtitle{\texttt{EXT:form}: Integration von vordefinierten Formularen (3)}

	% decrease font size for code listing
	\lstset{basicstyle=\tiny\ttfamily}

	\begin{itemize}

		\item Beispiel-Formular:

		\begin{lstlisting}
			plugin.tx_form.predefinedForms.contactForm = FORM
			plugin.tx_form.predefinedForms.contactForm {
			  enctype = multipart/form-data
			  method = post
			  prefix = contact
			  confirmation = 1
			  postProcessor {
			    1 = mail
			    1 {
			      recipientEmail = test@example.com
			      senderEmail = test@example.com
			      subject {
			        value = Contact form
			        lang.de = Kontakt Formular
			      }
			    }
			  }
			  10 = TEXTLINE
			  10 {
			    name = name
			...
		\end{lstlisting}

	\end{itemize}

\end{frame}

% ------------------------------------------------------------------------------

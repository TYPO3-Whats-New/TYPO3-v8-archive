% ------------------------------------------------------------------------------
% TYPO3 CMS 7.6 - What's New - Chapter "In-Depth Changes" (English Version)
%
% @author	Patrick Lobacher <patrick@lobacher.de> and Michael Schams <schams.net>
% @license	Creative Commons BY-NC-SA 3.0
% @link		http://typo3.org/download/release-notes/whats-new/
% @language	English
% ------------------------------------------------------------------------------
% LTXE-CHAPTER-UID:		7229f1b9-b481e9bc-09c46183-a86b6a7e
% LTXE-CHAPTER-NAME:	In-Depth Changes
% ------------------------------------------------------------------------------

\section{In-Depth Changes}
\begin{frame}[fragile]
	\frametitle{In-Depth Changes}

	\begin{center}\huge{Chapter 3:}\end{center}
	\begin{center}\huge{\color{typo3darkgrey}\textbf{In-Depth Changes}}\end{center}

\end{frame}


% ------------------------------------------------------------------------------
% LTXE-SLIDE-START
% LTXE-SLIDE-UID:		e915b68d-8712e2e5-2fc94711-57ab8673
% LTXE-SLIDE-ORIGIN:	b9b259e4-9d35ace7-edff1014-7390d1d1 English
% LTXE-SLIDE-TITLE:		Support PECL-memcached in MemcachedBackend
% LTXE-SLIDE-REFERENCE:	!Feature-69794-SupportPecl-memcachedInMemcachedBackend.rst
% ------------------------------------------------------------------------------
\begin{frame}[fragile]
	\frametitle{In-Depth Changes}
	\framesubtitle{Support PECL-memcached in MemcachedBackend}

	% decrease font size for code listing
	\lstset{basicstyle=\tiny\ttfamily}

	\begin{itemize}

		\item Support for the PECL module "memcached" has been added to the
			MemcachedBackend of the Caching Framework

		\item If both, "memcache" and "memcached" are installed, "memcache"
			is used to avoid being a breaking change.

		\item An integrator may set the option \texttt{peclModule} to use the
			preferred PECL module:

			\begin{lstlisting}
				$GLOBALS['TYPO3_CONF_VARS']['SYS']['caching']['cacheConfigurations']['my_memcached'] = [
				  'frontend' => \TYPO3\CMS\Core\Cache\Frontend\VariableFrontend::class
				  'backend' => \TYPO3\CMS\Core\Cache\Backend\MemcachedBackend::class,
				  'options' => [
				    'peclModule' => 'memcached',
				    'servers' => [
				      'localhost',
				      'server2:port'
				    ]
				  ]
				];
			\end{lstlisting}

	\end{itemize}

\end{frame}


% ------------------------------------------------------------------------------
% LTXE-SLIDE-START
% LTXE-SLIDE-UID:		31727c33-073c982a-e1caf670-9d4e9411
% LTXE-SLIDE-ORIGIN:	ef8c124e-f902453a-d48770b9-4cf86545 English
% LTXE-SLIDE-TITLE:		Native support for Symfony Console (1)
% LTXE-SLIDE-REFERENCE:	!Feature-73042-IntroduceNativeSupportForSymfonyConsole.rst
% ------------------------------------------------------------------------------
\begin{frame}[fragile]
	\frametitle{In-Depth Changes}
	\framesubtitle{Native support for Symfony Console (1)}

	% decrease font size for code listing
	\lstset{basicstyle=\tiny\ttfamily}

	\begin{itemize}

		\item TYPO3 supports the Symfony Console component out-of-the-box now by providing
			a new command line script located in \texttt{typo3/sysext/core/bin/typo3}.
			On TYPO3 instances installed via Composer, the binary is linked into the
			\texttt{bin}-directory, e.g. \texttt{bin/typo3}.

		\item The new binary still supports the existing command line arguments when no
			proper Symfony Console command was found as a fallback.

		\item Registering a command to be available via the \texttt{typo3} command line
			tool works by putting a \texttt{Configuration/Commands.php} file into any
			installed extension. This lists the Symfony/Console/Command classes to be
			executed by \texttt{typo3} is an associative array. The key is the name of
			the command to be called as the first argument to \texttt{typo3}.

	\end{itemize}

\end{frame}

% ------------------------------------------------------------------------------
% LTXE-SLIDE-START
% LTXE-SLIDE-UID:		880529e1-7744cb47-b424c066-b98ffd12
% LTXE-SLIDE-ORIGIN:	28b018bb-21804fa1-c578f0ff-fc16cbb1 English
% LTXE-SLIDE-TITLE:		Native support for Symfony Console (2)
% LTXE-SLIDE-REFERENCE:	!Feature-73042-IntroduceNativeSupportForSymfonyConsole.rst
% ------------------------------------------------------------------------------
\begin{frame}[fragile]
	\frametitle{In-Depth Changes}
	\framesubtitle{Native support for Symfony Console (2)}

	% decrease font size for code listing
	\lstset{basicstyle=\tiny\ttfamily}

	\begin{itemize}

		\item A required parameter when registering a command is the \texttt{class} property.
			Optionally the \texttt{user} parameter can be set so a backend user is logged
			in when calling the command.

		\item A \texttt{Configuration/Commands.php} could look like this:

			\begin{lstlisting}
				return [
				  'backend:lock' => [
				    'class' => \TYPO3\CMS\Backend\Command\LockBackendCommand::class
				  ],
				  'referenceindex:update' => [
				    'class' => \TYPO3\CMS\Backend\Command\ReferenceIndexUpdateCommand::class,
				    'user' => '_cli_lowlevel'
				  ]
				];
			\end{lstlisting}

	\end{itemize}

\end{frame}

% ------------------------------------------------------------------------------
% LTXE-SLIDE-START
% LTXE-SLIDE-UID:		c85243e0-c0cd3b34-28f75a47-7a1865f1
% LTXE-SLIDE-ORIGIN:	fc16cbb1-21804fa1-c578f0ff-28b018bb English
% LTXE-SLIDE-TITLE:		Native support for Symfony Console (3)
% LTXE-SLIDE-REFERENCE:	!Feature-73042-IntroduceNativeSupportForSymfonyConsole.rst
% ------------------------------------------------------------------------------
\begin{frame}[fragile]
	\frametitle{In-Depth Changes}
	\framesubtitle{Native support for Symfony Console (3)}

	% decrease font size for code listing
	\lstset{basicstyle=\tiny\ttfamily}

	\begin{itemize}

		\item An example call could look like:
			\begin{lstlisting}
				bin/typo3 backend:lock http://example.com/maintenance.html
			\end{lstlisting}

		\item For a non-Composer installation:
			\begin{lstlisting}
				typo3/sysext/core/bin/typo3 backend:lock http://example.com/maintenance.html
			\end{lstlisting}

	\end{itemize}

\end{frame}

% ------------------------------------------------------------------------------
% LTXE-SLIDE-START
% LTXE-SLIDE-UID:		3efb6e26-1db39c68-56ede98e-b5e317ac
% LTXE-SLIDE-ORIGIN:	1626edc6-8d93ff84-f64178bf-8e7650df English
% LTXE-SLIDE-TITLE:		Cryptographically secure pseudorandom number generator
% LTXE-SLIDE-REFERENCE:	!Feature-73050-AddACSPRNGAPI.rst
% ------------------------------------------------------------------------------
\begin{frame}[fragile]
	\frametitle{In-Depth Changes}
	\framesubtitle{Cryptographically secure pseudorandom number generator}

	% decrease font size for code listing
	\lstset{basicstyle=\tiny\ttfamily}

	\begin{itemize}

		\item A new cryptographically secure pseudo-random number generator (CSPRNG) has been
			implemented in the TYPO3 core.\newline
			It takes advantage of the new CSPRNG functions in PHP 7.

		\item The API resides in the class
			\texttt{\textbackslash
				TYPO3\textbackslash
				CMS\textbackslash
				Core\textbackslash
				Crypto\textbackslash
				Random}

		\item Example:

			\begin{lstlisting}
				use \TYPO3\CMS\Core\Crypto\Random;
				use \TYPO3\CMS\Core\Utility\GeneralUtility;

				// Retrieving random bytes
				$someRandomString = GeneralUtility::makeInstance(Random::class)->generateRandomBytes(64);

				// Rolling the dice..
				$tossedValue = GeneralUtility::makeInstance(Random::class)->generateRandomInteger(1, 6);
			\end{lstlisting}

	\end{itemize}

\end{frame}

% ------------------------------------------------------------------------------
% LTXE-SLIDE-START
% LTXE-SLIDE-UID:		0f881566-0568b589-6ecdd860-55898e51
% LTXE-SLIDE-ORIGIN:	f098bb23-81eee73f-d9eb0d80-50bbe58b English
% LTXE-SLIDE-TITLE:		Wizard component (1)
% LTXE-SLIDE-REFERENCE:	!Feature-73429-WizardComponentHasBeenAdded.rst
% ------------------------------------------------------------------------------
\begin{frame}[fragile]
	\frametitle{In-Depth Changes}
	\framesubtitle{Wizard component (1)}

	% decrease font size for code listing
	\lstset{basicstyle=\tiny\ttfamily}

	\begin{itemize}

		\item A new wizard component has been added. This component may be used for user-guided interactions

		\item The RequireJS module can be used by including \texttt{TYPO3/CMS/Backend/Wizard}

		\item The wizard supports straight forward actions only\newline
			(junctions are not possible yet)

		\item The API resides in class
			\texttt{\textbackslash
				TYPO3\textbackslash
				CMS\textbackslash
				Core\textbackslash
				Crypto\textbackslash
				Random}

		\item The wizard component has the following public methods:

			\begin{lstlisting}
				addSlide(identifier, title, content, severity, callback)
				addFinalProcessingSlide(callback)
				set(key, value)
				show()
				dismiss()
				getComponent()
				lockNextStep()
				unlockNextStep()
			\end{lstlisting}

	\end{itemize}

\end{frame}

% ------------------------------------------------------------------------------
% LTXE-SLIDE-START
% LTXE-SLIDE-UID:		636d225a-bfe89292-d8d3d287-580b32c2
% LTXE-SLIDE-ORIGIN:	f098bb23-81eee73f-d9eb0d80-50bbe58b English
% LTXE-SLIDE-TITLE:		Wizard component (2)
% LTXE-SLIDE-REFERENCE:	!Feature-73429-WizardComponentHasBeenAdded.rst
% ------------------------------------------------------------------------------
\begin{frame}[fragile]
	\frametitle{In-Depth Changes}
	\framesubtitle{Wizard component (2)}

	% decrease font size for code listing
	\lstset{basicstyle=\tiny\ttfamily}

	\begin{itemize}

		\item The event \texttt{wizard-visible} is fired when the wizard rendering has finished

		\item Wizards can be closed by firing the \texttt{wizard-dismiss} event

		\item Wizards fire the \texttt{wizard-dismissed} event if the wizard is closed

		\item You can integrate your own listener by using \texttt{Wizard.getComponent()}

	\end{itemize}

\end{frame}

% ------------------------------------------------------------------------------
% LTXE-SLIDE-START
% LTXE-SLIDE-UID:		2cce6dd7-d9d82a96-4bb8d743-629eb455
% LTXE-SLIDE-ORIGIN:	5f0d3565-8b6ad76b-44ec4d62-247401f7 English
% LTXE-SLIDE-TITLE:		Generated asset files moved
% LTXE-SLIDE-REFERENCE:	!Important-72580-PubliclyAccessibleGeneratedAssetFilesMovedToTypo3tempassets.rst
% ------------------------------------------------------------------------------
\begin{frame}[fragile]
	\frametitle{In-Depth Changes}
	\framesubtitle{Generated asset files moved}

	% decrease font size for code listing
	\lstset{basicstyle=\tiny\ttfamily}

	\begin{itemize}

		\item The folder structure within \texttt{typo3temp} changed to separate assets
			that need to be accessed by the client from the files that are temporary
			created for (e.g. for caching or locking purposes and require server-side
			access only).

		\item These assets were moved from folders:\newline
			\texttt{\_processed\_}, \texttt{compressor}, \texttt{GB}, \texttt{temp},
			\texttt{Language}, \texttt{pics}\newline
			and re-organized into:

			\begin{itemize}
				\item \texttt{typo3temp/assets/js/}
				\item \texttt{typo3temp/assets/css/},
				\item \texttt{typo3temp/assets/compressed/}
				\item \texttt{typo3temp/assets/images/}
			\end{itemize}

	\end{itemize}

\end{frame}

% ------------------------------------------------------------------------------
% LTXE-SLIDE-START
% LTXE-SLIDE-UID:		660e62fd-2935aba6-91734dbb-1384e8e7
% LTXE-SLIDE-ORIGIN:	a4d7bec0-6c19d8b2-6a6be951-5d5aa0c0 English
% LTXE-SLIDE-TITLE:		ImageMagick/GraphicsMagick changes (1)
% LTXE-SLIDE-REFERENCE:	!Breaking-43085-RenamedGraphicsProcessorSettings.rst
% ------------------------------------------------------------------------------
\begin{frame}[fragile]
	\frametitle{In-Depth Changes}
	\framesubtitle{ImageMagick/GraphicsMagick changes (1)}

	% decrease font size for code listing
	\lstset{basicstyle=\tiny\ttfamily}

	\begin{itemize}

		\item Graphics processor settings for Image- or GraphicsMagick have been renamed
			(file: \texttt{LocalConfiguration.php}).\newline
			OLD:\tabto{1.0cm} \texttt{im\_}\newline
			NEW:\tabto{1.0cm} \texttt{processor\_}

		\item Negative naming such as \texttt{noScaleUp} have been changed to positive counterparts.
			During the conversion, the previous configuration values are negated to reflect the
			changes in semantics of these options.

		\item In addition, references to specific versions of ImageMagick/GraphicsMagick have been
			removed from settings names and values.

	\end{itemize}

\end{frame}

% ------------------------------------------------------------------------------
% LTXE-SLIDE-START
% LTXE-SLIDE-UID:		fcbbb370-76d331be-ecebe6d5-109a7b67
% LTXE-SLIDE-ORIGIN:	d48d2af2-66b218a6-4a54e402-097ad85e English
% LTXE-SLIDE-TITLE:		ImageMagick/GraphicsMagick changes (2)
% LTXE-SLIDE-REFERENCE:	!Breaking-43085-RenamedGraphicsProcessorSettings.rst
% ------------------------------------------------------------------------------
\begin{frame}[fragile]
	\frametitle{In-Depth Changes}
	\framesubtitle{ImageMagick/GraphicsMagick changes (2)}

	% decrease font size for code listing
	\lstset{basicstyle=\tiny\ttfamily}

	\begin{itemize}

		\item The unused configuration option \texttt{image\_processing} has been removed without
			replacement

		\item The processor-specific configuration option \texttt{colorspace} has been \textit{namespaced}
			below the \texttt{processor\_} hierarchy

	\end{itemize}

\end{frame}

% ------------------------------------------------------------------------------
% LTXE-SLIDE-START
% LTXE-SLIDE-UID:		b064c133-fb51674c-a2d87b95-d54191a9
% LTXE-SLIDE-ORIGIN:	74e4b871-9f2ded00-f54bc3c1-85d22167 English
% LTXE-SLIDE-TITLE:		Hooks and Signals (1): post-processing of the preview URL
% LTXE-SLIDE-REFERENCE:	!Feature-54887-Post-processingOfThePreviewUrl.rst
% ------------------------------------------------------------------------------
\begin{frame}[fragile]
	\frametitle{In-Depth Changes}
	\framesubtitle{Hooks and Signals (1)}

	% decrease font size for code listing
	\lstset{basicstyle=\tiny\ttfamily}

	\begin{itemize}

		\item An additional hook has been added to method \texttt{BackendUtility::viewOnClick()}
			to post-process the preview URL

		\item Register a hook class which implements the method with the name \texttt{postProcess}:

			\begin{lstlisting}
				$GLOBALS['TYPO3_CONF_VARS']['SC_OPTIONS']['t3lib/class.t3lib_befunc.php']['viewOnClickClass'][] =
				  \VENDOR\MyExt\Hooks\BackendUtilityHook::class;
			\end{lstlisting}

	\end{itemize}

\end{frame}

% ------------------------------------------------------------------------------
% LTXE-SLIDE-START
% LTXE-SLIDE-UID:		a5fa5137-0d24faf0-a830fed5-e936128d
% LTXE-SLIDE-ORIGIN:	9c150d50-ee3ba19b-120ef653-e714135c English
% LTXE-SLIDE-TITLE:		Hooks and Signals (2): hook to override a record overlay
% LTXE-SLIDE-REFERENCE:	!Feature-72505-IntroduceHookToOverrideARecordOverlay.rst
% ------------------------------------------------------------------------------
\begin{frame}[fragile]
	\frametitle{In-Depth Changes}
	\framesubtitle{Hooks and Signals (2)}

	% decrease font size for code listing
	\lstset{basicstyle=\tiny\ttfamily}

	\begin{itemize}

		\item Prior TYPO3 CMS 7.6, it was possible to override a record overlay in \texttt{Web -> List}.
			A new hook in TYPO3 CMS 8.0 provides the old functionality.

		\item The hook is called with the following signature:
			\begin{lstlisting}
				/**
				 * @param string $table
				 * @param array $row
				 * @param array $status
				 * @param string $iconName
				 * @return string the new (or given) $iconName
				 */
				function postOverlayPriorityLookup($table, array $row, array $status, $iconName) { ... }
			\end{lstlisting}

		\item Register the hook class which implements the method with the name \texttt{postOverlayPriorityLookup}:

			\begin{lstlisting}
				$GLOBALS['TYPO3_CONF_VARS']['SC_OPTIONS'][IconFactory::class]['overrideIconOverlay'][] =
				  \VENDOR\MyExt\Hooks\IconFactoryHook::class;
			\end{lstlisting}

	\end{itemize}

\end{frame}

% ------------------------------------------------------------------------------
% LTXE-SLIDE-START
% LTXE-SLIDE-UID:		84ca5162-688a71ee-0615161e-4ceff3ce
% LTXE-SLIDE-ORIGIN:	37ba9f7c-fb51932d-cca9a99f-6bb7a961 English
% LTXE-SLIDE-TITLE:		Hooks and Signals (3): preProcessStorage signal
% LTXE-SLIDE-REFERENCE:	!Feature-72904-AddPreProcessStorageSignalToResourceFactory.rst
% ------------------------------------------------------------------------------
\begin{frame}[fragile]
	\frametitle{In-Depth Changes}
	\framesubtitle{Hooks and Signals (3)}

	% decrease font size for code listing
	\lstset{basicstyle=\tiny\ttfamily}

	\begin{itemize}

		\item A new signal has been implemented before a resource storage is initialized.

		\item Register the class which implements your logic in \texttt{ext\_localconf.php}:

			\begin{lstlisting}
				$dispatcher = \TYPO3\CMS\Core\Utility\GeneralUtility::makeInstance(
				  \TYPO3\CMS\Extbase\SignalSlot\Dispatcher::class);
				$dispatcher->connect(
				  \TYPO3\CMS\Core\Resource\ResourceFactory::class,
				  ResourceFactoryInterface::SIGNAL_PreProcessStorage,
				  \MY\ExtKey\Slots\ResourceFactorySlot::class,
				  'preProcessStorage'
				);
			\end{lstlisting}

		\item The method is called with the following arguments:

			\begin{itemize}
				\item int \texttt{\$uid} the uid of the record
				\item array \texttt{\$recordData} all record data as array
				\item string \texttt{\$fileIdentifier} the file identifier
			\end{itemize}

	\end{itemize}

\end{frame}

% ------------------------------------------------------------------------------
% LTXE-SLIDE-START
% LTXE-SLIDE-UID:		9ff2b1c8-e7034eb5-e6568210-7043c43b
% LTXE-SLIDE-ORIGIN:	2c656d9e-38b9be7a-d590cb6e-21eccd4d English
% LTXE-SLIDE-TITLE:		Password hashing algorithm: PBKDF2
% LTXE-SLIDE-REFERENCE:	!Feature-28230-AddSupportForPBKDF2ToSaltedpasswords.rst
% ------------------------------------------------------------------------------
\begin{frame}[fragile]
	\frametitle{In-Depth Changes}
	\framesubtitle{Password hashing algorithm: PBKDF2}

	\begin{itemize}

		\item A new password hashing algorithm "PBKDF2" has been added to the system extension "saltedpasswords"

		\item PBKDF2 stands for: Password-Based Key Derivation Function 2

		\item The algorithm is designed to be computationally expensive to resist brute force password cracking

	\end{itemize}

\end{frame}

% ------------------------------------------------------------------------------
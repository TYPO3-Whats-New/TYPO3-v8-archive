% ------------------------------------------------------------------------------
% TYPO3 CMS 8.5 - What's New - Chapter "Extbase & Fluid" (Serbian Version)
%
% @author	Michael Schams <schams.net>
% @license	Creative Commons BY-NC-SA 3.0
% @link		http://typo3.org/download/release-notes/whats-new/
% @language	English
% ------------------------------------------------------------------------------
% LTXE-CHAPTER-UID:		846d40ce-66fdc2ec-750dcf95-33ce93e0
% LTXE-CHAPTER-NAME:	Extbase & Fluid
% ------------------------------------------------------------------------------

\section{Extbase \& Fluid}
\begin{frame}[fragile]
	\frametitle{Extbase \& Fluid}

	\begin{center}\huge{Chapter 4:}\end{center}
	\begin{center}\huge{\color{typo3darkgrey}\textbf{Extbase \& Fluid}}\end{center}

\end{frame}

% ------------------------------------------------------------------------------
% LTXE-SLIDE-START
% LTXE-SLIDE-UID:		0338c8cf-7eb630f2-acae202e-adf729a0
% LTXE-SLIDE-ORIGIN:	c5986329-fb36f473-a5430d2e-6ba49fb1 English
% LTXE-SLIDE-TITLE:		#78116: Extbase support for Doctrine's native DBAL Statement and QueryBuilder
% ------------------------------------------------------------------------------
\begin{frame}[fragile]
	\frametitle{Extbase \& Fluid}
	\framesubtitle{Doctrine DBAL}

	% decrease font size for code listing
	\lstset{basicstyle=\tiny\ttfamily}

	\begin{itemize}
		\item Direct SQL query functionality also supports QueryBuilder objects and instances of
			\texttt{\textbackslash
			Doctrine\textbackslash
			DBAL\textbackslash
			Statement} as prepared statements
		\item The following example works in any Extbase repository using native Doctrine DBAL statements:

			\begin{lstlisting}
				$connection = $this->objectManager->get(ConnectionPool::class)->getConnectionForTable('mytable');
				$statement = $this->objectManager->get(
				  \Doctrine\DBAL\Statement::class,
				  'SELECT * FROM mytable WHERE uid=? OR title=?',
				  $connection
				);

				$query = $this->createQuery();
				$query->statement($statement, [$uid, $title]);
			\end{lstlisting}
	\end{itemize}

\end{frame}
% ------------------------------------------------------------------------------
% LTXE-SLIDE-START
% LTXE-SLIDE-UID:		5a9ad83e-eaf27f1e-d9f6d58f-54f80269
% LTXE-SLIDE-ORIGIN:	8d31d436-4397d4fd-aa0eed2e-df798aa3 English
% LTXE-SLIDE-TITLE:		#78002: Enforce cHash argument for Extbase actions
% ------------------------------------------------------------------------------

\begin{frame}[fragile]
	\frametitle{Extbase \& Fluid}
	\framesubtitle{\texttt{cHash} Argument}

	\begin{itemize}
		\item URIs to Extbase actions now require a valid \texttt{cHash} by default\newline
			(cached and uncached actions)
		\item The behavior can be disabled for all actions using the feature switch
			\texttt{requireCHashArgumentForActionArguments}
	\end{itemize}

\end{frame}
% ------------------------------------------------------------------------------
% LTXE-SLIDE-START
% LTXE-SLIDE-UID:		57614743-ab44f5bc-33403be0-51748699
% LTXE-SLIDE-ORIGIN:	50143a1e-53a17c8c-83ffe7ea-5e88cba0 English
% LTXE-SLIDE-TITLE:		#29399: OptionViewHelper and OptgroupViewHelper
% ------------------------------------------------------------------------------

\begin{frame}[fragile]
	\frametitle{Extbase \& Fluid}
	\framesubtitle{Content for ViewHelper \texttt{f:form.select}}

	% decrease font size for code listing
	\lstset{basicstyle=\tiny\ttfamily}

	\begin{itemize}
		\item Introduced two new ViewHelpers allowing the manual definition of all options and
		    optgroups for the \texttt{f:form.select} as tag content of the select field

			\begin{itemize}
				\item \texttt{OptionViewHelper}
				\item \texttt{OptgroupViewHelper}
			\end{itemize}

		\item Example:

			\begin{lstlisting}
				<f:form.select name="myproperty">
				  <f:form.select.option value="1">Option one</f:form.select.option>
				  <f:form.select.option value="2">Option two</f:form.select.option>
				  <f:form.select.optgroup>
				    <f:form.select.option value="3">Grouped option one</f:form.select.option>
				    <f:form.select.option value="4">Grouped option twi</f:form.select.option>
				  </f:form.select.optgroup>
				</f:form.select>
			\end{lstlisting}

		\end{itemize}

\end{frame}


% ------------------------------------------------------------------------------
% LTXE-SLIDE-START
% LTXE-SLIDE-UID:		e2d95b5c-c5e2cd2f-71b20357-773cef1a
% LTXE-SLIDE-ORIGIN:	d36b9935-fc7d9e60-742a3f42-ef15e856 English
% LTXE-SLIDE-TITLE:		#78415: Global Fluid ViewHelper Namespace
% ------------------------------------------------------------------------------
\begin{frame}[fragile]
	\frametitle{Extbase \& Fluid}
	\framesubtitle{Global Fluid ViewHelper Namespace}

	\begin{itemize}
		\item Global Fluid ViewHelper namespaces are configurable now:\newline
			\smaller
				\texttt{\$GLOBALS['TYPO3\_CONF\_VARS']['SYS']['fluid']['namespaces']}
			\normalsize
		\item This allows the namespaces to be manipulated as part of the site configuration
		\item Benefits:

			\begin{itemize}
				\item Third party ViewHelper packages can manipulate the global Fluid namespace \texttt{f:}
				\item Third party ViewHelper packages are able to register new global namespaces as required
				\item Template developers can use such global namespaces without importing them first
					and can use them in all Fluid templates regardless of context
			\end{itemize}

	\end{itemize}

\end{frame}


% ------------------------------------------------------------------------------
% LTXE-SLIDE-START
% LTXE-SLIDE-UID:		f0162e25-8cac6971-9afc06f5-6db7bdc2
% LTXE-SLIDE-ORIGIN:	e16e8cfd-26f35f09-23b473a2-442ff1e2 English
% LTXE-SLIDE-TITLE:		#78842: FLUIDTEMPLATE can mimic an actual Extbase web request
% ------------------------------------------------------------------------------
\begin{frame}[fragile]
	\frametitle{Extbase \& Fluid}
	\framesubtitle{\texttt{FLUIDTEMPLATE} can Mimic Extbase Web Requests}

	% decrease font size for code listing
	\lstset{basicstyle=\small\ttfamily}

	\begin{itemize}
		\item The \texttt{FLUIDTEMPLATE} content element can mimic an actual Extbase web request now
		\item This makes it possible to access submitted data for example:

			\begin{lstlisting}
				$view->getRenderingContext()
				  ->getControllerContext()
				  ->getRequest()
				  ->getArguments();
			\end{lstlisting}

	\end{itemize}

\end{frame}




% ------------------------------------------------------------------------------

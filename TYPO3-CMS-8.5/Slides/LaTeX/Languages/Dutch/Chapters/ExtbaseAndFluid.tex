% ------------------------------------------------------------------------------
% TYPO3 CMS 8.5 - What's New - Chapter "Extbase & Fluid" (Dutch Version)
%
% @author	Michael Schams <schams.net>
% @license	Creative Commons BY-NC-SA 3.0
% @link		http://typo3.org/download/release-notes/whats-new/
% @language	English
% ------------------------------------------------------------------------------
% LTXE-CHAPTER-UID:		846d40ce-66fdc2ec-750dcf95-33ce93e0
% LTXE-CHAPTER-NAME:	Extbase & Fluid
% ------------------------------------------------------------------------------

\section{Extbase \& Fluid}
\begin{frame}[fragile]
	\frametitle{Extbase \& Fluid}

	\begin{center}\huge{Hoofdstuk 4:}\end{center}
	\begin{center}\huge{\color{typo3darkgrey}\textbf{Extbase \& Fluid}}\end{center}

\end{frame}

% ------------------------------------------------------------------------------
% LTXE-SLIDE-START
% LTXE-SLIDE-UID:		1fd8fd16-06e4d566-11d651d4-aa3dce09
% LTXE-SLIDE-ORIGIN:	c5986329-fb36f473-a5430d2e-6ba49fb1 English
% LTXE-SLIDE-TITLE:		#78116: Extbase support for Doctrine's native DBAL Statement and QueryBuilder
% ------------------------------------------------------------------------------
\begin{frame}[fragile]
	\frametitle{Extbase \& Fluid}
	\framesubtitle{Doctrine DBAL}

	% decrease font size for code listing
	\lstset{basicstyle=\tiny\ttfamily}

	\begin{itemize}
		\item Directe SQL-query-functionaliteit ondersteunt ook QueryBuilder-objecten en instanties van
			\texttt{\textbackslash
			Doctrine\textbackslash
			DBAL\textbackslash
			Statement} als prepared statements
		\item Het volgende voorbeeld werkt in elke Extbase repository met pure Doctrine DBAL statements:

			\begin{lstlisting}
				$connectie = $this->objectManager->get(ConnectionPool::class)->getConnectionForTable('mijntabel');
				$statement = $this->objectManager->get(
				  \Doctrine\DBAL\Statement::class,
				  'SELECT * FROM mijtabel WHERE uid=? OR titel=?',
				  $connection
				);

				$query = $this->createQuery();
				$query->statement($statement, [$uid, $titel]);
			\end{lstlisting}
	\end{itemize}

\end{frame}
% ------------------------------------------------------------------------------
% LTXE-SLIDE-START
% LTXE-SLIDE-UID:		137efb55-723a570d-d61c8ab4-076b7eb7
% LTXE-SLIDE-ORIGIN:	8d31d436-4397d4fd-aa0eed2e-df798aa3 English
% LTXE-SLIDE-TITLE:		#78002: Enforce cHash argument for Extbase actions
% ------------------------------------------------------------------------------

\begin{frame}[fragile]
	\frametitle{Extbase \& Fluid}
	\framesubtitle{\texttt{cHash} Argument}

	\begin{itemize}
		\item URI's naar Extbase-acties vereisen nu standaard een geldige \texttt{cHash}\newline
			(gecachete en ongecachete acties)
		\item Het gedrag kan uitgeschakeld worden voor alle acties met de optie
			\texttt{requireCHashArgumentForActionArguments}
	\end{itemize}

\end{frame}
% ------------------------------------------------------------------------------
% LTXE-SLIDE-START
% LTXE-SLIDE-UID:		c925287a-db500aa4-d240b4e2-2d207790
% LTXE-SLIDE-ORIGIN:	50143a1e-53a17c8c-83ffe7ea-5e88cba0 English
% LTXE-SLIDE-TITLE:		#29399: OptionViewHelper and OptgroupViewHelper
% ------------------------------------------------------------------------------

\begin{frame}[fragile]
	\frametitle{Extbase \& Fluid}
	\framesubtitle{Inhoud voor ViewHelper \texttt{f:form.select}}

	% decrease font size for code listing
	\lstset{basicstyle=\tiny\ttfamily}

	\begin{itemize}
		\item Er zijn twee nieuwe ViewHelpers voor het handmatig definiëren van alle opties en
		 	optgroups voor de \texttt{f:form.select} als inhoud van een select-veld

			\begin{itemize}
				\item \texttt{OptionViewHelper}
				\item \texttt{OptgroupViewHelper}
			\end{itemize}

		\item Voorbeeld:

			\begin{lstlisting}
				<f:form.select name="myproperty">
				  <f:form.select.option value="1">Optie eem</f:form.select.option>
				  <f:form.select.option value="2">Optie twee</f:form.select.option>
				  <f:form.select.optgroup>
				    <f:form.select.option value="3">Gegroepeerde optie een</f:form.select.option>
				    <f:form.select.option value="4">Gegroupeerde optie twee</f:form.select.option>
				  </f:form.select.optgroup>
				</f:form.select>
			\end{lstlisting}

		\end{itemize}

\end{frame}


% ------------------------------------------------------------------------------
% LTXE-SLIDE-START
% LTXE-SLIDE-UID:		a37bd34d-794ba46c-a905b4e7-62800639
% LTXE-SLIDE-ORIGIN:	d36b9935-fc7d9e60-742a3f42-ef15e856 English
% LTXE-SLIDE-TITLE:		#78415: Global Fluid ViewHelper Namespace
% ------------------------------------------------------------------------------
\begin{frame}[fragile]
	\frametitle{Extbase \& Fluid}
	\framesubtitle{Globale Fluid ViewHelper Namespace}

	\begin{itemize}
		\item Globale Fluid ViewHelper namespaces zijn nu instelbaar:\newline
			\smaller
				\texttt{\$GLOBALS['TYPO3\_CONF\_VARS']['SYS']['fluid']['namespaces']}
			\normalsize
		\item Hiermee kunnen de namespaces bewerkt worden als onderdeel van de configuratie van een site
		\item Voordelen:

			\begin{itemize}
				\item ViewHelper pakketten van derden kunnen de globale Fluid namespace \texttt{f:} aanpassen
				\item ViewHelper pakketten van derden kunnen nieuwe globale namespaces registreren
				\item Ontwikkelaars van sjablonen kunnen die globale namespaces gebruiken zonder ze eerst te importeren
					en kunnen ze gebruiken in alle Fluid-sjablonen ongeacht de context
			\end{itemize}

	\end{itemize}

\end{frame}


% ------------------------------------------------------------------------------
% LTXE-SLIDE-START
% LTXE-SLIDE-UID:		2a0a1672-56f18db2-60b397cb-effebca2
% LTXE-SLIDE-ORIGIN:	e16e8cfd-26f35f09-23b473a2-442ff1e2 English
% LTXE-SLIDE-TITLE:		#78842: FLUIDTEMPLATE can mimic an actual Extbase web request
% ------------------------------------------------------------------------------
\begin{frame}[fragile]
	\frametitle{Extbase \& Fluid}
	\framesubtitle{\texttt{FLUIDTEMPLATE} kan een Extbase web-request nabootsen}

	% decrease font size for code listing
	\lstset{basicstyle=\small\ttfamily}

	\begin{itemize}
		\item Het \texttt{FLUIDTEMPLATE} inhoudselement kan een Extbase web request nabootsen
		\item Hiermee is het mogelijk om bijvoorbeeld verstuurde gegevens te benaderen:

			\begin{lstlisting}
				$view->getRenderingContext()
				  ->getControllerContext()
				  ->getRequest()
				  ->getArguments();
			\end{lstlisting}

	\end{itemize}

\end{frame}




% ------------------------------------------------------------------------------

% ------------------------------------------------------------------------------
% TYPO3 CMS 8.5 - What's New - Chapter "Extbase & Fluid" (Italian Version)
%
% @author	Michael Schams <schams.net>
% @license	Creative Commons BY-NC-SA 3.0
% @link		http://typo3.org/download/release-notes/whats-new/
% @language	English
% ------------------------------------------------------------------------------
% LTXE-CHAPTER-UID:		846d40ce-66fdc2ec-750dcf95-33ce93e0
% LTXE-CHAPTER-NAME:	Extbase & Fluid
% ------------------------------------------------------------------------------

\section{Extbase \& Fluid}
\begin{frame}[fragile]
	\frametitle{Extbase \& Fluid}

	\begin{center}\huge{Capitolo 4:}\end{center}
	\begin{center}\huge{\color{typo3darkgrey}\textbf{Extbase \& Fluid}}\end{center}

\end{frame}

% ------------------------------------------------------------------------------
% LTXE-SLIDE-START
% LTXE-SLIDE-UID:		b11e8cf3-20531dc0-c4d1260f-c13b1ff7
% LTXE-SLIDE-ORIGIN:	c5986329-fb36f473-a5430d2e-6ba49fb1 English
% LTXE-SLIDE-TITLE:		#78116: Extbase support for Doctrine's native DBAL Statement and QueryBuilder
% ------------------------------------------------------------------------------
\begin{frame}[fragile]
	\frametitle{Extbase \& Fluid}
	\framesubtitle{Doctrine DBAL}

	% decrease font size for code listing
	\lstset{basicstyle=\tiny\ttfamily}

	\begin{itemize}
		\item Le funzionalità delle query SQL dirette supportano anche gli oggetti di QueryBuilder e le funzionalità di
			\texttt{\textbackslash
			Doctrine\textbackslash
			DBAL\textbackslash
			Statement} come istruzioni preparate
		\item Gli esempi seguenti funzionano in tutti i repository Extbase che utilizzano dichiarazioni native di Doctrine DBAL:

			\begin{lstlisting}
				$connection = $this->objectManager->get(ConnectionPool::class)->getConnectionForTable('mytable');
				$statement = $this->objectManager->get(
				  \Doctrine\DBAL\Statement::class,
				  'SELECT * FROM mytable WHERE uid=? OR title=?',
				  $connection
				);

				$query = $this->createQuery();
				$query->statement($statement, [$uid, $title]);
			\end{lstlisting}
	\end{itemize}

\end{frame}
% ------------------------------------------------------------------------------
% LTXE-SLIDE-START
% LTXE-SLIDE-UID:		b6ce8818-7bf91650-86e2ff72-0368b154
% LTXE-SLIDE-ORIGIN:	8d31d436-4397d4fd-aa0eed2e-df798aa3 English
% LTXE-SLIDE-TITLE:		#78002: Enforce cHash argument for Extbase actions
% ------------------------------------------------------------------------------

\begin{frame}[fragile]
	\frametitle{Extbase \& Fluid}
	\framesubtitle{Parametri \texttt{cHash}}

	\begin{itemize}
		\item Le URI per le azioni Extbase ora richiedono una \texttt{cHash} valida di default\newline
			(azioni cached e uncached)
		\item Il comportamento può essere disabilitato per tutte le azioni che utilizzano le funzionalità di scambio
			\texttt{requireCHashArgumentForActionArguments}
	\end{itemize}

\end{frame}
% ------------------------------------------------------------------------------
% LTXE-SLIDE-START
% LTXE-SLIDE-UID:		459cdcbd-43671cbb-6a76ee5b-b5beccc9
% LTXE-SLIDE-ORIGIN:	50143a1e-53a17c8c-83ffe7ea-5e88cba0 English
% LTXE-SLIDE-TITLE:		#29399: OptionViewHelper and OptgroupViewHelper
% ------------------------------------------------------------------------------

\begin{frame}[fragile]
	\frametitle{Extbase \& Fluid}
	\framesubtitle{Contenuti per ViewHelper \texttt{f:form.select}}

	% decrease font size for code listing
	\lstset{basicstyle=\tiny\ttfamily}

	\begin{itemize}
		\item Sono stati introdotti due nuovi ViewHelper che permettono la definizione manuale di tutte le options e
		    optgroups per \texttt{f:form.select} come contenuto del tag del campo select

			\begin{itemize}
				\item \texttt{OptionViewHelper}
				\item \texttt{OptgroupViewHelper}
			\end{itemize}

		\item Esempio:

			\begin{lstlisting}
				<f:form.select name="myproperty">
				  <f:form.select.option value="1">Opzione uno</f:form.select.option>
				  <f:form.select.option value="2">Opzione due</f:form.select.option>
				  <f:form.select.optgroup>
				    <f:form.select.option value="3">Gruppo opzioni uno</f:form.select.option>
				    <f:form.select.option value="4">Gruppo opzioni due</f:form.select.option>
				  </f:form.select.optgroup>
				</f:form.select>
			\end{lstlisting}

		\end{itemize}

\end{frame}


% ------------------------------------------------------------------------------
% LTXE-SLIDE-START
% LTXE-SLIDE-UID:		e71a362a-251f9b31-86f8675d-825fac18
% LTXE-SLIDE-ORIGIN:	d36b9935-fc7d9e60-742a3f42-ef15e856 English
% LTXE-SLIDE-TITLE:		#78415: Global Fluid ViewHelper Namespace
% ------------------------------------------------------------------------------
\begin{frame}[fragile]
	\frametitle{Extbase \& Fluid}
	\framesubtitle{Namespace dei ViewHelper Fluid globali}

	\begin{itemize}
		\item I namespace dei ViewHelper di Fluid globali sono ora configurabili:\newline
			\smaller
				\texttt{\$GLOBALS['TYPO3\_CONF\_VARS']['SYS']['fluid']['namespaces']}
			\normalsize
		\item In questo modo i namespace possono essere manipolati come parte della configurazione del sito
		\item Benefici:

			\begin{itemize}
				\item Pacchetti di ViewHelper, di terze parti, possono intervenire su namespace di Fluid globali \texttt{f:}
				\item Pacchetti di ViewHelper, di terze parti, possono registrare nuovi namespace globali come necessario
				\item Gli sviluppatori di Template possono utilizzare namespace globali senza prima importarli
					e possono utilizzarli in tutti i template Fluid indipendentemente dal contesto
			\end{itemize}

	\end{itemize}

\end{frame}


% ------------------------------------------------------------------------------
% LTXE-SLIDE-START
% LTXE-SLIDE-UID:		6204bad8-1dee0a4d-607a127a-96428ad1
% LTXE-SLIDE-ORIGIN:	e16e8cfd-26f35f09-23b473a2-442ff1e2 English
% LTXE-SLIDE-TITLE:		#78842: FLUIDTEMPLATE can mimic an actual Extbase web request
% ------------------------------------------------------------------------------
\begin{frame}[fragile]
	\frametitle{Extbase \& Fluid}
	\framesubtitle{\texttt{FLUIDTEMPLATE} è in grado di simulare le richieste web di Extbase}

	% decrease font size for code listing
	\lstset{basicstyle=\small\ttfamily}

	\begin{itemize}
		\item L'elemento di contenuto \texttt{FLUIDTEMPLATE} è ora in grado di simulare una richiesta web di Extbase
		\item Questo permette di accedere ai dati sottomessi, ad esempio:

			\begin{lstlisting}
				$view->getRenderingContext()
				  ->getControllerContext()
				  ->getRequest()
				  ->getArguments();
			\end{lstlisting}

	\end{itemize}

\end{frame}




% ------------------------------------------------------------------------------

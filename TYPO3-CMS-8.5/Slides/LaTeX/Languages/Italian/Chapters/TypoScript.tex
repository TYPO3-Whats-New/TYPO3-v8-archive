% ------------------------------------------------------------------------------
% TYPO3 CMS 8.5 - What's New - Chapter "TypoScript" (Italian Version)
%
% @author	Michael Schams <schams.net>
% @license	Creative Commons BY-NC-SA 3.0
% @link		http://typo3.org/download/release-notes/whats-new/
% @language	English
% ------------------------------------------------------------------------------
% LTXE-CHAPTER-UID:		3a9852ea-e2360d9d-1ff5eec1-a7de3f9f
% LTXE-CHAPTER-NAME:	TypoScript
% ------------------------------------------------------------------------------

\section{TSconfig \& TypoScript}
\begin{frame}[fragile]
	\frametitle{TSconfig \& TypoScript}

	\begin{center}\huge{Capitolo 2:}\end{center}
	\begin{center}\huge{\color{typo3darkgrey}\textbf{TSconfig \& TypoScript}}\end{center}

\end{frame}

% ------------------------------------------------------------------------------
% LTXE-SLIDE-START
% LTXE-SLIDE-UID:		5fef87af-24a723a6-f0a2c8a5-d75f0d72
% LTXE-SLIDE-ORIGIN:	aee7c947-a271d451-cb640e43-101a5543 English
% LTXE-SLIDE-TITLE:		#78549: Override New Page Creation Wizard via page TSconfig
% ------------------------------------------------------------------------------

\begin{frame}[fragile]
	\frametitle{TSconfig \& TypoScript}
	\framesubtitle{Nuovo wizard per la creazione delle pagine}

	\begin{itemize}
		\item Nelle precedenti versioni di TYPO3 CMS, era possibile ignorare la "New Page Creation Wizard"
			con script personalizzati:\newline
			\small
				\texttt{mod.web\_list.newPageWiz.overrideWithExtension = myextension}
			\normalsize
		\item Il nuovo modo per gestire punti di entrata e script personalizzati è ora gestito tramite moduli/percorsi
			e l'opzione descritta sopra è stata rimossa
		\item La seguente configurazione TSconfig può essere utilizzata al suo posto:
			\small
				\texttt{mod.newPageWizard.override = my\_custom\_module}
			\normalsize

		\item Invece di impostare un opzione su una certa estensione, deve essere specificato un modulo personalizzato o
			un percorso 

	\end{itemize}

\end{frame}
% ------------------------------------------------------------------------------
% LTXE-SLIDE-START
% LTXE-SLIDE-UID:		0f5d69d7-c77cc146-92582688-2de33f8c
% LTXE-SLIDE-ORIGIN:	c56cb2e8-f50d7eed-e93dcb9e-ab7be017 English
% LTXE-SLIDE-TITLE:		#73626: Number of search results is configurable now
% ------------------------------------------------------------------------------
\begin{frame}[fragile]
	\frametitle{TSconfig \& TypoScript}
	\framesubtitle{Numero dei risultati della ricerca}

	\begin{itemize}
		\item Il numero massimo di risultati della ricerca può essere configurato in TypoScript:\newline
			\texttt{plugin.tx\_indexedsearch.settings.blind.numberOfResults}
		\item Questa impostazione memorizza un elenco di valori
		\item Se il numero di risultati della ricerca è passato nella richiesta e corrisponde ad uno di questi
			valori, viene usato questo numero
		\item Se nessun numero di risultati della ricerca è passato o il numero non corrisponde a nessuno dei
			valori configurati, viene utilizzato il primo della lista
		\item Per matenere la compatibilità a ritroso, i valori predefiniti sono:\newline
			\texttt{10}, \texttt{25}, \texttt{50} and \texttt{100}
	\end{itemize}

\end{frame}
% ------------------------------------------------------------------------------
% LTXE-SLIDE-START
% LTXE-SLIDE-UID:		49707cff-9dbf116e-eb7cdf35-6c00f01d
% LTXE-SLIDE-ORIGIN:	d9b9e73b-db575400-8276f4dd-810f4f4e English
% LTXE-SLIDE-TITLE:		#78672: Fluid data processor for menus (1)
% ------------------------------------------------------------------------------
\begin{frame}[fragile]
	\frametitle{TSconfig \& TypoScript}
	\framesubtitle{Fluid Data Processor per i menu (1)}

	\begin{itemize}
		\item I processori di menu utilizzano \texttt{HMENU} per generare una stringa di menu in JSON
			che deve essere decodificata nuovamente e assegnata a \texttt{FLUIDTEMPLATE}
		\item In più un DataProcessing viene supportato e applicato a ogni record
		\item Queste le opzioni supportate: \texttt{as}, \texttt{levels}, \texttt{expandAll}, \texttt{includeSpacer},
			\texttt{titleField}
			(vedi \href{https://docs.typo3.org/typo3cms/TyposcriptReference/ContentObjects/Hmenu/Index.html}{TyposcriptReference} per maggiori opzioni)
	\end{itemize}

\end{frame}

% ------------------------------------------------------------------------------
% LTXE-SLIDE-START
% LTXE-SLIDE-UID:		7de3542e-3bd50ac6-97dcc1f7-124369e6
% LTXE-SLIDE-ORIGIN:	03e756a0-07740056-a2009898-d6159f4b English
% LTXE-SLIDE-TITLE:		#78672: Fluid data processor for menus (2)
% ------------------------------------------------------------------------------
\begin{frame}[fragile]
	\frametitle{TSconfig \& TypoScript}
	\framesubtitle{Fluid Data Processor per i menu (2)}

	% decrease font size for code listing
	\lstset{basicstyle=\tiny\ttfamily}

	\begin{itemize}
		\item Esempio di configurazione TypoScript:

			\begin{lstlisting}
				10 = TYPO3\CMS\Frontend\DataProcessing\MenuProcessor
				10 {
				  special = list
				  special.value.field = pages
				  levels = 7
				  as = menu
				  expandAll = 1
				  includeSpacer = 1
				  titleField = nav_title // title
				  dataProcessing {
				    10 = TYPO3\CMS\Frontend\DataProcessing\FilesProcessor
				    10 {
				      references.fieldName = media
				    }
				  }
				}
			\end{lstlisting}

	\end{itemize}

\end{frame}

% ------------------------------------------------------------------------------
% LTXE-SLIDE-START
% LTXE-SLIDE-UID:		aa5abd54-da3ad7d5-508a87b3-0f4f4a11
% LTXE-SLIDE-ORIGIN:	4f6c99f3-cca1739f-205be2f4-3a0e3326 English
% LTXE-SLIDE-TITLE:		#72050: encapsLines does not render duplicate paragraphs anymore
% ------------------------------------------------------------------------------
\begin{frame}[fragile]
	\frametitle{TSconfig \& TypoScript}
	\framesubtitle{Funzione TypoScript \texttt{\_encapsLines}}

	\begin{itemize}
		\item La funzione TypoScript \texttt{\_encapsLines} renderizzava due paragrafi con una
			linea vuota nel contenuto. Questo è stato risolto.

		\item Questo cambiamento può influire nel frontend, se sono presenti più righe vuote 
			nel contenut RTE. L'ultimo paragrafo non è più renderizzato due volte
			nel frontend da TYPO3 CMS versione 8.5.

	\end{itemize}

\end{frame}

% ------------------------------------------------------------------------------

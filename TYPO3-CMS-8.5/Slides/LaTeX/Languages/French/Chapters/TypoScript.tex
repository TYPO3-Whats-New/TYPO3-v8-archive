% ------------------------------------------------------------------------------
% TYPO3 CMS 8.5 - What's New - Chapter "TypoScript" (French Version)
%
% @author	Michael Schams <schams.net>
% @license	Creative Commons BY-NC-SA 3.0
% @link		http://typo3.org/download/release-notes/whats-new/
% @language	French
% ------------------------------------------------------------------------------
% LTXE-CHAPTER-UID:		3a9852ea-e2360d9d-1ff5eec1-a7de3f9f
% LTXE-CHAPTER-NAME:	TypoScript
% ------------------------------------------------------------------------------

\section{TSconfig \& TypoScript}
\begin{frame}[fragile]
	\frametitle{TSconfig \& TypoScript}

	\begin{center}\huge{Chapitre 2~:}\end{center}
	\begin{center}\huge{\color{typo3darkgrey}\textbf{TSconfig \& TypoScript}}\end{center}

\end{frame}

% ------------------------------------------------------------------------------
% LTXE-SLIDE-START
% LTXE-SLIDE-UID:		aee7c947-a271d451-cb640e43-101a5543
% LTXE-SLIDE-TITLE:		#78549: Override New Page Creation Wizard via page TSconfig
% ------------------------------------------------------------------------------

\begin{frame}[fragile]
	\frametitle{TSconfig \& TypoScript}
	\framesubtitle{Assistant de création d'une nouvelle page}

	\begin{itemize}
		\item Dans les versions précédentes de TYPO3 CMS, il était possible de
			surcharger l'«~Assistant de création d'une nouvelle page~» à l'aide
			d'un script personnalisé~:\newline
			\small
				\texttt{mod.web\_list.newPageWiz.overrideWithExtension = myextension}
			\normalsize
		\item La nouvelle manière de gérer les points d'entrées et scripts personnalisés
			est faite via les modules et routes. L'option citée ci-dessus est retirée
		\item La nouvelle option TSconfig suivante s'utilise à la place~:
			\small
				\texttt{mod.newPageWizard.override = my\_custom\_module}
			\normalsize

		\item À la place de la spécification d'une clé d'extension, un module ou
			une route personnalisée doit être spécifié

	\end{itemize}

\end{frame}
% ------------------------------------------------------------------------------
% LTXE-SLIDE-START
% LTXE-SLIDE-UID:		c56cb2e8-f50d7eed-e93dcb9e-ab7be017
% LTXE-SLIDE-TITLE:		#73626: Number of search results is configurable now
% ------------------------------------------------------------------------------
\begin{frame}[fragile]
	\frametitle{TSconfig \& TypoScript}
	\framesubtitle{Nombre de résultats de recherche}

	\begin{itemize}
		\item Le nombre maximal de résultats de recherche se configure en TypoScript~:\newline
			\texttt{plugin.tx\_indexedsearch.settings.blind.numberOfResults}
		\item Cette option contient une liste de valeurs
		\item Si le nombre de résultats de recherche est passé dans la requête et qu'il
			correspond à l'une des valeurs configurées, ce nombre est utilisé
		\item Si le nombre de résultats de recherche n'est pas passé dans la requête ou
			qu'il ne correspond à aucune des valeurs configurées, la première valeur
			est utilisée
		\item Pour garder la compatibilité, les valeurs par défaut sont~:\newline
			\texttt{10}, \texttt{25}, \texttt{50} et \texttt{100}
	\end{itemize}

\end{frame}
% ------------------------------------------------------------------------------
% LTXE-SLIDE-START
% LTXE-SLIDE-UID:		d9b9e73b-db575400-8276f4dd-810f4f4e
% LTXE-SLIDE-TITLE:		#78672: Fluid data processor for menus (1)
% ------------------------------------------------------------------------------
\begin{frame}[fragile]
	\frametitle{TSconfig \& TypoScript}
	\framesubtitle{Processeur de données Fluid pour les menus (1)}

	\begin{itemize}
		\item Le processeur de menu utilise \texttt{HMENU} pour générer le menu sous forme
		 	encodée JSON qui est décodée et assignée au \texttt{FLUIDTEMPLATE}
		\item Les traitements supplémentaires sont possibles et appliqués à chaque élément
		\item Options supportées~: \texttt{as}, \texttt{levels}, \texttt{expandAll}, \texttt{includeSpacer},
			\texttt{titleField}
			(voir \href{https://docs.typo3.org/typo3cms/TyposcriptReference/ContentObjects/Hmenu/Index.html}{TyposcriptReference} pour plus d'options)
	\end{itemize}

\end{frame}

% ------------------------------------------------------------------------------
% LTXE-SLIDE-START
% LTXE-SLIDE-UID:		03e756a0-07740056-a2009898-d6159f4b
% LTXE-SLIDE-TITLE:		#78672: Fluid data processor for menus (2)
% ------------------------------------------------------------------------------
\begin{frame}[fragile]
	\frametitle{TSconfig \& TypoScript}
	\framesubtitle{Processeur de données Fluid pour les menus (2)}

	% decrease font size for code listing
	\lstset{basicstyle=\tiny\ttfamily}

	\begin{itemize}
		\item Exemple de configuration TypoScript~:

			\begin{lstlisting}
				10 = TYPO3\CMS\Frontend\DataProcessing\MenuProcessor
				10 {
				  special = list
				  special.value.field = pages
				  levels = 7
				  as = menu
				  expandAll = 1
				  includeSpacer = 1
				  titleField = nav_title // title
				  dataProcessing {
				    10 = TYPO3\CMS\Frontend\DataProcessing\FilesProcessor
				    10 {
				      references.fieldName = media
				    }
				  }
				}
			\end{lstlisting}

	\end{itemize}

\end{frame}

% ------------------------------------------------------------------------------
% LTXE-SLIDE-START
% LTXE-SLIDE-UID:		4f6c99f3-cca1739f-205be2f4-3a0e3326
% LTXE-SLIDE-TITLE:		#72050: encapsLines does not render duplicate paragraphs anymore
% ------------------------------------------------------------------------------
\begin{frame}[fragile]
	\frametitle{TSconfig \& TypoScript}
	\framesubtitle{Function TypoScript \texttt{\_encapsLines}}

	\begin{itemize}
		\item La fonction TypoScript \texttt{\_encapsLines} transformait le
			saut de ligne de fin en deux paragraphe dans le contenu. C'est corrigé.

		\item Le changement peut affecter l'apparence en frontend si plusieurs paragraphes
			de fin vides existent dans le contenu RTE. Le dernier paragraphe n'est plus
			généré deux fois dans le frontend depuis TYPO3 CMS version 8.5.

	\end{itemize}

\end{frame}

% ------------------------------------------------------------------------------

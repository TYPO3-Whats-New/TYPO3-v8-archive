% ------------------------------------------------------------------------------
% TYPO3 CMS 8.5 - What's New - Chapter "TypoScript" (Spanish Version)
%
% @author	Michael Schams <schams.net>
% @license	Creative Commons BY-NC-SA 3.0
% @link		http://typo3.org/download/release-notes/whats-new/
% @language	English
% ------------------------------------------------------------------------------
% LTXE-CHAPTER-UID:		3a9852ea-e2360d9d-1ff5eec1-a7de3f9f
% LTXE-CHAPTER-NAME:	TypoScript
% ------------------------------------------------------------------------------

\section{TSconfig \& TypoScript}
\begin{frame}[fragile]
	\frametitle{TSconfig \& TypoScript}

	\begin{center}\huge{Capítulo 2:}\end{center}
	\begin{center}\huge{\color{typo3darkgrey}\textbf{TSconfig \& TypoScript}}\end{center}

\end{frame}

% ------------------------------------------------------------------------------
% LTXE-SLIDE-START
% LTXE-SLIDE-UID:		69890047-875953b2-8cf980c8-08d823fb
% LTXE-SLIDE-ORIGIN:	aee7c947-a271d451-cb640e43-101a5543 English
% LTXE-SLIDE-TITLE:		#78549: Override New Page Creation Wizard via page TSconfig
% ------------------------------------------------------------------------------

\begin{frame}[fragile]
	\frametitle{TSconfig \& TypoScript}
	\framesubtitle{Nuevo Asistente de Creación de Página}

	\begin{itemize}
		\item En versiones previas de TYPO3 CMS, era posible sobreescribir el "Asistente de Creación de Nueva Página"
			a través de scripts personalizados:\newline
			\small
				\texttt{mod.web\_list.newPageWiz.overrideWithExtension = myextension}
			\normalsize
		\item La nueva manera de manejar puntos de entrada y scripts personalizados es ahora llevada a cabo a través de módulos/rutas
			y la opción listada arriba ha sido eliminada
		\item La siguiente nueva opción TSconfig puede ser usada en su lugar:
			\small
				\texttt{mod.newPageWizard.override = my\_custom\_module}
			\normalsize

		\item En lugar de configurar la opción para una cierta clave de extensión, un módulo personalizado o
			ruta necesita ser especificado

	\end{itemize}

\end{frame}
% ------------------------------------------------------------------------------
% LTXE-SLIDE-START
% LTXE-SLIDE-UID:		88285694-c66d3829-e224a654-70bda900
% LTXE-SLIDE-ORIGIN:	c56cb2e8-f50d7eed-e93dcb9e-ab7be017 English
% LTXE-SLIDE-TITLE:		#73626: Number of search results is configurable now
% ------------------------------------------------------------------------------
\begin{frame}[fragile]
	\frametitle{TSconfig \& TypoScript}
	\framesubtitle{Número de Resultados de Búsqueda}

	\begin{itemize}
		\item El número máximo de resultados de búsqueda puede ser configurado en TypoScript ahora:\newline
			\texttt{plugin.tx\_indexedsearch.settings.blind.numberOfResults}
		\item Este ajuste almacena una lista de valores
		\item Si el número de resultados de búsqueda se pasa en la petición e iguala uno de los valores
			configurados, se usa este número
		\item Si el número de resultados de búsqueda no se pasa en la petición o no iguala cualquiera de
			los valores configurados, el primer valor de la lista es usado
		\item Para mantener compatibilidad hacia atrás, los valores por defecto son:\newline
			\texttt{10}, \texttt{25}, \texttt{50} y \texttt{100}
	\end{itemize}

\end{frame}
% ------------------------------------------------------------------------------
% LTXE-SLIDE-START
% LTXE-SLIDE-UID:		87980cea-108d0779-4efdc20b-0eff61d3
% LTXE-SLIDE-ORIGIN:	d9b9e73b-db575400-8276f4dd-810f4f4e English
% LTXE-SLIDE-TITLE:		#78672: Fluid data processor for menus (1)
% ------------------------------------------------------------------------------
\begin{frame}[fragile]
	\frametitle{TSconfig \& TypoScript}
	\framesubtitle{Procesador de Datos Fluid para Menús (1)}

	\begin{itemize}
		\item El procesador de menú utiliza \texttt{HMENU} para generar una cadena de menú codificada en JSON
			que es decodificada otra vez y asignada a \texttt{FLUIDTEMPLATE}
		\item DataProcessing adicional es soportado y aplicado a cada registro
		\item Opciones soportadas: \texttt{as}, \texttt{levels}, \texttt{expandAll}, \texttt{includeSpacer},
			\texttt{titleField}
			(ver \href{https://docs.typo3.org/typo3cms/TyposcriptReference/ContentObjects/Hmenu/Index.html}{TyposcriptReference} para más opciones)
	\end{itemize}

\end{frame}

% ------------------------------------------------------------------------------
% LTXE-SLIDE-START
% LTXE-SLIDE-UID:		6085c39e-a6886134-f865a45b-7563b730
% LTXE-SLIDE-ORIGIN:	03e756a0-07740056-a2009898-d6159f4b English
% LTXE-SLIDE-TITLE:		#78672: Fluid data processor for menus (2)
% ------------------------------------------------------------------------------
\begin{frame}[fragile]
	\frametitle{TSconfig \& TypoScript}
	\framesubtitle{Procesador de Datos Fluid para Menús (2)}

	% decrease font size for code listing
	\lstset{basicstyle=\tiny\ttfamily}

	\begin{itemize}
		\item Ejemplo de configuración TypoScript:

			\begin{lstlisting}
				10 = TYPO3\CMS\Frontend\DataProcessing\MenuProcessor
				10 {
				  special = list
				  special.value.field = pages
				  levels = 7
				  as = menu
				  expandAll = 1
				  includeSpacer = 1
				  titleField = nav_title // title
				  dataProcessing {
				    10 = TYPO3\CMS\Frontend\DataProcessing\FilesProcessor
				    10 {
				      references.fieldName = media
				    }
				  }
				}
			\end{lstlisting}

	\end{itemize}

\end{frame}

% ------------------------------------------------------------------------------
% LTXE-SLIDE-START
% LTXE-SLIDE-UID:		8cf98f18-a677dd4f-9fe839b6-395469f1
% LTXE-SLIDE-ORIGIN:	4f6c99f3-cca1739f-205be2f4-3a0e3326 English
% LTXE-SLIDE-TITLE:		#72050: encapsLines does not render duplicate paragraphs anymore
% ------------------------------------------------------------------------------
\begin{frame}[fragile]
	\frametitle{TSconfig \& TypoScript}
	\framesubtitle{Función TypoScript \texttt{\_encapsLines}}

	\begin{itemize}
		\item La función TypoScript \texttt{\_encapsLines} renderizaba dos parágrafos para un
			salto de línea vacío arrastrado en el contenido. Esto se ha solucionado ahora.

		\item El cambio posiblemente afecta a la apariencia en el frontend, si múltiPLes parágrafos vacíos
			existen en el contenido RTE. El último parágrafo no es renderizado dos veces
			en el frontend desde la versión TYPO3 CMS 8.5.

	\end{itemize}

\end{frame}

% ------------------------------------------------------------------------------

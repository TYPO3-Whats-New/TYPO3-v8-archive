% ------------------------------------------------------------------------------
% TYPO3 CMS 8.5 - What's New - Chapter "Extbase & Fluid" (Spanish Version)
%
% @author	Michael Schams <schams.net>
% @license	Creative Commons BY-NC-SA 3.0
% @link		http://typo3.org/download/release-notes/whats-new/
% @language	English
% ------------------------------------------------------------------------------
% LTXE-CHAPTER-UID:		846d40ce-66fdc2ec-750dcf95-33ce93e0
% LTXE-CHAPTER-NAME:	Extbase & Fluid
% ------------------------------------------------------------------------------

\section{Extbase \& Fluid}
\begin{frame}[fragile]
	\frametitle{Extbase \& Fluid}

	\begin{center}\huge{Capítulo 4:}\end{center}
	\begin{center}\huge{\color{typo3darkgrey}\textbf{Extbase \& Fluid}}\end{center}

\end{frame}

% ------------------------------------------------------------------------------
% LTXE-SLIDE-START
% LTXE-SLIDE-UID:		c964eb26-cbdceb52-029c551b-8bd0463a
% LTXE-SLIDE-ORIGIN:	c5986329-fb36f473-a5430d2e-6ba49fb1 English
% LTXE-SLIDE-TITLE:		#78116: Extbase support for Doctrine's native DBAL Statement and QueryBuilder
% ------------------------------------------------------------------------------
\begin{frame}[fragile]
	\frametitle{Extbase \& Fluid}
	\framesubtitle{Doctrine DBAL}

	% decrease font size for code listing
	\lstset{basicstyle=\tiny\ttfamily}

	\begin{itemize}
		\item Funcionalidad de consulta directa SQL también soporta objetos QueryBuilder e instancias de
			\texttt{\textbackslash
			Doctrine\textbackslash
			DBAL\textbackslash
			Statement} como sentencias preparadas
		\item El siguiente ejemplo funciona en cualquier repositorio Extbase usando sentencias nativas Doctrine DBAL:

			\begin{lstlisting}
				$connection = $this->objectManager->get(ConnectionPool::class)->getConnectionForTable('mytable');
				$statement = $this->objectManager->get(
				  \Doctrine\DBAL\Statement::class,
				  'SELECT * FROM mytable WHERE uid=? OR title=?',
				  $connection
				);

				$query = $this->createQuery();
				$query->statement($statement, [$uid, $title]);
			\end{lstlisting}
	\end{itemize}

\end{frame}
% ------------------------------------------------------------------------------
% LTXE-SLIDE-START
% LTXE-SLIDE-UID:		6024cd83-4896a75b-38a39e96-b7dae918
% LTXE-SLIDE-ORIGIN:	8d31d436-4397d4fd-aa0eed2e-df798aa3 English
% LTXE-SLIDE-TITLE:		#78002: Enforce cHash argument for Extbase actions
% ------------------------------------------------------------------------------

\begin{frame}[fragile]
	\frametitle{Extbase \& Fluid}
	\framesubtitle{\texttt{cHash} Argument}

	\begin{itemize}
		\item URIs para acciones Extbase ahora requieren un \texttt{cHash} válido por defecto\newline
			(acciones cacheadas y no cacheadas)
		\item El comportamiento puede deshabilitarse para todas las acciones usando el cambio de característica
			\texttt{requireCHashArgumentForActionArguments}
	\end{itemize}

\end{frame}
% ------------------------------------------------------------------------------
% LTXE-SLIDE-START
% LTXE-SLIDE-UID:		1f4afbaf-1eeb3051-49423a6c-fff81a76
% LTXE-SLIDE-ORIGIN:	50143a1e-53a17c8c-83ffe7ea-5e88cba0 English
% LTXE-SLIDE-TITLE:		#29399: OptionViewHelper and OptgroupViewHelper
% ------------------------------------------------------------------------------

\begin{frame}[fragile]
	\frametitle{Extbase \& Fluid}
	\framesubtitle{Contenido para ViewHelper \texttt{f:form.select}}

	% decrease font size for code listing
	\lstset{basicstyle=\tiny\ttfamily}

	\begin{itemize}
		\item Introducidos dos nuevos ViewHelpers permitiendo la definición manual de todas las opciones y
		    grupos de opciones para el \texttt{f:form.select} como contenido de tag del campo select

			\begin{itemize}
				\item \texttt{OptionViewHelper}
				\item \texttt{OptgroupViewHelper}
			\end{itemize}

		\item Ejemplo:

			\begin{lstlisting}
				<f:form.select name="myproperty">
				  <f:form.select.option value="1">Option one</f:form.select.option>
				  <f:form.select.option value="2">Option two</f:form.select.option>
				  <f:form.select.optgroup>
				    <f:form.select.option value="3">Grouped option one</f:form.select.option>
				    <f:form.select.option value="4">Grouped option twi</f:form.select.option>
				  </f:form.select.optgroup>
				</f:form.select>
			\end{lstlisting}

		\end{itemize}

\end{frame}


% ------------------------------------------------------------------------------
% LTXE-SLIDE-START
% LTXE-SLIDE-UID:		d1dd178f-616e82bf-c725b5c2-9eb29638
% LTXE-SLIDE-ORIGIN:	d36b9935-fc7d9e60-742a3f42-ef15e856 English
% LTXE-SLIDE-TITLE:		#78415: Global Fluid ViewHelper Namespace
% ------------------------------------------------------------------------------
\begin{frame}[fragile]
	\frametitle{Extbase \& Fluid}
	\framesubtitle{Espacio de nombres Global Fluid ViewHelper}

	\begin{itemize}
		\item Espacios de nombres Global Fluid ViewHelper son configurables ahora:\newline
			\smaller
				\texttt{\$GLOBALS['TYPO3\_CONF\_VARS']['SYS']['fluid']['namespaces']}
			\normalsize
		\item Esto permite que los espacios de nombres sean manipulados como parte de la configuración de la página
		\item Beneficios:

			\begin{itemize}
				\item Paquetes de ViewHelper de terceros pueden manipular el espacio de nombres global Fluid \texttt{f:}
				\item Paquetes de ViewHelper de terceros son capaces de registrar nuevos espacios de nombres globales como requeridos
				\item Desarrolladores de plantilla pueden usar dichos espacios de nombres global sin importarlos primero
					y pueden usarlos en todas las plantillas Fluid independientemente del contexto
			\end{itemize}

	\end{itemize}

\end{frame}


% ------------------------------------------------------------------------------
% LTXE-SLIDE-START
% LTXE-SLIDE-UID:		ba267789-fc988dc5-70300a5a-7a8583f4
% LTXE-SLIDE-ORIGIN:	e16e8cfd-26f35f09-23b473a2-442ff1e2 English
% LTXE-SLIDE-TITLE:		#78842: FLUIDTEMPLATE can mimic an actual Extbase web request
% ------------------------------------------------------------------------------
\begin{frame}[fragile]
	\frametitle{Extbase \& Fluid}
	\framesubtitle{\texttt{FLUIDTEMPLATE} puede Imitar Peticiones Web Extbase}

	% decrease font size for code listing
	\lstset{basicstyle=\small\ttfamily}

	\begin{itemize}
		\item El elemento de contenido \texttt{FLUIDTEMPLATE} puede imitar una petición web Extbase real ahora
		\item Esto hace posible el acceder a datos entregados, por ejemplo:

			\begin{lstlisting}
				$view->getRenderingContext()
				  ->getControllerContext()
				  ->getRequest()
				  ->getArguments();
			\end{lstlisting}

	\end{itemize}

\end{frame}




% ------------------------------------------------------------------------------
